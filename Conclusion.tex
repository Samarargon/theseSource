Notre travail était orienté selon deux axes~: la compréhension et le contrôle du démouillage à l'état solide des couches d'argent. Nous allons discuter de ces deux aspects l'un après l'autre.\par 
Nous avons mis en évidence lors de notre revue bibliographique la grande diversité des systèmes étudiés pendant le démouillage, ainsi que la diversité des morphologies de démouillage observées. Nous nous plaçons dans le cadre du démouillage des couches polycristallines. Il est usuellement décrit en deux étapes~: l'induction et la propagation des trous. Pendant l'induction, les grains de la couche grossissent, les joints de grains forment des sillons de plus en plus profonds et finalement des trous. Pendant le propagation, ces trous se propagent. L'évolution du front de démouillage est discutée en termes de courbure locale. Ceci donne lieu à la formation d'un bourrelet de démouillage~: le front est plus épais que le reste de la couche. Le moteur du démouillage est supposé être la réduction de l'énergie de surface de la couche métallique.\par 
Seuls quelques modèles récents font état du rôle des grains en tant qu'entités discrètes pendant le démouillage. Ces modèles impliquent également l'existence d'un bourrelet (constitué de grains). Ils prennent en compte la réduction de l'énergie à l'interface métal-substrat en plus de la réduction de l'énergie de surface dans la force motrice du démouillage.\par 
Nous avons mené des expériences \textit{in situ} de suivi du démouillage par plusieurs méthodes~: l'AFM, le MEB et l'éllipsométrie. Dans un premier temps, nous avons procédé à cette étude dans des atmopshères contenant de l'oxygène (air ou 400~Pa). Dans ce cadre, nous avons observé que certains grains spécifiques entraient en croissance extraordinaire dès l'induction. Ces grains accumulent tout le matériau provenant du démouillage et pilotent la morphologie de la couche lors de sa rétraction. Ceci implique qu'il n'y a pas de bourrelet de propagation~; la propagation des trous est un transport de matériau entre les grains consommés au niveau du front vers les grains en croissance. Le reste de la couche n'est pas modifié pendant la propagation, il n'y a pas de bourrelet. Les grains qui croissent donnent lieu aux particules observées à la fin du démouillage.\par 
Nous avons également pu observer que la propagation des trous ne donnait pas lieu à un état final, au sens où les particules obtenues à ce moment-là n'avaient pas la forme attendue par des conceptions thermodynamiques. Il existe en réalité un troisième régime dans le démouillage, que nous appelons frittage. Le frittage a une cinétique très lente, il correspond à l'évolution des particules d'argent isolées vers leur forme d'équilibre (ou du moins une forme plus proche de l'équilibre attendu). Le début du frittage est marqué par un ralentissement de la cinétique de l'évolution du taux de couverture. En termes de morphologie, c'est le moment où il ne reste plus aucun grain \og intact\fg{}: tous on soit disparu, soit crû. \par 
En outre, nous avons observé que le démouillage avait un impact sur le substrat~: celui-ci se déformait au niveau du front lorsque ce dernier marquait le début du régime de frittage.\par 
Nous avons procédé à une comparaison du démouillage en faisant varier l'atmosphère de recuit. Nous avons démontré que l'absence d'oxygène changeait radicalement la morphologie et la cinétique du démouillage. Sans oxygène, nous n'observons pas de croissance extréordinaire de certains grains~: plus de grains peuvent croitre, mais de manière modérée. Ces grains bloquent localement la propagation du front, lui donnant une forme dendritique.\par 
Nous avons développé et appliqué une procédure de traitement d'image pour mesurer la courbure locale du front, ainsi que la vitesse locale de propagation. Les données ainsi obtenues nous ont permis de démontrer que la courbure locale de la surface n'est pas le moteur de la propagation. Nous proposons que la force motrice de la propagation est la réorganisation des grains dans la couche qui se poursuit. \par 
L'influence de l'oxygène tient dans le fait qu'il abaisse l'énergie de surface de l'argent. Ceci permet une nucléation de trous plus aisée, et augmente ainsi la densité de trou observée lors du démouillage. Une étude en variations progressive de la pression partielle d'oxygène montre que cet effet a lieu dès des pressions de $10^{-7}$~atm ($10^{-2}$~Pa). Cette réduction d'énergie étant anisotrope, elle peut participer à l'émergence de croissance extraordinaire que nous n'observons que sous oxygène.\par 
D'un point de vue industriel, l'argent est systématiquement encapsulé dans des empilements. Nous avons cependant démontré que l'oxygène pouvait être apporté par la sous-couche. Ainsi, l'observation de défauts de type \og dômes \fg{} semble indiquer une présence de croissance extraordinaire caractéristique du démouillage sous oxygène. Au contraire, un démouillage de type \og dendrites \fg{} est au contraire similaire au démouillage en absence d'oxygène. Cela tend à montrer que la stabilité des couches d'argent dans les empilements est fortement dépendante de la composition des autres couches, même si elles ne sont pas directement en contact. La présence d'oxygène pouvant diffuser dans l'empilement est un facteur promoteur du démouillage.\par 
Nous avons appliqué ces connaissances au démouillage sur des surfaces texturées. Nos expériences étaient en désaccord avec le modèle proposé par Giermann et Thompson~\cite{giermann2005solid, giermann2011requirements}, qui prédit l'impossibilité d'une organisation des particules métalliques si le volume de métal excède le volume des trous accueillant les particules. Notre modèle implique que c'est la présence d'un grain en croissance extraordinaire au sein d'un trou qui détermine si une particule s'y trouve à la fin. Ceci permet alors d'avoir des particules de volume supérieur au volume des trous de la texture tout en gardant une très bonne organisation spatiale.\par 
Cette connaissance nous a permis d'optimiser les dépôt afin de fabriquer les meilleurs réseaux de particules possibles. Cette optimisation repose qsur trois facteurs~: l'épaisseur de la couche initiale, la puissance électrique fournie à la cathode lors du dépôt et la pression de gaz neutre constituant le plasma du dépôt.\par
Nous avons observé que les particules obtenues sur surface texturées épousent la forme des trous dans lesquels elles se situent. Elles présentent donc une partie en forme de pyramide, surmontée d'une hémisphère. Cette forme est différente de ce qu'observaient Giermann et Thompson.\par 
Nous avons mesuré la réponse optique des particules obtenues. Sur une surface plane, si l'on étudie l'absorption, seul un pic à 420~nm est visible. Sur une surface texturée, un deuxième pic émerge. La finesse de ce pic est dépendante de la qualité du réseau de particules, et sa position dépend de la période du réseau avec une relation linéaire. \par 
Des simulation optiques ont consolidé notre étude expérimentale. Elles retrouvent les même comportements, et indiquent la présence d'un troisième pic, lui aussi dépendant linéairement de la période du réseau. Ces simulations et ces observations nous permettent d'avancer que les pics d'absorption observés sont dûs au couplage de la lumière, via la diffraction, aux particules.\par 
Ce système est prometteur en termes d'applications~: il permet de positionner très précisément un pic d'absorption. Par exemple, en le positionnant dans l'IR, ceci permettrait de renforcer une propriétés de basse-émissivité d'un système. Pour le photovoltaïque, cette approche permettrait également de positionner un pic d'absorption en fonction du spectre solaire, afin de convertir le maximum de lumière en paires électrons-trous.\par 
\vspace{12pt}
La configuration de cette thèse était originale, puisqu'elle abordait la question du démouillage par un angle industriel. Cette approche imposait un système d'étude loin des cas modèles (une couche d'argent polycristalline sur une substrat amorphe), mais c'est précisément cette distance à l'idéalité qui nous a permis d'améliorer la compréhension du démouillage. En effet, observer d'aussi grandes différences morphologiques lors du démouillage en changeant seulement d'atmosphère est un résultat nouveau. Ceci nous a permis de proposer un nouveau modèle du démouillage.\par 
La connaissance du phénomène est précieuse en vue de l'amélioration des propriétés optiques des réseaux de particules obtenus par démouillage. Nous avons montré que le pic d'absorption généré par le réseau est indépendant du matériau qui constitue les particules. Ainsi, la recherche d'un autre métal à faire démouiller permettrait, tout en gardant le pic du réseau dont la position est contrôlée, de faire disparaître l'absorption dans les visible.\par 
Nous disposons d'un bon panorama des paramètres à prendre en compte en vue d'applications industrielles. À l'heure actuelle, la principale limite est la texturation de surface, qui ne s'applique pas à l'échelle indsutrielle (les dimensions typiques sont 6x3~m$^2$). Si ce verrou technologique est amené à disparaitre, nous disposons d'une base de travail solide pour optimiser la production de systèmes à forte valeur ajoutée. Les applications les plus immédiates correspondent au Bas-Émissif ainsi qu'aux cellules photovoltaïque.\par 

\vspace{12pt}
Du point de vue de la compréhension fondamentale, certaines zones d'ombre demeurent. Pourquoi certains grains croissent plutôt que d'autres, notamment. Il peut s'agir d'une délicate compétition de facteurs locaux (énergies de surface et d'interface au joints de grains) dont la mesure expérimentale est ardue. Notons de plus que la microscopie électronique, à ces échelles-là, perturbe beaucoup le phénomène. Des compléments de compréhension pourraient être apportés par des modélisations. La difficulté réside dans l'aspect multi-échelle du phénomène, aussi les simulations devront-elles sans doute être issues de différentes méthodes. Pour le moment, une simulation de type Vertex~\cite{weygand1998vertex} qui permet de prendre en compte la structure cristalline de la couche semble être un piste prometteuse.\par 
Concernant l'optique des réseaux, nous avons trouvé un modèle satisfaisant, mais la manière la plus directe de montrer sa pertinence serait de réaliser des études avec des angles d'incidence variables. Une étude dans ce sens, comme par exemple ce que l'ellipsométrie permet de réaliser~\cite{brakstad2015dispersion}, apporterait beaucoup d'éléments de compréhension.\par  
\newpage
\bibliographystyle{ieeetr}
\bibliography{biblio}
