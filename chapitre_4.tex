\minitoc
\newpage

Dans ce chapitre, nous allons discuter de l’influence de l’atmosphère de recuit sur le démouillage. L’intérêt de cette étude s’est imposé de lui-même dès les premières expériences réalisées, tant les observations diffèrent en fonction de l’atmosphère de recuit. Nous présenterons très globalement les changements observés en fonction de la présence d’oxygène lors du démouillage. Puis, pour plus de clarté, les observations et interprétations relatives à l’induction et à la propagation seront traitées dans deux parties successives. Ensuite, nous étudierons les effets d’un changement progressif d’atmosphère.\par 

\section{Changement d’atmosphère, observations cinétiques et statistiques}

		\subsubsection{Comparaison : vide et oxygène}
Les vidéos 1 et 4, représentent l’évolution d’une couche d’argent de 40 nm recuite respectivement dans une atmosphère d’oxygène ou sous vide, à un grossissement relativement faible (x3 000). Des images représentatives extraites de ces vidéos sont présentées sur les figures~\ref{MEBinSituLow} et \ref{MEBinSituLowVac} de la page~\ref{MEBinSituLow}. Le démouillage sous vide est très différent de celui que nous avons étudié dans le chapitre précédent. En termes de morphologie, les trous ont une forme irrégulière et présentent des dendrites. De la même manière que sous oxygène, nous pouvons également étudier l’évolution de quantités statistiques, tels que le taux de couverture et la densité de trous. Les résultats sont représentés sur la figure~\ref{MEBcomparaison} de la page~\pageref{MEBcomparaison}.\par 
\begin{figure}[!p]
\centering
\includegraphics[width = 0.8\textwidth]{MEBinSituLow}
\caption{Séquence d’images représentatives de la vidéo 1 (déjà utilisée dans le chapitre 3). Couche d’argent de 40 nm chauffée à 390~$^\circ$C sous 400~Pa d’oxygène, observée en MEB \textit{in situ} à un grossissement de x3000. Par rapport à l’image a), les images suivantes sont enregistrées avec un délai de b) 14~s, c) 28~s, d) 47~s et e) 1307~s.}
\label{MEBinSituLow}
\end{figure}
\begin{figure}[!p]
\centering
\includegraphics[width = 0.8\textwidth]{MEBinSituLowVac}
\caption{Séquence d’images représentatives de la vidéo 4. Couche d’argent de 40 nm chauffée à 390~$^\circ$C sous vide, observée en MEB \textit{in situ} à un grossissement de x3000. Par rapport à l’image a), les images suivantes sont enregistrées avec un délai de b) 39~s, c) 89~s, d) 142~s et e) 440~s.}
\label{MEBinSituLowVac}
\end{figure}
\begin{figure}[!p]
\centering
\includegraphics[width = 0.7\textwidth]{MEBcomparaison}
\caption{Évolution du taux de couverture et de la densité de trous en fonction du temps lors du démouillage d’une couche d’argent de 40 nm recuite à 350~$^\circ$C.}
\label{MEBcomparaison}
\end{figure}
Dans les deux cas, l’origine de l’axe des temps est pris au début de la propagation de trous. En comparant les données obtenues lors d’un recuit sous oxygène et sous vide, bien que l’allure soit semblable, plusieurs différences apparaissent. Le taux de couverture atteint à la fin de la propagation diffère légèrement (35~\% sous vide contre 32~\% sous oxygène). La densité de trous maximale sous vide est beaucoup plus faible que sous oxygène (0,10~\micro\meter$^{-2}$ contre 0,52~\micro\meter$^{-2}$). Enfin, la percolation des trous ainsi que le ralentissement des variations du taux de couverture arrivent beaucoup plus tard et de manière beaucoup plus progressive sous vide que sous oxygène.\par 
Cette différence dans la morphologie du démouillage n’avait jamais été observée pour un même système. Des comportement semblables ont été décrits par Kwon \textit{et al.}~\cite{kwon2003comparison} qui comparaient le démouillage de deux métaux différents, le cuivre et l’or, sur de la silice amorphe. Nous l’observons uniquement en changeant l’atmosphère de recuit. À cet égard, notre système est particulièrement intéressant, puisqu’il permet de faire le lien entre deux morphologies de démouillage très différentes en variant uniquement l’atmosphère de recuit.\par 
Notre objectif est donc d’expliquer, par le seul changement d’atmosphère, ce changement de la densité de trous, de leur forme et de leur propagation.\par 

		\subsubsection{Autres gaz dans l’atmosphère}
Avant d’aller plus loin, il convient de démontrer que l’oxygène est effectivement à l’origine de ces changements, et pas la simple présence de gaz. Pour cela, nous avons également procédé à un recuit sous atmosphère d’argon hydrogéné d’une couche d’argent. Une image MEB d’une couche de 80 nm recuit à 500~$^\circ$C (image prise dès la température atteinte) sous une atmosphère réductrice d’argon hydrogéné (Ar:H2 4~\%, 400~Pa) est présentée sur la figure~\ref{MEBargon} de la page~\pageref{MEBargon}. Le démouillage présente les mêmes caractéristiques que sous vide~: peu de trous, démouillage lent. Nous en déduisons que c’est bien l’oxygène qui est responsable des changements observés (pas la simple présence de gaz), ni les potentielles impuretés présentes dans les mélanges gazeux employés dans la chambre du microscope. Cette affirmation sera confirmée lors de la variation progressive de la pression partielle d'oxygène (section~D de ce chapitre).\par 
\begin{figure}[!p]
\centering
\includegraphics[width = 0.7\textwidth]{MEBargon}
\caption{Image MEB d’une couche d’argent de 80 nm recuite à 500~$^\circ$C dans une atmosphère d’argon hydrogéné (4~\%, 400~Pa).}
\label{MEBargon}
\end{figure}

\section{Influence de l'oxygène sur l'induction}

		\subsection{Observations expérimentales~: croissance cristalline}
Dans le chapitre III, nous avons présenté les mesures AFM \textit{in situ} réalisées sur une couche durant l’induction lors d’un recuit à l’air. Nous avions alors observé un grossissement de la taille moyenne des grains, ainsi qu’une augmentation de rugosité. Pour étudier l’induction sous vide, nous avons employé la technique ASTAR (voir le chapitre 2 pour plus d’explications). Une couche de 15 nm d’argent a été recuite dans la chambre du TEM à 75 puis 125~$^\circ$C. Nous avons réalisé in situ trois cartographies des orientations : dans l’état initial, après le premier recuit et après le second recuit. Ces cartographies sont représentées sur la figure 1.5. Bien que l’image C soit très différentes des deux autres, un repérage soigneux en imagerie directe nous a permis de nous assurer qu’il s’agissait bien de la même zone, malgré de possibles effets de dérive thermique. Nous pouvons faire plusieurs observations :
\begin{itemize}
\item la couche présente une multitude de grains dont la taille peut varier de quelques nanomètres à plusieurs centaines de nanomètres ;
\item à l'échelle de la statistique des images, il n’y a pas d’orientation cristalline privilégiée en apparence (avant ou pendant le recuit) ;
\item après le premier recuit, certains grains ont grossi (voir le grain pointé par la flèche rouge sur les images A et B), tandis que d’autres se sont scindés (voir grain pointé par la flèche bleue sur les images A et B) ;
\item après le second recuit (image C), les orientations de beaucoup de grains, observées localement, ont varié.
\end{itemize}
En outre, la taille mesurée moyenne des domaines cristallins a cru après les deux recuits, évoluant de 12 nm (état initial) à 17 nm (après le second recuit). Cette croissance moyenne reste conforme à ce qui est attendu, mais les valeurs numériques sont à considérer avec précaution : la validité statistique n’est pas vérifiable à cette échelle.\par 
Considérons les représentativités relatives des orientations dans la couche. Sur les cartographies, malgré la faible anisotropie, l’orientation (110) est moins représentée que les autres, mais quelques grains présentant cette orientation sont observables. Cela peut s’expliquer par la hiérarchie des énergies de surface sur les différentes facettes de l’argent, qui est la suivante : $\gamma(111) < \gamma(100) < \gamma(110)$ avec une faible anisotropie, de l’ordre de 1,08 entre les deux extrêmes~\cite{stankic2013equilibrium, molina2011size}. Cette observation est également cohérente avec les diffractogrammes DRX du chapitre précédent, qui indiquent que la fréquence d'apparition de cette orientation est extrêmement faible.\par 
Les cartographies ASTAR montrent qu’il y a une très importante réorganisation de la structure cristalline de la couche pendant l’induction. Cette réorganisation a lieu à très basse température : dès 75~$^\circ$C. Nous observons une croissance modérée de la taille moyenne des domaines cristallins, mais surtout une modification des orientations cristallines. Il est également surprenant de constater que si certains grains peuvent croître, d’autres peuvent se scindér.\par 
Par ailleurs, les mesures effectuées sur les séquences d’images (cf. figure~\ref{MEBcomparaison}) indiquent une grande différence dans la densité de trous formés, en fonction de l’atmosphère. Afin d’étudier la topographie de la couche, nous avons de plus réalisé des images AFM \textit{post mortem} d’une couche d’argent de 40 nm recuite à 150~$^\circ$C pendant 15~min sous air ou 300~$^\circ$C pendant 30~min sous vide. Les images AFM sont présentées sur la figure~\ref{AFMcomparaison}. Les densités de trous mesurées sont respectivement de 1,1~\micro\meter$^{-2}$ et 4,9.10$^{-3=$~\micro\meter$^{-2}$ (la densité de trous sous vide a été estimée sur une image MEB du même échantillon à plus faible grossissement). Il est intéressant de remarquer que la taille des grains, dans la couche non démouillée recuite sous vide, est très imposante : encore plus que sous oxygène. Ceci signifie que la réorganisation cristalline de la couche, que nous avons observée en ASTAR, continue d’avoir lieu. En effet, elle n'est pas perturbée par la formation et la propagation de trous dans la couche.
\begin{figure}[!htb]
\centering
\includegraphics[width = 0.7\textwidth]{AFMcomparaison}
\caption{Images AFM obtenues sur une couche d’argent de 40 nm recuite a) à 150~$^\circ$C pendant 15~min sous air et b) à 300~$^\circ$C pendant 30~min sous vide.}
\label{AFMcomparaison}
\end{figure}
\paragraph*{Remarque~:} Les densités de trous mesurées ici différent de celles mesurées en MEB \textit{in situ}. Ceci provient de la différence des conditions de recuit (température et durée), et du fait que l’étude \textit{post mortem} a été réalisée sur une série différente d’échantillons (voir chapitre III, section~A.\ref{sPostMortem}). Cette remarque ne modifie en rien l’observation d’une très forte réorganisation de la structure cristalline, quelle que soit l’atmosphère, ni la comparaison de la taille des grains, qui continue d’augmenter sous vide.\par 

		\subsection{Discussion}
Le concept d'induction a été proposé par Presland~\cite{presland1972hillock}, qui observait la croissance de buttes. Nous avons démontré dans le chapitre précédent que ces buttes étaient en réalité des grains en croissance extraordinaire. Les grains croissent par une réorganisation de la structure cristalline de la couche, que nous avons observée par AFM lors d’un recuit sous air, ou par ASTAR lors d’un recuit sous vide. Quelle que soit l’atmosphère, cette réorganisation a bien lieu.\par 
Cependant, nous avons observé une grande différence dans la densité de trous apparaissant lors du démouillage suivant l’atmosphère de recuit. Dans cette mesure, la réorganisation de la couche mène à des structures très différentes. Pour expliquer cela, nous pouvons nous référer à la théorie de Mullins~\cite{mullins1957theory} sur le sillonnement des joints de grains. À la ligne triple du joint de grains (contact entre les deux grains et le milieu ambiant), l’angle de contact $\theta_{GB}$ (voir figure~\ref{schemaGrainBoundary}) est déterminé par un rapport entre l’énergie de l’interface du joint de grain $\gamma_{GB}$ et l’énergie de surface du métal $\gamm_m$ par la loi de Young :
\begin{equation}
\gamma_{GB}/\gamma_m = 2\cos(\theta_{GB}/2).
\end{equation}
Ainsi, l’angle de contact dépend de l’énergie de surface du métal. Or, plusieurs travaux expérimentaux~\cite{buttner1952adsorption} ou théoriques~\cite{molina2011size} montrent que l’oxygène réduit l’énergie de surface de l’argent. Ainsi, la présence d’oxygène modifie l’angle de contact aux joints de grains. Si $\gamma_m$ diminue, alors le cosinus augmente, amenant $\theta_{GB}$ à diminuer. En d’autres termes, si on diminue l’énergie de surface $\gamma_m$, on privilégie la création de surface de métal libre entre deux grains, plutôt que de l’interface du joint de grains.\par 
\begin{figure}[!htb]
\centering
\includegraphics[width = 0.4\textwidth]{schemaGrainBoundary}
\caption{Schéma d’un joint de grain.}
\label{schemaGrainBoundary}
\end{figure}
Mais il faut souligner que la théorie de Mullins prévoit une évolution progressive du sillonnement des joints de grains. Dans notre cas, ce n’est pas ce que nous observons. Les réorganisations de la structure cristalline sont importantes, brusques et nombreuses. Ces réorganisations répondant à une minimisation de l’énergie, il est logique que les angles de contact apparaissant lors de l’induction tendent à s’approcher de l’angle d’équilibre théorique.\par 
Cette différence est schématisée sur la figure~\ref{schemaAngle}. Si l’on considère une couche constituée de grains, et qu’une réorganisation a lieu sous la forme de la fusion de deux grains, alors elle ne donnera pas le même résultat suivant l’atmosphère de recuit. Sous vide, l’angle de contact serait large, et donc les sillons des joints de grains peu profonds. Sous atmosphère d’oxygène, au contraire, cet angle serait petit et favoriserait le sillonnement.\par 
\begin{figure}[!htb]
\centering
\includegraphics[width = 0.9\textwidth]{schemaAngle}
\caption{Schéma de la reconstruction des grains dans la couche d’argent pendant l’induction, sous vide ou sous atmosphère d’oxygène.}
\label{schemaAngle}
\end{figure}
Cette conception tient bien compte des deux éléments mis en avant par les résultats expérimentaux : une très forte réorganisation de la structure, quelle que soit l’atmosphère, mais une tendance bien plus forte à former des trous sous oxygène. Nous avons cherché à quantifier cet effet. Pour cela, nous avons besoin des valeurs de l’énergie de surface et de l’énergie à l’interface des joints de grains. Plusieurs travaux de la littérature traitent de ce sujet, mais les valeurs peuvent varier énormément. Les travaux qui cherchent à quantifier ces valeurs mesurent en réalité les angles aux joints de grains, en déduisent les ratios  GB=m, et supposent la valeur de m. Sur la figure 1.9 sont représentées les valeurs prises dans les références~\cite{buttner1952adsorption} pour la surface libre et~\cite{inman1963interfacial, fiala1975surface, kudrman1969relative} pour les ratios  $\gamma_{GB}/\gamma_m$, pour l’argent. Nous avons sélectionné l’énergie de surface de Buttner comme référence, bien que celui-ci supposait déjà une valeur de $\gamma_{GB}$, car c’est le seul qui décrivait une évolution dans l’oxygène. Nous avons calculé les valeurs théoriques des angles de contact aux joints de grains en fonction de l’activité d’oxygène dans l’atmosphère, avec les différents ratios $\gamma_{GB}/\gamma_m$ de la littérature (voir figure~\ref{energiesFr}.\par 
\begin{figure}[!htb]
\centering
\includegraphics[width = 0.7\textwidth]{energiesFr}
\caption{Évolution des énergies de surfaces et de l’interface aux joints de grains en fonction de l’activité d’oxygène dans l’atmosphère, ainsi que l’angle d’équilibre prévu aux joints de grains. Les valeurs sont tirées de \cite{buttner1952adsorption, inman1963interfacial, fiala1975surface, kudrman1969relative} et concernent l’argent. L'oxygène est considéré comme un gaz idéal et sont activité est égale à sa pression partielle normalisée par $P^0$.}
\label{energiesFr}
\end{figure}
Buttner \textit{et al.}~\cite{buttner1952adsorption} prévoient une diminution de l’énergie de surface à partir d’une activité d’oxygène de $10^{-5}$. Comme eux, nous supposons que les énergies aux joints de grains ne sont pas soumises à cette évolution. Dans les trois travaux cités~\cite{inman1963interfacial, fiala1975surface, kudrman1969relative, un facteur~4 est observé entre l’énergie la plus haute et l’énergie la plus basse à l'interface des joints de grains. Un calcul en tenant compte de ces différentes valeurs donne des angles de joints de grains qui ont effectivement tendance à se réduire (et donc le joint à s’approfondir). La valeur la plus extrême (par Inman \textit{et al.}~\cite{inman1963interfacial}) prédirait même un angle de 0$^\circ$ sous air : une dissociation totale des grains.\par 
Les valeurs de la littérature ne permettent pas facilement de trancher sur une valeur numérique, tant elles sont diverses. Dans tous les cas, augmenter l’activité de l’oxygène tend à favoriser l’apparition de trous. Notons de plus que jusqu’ici, nous considérions que les énergies étaient isotropes et homogènes, ce que la présence de facettes dément~\cite{wulff1901xxv}. Le fait de travailler sur un substrat amorphe contribue à augmenter les degrés de liberté quant à l’orientation cristalline des grains les uns par rapport aux autres.\par 
Par exemple, l’énergie de surface des facettes est relativement isotrope sous vide, mais connaît une diminution plus prononcée pour la surface (100)~\cite{molina2011size}. Si c’est la surface exposée lors du sillonnement d’un joint de grain, le phénomène sera d’autant plus amplifié. Mais comme nous l’avions expliqué dans le chapitre précédent, il ne nous est pas expérimentalement possible de sonder les orientations relatives de grains au cours du démouillage. Une mise en oeuvre de modélisation pourrait, à cet égard, s’avérer fructueuse pour aller plus loin.\par 
\conclusion{Pendant l’induction, la structure cristalline de la couche se réorganise. Ces réorganisations sont des changements drastiques des orientations cristallines des grains, pas un simple sillonnement progressif. La présence d’oxygène dans l’atmosphère favorise les réorganisations qui donnent naissance à des trous. Notre modèle est cohérent avec les valeurs énergétiques de la littérature. Cependant il n’est pas possible de rentrer dans des considérations quantitatives.\par}
Notons enfin que Kwon \textit{et al.}~\cite{kwon2003comparison} observaient cette différence de densité de trous, qu’ils assignaient à un changement de mécanisme de formation. Si la densité de trous était importante (cas du cuivre), alors les trous provenaient du sillonnement des joints de grains, tandis que dans l’autre configuration, ils semblaient provenir d’accumulation de lacunes à un joint de grains (cas de l’or). Suite à nos travaux, nous sommes en mesure de proposer un unique mécanisme dépendant de l’atmosphère de recuit, par la variation de l'énergie de surface du métal. Dans cette mesure, le choix du système Ag/SiO$_2$ peut permettre d'appréhender sous un nouvel angle l’étude du démouillage d’autres couches minces polycristallines.\par 
\section{Influence de l'oxygène sur la propagation}
		\subsection{Description de notre approche}
			\subsubsection{Définition des objets étudiés}
			\subsubsection{Représentations des données et cas modèles}
		\subsection{Observations expérimentales~: courbure et vitesse de propagation}
		\subsection{Discussion}
			\subsubsection{Détermination de la force motrice du démouillage}
			\subsubsection{La diffusion, un facteur non limitant}
\section{Variations progressives d'atmosphère}
	\subsection{Densité de trous en fonction de la pression partielle d'oxygène}
	\subsubsection{Présence d'oxygène dans les empilements}
\subsection{Conclusion}
	
\newpage
\bibliographystyle{ieeetr}
\bibliography{biblio}

