\minitoc
\newpage

Dans ce chapitre, nous allons discuter de l’influence de l’atmosphère de recuit sur le démouillage. L’intérêt de cette étude s’est imposé de lui-même dès les premières expériences réalisées, tant les observations diffèrent en fonction de l’atmosphère de recuit. Nous présenterons très globalement les changements observés en fonction de la présence d’oxygène lors du démouillage. Puis, pour plus de clarté, les observations et interprétations relatives à l’induction et à la propagation seront traitées dans deux parties successives. Ensuite, nous étudierons les effets d’un changement progressif d’atmosphère.\par 

\section{Changement d’atmosphère, observations cinétiques et statistiques}

		\subsubsection{Comparaison : vide et oxygène}
Les vidéos 1 et 4, représentent l’évolution d’une couche d’argent de 40 nm recuite respectivement dans une atmosphère d’oxygène ou sous vide, à un grossissement relativement faible (x3 000). Des images représentatives extraites de ces vidéos sont présentées sur les figures~\ref{MEBinSituLow} et \ref{MEBinSituLowVac} de la page~\ref{MEBinSituLow}. Le démouillage sous vide est très différent de celui que nous avons étudié dans le chapitre précédent. En termes de morphologie, les trous ont une forme irrégulière et présentent des dendrites. De la même manière que sous oxygène, nous pouvons également étudier l’évolution de quantités statistiques, tels que le taux de couverture et la densité de trous. Les résultats sont représentés sur la figure~\ref{MEBcomparaison} de la page~\pageref{MEBcomparaison}.\par 
\begin{figure}[!p]
\centering
\includegraphics[width = 0.8\textwidth]{MEBinSituLow}
\caption{Séquence d’images représentatives de la vidéo 1 (déjà utilisée dans le chapitre 3). Couche d’argent de 40 nm chauffée à 390~$^\circ$C sous 400~Pa d’oxygène, observée en MEB \textit{in situ} à un grossissement de x3000. Par rapport à l’image a), les images suivantes sont enregistrées avec un délai de b) 14~s, c) 28~s, d) 47~s et e) 1307~s.}
\label{MEBinSituLow}
\end{figure}
\begin{figure}[!p]
\centering
\includegraphics[width = 0.8\textwidth]{MEBinSituLowVac}
\caption{Séquence d’images représentatives de la vidéo 4. Couche d’argent de 40 nm chauffée à 390~$^\circ$C sous vide, observée en MEB \textit{in situ} à un grossissement de x3000. Par rapport à l’image a), les images suivantes sont enregistrées avec un délai de b) 39~s, c) 89~s, d) 142~s et e) 440~s.}
\label{MEBinSituLowVac}
\end{figure}
\begin{figure}[!p]
\centering
\includegraphics[width = 0.7\textwidth]{MEBcomparaison}
\caption{Évolution du taux de couverture et de la densité de trous en fonction du temps lors du démouillage d’une couche d’argent de 40 nm recuite à 350~$^\circ$C.}
\label{MEBcomparaison}
\end{figure}
Dans les deux cas, l’origine de l’axe des temps est pris au début de la propagation de trous. En comparant les données obtenues lors d’un recuit sous oxygène et sous vide, bien que l’allure soit semblable, plusieurs différences apparaissent. Le taux de couverture atteint à la fin de la propagation diffère légèrement (35~\% sous vide contre 32~\% sous oxygène). La densité de trous maximale sous vide est beaucoup plus faible que sous oxygène (0,10~\micro\meter$^{-2}$ contre 0,52~\micro\meter$^{-2}$). Enfin, la percolation des trous ainsi que le ralentissement des variations du taux de couverture arrivent beaucoup plus tard et de manière beaucoup plus progressive sous vide que sous oxygène.\par 
Cette différence dans la morphologie du démouillage n’avait jamais été observée pour un même système. Des comportement semblables ont été décrits par Kwon \textit{et al.}~\cite{kwon2003comparison} qui comparaient le démouillage de deux métaux différents, le cuivre et l’or, sur de la silice amorphe. Nous l’observons uniquement en changeant l’atmosphère de recuit. À cet égard, notre système est particulièrement intéressant, puisqu’il permet de faire le lien entre deux morphologies de démouillage très différentes en variant uniquement l’atmosphère de recuit.\par 
Notre objectif est donc d’expliquer, par le seul changement d’atmosphère, ce changement de la densité de trous, de leur forme et de leur propagation.\par 

		\subsubsection{Autres gaz dans l’atmosphère}
Avant d’aller plus loin, il convient de démontrer que l’oxygène est effectivement à l’origine de ces changements, et pas la simple présence de gaz. Pour cela, nous avons également procédé à un recuit sous atmosphère d’argon hydrogéné d’une couche d’argent. Une image MEB d’une couche de 80 nm recuit à 500~$^\circ$C (image prise dès la température atteinte) sous une atmosphère réductrice d’argon hydrogéné (Ar:H2 4~\%, 400~Pa) est présentée sur la figure~\ref{MEBargon} de la page~\pageref{MEBargon}. Le démouillage présente les mêmes caractéristiques que sous vide~: peu de trous, démouillage lent. Nous en déduisons que c’est bien l’oxygène qui est responsable des changements observés (pas la simple présence de gaz), ni les potentielles impuretés présentes dans les mélanges gazeux employés dans la chambre du microscope. Cette affirmation sera confirmée lors de la variation progressive de la pression partielle d'oxygène (section~D de ce chapitre).\par 
\begin{figure}[!p]
\centering
\includegraphics[width = 0.7\textwidth]{MEBargon}
\caption{Image MEB d’une couche d’argent de 80 nm recuite à 500~$^\circ$C dans une atmosphère d’argon hydrogéné (4~\%, 400~Pa).}
\label{MEBargon}
\end{figure}

\section{Influence de l'oxygène sur l'induction}

		\subsection{Observations expérimentales~: croissance cristalline}
Dans le chapitre III, nous avons présenté les mesures AFM \textit{in situ} réalisées sur une couche durant l’induction lors d’un recuit à l’air. Nous avions alors observé un grossissement de la taille moyenne des grains, ainsi qu’une augmentation de rugosité. Pour étudier l’induction sous vide, nous avons employé la technique ASTAR (voir le chapitre 2 pour plus d’explications). Une couche de 15 nm d’argent a été recuite dans la chambre du TEM à 75 puis 125~$^\circ$C. Nous avons réalisé in situ trois cartographies des orientations : dans l’état initial, après le premier recuit et après le second recuit. Ces cartographies sont représentées sur la figure 1.5. Bien que l’image C soit très différentes des deux autres, un repérage soigneux en imagerie directe nous a permis de nous assurer qu’il s’agissait bien de la même zone, malgré de possibles effets de dérive thermique. Nous pouvons faire plusieurs observations :
\begin{itemize}
\item la couche présente une multitude de grains dont la taille peut varier de quelques nanomètres à plusieurs centaines de nanomètres ;
\item à l'échelle de la statistique des images, il n’y a pas d’orientation cristalline privilégiée en apparence (avant ou pendant le recuit) ;
\item après le premier recuit, certains grains ont grossi (voir le grain pointé par la flèche rouge sur les images A et B), tandis que d’autres se sont scindés (voir grain pointé par la flèche bleue sur les images A et B) ;
\item après le second recuit (image C), les orientations de beaucoup de grains, observées localement, ont varié.
\end{itemize}
En outre, la taille mesurée moyenne des domaines cristallins a cru après les deux recuits, évoluant de 12 nm (état initial) à 17 nm (après le second recuit). Cette croissance moyenne reste conforme à ce qui est attendu, mais les valeurs numériques sont à considérer avec précaution : la validité statistique n’est pas vérifiable à cette échelle.\par 
Considérons les représentativités relatives des orientations dans la couche. Sur les cartographies, malgré la faible anisotropie, l’orientation (110) est moins représentée que les autres, mais quelques grains présentant cette orientation sont observables. Cela peut s’expliquer par la hiérarchie des énergies de surface sur les différentes facettes de l’argent, qui est la suivante : $\gamma(111) < \gamma(100) < \gamma(110)$ avec une faible anisotropie, de l’ordre de 1,08 entre les deux extrêmes~\cite{stankic2013equilibrium, molina2011size}. Cette observation est également cohérente avec les diffractogrammes DRX du chapitre précédent, qui indiquent que la fréquence d'apparition de cette orientation est extrêmement faible.\par 
Les cartographies ASTAR montrent qu’il y a une très importante réorganisation de la structure cristalline de la couche pendant l’induction. Cette réorganisation a lieu à très basse température : dès 75~$^\circ$C. Nous observons une croissance modérée de la taille moyenne des domaines cristallins, mais surtout une modification des orientations cristallines. Il est également surprenant de constater que si certains grains peuvent croître, d’autres peuvent se scindér.\par 
Par ailleurs, les mesures effectuées sur les séquences d’images (cf. figure~\ref{MEBcomparaison}) indiquent une grande différence dans la densité de trous formés, en fonction de l’atmosphère. Afin d’étudier la topographie de la couche, nous avons de plus réalisé des images AFM \textit{post mortem} d’une couche d’argent de 40 nm recuite à 150~$^\circ$C pendant 15~min sous air ou 300~$^\circ$C pendant 30~min sous vide. Les images AFM sont présentées sur la figure~\ref{AFMcomparaison}. Les densités de trous mesurées sont respectivement de 1,1~\micro\meter$^{-2}$ et 4,9.10$^{-3=$~\micro\meter$^{-2}$ (la densité de trous sous vide a été estimée sur une image MEB du même échantillon à plus faible grossissement). Il est intéressant de remarquer que la taille des grains, dans la couche non démouillée recuite sous vide, est très imposante : encore plus que sous oxygène. Ceci signifie que la réorganisation cristalline de la couche, que nous avons observée en ASTAR, continue d’avoir lieu. En effet, elle n'est pas perturbée par la formation et la propagation de trous dans la couche.
\begin{figure}[!htb]
\centering
\includegraphics[width = 0.7\textwidth]{AFMcomparaison}
\caption{Images AFM obtenues sur une couche d’argent de 40 nm recuite a) à 150~$^\circ$C pendant 15~min sous air et b) à 300~$^\circ$C pendant 30~min sous vide.}
\label{AFMcomparaison}
\end{figure}
\paragraph*{Remarque~:} Les densités de trous mesurées ici différent de celles mesurées en MEB \textit{in situ}. Ceci provient de la différence des conditions de recuit (température et durée), et du fait que l’étude \textit{post mortem} a été réalisée sur une série différente d’échantillons (voir chapitre III, section~A.\ref{sPostMortem}). Cette remarque ne modifie en rien l’observation d’une très forte réorganisation de la structure cristalline, quelle que soit l’atmosphère, ni la comparaison de la taille des grains, qui continue d’augmenter sous vide.\par 

		\subsection{Discussion}
Le concept d'induction a été proposé par Presland~\cite{presland1972hillock}, qui observait la croissance de buttes. Nous avons démontré dans le chapitre précédent que ces buttes étaient en réalité des grains en croissance extraordinaire. Les grains croissent par une réorganisation de la structure cristalline de la couche, que nous avons observée par AFM lors d’un recuit sous air, ou par ASTAR lors d’un recuit sous vide. Quelle que soit l’atmosphère, cette réorganisation a bien lieu.\par 
Cependant, nous avons observé une grande différence dans la densité de trous apparaissant lors du démouillage suivant l’atmosphère de recuit. Dans cette mesure, la réorganisation de la couche mène à des structures très différentes. Pour expliquer cela, nous pouvons nous référer à la théorie de Mullins~\cite{mullins1957theory} sur le sillonnement des joints de grains. À la ligne triple du joint de grains (contact entre les deux grains et le milieu ambiant), l’angle de contact $\theta_{GB}$ (voir figure~\ref{schemaGrainBoundary}) est déterminé par un rapport entre l’énergie de l’interface du joint de grain $\gamma_{GB}$ et l’énergie de surface du métal $\gamm_m$ par la loi de Young :
\begin{equation}
\gamma_{GB}/\gamma_m = 2\cos(\theta_{GB}/2).
\end{equation}
Ainsi, l’angle de contact dépend de l’énergie de surface du métal. Or, plusieurs travaux expérimentaux~\cite{buttner1952adsorption} ou théoriques~\cite{molina2011size} montrent que l’oxygène réduit l’énergie de surface de l’argent. Ainsi, la présence d’oxygène modifie l’angle de contact aux joints de grains. Si $\gamma_m$ diminue, alors le cosinus augmente, amenant $\theta_{GB}$ à diminuer. En d’autres termes, si on diminue l’énergie de surface $\gamma_m$, on privilégie la création de surface de métal libre entre deux grains, plutôt que de l’interface du joint de grains.\par 
\begin{figure}[!htb]
\centering
\includegraphics[width = 0.4\textwidth]{schemaGrainBoundary}
\caption{Schéma d’un joint de grain.}
\label{schemaGrainBoundary}
\end{figure}
Mais il faut souligner que la théorie de Mullins prévoit une évolution progressive du sillonnement des joints de grains. Dans notre cas, ce n’est pas ce que nous observons. Les réorganisations de la structure cristalline sont importantes, brusques et nombreuses. Ces réorganisations répondant à une minimisation de l’énergie, il est logique que les angles de contact apparaissant lors de l’induction tendent à s’approcher de l’angle d’équilibre théorique.\par 
Cette différence est schématisée sur la figure~\ref{schemaAngle}. Si l’on considère une couche constituée de grains, et qu’une réorganisation a lieu sous la forme de la fusion de deux grains, alors elle ne donnera pas le même résultat suivant l’atmosphère de recuit. Sous vide, l’angle de contact serait large, et donc les sillons des joints de grains peu profonds. Sous atmosphère d’oxygène, au contraire, cet angle serait petit et favoriserait le sillonnement.\par 
\begin{figure}[!htb]
\centering
\includegraphics[width = 0.9\textwidth]{schemaAngle}
\caption{Schéma de la reconstruction des grains dans la couche d’argent pendant l’induction, sous vide ou sous atmosphère d’oxygène.}
\label{schemaAngle}
\end{figure}
Cette conception tient bien compte des deux éléments mis en avant par les résultats expérimentaux : une très forte réorganisation de la structure, quelle que soit l’atmosphère, mais une tendance bien plus forte à former des trous sous oxygène. Nous avons cherché à quantifier cet effet. Pour cela, nous avons besoin des valeurs de l’énergie de surface et de l’énergie à l’interface des joints de grains. Plusieurs travaux de la littérature traitent de ce sujet, mais les valeurs peuvent varier énormément. Les travaux qui cherchent à quantifier ces valeurs mesurent en réalité les angles aux joints de grains, en déduisent les ratios  GB=m, et supposent la valeur de m. Sur la figure 1.9 sont représentées les valeurs prises dans les références~\cite{buttner1952adsorption} pour la surface libre et~\cite{inman1963interfacial, fiala1975surface, kudrman1969relative} pour les ratios  $\gamma_{GB}/\gamma_m$, pour l’argent. Nous avons sélectionné l’énergie de surface de Buttner comme référence, bien que celui-ci supposait déjà une valeur de $\gamma_{GB}$, car c’est le seul qui décrivait une évolution dans l’oxygène. Nous avons calculé les valeurs théoriques des angles de contact aux joints de grains en fonction de l’activité d’oxygène dans l’atmosphère, avec les différents ratios $\gamma_{GB}/\gamma_m$ de la littérature (voir figure~\ref{energiesFr}.\par 
\begin{figure}[!htb]
\centering
\includegraphics[width = 0.7\textwidth]{energiesFr}
\caption{Évolution des énergies de surfaces et de l’interface aux joints de grains en fonction de l’activité d’oxygène dans l’atmosphère, ainsi que l’angle d’équilibre prévu aux joints de grains. Les valeurs sont tirées de \cite{buttner1952adsorption, inman1963interfacial, fiala1975surface, kudrman1969relative} et concernent l’argent. L'oxygène est considéré comme un gaz idéal et sont activité est égale à sa pression partielle normalisée par $P^0$.}
\label{energiesFr}
\end{figure}
Buttner \textit{et al.}~\cite{buttner1952adsorption} prévoient une diminution de l’énergie de surface à partir d’une activité d’oxygène de $10^{-5}$. Comme eux, nous supposons que les énergies aux joints de grains ne sont pas soumises à cette évolution. Dans les trois travaux cités~\cite{inman1963interfacial, fiala1975surface, kudrman1969relative, un facteur~4 est observé entre l’énergie la plus haute et l’énergie la plus basse à l'interface des joints de grains. Un calcul en tenant compte de ces différentes valeurs donne des angles de joints de grains qui ont effectivement tendance à se réduire (et donc le joint à s’approfondir). La valeur la plus extrême (par Inman \textit{et al.}~\cite{inman1963interfacial}) prédirait même un angle de 0$^\circ$ sous air : une dissociation totale des grains.\par 
Les valeurs de la littérature ne permettent pas facilement de trancher sur une valeur numérique, tant elles sont diverses. Dans tous les cas, augmenter l’activité de l’oxygène tend à favoriser l’apparition de trous. Notons de plus que jusqu’ici, nous considérions que les énergies étaient isotropes et homogènes, ce que la présence de facettes dément~\cite{wulff1901xxv}. Le fait de travailler sur un substrat amorphe contribue à augmenter les degrés de liberté quant à l’orientation cristalline des grains les uns par rapport aux autres.\par 
Par exemple, l’énergie de surface des facettes est relativement isotrope sous vide, mais connaît une diminution plus prononcée pour la surface (100)~\cite{molina2011size}. Si c’est la surface exposée lors du sillonnement d’un joint de grain, le phénomène sera d’autant plus amplifié. Mais comme nous l’avions expliqué dans le chapitre précédent, il ne nous est pas expérimentalement possible de sonder les orientations relatives de grains au cours du démouillage. Une mise en oeuvre de modélisation pourrait, à cet égard, s’avérer fructueuse pour aller plus loin.\par 
\conclusion{Pendant l’induction, la structure cristalline de la couche se réorganise. Ces réorganisations sont des changements drastiques des orientations cristallines des grains, pas un simple sillonnement progressif. La présence d’oxygène dans l’atmosphère favorise les réorganisations qui donnent naissance à des trous. Notre modèle est cohérent avec les valeurs énergétiques de la littérature. Cependant il n’est pas possible de rentrer dans des considérations quantitatives.\par}
Notons enfin que Kwon \textit{et al.}~\cite{kwon2003comparison} observaient cette différence de densité de trous, qu’ils assignaient à un changement de mécanisme de formation. Si la densité de trous était importante (cas du cuivre), alors les trous provenaient du sillonnement des joints de grains, tandis que dans l’autre configuration, ils semblaient provenir d’accumulation de lacunes à un joint de grains (cas de l’or). Suite à nos travaux, nous sommes en mesure de proposer un unique mécanisme dépendant de l’atmosphère de recuit, par la variation de l'énergie de surface du métal. Dans cette mesure, le choix du système Ag/SiO$_2$ peut permettre d'appréhender sous un nouvel angle l’étude du démouillage d’autres couches minces polycristallines.\par 
\section{Influence de l'oxygène sur la propagation}
		\subsection{Description de notre approche}
L’une des principales différences induite par la modification de l’atmosphère est la forme des trous : nous observons des dendrites sous vide, mais pas sous oxygène. Au delà de cette observation qualitative, est-il possible de différencier de manière explicite la propagation des trous en fonction de l’atmosphère ? Pour répondre à cette question, nous avons développé un méthode d’analyse spécifique que nous allons décrire maintenant.\par 
Les images obtenues à un grossissement de x3~000 ne permettent pas l'évolution de la forme du front de démouillage de manière statisfaisante, car la délimitation  obtenue par segmentation n'est pas repérée de façon suffisamment précise. Nous travaillons donc avec des séquences d'images réalisées à un grossissement x10~000 ou x20~000, suivant les cas.\par 
Notre étude s’est portée en particulier sur la vitesse de propagation locale du front en fonction de sa courbure locale. En effet, la courbure semble varier à première vue suivant l’atmosphère de recuit, notamment à cause de la présence des digitations. Nous allons ici aborder le développement d’une méthode d’analyse construite spécifiquement pour étudier spécifiquement le démouillage sous cet angle.\par 

			\subsubsection{Définition des objets étudiés}
\paragraph*{La courbure :} La courbure $\kappa$ d’un arc $\mathcal{C}$ dans le plan se définit de plusieurs manières équivalentes~; nous retiendrons pour plus de facilité qu’il s’agit de l’inverse du rayon de courbure $r$, rayon du cercle tangent à l’arc au point considéré. Dans le plan, il est possible de donner un sens à la courbure en fonction de la direction de la concavité de l’arc.\par 
La courbure d’une surface S dans l’espace peut être définie de plusieurs manières. Dans notre cas, nous référons aux travaux de Taylor~\cite{taylor1992ii}. Nous considérons la courbure moyenne, qui se définit par rapport aux \og courbures principales \fg~:
\begin{equation}
\kappa = \dfrac{\kappa_1+\kappa_2}{2}.
\label{eCourb}
\end{equation}
Le schéma présenté sur la figure~\ref{schemaPlans} permet d’illustrer la manière de trouver les courbures principales d’une surface. On définit d’abord le plan tangent à la surface au point considéré. Perpendiculairement à ce plan tangent, tout plan interceptera la surface $\mathcal{S}$ par un arc, dont la courbure est facilement accessible. Les courbures principales sont les courbures maximale et minimale, obtenues dans des plans perpendiculaires entre eux (et perpendiculaires au plan tangent).\par
Il existe de plus un théorème stipulant que la courbure moyenne ne dépend pas du choix des plans perpendiculaires, tant qu’il sont orthogonaux entre eux ~\cite{taylor1992ii}. Le schéma~\ref{schemaPlansApplique} illustre l’application de ce théorème à la surface d'un domaine d’argent, au niveau du front de démouillage. Nous considérons la courbure dans le plan du substrat, $\kappa_\parallel$ et la courbure perpendiculaire à ce plan, $\kappa_\perp$. Cette hypothèse implique qu’au niveau du front de démouillage, l’angle de contact entre la couche et le substrat est de 90$^\circ$, pour que le plan tangent soit perpendiculaire au substrat. Il est difficile de mesurer cet angle expérimentalement, ce qui nous oblige à simplement accepter cette approximation. C’est par ailleurs ce qui a été implicitement supposé dans les modèles de la littérature~\cite{brandon1966mobility, jiran1990capillary}.\par 
lorsqu’on observe la couche en vue de dessus, nous accédons uniquement à la courbure $\kappa_\parallel$. Par convention, la courbure sera considérée comme positive si la concavité de l’arc est orientée à l’intérieur du métal. Une particule d’argent aura une courbure positive, tandis qu’un trou dans un milieu continu aura une courbure négative. Sur la figure~\ref{schemaPlansApplique}, les deux courbures sont positives. Sur le schéma~\ref{schemaPlans}, elles sont de signes opposés.\par 

\begin{figure}[!htb]
\centering
\includegraphics[width = 0.7\textwidth]{schemaPlans}
\caption{Schéma montrant pour une surface les plans tangent et perpendiculaires. L’interception de la surface par les plans perpendiculaires est représentée par les arcs pointillés, dont les courbures au point considéré sont les courbures principales.}
\label{schemaPlans}
\end{figure}
\begin{figure}[!htb]
\centering
\includegraphics[width = 0.7\textwidth]{schemaPlansApplique}
\caption{Schéma de l'application de la définition de la courbure à une surface d’un domaine d’argent.}
\label{schemaPlansApplique}
\end{figure}

\paragraph*{Remarque~:} La définition de la courbure a été discutée par Taylor~\cite{taylor1992ii}, qui néglige pour des raisons de traitement mathématique le facteur 1/2 dans l’expression~\ref{eCourb}, et qui introduit le concept de \og courbure moyenne pondérée \fg. Cette courbure concerne les surfaces présentant une anisotropie d’énergie, et repose sur des questions énergétiques. La courbure devient alors~:
\begin{equation}
\kappa = a_1\kappa_1+a_2\kappa_2
\end{equation}
où les facteurs $a_i$ dépendent de l’énergie de surface dans la direction considérée, ainsi que sa variation dans la direction orthogonale. Dans un repère polaire d'un arc de courbure principale, pour un point de la surface de coordonnées $(r,\theta)$, $a=\gamma+\partial\gamma/\partial\theta$. Précisons que par \og direction  \fg, Taylor entend toutes les directions mathématiques possibles, pas seulement les directions correspondant aux faibles indices de Miller. Il ne nous est pas possible d'accéder expérimentalement aux coefficients $a_i$ car l'orientation cristalline de la surface à cette échelle nous est inaccessible. Nous mettons de côté ces raffinements, et nous rediscuterons de cette hypothèse par la suite.\par 

\paragraph*{La vitesse locale~:} Considérons deux images successives, $n$ et $n + 1$ d'une séquence de démouillage. Entre ces deux images, il y a eu une petite évolution du front. Sur chaque point du front l’image $n$, nous pouvons calculer le vecteur normal au front. En propageant ce vecteur jusqu’au front de l’image $n+1$, nous obtenons une distance, qui divisée par la temps d'acquisition, donne la vitesse de propagation. La figure~\ref{schemaCourbure} illustre le principe de cette mesure.\par 
Ce calcul simple est valable dans le cas où les images sont suffisamment proches. S’il y a trop d’évolution du front entre deux images, les vecteurs normaux de plusieurs points du contour de l’image $n$ risquent de se croiser avant d’atteindre le contour de l’image $n + 1$ ; cela donne des distances surévaluées. Nous veillons donc à prendre une fréquence d’acquisition d’images relativement élevée afin de garder une distance petite entre deux fronts successifs.\par 

\begin{figure}[!htb]
\centering
\includegraphics[width = 0.9\textwidth]{schemaCourbure}
\caption{Schéma présentant le principe de la mesure de la courbure locale et de la vitesse de propagation en un point du contour d’une image $n$.}
\label{schemaCourbure}
\end{figure}

			\subsubsection{Représentations des données et cas modèles}
Pour chaque point du front sur chaque image, nous obtenons à partir de l'analyse d'images MEB deux valeurs : la courbure locale $\kappa_\parallel$ et la vitesse de propagation $v$. Pour une séquence d’images, nous comptons le nombre de points obtenus par couple de coordonnées $(\kappa_\parallel,v)$. Ceci nous donne un histogramme, que nous choisissons de projeter dans le plan des coordonnées, en attribuant un code couleur à l’intensité.\par 
Enfin, il est nécessaire de normaliser les données obtenues. En effet, les courbures ne sont pas isoprobables sur le front, et ce déséquilibre peut entraîner une mauvaise interprétation. Deux expériences de pensée permettent d’illustrer ce phénomène.\par 
Premièrement, considérons un front circulaire se propageant à vitesse constante $v_0$ (c’est-à-dire que la vitesse est indépendante de la courbure). La longueur du front va augmenter au fur et à mesure du temps, ce qui va augmenter le nombre de points considérés sur le front (la distance $d$ entre ces points reste constante dans notre analyse). Ainsi, plus le rayon $r$ du cercle sera grand, plus la quantité de points de coordonnées $(1/r,v_0)$ sera grande. Si l’on ne normalise pas l’histogramme, on pourrait donc croire que la vitesse augmente avec le rayon. Il est bien important de remarquer que la surface du disque va bien croître de manière quadratique avec le temps, mais que nous ne considérons ici que des données locales : un point individuel sur le front, lorsqu’il se déplace, balaye toujours la même surface $d \cdot v_0 \cdot t$ avec $d$ la distance entre points, et $t$ l’intervalle de temps entre deux images successives.\par
Deuxièmement, considérons un front dont l’allure est représentée sur la figure~\ref{schemaNormalisation}. Il s'agit de quatre digitations, trois ont une courbure égale $\kappa_b$ et la dernière a une courbure différente $\kappa_a$. L'image a) est une situation initiale. En b), c) et d), la digitation de courbure $\kappa_a$ se propage à une vitesse arbitraire $v_0$. On aurait donc $n(\kappa_a,v_0)=3$. Par ailleurs, les digitations de courbure $\kappa_b$ se propagent une fois chacune. On a donc $n(\kappa_b,v_0)=3$. Cependant, on constate que la digitation de courbure $\kappa_a$ se propage plus rapidement (car plus fréquemment). Il ne faut donc pas considérer le compte total, mais normaliser ce compte par la fréquence d'apparition des courbures considérées.\par 
\begin{figure}[!htb]
\centering
\includegraphics[width = 0.6\textwidth]{schemaNormalisation}
\caption{Schéma d'un front constitué de quatre digitations se propageant à des vitesse différentes.}
\label{schemaNormalisation}
\end{figure}
Ainsi, nous consiédrons la quantité~:
\begin{equation}
P(v|\kappa) = \dfrac{n(\kappa,v)}{n(\kappa)},
\end{equation}
qui n'est autre que l'expression mathématique de la probabilité conditionnelle, sachant la courbure, d'observer une vitesse donnée.\par \vspace{12pt}
Pour plus de facilité pour la lecture des résultats expérimentaux, nous  présentons d'abord des cas modèles. Ces cas font référence à la figure~\ref{modelesFr}, où sont représentés des front de démouillage théoriques en cours de propagation et les histogrammes associés. Dans le cas a), le front est linéaire et se propage à une vitesse $v_0$. la courbure dans le plan $\kappa_\parallel = 0$. Ses coordonnées sont donc dans un unique point : $(0, v_0)$. Dans le cas b), le front est circulaire (correspondant à un trou se propageant), de courbure $\kappa =1/r(t)$ et se déplace à une vitesse $v_0$. À $t$, il est donc représenté en un unique point $(1/r(t), v_0)$. Lorsque $r(t)$ augmente, la courbure diminue, le point se rapproche de l’axe $\kappa = 0$. Dans le cas c), on s’intéresse à la propagation d’un trou en forme de doigt à une vitesse $v_0$. Le front est de courbure nulle sur les côtés et circulaire au bout, avec un rayon $r$. La courbure est donc soit nulle, soit de $1/r$. Les points où la propagation a lieu se situent à l’extrémité seulement, où ils ont tous une courbure $1/r$. La vitesse dépend cependant de la position des points sur la partie circulaire du doigt; la propagation du front est donc représentée sur l'histogramme par un ensemble de points tels que $\kappa = 1/r$ et $0 \geq v \geq v_0$.\par 

\begin{figure}[!htb]
\centering
\includegraphics[width = 0.8\textwidth]{modelesFr}
\caption{Cas modèles de propagation de front et les histogrammes de $P(v|Ç\kappa)$ associés.}
\label{modelesFr}
\end{figure}

		\subsection{Observations expérimentales~: courbure et vitesse de propagation}
Comme nous l’avions vu dans le chapitre précédent, la morphologie du démouillage est indépendante de la température de recuit dans une large gamme. Nous considérons qu’elle ne varie pas tant que l’argent n’est pas sublimé. Or, ce phénomène n’est significatif qu’à partir d’une température de 700~$^\circ$C. De plus, dans notre étude, il est difficile expérimentalement de se placer dans des conditions de recuit identiques pour comparer le démouillage sous vide ou sous oxygène. En effet, si l’on voulait observer un temps de recuit raisonnable sous vide, alors le démouillage sous oxygène est trop rapide.\par 
Nous n’avons donc pas cherché à nous placer dans des conditions de recuit identiques entre les deux atmosphères, mais seulement à favoriser la progression du démouillage, quitte à augmenter la température sous vide. Des couches d’argent de 60 nm (respectivement 80~nm) ont été observées au MEB lors d’un recuit à 230~$^\circ$C (respectivement 450~$^\circ$C) sous oxygène (respectivement vidéos 5 et 7)  ou à 480~$^\circ$C (respectivement 570~$^\circ$C) sous vide (respectivement vidéos 6 et 8). Les résultats sont représentés sur la figure~\ref{proba60} (respectivement \ref{proba80}) de la page~\pageref{proba60}.
Des séquences d'images des vidéos~7 et 8 (couches de 80~nm) sont présentées sur les figures~\ref{MEBgrainsOx} et \ref{MEBgrainsVac} de la page~\pageref{MEBgrainsOx}.\par 
\begin{figure}[!p]
\centering
\includegraphics[width = 0.5\textwidth]{probaOx60fr}\includegraphics[width = 0.5\textwidth]{probaVac60fr}
\caption{Probabilité sachant la courbure d’observer une vitesse de propagation locale donnée $P(v|\kappa)$ observée lors de la propagation d’un front de démouillage sur une couche d’argent de 60 nm. Gauche : sous oxygène, à 230~$^\circ$C. Droite : sous vide, à 430~$^\circ$C.}
\label{proba60}
\end{figure}
\begin{figure}[!p]
\centering
\includegraphics[width = 0.5\textwidth]{probaOx80fr}\includegraphics[width = 0.5\textwidth]{probaVac80fr}
\caption{Probabilité sachant la courbure d’observer une vitesse de propagation locale donnée $P(v|\kappa)$ observée lors de la propagation d’un front de démouillage sur une couche d’argent de 80 nm. Gauche : sous oxygène, à 450~$^\circ$C. Droite : sous vide, à 570~$^\circ$C.}
\label{proba80}
\end{figure}

\begin{figure}[!p]
\centering
\includegraphics[width = \textwidth]{MEBgrainsOx}
\caption{Images extraites de la vidéo 7 : couche d’argent de 80 nm démouillant sous atmosphère de 100~Pa d’oxygène à 450~$^\circ$C. Sur les images du bas, des grains ont été surlignés en bleu, les flèches rouges indiquent les directions possibles de propagation. L'intervalle de temps d'acquisition est de 40~s entre les deux premières images et de 1~min entre les deux dernières.}
\label{MEBgrainsOx}
\end{figure}
\begin{figure}[!p]
\centering
\includegraphics[width = \textwidth]{MEBgrainsVac}
\caption{Images extraites de la vidéo 8 : couche d’argent de 80 nm démouillant sous vide à 570~$^\circ$C. Sur les images du bas, des grains ont été surlignés en bleu, les flèches rouges indiquent les directions possibles de propagation. Les images ont été acquises avec un intervalle de 20~s entre chacune.}
\label{MEBgrainsVac}
\end{figure}
Considérons tout d'abord les histogrammes (figures~\ref{proba60} et \ref{proba80}). Les deux épaisseurs étudiées montrent des tendances similaires, dans les deux atmosphères de recuit. Sous vide, plus la courbure $\kappa$ tend vers des valeurs très négatives, plus la vitesse de propagation et la probabilité d’évolution augmentent. Cette augmentation est valable jusqu’à des valeurs de l’ordre de -18~\micro\meter$^{-1}$, soit un rayon de courbure de 55~nm. Sous oxygène, la probabilité semble être relativement constante, indépendamment de la courbure $\kappa$.\par 
Ceci montre que la forme du front est bien dépendante de l’atmosphère. Sous vide, la propagation au fortes courbures négatives est interprétée comme la propagation de doigts (Cf.les cas modèles de la figure~\ref{modelesFr}), tandis qu’elle semble isotrope sous oxygène.\par 
Considérons maintenant les séquences d'images (figures~\ref{MEBgrainsOx} et \ref{MEBgrainsVac}). Ces séquences mettent en lumière une autre différence est remarquable si l’on compare les démouillages sous vide ou sous oxygène. Nous avons discuté, dans le chapitre précédent, de la croissance extraordinaire d’un nombre restreint de grains qui pilotaient la morphologie du démouillage. Cette croissance, observée sous oxygène, n’est plus observée sous vide.\par
Afin de mieux visualiser ce comportement, sur les images des figures~\ref{MEBgrainsOx} et \ref{MEBgrainsVac}, nous avons représenté des images successives en surlignant en bleu les limites des plus gros grains. Les flèches rouges indiquent les directions possibles de propagation, c’est-à-dire les parties du front en contact avec les grains dont la hauteur n’a pas augmenté. Si un grain grossit, il devient un obstacle, et le front doit le contourner pour progresser. Sous vide, ces grains sont nombreux et sont à une grande proximité du front ; celui-ci ne peut alors progresser qu’entre des grains, ce qui donne lieu à la digitation. En revanche, sous oxygène, du fait de la sélection d’un nombre très restreint de grains, le front n’est pas bloqué et peut se propager beaucoup plus librement.\par 
Ces observations expérimentales montrent bien que d’importantes modifications sont apportées par l’oxygène pendant la propagation : la forme des trous est changée. Cela rappelle beaucoup les travaux effectuées par Kwon\textit{ et al.}~\cite{kwon2003comparison} qui comparaient de l’or et du cuivre, mais dans notre cas, un seul paramètre change : la présence d’oxygène. Dans la partie suivante, nous allons chercher à expliquer ces différences avec un modèle simple.\par 

\paragraph*{Remarque (relative à la définition de la courbure)~:}
Nous avions évoqué l’approximation que constituait notre définition de la courbure vis-à-vis d’un modèle tenant compte des anisotropies, tel que défini par Taylor~\cite{taylor1992ii}. Il faudrait, en toute rigueur, tenir compte des variations de l’énergie de surface ainsi que sa dérivée dans les différentes directions (le coefficient $a=\gamma+\partial\gamma/\partial\theta$). Cependant, une cartographie ASTAR réalisée sur une couche d’argent de 20 nm recuite à 250~$^\circ$C pendant 12~min sous vide, présentée sur la figure~\ref{astarVacDem} nous permet de constater que l’orientation cristalline est toujours aussi aléatoire au bout des doigts que sur les bords. Nous pouvons donc penser que l’orientation cristalline reste suffisamment aléatoire pour que la légère anisotropie des énergies n’influence pas le démouillage (c'est-à-dire que les coefficients $a_i$ soient semblables en tout point), et donc que notre définition de la courbure reste valable.\par 

\begin{figure}[!htb]
\centering
\includegraphics[width=0.75\textwidth]{AstarVacDem}
\caption{Cartographie Astar d’une couche d’argent de 20 nm recuite à 250~$^\circ$C pendant 12~min sous vide.}
\label{astarVacDem}
\end{figure}

		\subsection{Discussion}

			\subsubsection{Détermination de la force motrice du démouillage}
Nous avons montré que la propagation des trous était différente selon l’atmosphère de recuit. Dans beaucoup de travaux~\cite{presland1972hillock, jiran1990capillary, zucker2013model}, la courbure de la surface au niveau du front est considérée comme le moteur du démouillage. La diffusion solide est reliée à une différence de potentiel chimique, le potentiel s’exprimant par~:
\begin{equation}
\mu = \mu^0 +\gamma \Omega \kappa
\end{equation}
avec $\gamma$ l’énergie de surface et $\Omega$ le volume atomique. Jiran et Thompson~\cite{jiran1990capillary} avaient notamment proposé un modèle de démouillage où le front se propageait aux endroits où il était le plus mince, car la courbure $\kappa_\perp$ y était maximale. Ils supposaient alors que $\kappa_\perp \gg \kappa_\parallel$. Nos observations montrent d’une part que cette hypothèse n’est pas vraie, mais surtout que le modèle de courbure n’est pas applicable. En effet, si l’on regarde la vitesse de propagation du front sous vide, on s’aperçoit qu’elle est maximale précisément lorsque $\kappa\parallel = 1/h$, avec $h$ l’épaisseur du film. Or, $h$ est le rayon de courbure minimal dans la direction orthogonale, donc la courbure $\kappa_\perp\leq 1/h$. Si l’on regarde maintenant la courbure moyenne, ceci implique :
\begin{equation}
\kappa = \dfrac{\kappa_\perp+\kappa_\parallel}{2}\leq 0.
\end{equation}
Ainsi, un modèle basé sur la variation du potentiel chimique lié à la courbure prévoit que l’extrémité des doigts sous vide serait l’endroit où le potentiel est le plus bas, donc une zone d’agglomération de la matière. Nous observons expérimentalement l’inverse : c’est l’endroit où le démouillage est le plus rapide.\par 
\conclusion{Nous concluons donc que la courbure « macroscopique » (dont l’inverse est comparable à l’épaisseur de la couche) n’est pas responsable de la morphologie du démouillage. Ceci implique deux choses : le modèle supposant que c’est la force motrice de la propagation n’est pas valable, et l’explication des différences en fonction de l’atmosphère est à trouver ailleurs.\par}
Or, nous observons que la croissance de grains est radicalement différente en fonction de l’atmosphère de recuit (voir les figures page~\pageref{MEBgrainsOx}) : sous oxygène, peu de grains concentrent la matière provenant du démouillage, tandis que sous vide, l’agglomération est beaucoup plus répartie entre des grains proches du front. Nous considérons que la propagation se fait exclusivement sur les zones de la couche qui ont gardé l’épaisseur initiale, ce qui est équivalent à dire que les grains, une fois qu’ils ont grossi, bloquent la propagation du front.\par 
Cette simple considération permet d’expliquer pourquoi, sous oxygène, la vitesse est indépendante de la courbure : le front se propage dans toutes les directions, il est rarement contraint par des gros grains. Sous vide, la raison pour laquelle la propagation se fait là où la courbure est la plus négative est moins immédiate, mais néanmoins justifiée : c’est au bout des doigts que l’on trouve plus facilement un couche encore intacte, tandis que l’argent s’accumule majoritairement sur les bords.\par 
Considérons à nouveau les modèles de la littérature, résumés sur la figure~\ref{schemaPropagation}. Dans un premier temps, Jiran et Thompson~\cite{jiran1990capillary} considéraient que la propagation se faisait au niveau des zones d’amincissement du bourrelet du front, qui étaient situées au bout des doigts (figure~\ref{schemaPropagation}-a). Kosinova \textit{et al.}~\cite{kosinova2014role} et Atiya \textit{et al.}~\cite{atiya2014role}, plus récemment, ont mis en avant le rôle des grains dans le démouillage. Ce faisant, ils gardent l’idée d’un bourrelet qui progresse dans la couche (figure~\ref{schemaPropagation}-b). La progression serait assurée par une succession de croissance puis réduction des grains au niveau du front.\par 
Notre description du démouillage ne correspondent à aucun de ces modèles. Sous vide, si nos observations sont en accord avec celles de Jiran, nous avons montré en revanche que le modèle de courbure n’était pas applicable. Parallèlement, nous sommes en accord sur l’importance des grains dans le modèle employé par Kosinova \textit{et al.} ou Atiya \textit{et al.}, mais leur manière de décrire la progression du front n’est pas en accord avec nos observations : il n’y a pas de bourrelet se déplaçant au niveau du front. Notre conception du démouillage se démarque donc des deux modèles précédents.\par 
\conclusion{Le démouillage n’est pas induit par la courbure locale. Il correspond à la disparition des grains les plus petits, dont la matière s’accumule dans d’autres grains. Les grains ayant grossi bloquent la propagation du front, qui se fait donc là où la couche n’a pas gagné en épaisseur : sous vide, la propagation a majoritairement lieu au bout des doigts (figure~\ref{schemaPropagation}-c). Sous oxygène, elle est très peu contrainte (figure~\ref{schemaPropagation}-d).\par }
\begin{figure}[!htb]
\centering
\includegraphics[width=0.8\textwidth]{schemaPropagation}
\caption{Modèles pour décrire la propagation du front lors du démouillage a) selon Jiran et Thompson~\cite{jiran1990capillary}, b) selon Kosinova \textit{et al.} ou Atiya \textit{et al.}~\cite{kosinova2014role, atiya2014role}, c) selon nos travaux, sous vide et d) selon nos travaux, sous oxygène.}
\label{schemaPropagation}
\end{figure}

\paragraph*{Remarque~:} Nous avons conclu que la courbure n’est pas la force motrice de la transformation. Précisons que cette assertion est vraie pour la propagation, mais pas pour le frittage. En effet, le frittage, tel qu’observé dans le chapitre III et la section B.2.\ref{sFrittage}, au sein d’un îlot d’argent, amène les zones de fortes courbures (petites protubérances) à alimenter les zones de faibles courbures. Cette transformation est illustrée sur la figure~\ref{MEBfrittage} de la page~\ref{MEBfrittage}. Or, nous avons vu que le frittage était très lent par rapport à la propagation. Cette observation corrobore la présence, pendant la propagation, d’une force motrice d’une autre nature, qui accélère le processus.\par \vspace{12pt}

La question de la nature de cette force motrice pendant la propagation demeure donc. Les travaux de Kosinova \textit{et al.} et Atiya \textit{et al.}~\cite{kosinova2014role, atiya2014role} définissent une force motrice liée aux différentes énergies de surface et d'interface du système. Pour des grains cubiques de taille identique, elle s’écrit $\Delta\gamma = \gamma_m+\gamma_i-\gamma_S$, avec les énergies respectivement de surface du métal, de l’interface et de surface du substrat. Cette description est en contradiction avec nos observations : une diminution de $\gamma_m$ entraînerait une diminution de la force motrice, or la propagation est plus rapide sous oxygène, lorsque l’énergie de surface est la plus faible (notons que nous considérons $\gamma_i$ indépendante de l’atmosphère comme Buttner~\cite{buttner1952adsorption}, et Parikh~\cite{parikh1958effect} a observé par mesures mécaniques que $\gamma_s$ ne varie pas en fonction de la présence d’oxygène). Ainsi, la diminution d’énergie de surface du système ne semble pas suffisante pour expliquer nos observations.\par 
Pendant le démouillage, un autre processus n’est pas gouverné par la diminution de la surface libre : c’est l’induction. Pendant cette étape, la réorganisation des grains est certes influencée par l’énergie de surface, mais beaucoup plus par les énergies aux joints de grains, ainsi qu’à l’interface. De ce point de vue, nous considérons la propagation comme la prolongation de l’induction, mais avec un phénomène supplémentaire : la propagation des trous.\par 
\conclusion{Ceci implique que la réduction au cours du temps de l’énergie aux joints de grains et à l’interface est le
véritable moteur de la propagation. La manière dont cette réduction s'opère dépend de l'énergie de surface du métal, et donc de l'atmosphère de recuit dans le case de l'argent.\par }

			\subsubsection{La diffusion, un facteur non limitant}
Durant toute la discussion, nous n’avons pas abordé la question de la diffusion. Plusieurs travaux suggèrent que l’oxygène augmente l’auto-diffusion de l’argent sur l’argent~\cite{rhead1963surface, yoshihara1979effect}, comme étudié par exemple dans le cas du cuivre~\cite{bradshaw1964surface}. Pourtant, nous n’avons pas retenu cette hypothèse comme déterminante dans notre modèle, et cela pour deux raisons.\par 
La première est liée à l’affirmation de l’augmentation de la diffusion ne présence d'oxygène elle-même : dans les publications traitantdu sujet, la méthode employée pour mesurer la variation de diffusion est la mesure cinétique du sillonnement des joints de grains. Ce phénomène, rationalisé par Mullins~\cite{mullins1957theory} (cf. chapitre I), indique que le sillonnement suit une loi :
\begin{equation}
w = 4,6(Bt)^{1/4},
\end{equation}
avec $w$ l’écartement entre les deux maxima avoisinant le joint, et $B = D\gamma_m \Omega^2\nu/kT$, avec $D$ le
coefficient d’auto-diffusion, $\gamma_m $ l’énergie de surface, $\Omega$ le volume atomique, $\nu$ la densité surfacique
d’adatomes et $kT$ dans son acception usuelle en thermodynamique. Or, dans les différents travaux, une amplification de l’évolution de $w$ sous gaz est interprétée comme une augmentation de $D$, en supposant $\gamma_M$ constante. Au vu de la discussion dans les travaux de Buttner~\cite{buttner1952adsorption}, cette hypothèse n’est pas vérifiée, et il est très difficile de conclure sur un réelle augmentation de $D$. Notons que Rhead~\cite{rhead1963surface} évoque dans sa publication le problème de l’apparition de nouvelles facettes lors de recuits sous oxygène, liée à la modification des énergies de surface. Dans son étude, il ne caractérise donc que les joints de grains où il n’observe pas, en apparence, ce phénomène de facettage, mais rien n'indique que l'augmentation de $w$ ne soit pas liée au même phénomène. Nous en concluons que l’augmentation de la diffusion, en plus d’être mal quantifiée, n’est pas absolument certaine. Nous préférons à ce titre retenir la diminution des énergies de surface comme le facteur déterminent.\par 
La seconde raison nous appelle à comparer notre système à celui de Kwon \textit{et al.}~\cite{kwon2003comparison} Par comparaison des morphologies, nous mettons en parallèle le démouillage de l’argent sous vide avec celui de l’or d’une part (on parlera du démouillage de type \og vide \fg), et le démouillage de l’argent sous oxygène avec celui du cuivre d’autre part (on parlera du type \og oxygène \fg). Remarquons que cette comparaison est légitime dans la mesure ou il s’agit de systèmes semblables : couches polycristallines, métaux cfc, substrat de silice amorphe. Dans les différents systèmes étudiés, nous pouvons comparer les morphologies et les coefficients de diffusion : les résultats sont présentés dans le tableau~\ref{tDiffusion}. On remarque rapidement que les coefficients de diffusion jouent en sens contraire : le coefficient de diffusion de l’or à 700~$^\circ$C (type \og vide \fg) est plus important que celui du cuivre à 300~$^\circ$C (type \og oxygène \fg). Or, si on garde l’hypothèse d’une augmentation du coefficient de l’auto-diffusion de l’argent sous oxygène, il devient donc plus grand dans le type \og oxygène \fg{} que dans le type \og vide \fg. L’effet est donc opposé ; la diffusion n’est pas la raison principale du changement de morphologie.\par 
\conclusion{Pour ces deux raisons, nous considérons que la diffusion n’est pas un facteur limitant,
quelle que soit l’atmosphère.\par}
\begin{table}
\centering
\begin{tabular}{ccccc}
\hline
système & Ag (vide, 200~$^\circ$C) & Au (atm. réduc., 700~$^\circ$C) & Ag (air ambiant, 200~$^\circ$C) & Cu (atm. réduc., 300~$^\circ$C)\\
\hline
morphologie & type vide & type vide & type oxygène & type oxygène\\
D (cm2.s$^{-1}$) &  $\approx$ 10$^{-8}$ & 4,6.10$^{-6}$ & $\approx$ 10$^{-6}$* & 7,8.10$^{-7}$\\
\end{tabular}
\caption{Morphologies observées pendant le démouillage et valeurs des coefficients d’autodiffusion des métaux obtenues par calcul dans la publication~\cite{agrawal2002predicting}. *) Si l’on conserve l’hypothèse d’un coefficient plus important sous oxygène, en extrapolant les travaux de Rhead~\cite{rhead1963surface}, comme expliqué dans le chapitre I, section C.}
\label{tDiffusion}
\end{table}
Ainsi, lorsqu’on observe une sélection des grains sous oxygène, dont la croissance a lieu à un micron du front, et pas sous vide, nous pensons que la différence n’est pas expliquée par la diffusion. D’autres pistes sont à envisager :
\begin{itemize}
\item une limitation par l’adsorption d’oxygène : certains travaux estiment que l’oxygène ne s’adsorbe pas sur les facettes (111)~\cite{engelhardt1976adsorption}. L’adsorption d’oxygène pourrait limiter l’agglomération de l’argent partout ailleurs que sur ces faces \og nues \fg ;
\item une limitation par l’énergie de surface : selon d’autres travaux~\cite{molina2011size}, la présence d'oxygène abaisserait beaucoup plus l'énergie de la surface (100) que celle des autres orientations, ce qui pourrait expliquer une croissance préférentielle dans cette orientation.
\end{itemize}
Toute la question demeure dans la facilité qu’a un atome d’argent de s’extraire d’une surface pour devenir un adatome libre, diffusant vers d’autres sites plus favorables~\cite{combe00}. Si cette étape est facile, alors la sélectivité de la croissance correspondrait à un équilibre gouverné par les rapports d’énergie uniquement. Si au contraire l’extraction d’adatomes est limitante, alors le processus ne serait pas équilibré et la sélection serait uniquement liée au premier endroit où les adatomes peuvent se fixer : là où il n’y a pas d’adsorbat. Il ne nous est pas possible de trancher entre ces deux possibilités, il est également possible que les deux coexistent.\par
Notons que la morphologie observée du démouillage de l’argent sous oxygène est très proche de celle du fer sous hydrogène~\cite{kovalenko2013solid} . Dans cette étude, Kovalenko \textit{et al.} attribuaient la sélection à un résultat de la diffusion à l’interface ou aux joints de grains, arguant que la diffusion de surface aurait tendance à gommer les trop grandes différences de tailles entre grains voisins si elle avait lieu. Nous avons montré dans le chapitre III que la diffusion de surface restait prépondérante ; nous n’adhérons pas à leurs arguments. En revanche, il est intéressant de noter que Grenga \textit{et al.}~\cite{grenga1976surface}, en observant les modifications de particules de Fe en présence de $H_2$, on noté deux effets : à la fois une adsorption différente suivant les facettes et un changement de l’anisotropie de l’énergie de surface, observations cohérentes avec nos hypothèses dans le cas de l'oxygène sur l'argent.\par 

\section{Variations progressives d'atmosphère}
Jusqu’ici, nous avons considéré deux cas extrêmes d’atmosphère de recuit, entre le vide secondaire et une atmosphère « riche » en oxygène (pression partielle >100 Pa). Nous allons ici présenter ce qu’il advient sous des pressions partielles d’oxygène intermédiaires, ou dans des empilements enrichis en oxygène.\par 
	\subsection{Densité de trous en fonction de la pression partielle d'oxygène}
Nous avons étudié en MEB in situ le démouillage sous plusieurs pressions partielles d’oxygène. Étant donné que le démouillage connaît de grandes variations de cinétique en fonction de l’atmosphère de recuit, nous n’avons pas pris le parti d’étudier le démouillage en intégralité pour chaque atmosphère. À la place, nous avons fixé arbitrairement une température (350~$^\circ$C) et observé l’évolution d’une couche de 40 nm sous différentes atmosphères. Ces atmosphères sont obtenues en introduisant dans la chambre du MEB de l'air, de l'oxygène pur, un mélange $N_2:O_2$ (1000~ppm) ou un mélange $N_2:O_2$ (100~ppm), ainsi qu'en variant la pression entre 400 et 10~Pa. Lors du démouillage, des trous apparaissent. Nous avons simplement mesuré la densité maximale de trous au court d’une rampe de température de 10~$^\circ$C/min jusqu’à la température finale. Cette densité maximale était soit celle mesurée à 350~$^\circ$C (si le démouillage est lent), soit celle mesurée au cours du recuit (si le démouillage est rapide, les trous percolent avant 350~$^\circ$C). La vitesse de la rampe a été fixée, car nous observons que la densité maximale dépend de la rampe. Une rampe lente favorise davantage la propagation des premiers trous formés face à la création de nouveaux trous ; nous avons donc travaillé avec une rampe aussi rapide que les conditions techniques le permettent. Les résultats sont présentés dans la figure~\ref{pressionPartielle}.\par 
\begin{figure}[!htb]
\centering
\includegraphics[width=0.7\textwidth]{pressionPartielle}
\caption{Densité  maximale de trous maximale mesurée lors du démouillage de couches d’argent de 40~nm lors d’une rampe jusqu’à 350~$^\circ$C (10~$^\circ$C/min) en fonction de l’activité d’oxygène dans l’atmosphère de recuit. Deux séries ont été réalisées (symboles verts et bleus).}
\label{pressionPartielle}
\end{figure}
Deux séries différentes ont subi ce traitement (points bleus et points verts)~; nous pouvons en observer la cohérence sur la figure 1.21. Aux très basses pressions d’oxygène (activité <5.10$^{-7}$), la densité maximale de trous ne varie pas. Elle augmente de manière linéaire à partir d’un seuil aux alentours de 5.10$^{-7}$-10$^{-6}$, jusqu’à la pression d’oxygène dans l’air.\par 
Comme nous l’avions calculé auparavant (voir figure~\ref{energiesFr}), l’énergie de surface a une influence sur l’angle de contact au niveau des joints de grains. Une diminution de l’énergie mène à un plus grand nombre de trous dans la couche. Ainsi, les variations de densité de trous que nous mesurons sont révélatrices de l’abaissement de l’énergie de surface de l’argent en présence d'une pression partielle d'oxygène qui conduit à un taux de recouvrement significatif des surfaces/interfaces d'argent. Cet effet est visible pour une activité aussi faible que 5.10$^{-7}$, ce qui est un seuil plus bas de deux décades que celui proposé par Buttner~\cite{buttner1952adsorption}, cf. figure~\ref{energiesFr}. Par ailleurs, des mesures d'angles de contact de l'argent liquide sur un substrat de sapphire pouvaient permettaient de constater une baisse de cette énergie pour une activité comprise entre 10$^{-3}$~\cite{chatain94} ou 1010$^{-6}$~\cite{muolo08}.  À notre connaissance, une modification à si faible activité n’avais pas été observée dans la littérature dans cette gamme de température (c'est-à-dire pour de l'argent solide).\par 
Il serait trop hasardeux d’estimer numériquement la variation d’énergie : cela nécessiterait de construire un modèle à multiples paramètres (distribution d’épaisseur des joints de grains, densité des joints de grains, configuration locale des orientations cristallines, croissance en fonction de la température, etc.) dont la plupart sont inconnus.\par 
\conclusion{L’influence de l’oxygène sur le démouillage est très progressive et a lieu dès de très
faibles activités.\par }
	\subsubsection{Présence d'oxygène dans les empilements}
Dans le cas des empilements, certains paramètres sont différents : il n’y a plus de surface libre du métal, donc pas d’atmosphère ni de diffusion de surface. Pourtant, suivant les empilements, l’aspect des défauts peut varier. Dans certains cas, on observe plutôt des \og dendrites \fg, tandis que dans d’autres, ce sont plutôt des « dômes ». Comme nous l’avons expliqué plus haut, la présence de croissance extraordinaire de grains est liée à la présence d’oxygène, induisant une forte sélectivité des facettes qui peuvent croître. La question est alors de savoir si la présence d’oxygène dans l’empilement est envisageable.\par 
Nous avons donc procédé à une expérience en faisant varier la nature du substrat. C’est une couche de SnZnO$_x$, qui dans un cas contient de l’oxygène en proportions stœchiométriques, et dans l’autre contient de l’oxygène surnuméraire. Cette couche s’obtient aisément lors d’un dépôt magnétron réactif : en fonction de la proportion de dioxygène dans le plasma de dépôt, il est possible de contrôler les proportions dans la couche finale (voir partie expérimentale pour les détails du dépôt).\par 
Nous avons procédé à un recuit sous un vide secondaire à 300~$^\circ$C dans les deux cas et comparons l’état de la couche. La figure~\ref{MEBempilOx} présente les états de la couche obtenus en fonction du substrat. Sur la couche d’oxyde en proportions stoechiométriques (image a), la morphologie est très semblable à ce que l’on observe sous vide avec de la silice amorphe. En revanche, lorsque la couche d’argent démouille sur la couche d’oxyde enrichie en oxygène, on observe une morphologie proche de celle observée après un recuit sous oxygène. Notons que la couche a atteint cet état très vite (avant même de réaliser la première mise au point pendant la rampe), mais qu’elle a cessé d’évoluer. Ce comportement semble indiquer deux choses : le substrat est capable de fournir de l’oxygène à la couche d’argent lors du démouillage, modifiant son comportement, mais cet apport est limité.\par 
\conclusion{Ainsi, même dans les empilements, de l’oxygène peut être mis en contact avec la couche d’argent, influençant sa stabilité et son démouillage. La nature des couches directement en contact avec la couche d’argent, mais aussi la capactité de l'empilement à relâcher de l'oxygène au travers de l'argent est donc un facteur important pour comprendre le démouillage.\par}
\begin{figure}[!htb]
\centering
\includegraphics[width=0.7\textwidth]{MEBempilOx}
\caption{Couche d’argent de 40 nm recuite à 300~$^\circ$C sous vide sur une couche de SnZnO$_x$ amorphe a) stroechiométrique en oxygène b) avec un excès d'oxygène.}
\label{MEBempilOx}
\end{figure}
\section{Conclusion}
Le démouillage se décompose en trois étapes : l’induction, la propagation et le frittage.\par 
\textbf{Pendant l’induction}, la structure cristalline connaît de grands changements : augmentation de la taille moyenne des grains, de la rugosité et de façon surprenante, changement des orientations cristallines par rapport au substrat. Selon la littérature, Ces changements sont gouvernés par la réduction de l’énergie de surface, des contraintes à l’interface et très localement de l’énergie à l’interface des joints de grains.\par 
L’oxygène favorise à la fois une croissance extraordinaire de certains grains spécifiques, auparavant appelés \og buttes \fg, mais aussi l’apparition de trous dans la couche. En absence d’oxygène, les équilibres locaux ont tendance à ne pas créer de trous, mais la transformation durant l’induction a bel et bien lieu.\par 
Pendant la propagation, les trous se progressent dans la couche, diminuant le taux de couverture. La propagation est assurée par la disparition des grains les plus petits, qui alimentent la croissance d’autres grains. Si un grain a cru, il est beaucoup moins susceptible d’être modifié par le démouillage. Sous oxygène, la croissance ne concerne qu’un petit nombre de grains sélectionnés durant l’induction. Elle peut avoir lieu à une distance de l’ordre du micron du front de démouillage. Ceci assure au trou la possibilité de se propager quasiment librement au sein de la couche. Sous vide, l’agglomération est beaucoup plus répartie entre des grains proches du front, contraignant la progression du front à s’adapter sans cesse à l’évolution locale qu’elle engendre.\par 
Aucun bourrelet de propagation n’est visible lors du démouillage, quelle que soit l’atmosphère de recuit. Nous avons montré que la courbure locale n’est pas le la force motrice de la propagation. La propagation est liée aux mêmes forces motrices que l’induction : c’est une réorganisation de la structure cristalline qui a lieu à une échelle plus grande. La présence de trous accélère cette réorganisation.\par 
Lorsque toute la couche a été réorganisée par la propagation, la dernière étape prend place : le frittage. Cette étape est gouvernée par l’énergie de surface des îlots d’argent restants, elle a une cinétique beaucoup plus lente que les précédentes. Le frittage tend à amener les îlots vers leur forme d’équilibre, sans que ceux-ci n’échangent de matière entre eux : à l'échelle de nos mesures, le frittage ne conduit pas à une réduction de la densité des particules. Il est important de noter que la différence entre frittage et propagation n’est pas uniquement cinétique, mais résulte bien d'un changement de force motrice. Enfin, notons que le frittage n’est pas exclus pendant la propagation, il est simplement trop lent pour être significatif à ce moment-là. Notons enfin que le frittage n'a pas (ou peu) été observé sous vide car sa cinétique est extrêmement lente.\par \vspace{12pt}
Toutes ces conclusions ont pu être formulées grâce à la coïncidence de plusieurs facteurs. Le premier est la possibilité d’avoir pu étudier en MEB le démouillage \textit{in situ}, en temps réel et en atmosphère contrôlée. À ce titre, le MEB \textit{in situ} est un outil précieux pour l’étude du démouillage. Le second est le choix du système (argent polycristallin sur substrat amorphe), qui permet de s'affranchir des effets d'épitaxie qui interviennent sur les substrats cristallins. Ce choix correspond à une volonté de s’approcher de la thématique industrielle développée par Saint-Gobain. Le fait de s’intéresser à ce système original nous a permis d’observer un comportement inédit, et d’approfondir la compréhension fondamentale du démouillage. La démarche d’application de recherche fondamentale dans le cadre d’un système industriel s’est révélée extrêmement riche. Cette dernière considération s’applique bien au-delà de la seule thématique du démouillage.\par 

\newpage
\bibliographystyle{ieeetr}
\bibliography{biblio}

