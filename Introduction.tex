\newpage\null\thispagestyle{empty}\newpage
\paragraph*{Contexte industriel~: les couches minces d'argent\newline}

Après de nombreux développements techniques et industriels et afin de répondre à des marchés et des applications de plus en plus diversifiés, les vitrages d'aujourd'hui sont devenus des systèmes complexes. Le verre qui les constitue est bien souvent fonctionnalisé, notamment par des empilements de couches composées de différents matériaux et dont l'épaisseur est de l'ordre de quelques nanomètres à quelques centaines de nanomètres. On les désigne par le terme \og couches minces \fg.\par 
Les couches minces répondent toutes à des fonctions précises, qui peuvent être d'ordre mécanique (prévenir prévenir l'endommagement de la surface du verre), chimique (catalyser la dégradation des impuretés à la surface pour la garder propre) ou optique (absorber ou réfléchir certaines longueurs d'onde choisies).\par 
Un matériau est particulièrement employé en couche mince~: l'argent. Il présente plusieurs propriétés intéressantes qui font de lui un candidat de choix. C'est un un excellent conducteur électrique, son prix est relativement abordable et ses propriétés optiques, décrites par la suite, sont très intéressantes.\par 
\begin{figure}[!htb]
\centering
\includegraphics[width=0.3\textwidth]{TEMempilement}
\caption{Image TEM (Microscopie Électronique à Transmission) d'une couche d'argent dans un empilement produit par Saint-Gobain. Image fournie par Saint-Gobain Recherche.}
\label{TEMthinLayer}
\end{figure}
Lorsque l'argent est présent sous la forme d'une couche de quelques nanomètres d'épaisseur dans un empilement verrier, comme présenté sur la figure~\ref{TEMthinLayer}, la transmission lumineuse obtenue est modifiée (voir figure~\ref{transmissionStack}). Elle est forte dans la gamme visible (entre 400 et 800~nm), mais diminue très vite dans les infrarouges (IR, au delà de 800~nm). Ceci se traduit donc par un vitrage toujours transparent dans le visible, mais réfléchissant dans les IR, qui sont responsables de la transmission radiative de chaleur. Les vitrages ainsi fonctionnalisés sont appelés vitrages bas-émissifs~; ils représentent une forte part des ventes de vitrage dans le bâtiment, mais aussi pour l'automobile.\par 
\begin{figure}[!htb]
\centering
\includegraphics[width=0.5\textwidth]{spectreBasE.pdf}
\caption{Mesure de la transmission (T), de la réflexion (R) et de l'absorption (A) lumineuses d'un empilement contenant une couche d'argent. Produit Saint-Gobain standard (Thermocontrol R).}
\label{transmissionStack}
\end{figure}

Au sein d'usines dédiées, les couches sont fabriquées industriellement sur du verre plat par pulvérisation cathodique magnétron. Cette technique de dépôt est versatile en termes de matériaux. Une problématique particulièrement contraignante pour l'industrie provient de la nécessité de recuire le verre plat, après le dépôt des couches, pour un usage ultérieur. Par exemple la trempe, qui consiste à chauffer le verre, puis le refroidir très rapidement, améliore grandement sa résistance mécanique. Elle permet aussi de diminuer la taille des éclats produits s'il venait à se briser, le rendant moins dangereux  pour des applications du domaine automobile. Ou encore le formage, c'est-à-dire la déformation du verre en vu d'un usage particulier, comme les pare-brise pour l'automobile, par exemple. Cette chauffe se fait légèrement au-dessus de la température de transition vitreuse du verre (typiquement à 650~$^\circ$C). Étant donné la méthode de croissance des couches minces, l'étape de chauffe ne peut se faire qu'après le dépôt.\par
Or la chauffe n'est pas neutre pour les couches minces, notamment celles d'argent. Des défauts apparaissent~: selon les empilements, ils peuvent prendre la forme de trous dendritiques ou de dômes, comme illustré sur les figures~\ref{MEBintroDewetting}-a et b. Les zones sombres sont dépourvues d'argent, tandis que les zones blanches sont des points d'agglomération. Comme ces défauts ont une taille de l'ordre du micromètre, ils diffusent la lumière et donne lieu à un effet de flou, particulièrement visible et donc néfaste pour la commercialisation des produits.\par 
\begin{figure}[!htb]
\centering
\includegraphics[width=\textwidth]{MEBintroDewetting}
\caption{Images MEB (Microscopie Electronique à Balayage) de défauts apparus après recuit, dans un produit Saint-Gobain a)  sous forme de \og dendrites \fg{} b) ou de \og dômes \fg (images fournies par Saint-Gobain Recherche). c) Couche d'argent de 20~nm sur un substrat de silice (hors d'un empilement), après recuit.}
\label{MEBintroDewetting}
\end{figure}
Si l'on observe l'évolution d'une couche mince d'argent recuite sur un substrat, sans les autres couches qui l'encapsulent usuellement dans un empilement, l'évolution est encore plus marquée, comme on peut l'observer sur la figure~\ref{MEBintroDewetting}~: la couche se décompose en particules isolées. Ce phénomène est très semblable au comportement de l'eau sur une surface hydrophobe, appelé \og démouillage \fg. Cependant, il se produit au dessous de la température de fusion de l'argent, donc à l'état solide. Il a donc été baptisé \textbf{démouillage à l'état solide}.

\paragraph*{Utilisation du démouillage\newline}
Nous venons de voir que du point de vue industriel, le démouillage est un problème que l'on souhaite prévenir. Mais le démouillage présente également une forte potentialité en tant que mode de fabrication de particules métalliques, qui peuvent être employées pour colorer du verre~\cite{freestone2007lycurgus}, pour le photovoltaïque~\cite{morawiec2013self, atwater2010plasmonics} ou d'autres applications un peu plus éloignées des vitrages, comme des bio-capteurs~\cite{grochowska2016properties}.\par 
Pour toutes ces applications, un contrôle du démouillage est nécessaire. C'est pourquoi les études visant à l'améliorer ont gagné en importance ces dernières années~\cite{thompson2012solid}. Une stratégie en particulier vise à organiser spatialement les particules métalliques sur une surface. Elle a été développée par Giermann et Thompson~\cite{giermann2005solid}, voir figure~\ref{MEBorganisation}-a. Elle consiste à texturer préalablement le substrat par lithographie électronique avec des trous en forme de pyramides inversées organisées en réseau. Lors du démouillage, le métal s'agglomère dans ces trous, donc les particules forment elles aussi un réseau.\par 
Pour que cette organisation soit possible, Giermann et Thompson ont défini une condition nécessaire~: le volume de métal ne doit pas excéder le volume des trous pour que l'organisation soit possible.\par 
 \begin{figure}[!htb]
\centering
\includegraphics[width=0.45\textwidth]{giermannControl}\includegraphics[width=0.45\textwidth]{lebrisOrganisation}
\caption{Images MEB de particules métalliques organisées a) par Giermann et Thompson~\cite{giermann2005solid} et b) par Lebris \textit{et al.}~\cite{le2014self}.}
\label{MEBorganisation}
\end{figure}

\paragraph*{Problématique\newline}
Des travaux récents au laboratoire CNRS/Saint-Gobain \textit{Surface du Verre et Interface} (SVI) ont permis de reproduire cette organisation spatiale de particules (voir image~\ref{MEBorganisation}-b), avec des couches d'argent sur des substrats de silice. Le succès de la méthode était cependant surprenant par rapport aux prédictions de Giermann et Thompson~: le volume d'argent s'organisant était supérieur au volume des trous. La simple description en termes de volumes ne permet pas de rendre compte de nos systèmes, appelant donc à une meilleure compréhension du phénomène de démouillage.

\conclusion{L'objet de cette thèse est double~: progresser dans la compréhension du démouillage des couches d'argent et améliorer le contrôle de ce phénomène par texturation du substrat. Notre travail s'est focalisé sur des couches d'argent polycristallines sur un substrat de silice amorphe, afin de concilier la problématique industrielle et la méthode de contrôle développée au laboratoire SVI.\par }


\paragraph*{Structure du manuscrit\newline}
Le premier chapitre est une revue de la littérature concernant les différents domaines abordés dans les autres chapitres~: compréhension et modélisation du démouillage à l'état solide (particulièrement pour les couches polycristallines), étude de l'argent et de ses interactions avec l'oxygène, contrôle du démouillage et propriétés optiques de structures métalliques.\par 
Le second chapitre est dédié à la description des différentes méthodes déployées au cours de nos travaux. Nous y décrivons les procédures spécifiques pour produire des échantillons, ainsi que les méthodes de caractérisation. Étant donné qu'une majeure partie de notre travail repose sur du traitement d'images, les codes écrits pendant la thèse y sont aussi décrits.\par 
Le troisième chapitre est une étude \textit{in situ} du démouillage sous atmosphère contenant de l'oxygène. Nous y mettons en évidence le rôle de certains grains en croissance extraordinaire qui pilotent le morphologie du démouillage. Plusieurs méthodes sont utilisées~: AFM (Microscope à Force Atomique), MEB (Microscope Electronique à Balayage) et ellipsométrie. Cette étude est transposée aux surfaces texturées, ce qui nous permet de répondre en partie à la problématique de contrôle développée dans l'introduction.\par 
L'étude \textit{in situ} a été poursuivie en changeant d'atmosphère~: en absence d'oxygène, le démouillage prend un aspect radicalement différent. Observer autant de variations dues uniquement à la présence ou non d'oxygène nous a permis de comprendre la force motrice du démouillage~: la réorganisation des grains. Pour cela, nous avons en particulier étudié les effets de courbure locale dans la cadre d'une vision capillaire du démouillage. Ces développements sont exposés dans le quatrième chapitre.\par 
Enfin, nous avons appliqué nos connaissances du mécanisme du démouillage pour améliorer le contrôle que nous en avons. La méthodologie d'optimisation de l'organisation spatiale des structures d'argent obtenues est décrite. Nous avons ensuite étudié ces systèmes d'un point de vue optique. LEs résultats expérimentaux ont été consolidés par un travail de simulation qui nous a permis de mettre en évidence l'influence du réseau sur la réponse optique des particules.\par 
  

\bibliographystyle{ieeetr}
\bibliography{biblio}