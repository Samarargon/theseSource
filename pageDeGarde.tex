% - - - - - - - début de la page 
\thispagestyle{empty}

\begin{figure}
\minipage{0.32\textwidth}
  \includegraphics[width=\linewidth]{upmc-logo.png}
\endminipage\hfill
\minipage{0.32\textwidth}
\centering
  \includegraphics[width=0.5\linewidth]{insp-logo}
\endminipage\hfill
\minipage{0.32\textwidth}%
  \includegraphics[width=\linewidth]{logo_sg_rvb}
\endminipage
\end{figure}

%{\large

\vspace*{1cm}

\begin{center}

{\bf TH\`ESE DE DOCTORAT DE \\ l'UNIVERSIT\'E PIERRE ET MARIE CURIE}

\vspace*{0.5cm}

%Sp\'ecialit\'e \\ [2ex]
%{\bf Informatique}\ \\ 

École doctorale Chimie et Physico-chimie des matériaux (397)

\vspace*{0.2cm}


Pr\'esent\'ee par \ \\


\vspace*{0.5cm}


{\Large {\bf Paul Jacquet}}

\vspace*{1cm}
Pour obtenir le grade de \ \\[1ex]
{\bf DOCTEUR de l'UNIVERSIT\'E PIERRE ET MARIE CURIE} \ \\

\vspace*{1cm}

\end{center}

\begin{flushleft}
Sujet de la th\`ese :\ \\
\ \\
{\Large {\bf Vers la compréhension et le contrôle du démouillage des couches d'argent \\ }}
  

\vspace*{0.5cm} 
\flushleft{soutenue le 14 septembre 2017}\\[2ex]
\flushleft{devant le jury composé de :  }\\[1ex]
\flushleft{\begin{tabular}{r@{\ }ll}
  & M. Rémi {\sc Lazzari} & Directeur de thèse\\
  & Mme Dominique {\sc Chatain} & Rapporteuse \\
  & Mme Béatrice {\sc Dagens} & Rapporteuse  \\
  & M. Yves {\sc Bréchet} & Examinateur  \\
  & M. Nicolas {\sc Menguy} & Examinateur  \\
  & Mme Iryna {\sc Gozhyk} & Invitée  \\
  & M. Jérémie {\sc Teisseire} & Invité  \\
\end{tabular}}

\end{flushleft}\newpage
% - - - - - - - fin de la page 