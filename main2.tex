\documentclass[12pt,french]{report} 
\usepackage[utf8]{inputenc}          % Les packages `inputenc' et `fontenc'
\usepackage[T1]{fontenc}             % doivent être parmi les 1ers chargés. 
\usepackage[left=2.5cm,right=2.5cm,top=2.5cm,bottom=2.5cm]{geometry}  
     
\usepackage{minitoc}
%\usepackage{french}
\usepackage{babel}
\usepackage{lmodern}
\usepackage{graphicX}

\newcommand {\conclusion }[1]{\begin{center}
\fbox{
\begin{minipage}{0.9\textwidth}
   #1
\end{minipage}
}\end{center}\par}

\usepackage{amsmath,amssymb}
\usepackage[squaren,Gray]{SIunits}
\graphicspath{{Figures/}}

\usepackage{titlesec}

\renewcommand{\thechapter}{\Roman{chapter}}
\renewcommand{\thesection}{\Alph{section}}
\renewcommand{\thesubsection}{\arabic{subsection}}
\renewcommand{\thesubsubsection}{\alph{subsubsection}}

\linespread{1.1}
\setcounter{secnumdepth}{4}
\setcounter{minitocdepth}{3}
\setcounter{tocdepth}{1}
\usepackage{fancyhdr}
\pagestyle{fancy}
\fancyhf{}
\fancypagestyle{body}{%
	\fancyhead{}
	\fancyhead[R]{Chapitre \thechapter}
	\fancyhead[L]{\rightmark}
	\fancyfoot[C]{\thepage}}

\fancypagestyle{Conclusion}{%
  \fancyhead{} % clear all header fields
  \fancyhead[L]{\large{\textsc{Conclusion}}}}


\fancypagestyle{Introduction}{%
  \fancyhead{} % clear all header fields
  \fancyhead[L]{\large{\textsc{Introduction}}}}


\usepackage[sectionbib]{chapterbib}

\begin{document}
%% - - - - - - - début de la page 
\thispagestyle{empty}

\begin{figure}
\minipage{0.32\textwidth}
  \includegraphics[width=\linewidth]{upmc-logo.png}
\endminipage\hfill
\minipage{0.32\textwidth}
\centering
  \includegraphics[width=0.5\linewidth]{insp-logo}
\endminipage\hfill
\minipage{0.32\textwidth}%
  \includegraphics[width=\linewidth]{logo_sg_rvb}
\endminipage
\end{figure}

%{\large

\vspace*{1cm}

\begin{center}

{\bf TH\`ESE DE DOCTORAT DE \\ l'UNIVERSIT\'E PIERRE ET MARIE CURIE}

\vspace*{0.5cm}

%Sp\'ecialit\'e \\ [2ex]
%{\bf Informatique}\ \\ 

École doctorale Chimie et Physico-chimie des matériaux (397)

\vspace*{0.2cm}


Pr\'esent\'ee par \ \\


\vspace*{0.5cm}


{\Large {\bf Paul Jacquet}}

\vspace*{1cm}
Pour obtenir le grade de \ \\[1ex]
{\bf DOCTEUR de l'UNIVERSIT\'E PIERRE ET MARIE CURIE} \ \\

\vspace*{1cm}

\end{center}

\begin{flushleft}
Sujet de la th\`ese :\ \\
\ \\
{\Large {\bf Vers la compréhension et le contrôle du démouillage des couches d'argent \\ }}
  

\vspace*{0.5cm} 
\flushleft{soutenue le 14 septembre 2017}\\[2ex]
\flushleft{devant le jury composé de :  }\\[1ex]
\flushleft{\begin{tabular}{r@{\ }lll}
  & M. Rémi {\sc Lazzari} & DR, Université Pierre et Marie Curie & Directeur de thèse\\
  & Mme Dominique {\sc Chatain} & DR, Université Aix-Marseille & Rapporteuse \\
  & Mme Béatrice {\sc Dagens} & DR, Université Paris-Saclay & Rapporteuse  \\
  & M. Yves {\sc Bréchet} & Pr, INP Grenoble & Examinateur  \\
  & M. Nicolas {\sc Menguy} & Pr, Université Pierre et Marie Curie & Examinateur  \\
  & Mme Iryna {\sc Gozhyk}& Ingénieure, Saint-Gobain Recherche & Invitée  \\
  & M. Jérémie {\sc Teisseire} & Ingénieur, Saint-Gobain Recherche  & Invité  \\
\end{tabular}}

\end{flushleft}\newpage
% - - - - - - - fin de la page 

\newpage
\dominitoc
\tableofcontents
%%%%%%%%%%%%%%%%%%%%%%%%%%%%%%%%%%%%%%%%%%%%%%%%%%%%%%
%%
%% Chapitre O : Introduction
%%
%%%%%%%%%%%%%%%%%%%%%%%%%%%%%%%%%%%%%%%%%%%%%%%%%%%%%%
%\chapter*{Introduction}
%\pagestyle{Introduction}
%\newpage\null\thispagestyle{empty}\newpage
\paragraph*{Contexte industriel~: les couches minces d'argent\newline}

Après de nombreux développements techniques et industriels et afin de répondre à des marchés et des applications de plus en plus diversifiés, les vitrages d'aujourd'hui sont devenus des systèmes complexes. Le verre qui les constitue est bien souvent fonctionnalisé, notamment par des empilements de couches composées de différents matériaux et dont l'épaisseur est de l'ordre de quelques nanomètres à quelques centaines de nanomètres. On les désigne par le terme \og couches minces \fg.\par 
Les couches minces répondent toutes à des fonctions précises, qui peuvent être d'ordre mécanique (prévenir prévenir l'endommagement de la surface du verre), chimique (catalyser la dégradation des impuretés à la surface pour la garder propre) ou optique (absorber ou réfléchir certaines longueurs d'onde choisies).\par 
Un matériau est particulièrement employé en couche mince~: l'argent. Il présente plusieurs propriétés intéressantes qui font de lui un candidat de choix. C'est un un excellent conducteur électrique, son prix est relativement abordable et ses propriétés optiques, décrites par la suite, sont très intéressantes.\par 
\begin{figure}[!htb]
\centering
\includegraphics[width=0.3\textwidth]{TEMempilement}
\caption{Image TEM (Microscopie Électronique à Transmission) d'une couche d'argent dans un empilement produit par Saint-Gobain. Image fournie par Saint-Gobain Recherche.}
\label{TEMthinLayer}
\end{figure}
Lorsque l'argent est présent sous la forme d'une couche de quelques nanomètres d'épaisseur dans un empilement verrier, comme présenté sur la figure~\ref{TEMthinLayer}, la transmission lumineuse obtenue est modifiée (voir figure~\ref{transmissionStack}). Elle est forte dans la gamme visible (entre 400 et 800~nm), mais diminue très vite dans les infrarouges (IR, au delà de 800~nm). Ceci se traduit donc par un vitrage toujours transparent dans le visible, mais réfléchissant dans les IR, qui sont responsables de la transmission radiative de chaleur. Les vitrages ainsi fonctionnalisés sont appelés vitrages bas-émissifs~; ils représentent une forte part des ventes de vitrage dans le bâtiment, mais aussi pour l'automobile.\par 
\begin{figure}[!htb]
\centering
\includegraphics[width=0.5\textwidth]{spectreBasE.pdf}
\caption{Mesure de la transmission (T), de la réflexion (R) et de l'absorption (A) lumineuses d'un empilement contenant une couche d'argent. Produit Saint-Gobain standard (Thermocontrol R).}
\label{transmissionStack}
\end{figure}

Au sein d'usines dédiées, les couches sont fabriquées industriellement sur du verre plat par pulvérisation cathodique magnétron. Cette technique de dépôt est versatile en termes de matériaux. Une problématique particulièrement contraignante pour l'industrie provient de la nécessité de recuire le verre plat, après le dépôt des couches, pour un usage ultérieur. Par exemple la trempe, qui consiste à chauffer le verre, puis le refroidir très rapidement, améliore grandement sa résistance mécanique. Elle permet aussi de diminuer la taille des éclats produits s'il venait à se briser, le rendant moins dangereux  pour des applications du domaine automobile. Ou encore le formage, c'est-à-dire la déformation du verre en vu d'un usage particulier, comme les pare-brise pour l'automobile, par exemple. Cette chauffe se fait légèrement au-dessus de la température de transition vitreuse du verre (typiquement à 650~$^\circ$C). Étant donné la méthode de croissance des couches minces, l'étape de chauffe ne peut se faire qu'après le dépôt.\par
Or la chauffe n'est pas neutre pour les couches minces, notamment celles d'argent. Des défauts apparaissent~: selon les empilements, ils peuvent prendre la forme de trous dendritiques ou de dômes, comme illustré sur les figures~\ref{MEBintroDewetting}-a et b. Les zones sombres sont dépourvues d'argent, tandis que les zones blanches sont des points d'agglomération. Comme ces défauts ont une taille de l'ordre du micromètre, ils diffusent la lumière et donne lieu à un effet de flou, particulièrement visible et donc néfaste pour la commercialisation des produits.\par 
\begin{figure}[!htb]
\centering
\includegraphics[width=\textwidth]{MEBintroDewetting}
\caption{Images MEB (Microscopie Electronique à Balayage) de défauts apparus après recuit, dans un produit Saint-Gobain a)  sous forme de \og dendrites \fg{} b) ou de \og dômes \fg (images fournies par Saint-Gobain Recherche). c) Couche d'argent de 20~nm sur un substrat de silice (hors d'un empilement), après recuit.}
\label{MEBintroDewetting}
\end{figure}
Si l'on observe l'évolution d'une couche mince d'argent recuite sur un substrat, sans les autres couches qui l'encapsulent usuellement dans un empilement, l'évolution est encore plus marquée, comme on peut l'observer sur la figure~\ref{MEBintroDewetting}~: la couche se décompose en particules isolées. Ce phénomène est très semblable au comportement de l'eau sur une surface hydrophobe, appelé \og démouillage \fg. Cependant, il se produit au dessous de la température de fusion de l'argent, donc à l'état solide. Il a donc été baptisé \textbf{démouillage à l'état solide}.

\paragraph*{Utilisation du démouillage\newline}
Nous venons de voir que du point de vue industriel, le démouillage est un problème que l'on souhaite prévenir. Mais le démouillage présente également une forte potentialité en tant que mode de fabrication de particules métalliques, qui peuvent être employées pour colorer du verre~\cite{freestone2007lycurgus}, pour le photovoltaïque~\cite{morawiec2013self, atwater2010plasmonics} ou d'autres applications un peu plus éloignées des vitrages, comme des bio-capteurs~\cite{grochowska2016properties}.\par 
Pour toutes ces applications, un contrôle du démouillage est nécessaire. C'est pourquoi les études visant à l'améliorer ont gagné en importance ces dernières années~\cite{thompson2012solid}. Une stratégie en particulier vise à organiser spatialement les particules métalliques sur une surface. Elle a été développée par Giermann et Thompson~\cite{giermann2005solid}, voir figure~\ref{MEBorganisation}-a. Elle consiste à texturer préalablement le substrat par lithographie électronique avec des trous en forme de pyramides inversées organisées en réseau. Lors du démouillage, le métal s'agglomère dans ces trous, donc les particules forment elles aussi un réseau.\par 
Pour que cette organisation soit possible, Giermann et Thompson ont défini une condition nécessaire~: le volume de métal ne doit pas excéder le volume des trous pour que l'organisation soit possible.\par 
 \begin{figure}[!htb]
\centering
\includegraphics[width=0.45\textwidth]{giermannControl}\includegraphics[width=0.45\textwidth]{lebrisOrganisation}
\caption{Images MEB de particules métalliques organisées a) par Giermann et Thompson~\cite{giermann2005solid} et b) par Lebris \textit{et al.}~\cite{le2014self}.}
\label{MEBorganisation}
\end{figure}

\paragraph*{Problématique\newline}
Des travaux récents au laboratoire CNRS/Saint-Gobain \textit{Surface du Verre et Interface} (SVI) ont permis de reproduire cette organisation spatiale de particules (voir image~\ref{MEBorganisation}-b), avec des couches d'argent sur des substrats de silice. Le succès de la méthode était cependant surprenant par rapport aux prédictions de Giermann et Thompson~: le volume d'argent s'organisant était supérieur au volume des trous. La simple description en termes de volumes ne permet pas de rendre compte de nos systèmes, appelant donc à une meilleure compréhension du phénomène de démouillage.

\conclusion{L'objet de cette thèse est double~: progresser dans la compréhension du démouillage des couches d'argent et améliorer le contrôle de ce phénomène par texturation du substrat. Notre travail s'est focalisé sur des couches d'argent polycristallines sur un substrat de silice amorphe, afin de concilier la problématique industrielle et la méthode de contrôle développée au laboratoire SVI.\par }


\paragraph*{Structure du manuscrit\newline}
Le premier chapitre est une revue de la littérature concernant les différents domaines abordés dans les autres chapitres~: compréhension et modélisation du démouillage à l'état solide (particulièrement pour les couches polycristallines), étude de l'argent et de ses interactions avec l'oxygène, contrôle du démouillage et propriétés optiques de structures métalliques.\par 
Le second chapitre est dédié à la description des différentes méthodes déployées au cours de nos travaux. Nous y décrivons les procédures spécifiques pour produire des échantillons, ainsi que les méthodes de caractérisation. Étant donné qu'une majeure partie de notre travail repose sur du traitement d'images, les codes écrits pendant la thèse y sont aussi décrits.\par 
Le troisième chapitre est une étude \textit{in situ} du démouillage sous atmosphère contenant de l'oxygène. Nous y mettons en évidence le rôle de certains grains en croissance extraordinaire qui pilotent le morphologie du démouillage. Plusieurs méthodes sont utilisées~: AFM (Microscope à Force Atomique), MEB (Microscope Electronique à Balayage) et ellipsométrie. Cette étude est transposée aux surfaces texturées, ce qui nous permet de répondre en partie à la problématique de contrôle développée dans l'introduction.\par 
L'étude \textit{in situ} a été poursuivie en changeant d'atmosphère~: en absence d'oxygène, le démouillage prend un aspect radicalement différent. Observer autant de variations dues uniquement à la présence ou non d'oxygène nous a permis de comprendre la force motrice du démouillage~: la réorganisation des grains. Pour cela, nous avons en particulier étudié les effets de courbure locale dans la cadre d'une vision capillaire du démouillage. Ces développements sont exposés dans le quatrième chapitre.\par 
Enfin, nous avons appliqué nos connaissances du mécanisme du démouillage pour améliorer le contrôle que nous en avons. La méthodologie d'optimisation de l'organisation spatiale des structures d'argent obtenues est décrite. Nous avons ensuite étudié ces systèmes d'un point de vue optique. LEs résultats expérimentaux ont été consolidés par un travail de simulation qui nous a permis de mettre en évidence l'influence du réseau sur la réponse optique des particules.\par 
  

\bibliographystyle{ieeetr}
\bibliography{biblio}

%%%%%%%%%%%%%%%%%%%%%%%%%%%%%%%%%%%%%%%%%%%%%%%%%%%%%%
%%
%% Chapitre 1 : État de l'art
%%
%%%%%%%%%%%%%%%%%%%%%%%%%%%%%%%%%%%%%%%%%%%%%%%%%%%%%%
\pagestyle{body}
%\chapter{État de l'art}
%
%\begin{flushleft}
%\textit{Mais quant au peuple obscur des petits, des insectes,\\
%Qu’elle les aime ou non, nul ne veut le savoir.\\
%Poussière d’avortons nés de larves infectes,\\
%Nous les méprisons trop pour chercher à les voir.\\}
%\end{flushleft}
%\begin{flushright}
%\textbf{Jean Richepin}, \textit{À la gloire des insectes, 1881.}
%\end{flushright}
%\newpage
%\newpage\null\thispagestyle{empty}\newpage
\setcounter{minitocdepth}{2}
\minitoc
\setcounter{minitocdepth}{3}
\newpage

Ce chapitre est dédié à l'étude bibliographique concernant les travaux exposés dans les chapitres suivants. Il s'articule en quatre parties~: \textit{(A)} une brève revue du démouillage à l'état solide, \textit{(B)} des caractéristiques physiques de l'argent, \textit{(C)} une présentation des approches pour contrôler le démouillage et \textit{(D)} une introduction à la réponse optique de structures métalliques semblables à celles obtenues par démouillage.


\section{Le démouillage à l'état solide}
Le démouillage solide d'un film mince est la transformation du film activée par chauffage vers son état d'équilibre thermodynamique~: des particules isolées. L'état initial de film est métastable~; ce retour dicté par la réduction d'énergie des interfaces en présence (métal/environnement ou métal/substrat).\par 
	\subsection{Bref historique du démouillage}
\label{sHistoire}
Avant de nous plonger dans le détail des travaux portant sur le démouillage à l’état solide, ce paragraphe a pour but d'expliquer comment l'étude de la stabilité de couches minces a finalement débouché sur le démouillage en tant qu'objet d'intérêt. Les modèles évoqués dans ce paragraphe seront décrits plus précisément par la suite.\par 
L'intérêt porté aux couches minces a débuté dans les années soixante. À cette date, les transistors à couches (TFT) ou transistors à effet de champ à grille isolée (MOSFET) sont inventés. De fines couches métalliques servent alors de contact électrique, ce qui permet de relier plusieurs transistors en circuit intégrés. La structure, la qualité et la durabilité de ces couches sont étudiées durant cette période~\cite{pennebaker1969hillock,dheurle1968structure,caswell1963low}. Une des premières observations concerne la formation de petites buttes en surface de la couche, dès lors que le système est chauffé. Les températures nécessaires à la transformation sont bien moindres que les températures de fusion des métaux~:
\textbf{tout se produit à l'état solide}.\par 
À la même période, Mullins~\cite{mullins1957theory} étudia un autre phénomène d'évolution des couches minces polycristallines~: le sillonnement des joints de grains. Un joint de grains tel qu'étudié par Mullins est représenté sur la figure~\ref{schemaMullins}. Après développement d'un modèle, Mullins obtient une loi de vitesse de sillonnement des joints de grains, reliant leur profondeur au temps de recuit. Suspecté d'être à l'origine de nombreux défauts dans les couches minces, ce phénomène sera étudié expérimentalement par Rhead~\cite{rhead1963surface,rhead1965surface} dans les couches d'argent. Il remarque alors que les sillons se creusent beaucoup plus rapidement lorsque l'atmosphère de recuit contient de l'oxygène, et estime un rapport de coefficients de diffusion de plusieurs ordres de grandeur par rapport à une atmosphère réductrice. À ce moment-là, le phénomène de \og démouillage \fg{} n'est pas encore un objet d'étude.\par 
\begin{figure}[!htb]
	\centering
	\includegraphics[width=0.6\textwidth]{schemaMullins}
	\caption{Joint de grains dans l'étude de Mullins~\cite{mullins1957theory}.}
	\label{schemaMullins}
\end{figure}
Brandon et Bradshaw~\cite{brandon1966mobility}, un peu plus tard, cherchent à remonter au coefficient d'autodiffusion de l'argent. Ils souhaitent notamment comparer la méthode de Mullins à d'autres mesures. Pour cela, ils étudient un autre phénomène~: la rétraction d'un front de métal sur un substrat. Ils proposent eux aussi un modèle, principalement basé sur la courbure locale, pour expliquer cette rétraction. Grâce à ce modèle, qui adopte une description continue et isotrope
du matériau, ils estiment un coefficient de diffusion. Il est important de noter que le domaine de la \og matière molle \fg{}, étudiant notamment le démouillage de polymères, est en effervescence à ce moment-là. L'approche de Brandon et Bradshaw est très largement inspirée du même formalisme.\par 
Dans leur étude, Brandon et Bradshaw comparent différentes manières de mesurer un coefficient de diffusion. Comme la mesure obtenue grâce à leur nouveau modèle est cohérente avec les résultats d'autres méthodes, ce modèle a été adopté par la suite. Bien que le phénomène étudié soit bien le démouillage, il faut attendre les années 1990 pour que le démouillage lui-même
devient un objet d'étude~: le terme \og démouillage à l'état solide \fg{} n'apparaît d'ailleurs que vers la fin des années 90. Ces études vont apporter des modifications au modèle initial, mais encore aujourd'hui des questions demeurent. Ce travail de thèse s'inscrit dans cette lignée et vise à acquérir une meilleure compréhension du mécanisme de démouillage.\par 

	\subsection{Diversité des systèmes}
La morphologie du démouillage dépend beaucoup du système étudié~\cite{thompson2012solid}. Nous allons présenter différents systèmes par ordre décroissant de complexité liée au désordre structural~: couches polycristallines non-épitaxiées, couches polycristallines épitaxiées, couches monocristallines.

\paragraph*{Couches polycristallines~:} les couches polycristallines correspondent aux études historiques du démouillage~\cite{presland1972hillock}. Plusieurs matériaux ont été étudiés~: l'argent~\cite{presland1972hillock, presland1972role, sharma1980agglomeration, sharma1980hillock, sharma1986hillock, dannenberg2000situ}, l'or~\cite{jiran1990capillary, kwon2003comparison, muller2010microstructure,jang2015crystallographic}, le cuivre~\cite{kwon2003comparison, gadkari2005comparison} ou encore le platine~\cite{gadkari2005comparison, atiya2014solid, jahangir2014situ}. Dans les couches
polycristallines, les joints de grains peuvent jouer un rôle dans l'évolution de la couche. Le rôle individuel des grains a récemment été mis en lumière dans le mécanisme de démouillage~\cite{atiya2014role, kosinova2014role}.\par 
Dans leurs travaux, Kwon\textit{ et al.}~\cite{kwon2003comparison} ont étudié le comportement de l'or et du cuivre sur silice. Sur la figure~\ref{kwonCompAu} et \ref{kwonCompCu} de la page~\pageref{kwonCompAu}, nous pouvons comparer l'évolution de deux couches d'épaisseur identique de 50~nm, déposées sur de la silice amorphe et recuites dans des atmosphères réductrices similaires. La première, d'or est recuite à 700~$^\circ$C et la seconde, de cuivre, est recuite à 300~$^\circ$C. Dans le cas de l'or, nous observons que peu de trous croissent en formant des dendrites, tandis que dans le cas du cuivre, beaucoup de trous apparaissent et se propagent sans former de dendrites. Nous observons deux comportements très différents en changeant le métal, \textbf{et la température de démouillage varie}.\par 

\begin{figure}[!p]
\centering
\includegraphics[width=0.6\textwidth]{kwonCompAu}
\caption{Évolution de la morphologie d'une couche d'or de 50~nm d'épaisseur recuite à 700~$^\circ$C dans une atmosphère
d'argon hydrogéné pendant a) 5 min, b) 30 min, c) 60 min,d) 90 min, e) 120 min, f) 150 min.~\cite{kwon2003comparison}}
\label{kwonCompAu}
\end{figure}


\begin{figure}[!p]
\centering
\includegraphics[width=0.6\textwidth]{kwonCompCu}
\caption{Évolution de la morphologie d'une couche de cuivre de 50~nm d'épaisseur recuite à 300~$^\circ$C dans une atmosphère d'argon hydrogéné pendant a) 0 min, b) 10 min, c) 20 min, d) 40 min, e) 60 min, f) 120 min~\cite{kwon2003comparison}.}
\label{kwonCompCu}
\end{figure}


\paragraph*{Couches épitaxiées} Si le substrat est cristallin et qu'il présente une correspondance avec la maille du métal qui est déposé dessus, le métal s'oriente de manière à faire correspondre sa propre maille avec celle du substrat. L'orientation hors-plan des grains qui constituent la couches métallique est alors déterminée. Ce phénomène est appelé épitaxie. Dans l'étude du démouillage, les systèmes épitaxiés courants sont Au/$\alpha$-Al$_2$0$_3$(0001)~\cite{kosinova2014role} ou Pt/SrTiO$_3$ (100)~\cite{atiya2014solid}. L'épitaxie stabilisant (dans une certaine mesure) l'interface, le démouillage sur les couches épitaxiées nécessite une plus haute température de recuit. Elle change également la nature des joints de grains au travers de l'orientation hors plan.\par 

\paragraph*{Remarque~:} Dans les couches utilisées dans l'industrie verrière, l'argent est stabilisée par une épitaxie sur la face basale de ZnO wurtzite. En dépit d'un désaccord paramétrique important (-11\%) et l'existence d'une autre orientation, tournée de 30$^\circ$ à désaccord plus faible  (2.5\%), l'argent adopte une orientation hexagone/hexagone sur les faces polaires de ZnO~\cite{jedrecy05,jedrecy05b}, impliquant une orientation (111) de l'argent. Cette orientation semble pilotée par la nucléation/croissance en bord de marches polaires~\cite{chernyshevat}.


\paragraph*{Couches monocristallines~:} On peut encore augmenter l'ordre du système en considérant des couches monocristallines. Plusieurs systèmes ont été étudiés~: citons par exemple Ni(100)/MgO(100)~\cite{zucker2016mechanism}, Ni(111)/$\alpha$-Al$_2$0$_3$(0001)~\cite{rabkin2014solid}, Ge(100) ou Si(100)/a-SiO$_2$~\cite{leroy2013influence, cheynis2013dynamics}. La morphologie du démouillage de monocristaux présente des symétries très caractéristiques de la structure cristalline \textbf{du film} elle-même, comme par exemple le démouillage de Ge présenté sur la figure~\ref{cheynisGeDewet}.\par
\begin{figure}[!htb]
\centering
\includegraphics[width=0.8\textwidth]{cheynisGeDewet}
\caption{Propagation d'un trou dans un monocristal de germanium observé en LEEM (Low Energy Electron Microscopy) (a-e) ou en AFM (f). Images issues de~\cite{cheynis2013dynamics}.}
\label{cheynisGeDewet}
\end{figure}
Dans le cas des monocristaux, la composante \og joints de grains \fg{} ne fait plus partie du problème, c'est pourquoi ces systèmes servent plus facilement de modèles. En effet, l'absence de joints de grains permet de considérer que le matériau est homogène. De plus, le fait de connaître l'orientation du cristal permet de construire plus aisément des modèles en fonction de l'anisotropie des énergies des surfaces qui peuvent exister~\cite{zucker2013model}. Dans ce système, l'état initial est très bien déterminé, contrairement aux couches polycristallines.\par


\conclusion{Les couches métalliques présentent une très grande variété de morphologies lors du démouillage, déterminées par leur nature chimique, leur structure et l'énergétique \textbf{(énergie et anisotropie des énergies)} de leurs surfaces/interfaces. \textbf{La température de démouillage augmente avec la température de fusion du métal considéré.} De plus la cinétique du phénomène est intimement liée à la microstructure initiale du film donc à la méthode et aux conditions de dépôt.\par 

Notre étude se situe dans la première catégorie, où l'ordre structural est le plus faible~: une couche polycristalline (argent) sur un substrat amorphe (silice). Ce choix se justifie par \textit{(i)} la volonté de rester proche des contraintes industrielles, où l'argent est déposé par pulvérisation magnétron, \textit{(ii)} mais aussi d'avoir un système simplifié (en éliminant la couche de ZnO) afin de tester les approches capillaires du phénomène, \textit{(iii)} et enfin de permettre la texturation de surface en vue du contrôle du démouillage. Au cours de nos travaux, nous remettrons nos résultats en perspectives avec les systèmes industriels.\par}


	\subsection{Le mécanisme du démouillage}
	\label{sBiblioMeca}
Pour la suite, nous allons centrer notre étude bibliographique sur les couches polycristallines. Nous décrirons peu dans ces pages les modèles portant sur les monocristaux. Certaines grandes caractéristiques du démouillage sont observées systématiquement~: il est d'autant plus rapide que la température est élevée (tout en restant sous le point de fusion)~\cite{simrick2012thermal, morawiec2013self}, et la couche métallique démouille \textbf{à d'autant plus basse température} qu'elle est fine~\cite{morawiec2013self, kim2002thickness, manuela2013dewetting}. Au-delà de ces similitudes apparentes, de nombreuses différences apparaissent en fonction du système considéré.\par 
		\subsubsection{Induction et croissance cristalline}

Presland~\cite{presland1972hillock} a observé, lors de ses travaux historiques, que le démouillage débute par une modification de la couche, avant même que les trous ne se forment. Cette période est appelée \og temps d'induction \fg. Nous allons d'abord nous concentrer sur ces modifications, qui sont de deux ordres. D'une part, la structure des grains d'une couche polycristalline peut évoluer pendant le recuit~\cite{thompson1990grain} (orientation, taille, forme), d'autre part des buttes apparaissent~\cite{sharma1980hillock, sharma1986hillock}.\par 

\paragraph*{Réorganisation cristalline~:} L'interface entre deux grains, appelée joint de grains, est une structure énergétiquement défavorable~: c'est la frontière entre deux domaines cristallins dont l'orientation varie. Elle est donc le siège d'une discontinuité structurale. On y observe notamment des dislocations. Ainsi, une structure polycristalline sera d'autant plus stable que le nombre de joints de grains est réduit, et que leur énergie associée est faible. Il est important de noter qu'il n'existe pas de relation simple entre l'énergie d'un joint de grains et la désorientation des deux domaines cristallins~\cite{hull2001introduction}; Il convient donc de calculer ou mesurer cette énergie pour chaque orientation considérée.\par 
La croissance cristalline est gouvernée par une diminution des énergies à l'interface des joints de grains. Nous allons nous intéresser plus spécifiquement à la croissance dans des couches minces~\cite{thompson1990grain}. Il existe deux types de croissances~: la croissance normale et la croissance anormale. Dans le cadre de la croissance normale, on peut formuler une condition d'autosimilarité~\cite{mullins1986statistical}~: la taille des grains augmente mais reste homogène (\og monodisperse \fg) au cours du processus. Dans le cadre d'une croissance anormale, certains grains croissent plus rapidement que leurs voisins~\cite{thompson1990grain}, menant à l'apparition de deux populations distinctes de grains. Ce processus est représenté sur la figure~\ref{thompsonSchemaGrowth}.\par 
\begin{figure}[!htb]
\centering
\includegraphics[width=0.6\textwidth]{thompsonSchemaGrowth}
\caption{Croissance normale (a) et croissance anormale d'un grain (b) dans une couche mince~\cite{thompson1990grain}.}
\label{thompsonSchemaGrowth}
\end{figure}
Trois facteurs orientent vers la croissance anormale~\cite{hillert1965theory}~: \textit{(i)} une grande disparité dans les énergies interfaciales au niveau des joints (due par exemple à l'anisotropie des énergies de surface), \textit{(ii)} un mouvement de certains joints de grains ralenti et \textit{(iii)} une disparité initiale dans la taille des grains.\par 
Le mouvement des joints de grains peut être ralenti de deux manières, ce qui influence donc la croissance~: l'épinglage de Zener par des impuretés~\cite{couturier20053d} et la présence de profonds sillons de joints de grains~\cite{mullins1958effect}. Dans les deux cas, un obstacle énergétique se présente pour empêcher le mouvement du joint de grains.\par

\paragraph*{La croissance de buttes~:} La formation de buttes a été observées dans diverses couches minces
(or, aluminium, plomb, argent~\cite{pennebaker1969hillock, dheurle1968structure, caswell1963low}). Pennebaker~\cite{pennebaker1969hillock} attribue la formation de buttes à la
relaxation de contraintes à l'interface, engendrées par la dilatation thermique. Ces contraintes sont écrites de la manière suivante~:
\begin{equation}
\Delta\sigma = (\alpha_M - \alpha_S)\cdot\Delta T\dfrac{E}{1-\nu},
\end{equation}

avec $\Delta\sigma$ la différence de contraintes entre deux températures, $\alpha_M$ et $\alpha_S$ les coefficients d'expansion thermique (respectivement de la couche métallique et du substrat), $\Delta T$ la différence de température, $E$ le module d'Young et $\nu$ le coefficient de Poisson de la couche métallique. Les coefficients d'expansion des métaux étant plus grands que ceux des diélectriques, une couche métallique supportée a donc tendance à subir des contraintes en compression si on la chauffe. Ces contraintes se relaxent lors du traitement thermique, et des buttes apparaissent à la surface du métal. Si l'on refroidit le système après la formation des buttes, on observe une contrainte en tension à l'interface~\cite{pennebaker1969hillock}. La formation de buttes en surface semble donc être due à une relaxation des contraintes à l'interface.\par 
Presland a observé la croissance de buttes lors du recuit d'une couche d'argent~\cite{presland1972hillock}. Scharma et Spritz étudient le même métal en variant les modes de dépôt~\cite{sharma1980agglomeration, sharma1980hillock, sharma1986hillock}. De ces études, plusieurs conclusions se dégagent~:
\begin{itemize}
\item L'oxygène favorise la croissance des buttes ;
\item La densité de buttes semble davantage liée à des inhomogénéités initiales de la couche qu'aux conditions de recuit ;
\item Les buttes sont bien des calottes pleines, pas des cloques comme on peut l'observer dans d'autres systèmes~\cite{gadkari2005comparison}.
\end{itemize}
Ainsi, la surface de contact métal-substrat ne varie pas, mais la surface libre du métal augmente. Ces observations pointent elles aussi dans la direction d'une relaxation des contraintes à l'interface. La nature des buttes elle-même n'est pas déterminée dans ces travaux.\par 

\paragraph*{Remarque~:} À la même période, un autre phénomène très semblable à la croissance des buttes a été observé~: l'apparition de whiskers (moustaches de chat). Les whiskers apparaissent grâce à un mode de croissance d'un grain dans une direction spécifique qui lui donne une forme de fil. Leur présence est particulièrement désastreuse dans des systèmes électroniques, car ils peuvent engendrer des courts-circuits. Les conditions d'apparition des whiskers sont comparables avec celles d'apparition des buttes~\cite{rhead1963surface}. \textbf{La similitude des deux phénomènes est encore étudiée de nos jours~\cite{sarobol2013whisker}.}

\paragraph*{Apparition des trous~:} Un des mécanismes possibles pour l'apparition des trous est lié au phénomène évoqué dans la section~\ref{sHistoire} et étudié par Mullins~\cite{mullins1957theory}~: le sillonnement des joints de grains. L'idée générale est de supposer que lorsque le sillon atteint le substrat, un trou se forme.\par 
Dans son étude, Mullins étudie le joint en 2D (voir la figure~\ref{schemaMullins}), considérant uniquement le projeté des surfaces de deux grains voisins dans un plan. Il propose deux mécanismes pour le phénomène~: l'évaporation-condensation ou la diffusion de surface. Le premier mécanisme est négligeable dans notre système (car le taux d'évaporation de l'argent est trop faible dans nos conditions de recuit). En revanche, le second va nourrir beaucoup d'études postérieures, car il correspond à la majorité des cas étudiés. Mullins calcule le potentiel chimique $\mu$ d'un atome de surface du métal~: 
\begin{equation}
\mu = \mu_{ref} + \kappa \gamma \Omega,
\end{equation}
en fonction de la courbure $\kappa$ et de l'énergie de surface $\gamma$ (supposée isotrope et uniforme) par rapport à un potentiel de référence $\mu_{ref}$ et $\Omega$, le volume atomique. Cette différence de potentiel chimique induit un flux diffusif sur un arc arbitraire s~:
\begin{equation}
J = -\dfrac{D_s\nu}{kT}\cdot\dfrac{\partial\mu}{\partial s}\newline
 = -\dfrac{D_s\nu\gamma\Omega}{kT}\cdot\dfrac{\partial\kappa}{\partial s},
\label{eDiffFlux}
\end{equation}
avec $D_s$ le coefficient d'auto-diffusion du métal, $\nu$ la concentration d'adatomes en surface, $k$ la constante de Boltzmann et $T$ la température. Après développement mathématique, Mullins obtient une loi de vitesse de sillonnement des joints de grains. Leur profondeur d varie selon la loi~:
\begin{equation}
d = 0,973m\left(\dfrac{D_s\Omega^2\nu}{kT}t\right)^{1/4},
\end{equation}
où $m$ est la tangente de l'angle $\beta$ du sillon du joint de grains (cf. fig.~\ref{schemaMullins}), $t$ le temps. Cette équation permet donc, dans un cadre que nous appellerons \og théorie de Mullins \fg{}, de relier le coefficient de diffusion à la vitesse de sillonnement des joints de grains. En étudiant la longueur du temps d'induction en fonction de l'épaisseur initiale de la couche, Presland~\cite{presland1972hillock} trouve bien une relation en $t^{1/4}$. Il en conclut que ce phénomène est bien à l'origine de la formation des trous.\par 
Depuis la formulation du modèle de Mullins, quelques modifications ont été proposées. Génin \textit{et al.}~\cite{genin1992capillary} ont étudié dans des hypothèses comparables le comportement des jonctions de trois grains. Ils ont conclu que l'approfondissement de ces points de jonction était beaucoup plus prononcé que celui des joints de grains ; ils jouent sans doute une rôle prépondérant dans la formation de trous. Dans le cas d'une grande anisotropie de surface~\cite{klinger2011kinetics}, où la formation de facettes modifiant le profil du sillon a été observée, le sillonnement est modifié. Enfin, la diffusion à l'interface a été ajoutée au modèle pour décrire des profils qui ne correspondaient pas au modèle de Mullins~\cite{amram2014grain}. Ces ajouts tendent à remettre en question la manière dont s'effectue le sillonnement et sa cinétique, mais pas le fait que c'est bien le sillonnement qui est à l'origine des trous dans la couche.\par 
Dans le cas de l'argent, des études TEM (Microscopie Électronique à Transmission) de Dannenberg \textit{et al.}~\cite{dannenberg2000situ} ont précisément observé la formation d'un trou au niveau de joints de grains. Sur la figure~\ref{dannenbergTEMgroove}, une image TEM d'une couche d'argent de 80~nm est représentée. Sur cette image, on discerne entre deux grains une fine bande blanche qui signifie que l'épaisseur du film d'argent est nulle. Dannenberg interprète cet amincissement comme étant dû au sillonnement du joint de grains.\par
\begin{figure}[!htb]
\centering
\includegraphics[width=0.6\textwidth]{dannenbergTEMgroove}
\caption{Image TEM en vue plane d'une couche d'argent de 80~nm recuite à 400~$^\circ$C~\cite{dannenberg2000situ}.}
\label{dannenbergTEMgroove}
\end{figure}
Considérons à nouveau les travaux de Kwon \textit{et al.}~\cite{kwon2003comparison}, qui compare le démouillage de l'or à celui du cuivre. Nous avons vu que le nombre de trous formés dans la couche était extrêmement différent en fonction du système considéré. Kwon \textit{et al.} interprètent cette différence comme étant due à deux mécanismes différents de formation de trous. Dans le cas du cuivre, ils attribuent effectivement la création de trous au sillonnement des joints de grains, mais dans le cas de l'or, ils proposent une autre origine~: une agglomération des lacunes présentes dans le cristal et proches des joints de grains. Dans ce système, ils sont donc en désaccord avec le modèle de sillonnement des joints de grains, suggéré jusque-là~\cite{frost1990simulation}.\par 
Notons également que dans le cadre de la réorganisation cristalline, les plus petits grains peuvent subir une \og annihilation \fg{}~\cite{frost1990simulation}, c'est-à-dire qu'ils subissent une réorganisation les faisant disparaître, la matière étant transférée vers les grains voisins. Cette annihilation peut également donner lieu à la formation de trous~\cite{derkach2014grain}.\par 

\conclusion{Trois hypothèses coexistent concernant la formation de trous~: le sillonnement de joints de grains, l'annihilation des petits grains et la nucléation de lacunes. Dans le cas de l'argent polycristallin, le sillonnement de joints de grains semble prépondérant.}


		\subsubsection{Propagation des trous}
Comme nous l'avons expliqué dans la section~\ref{sHistoire}, la rétraction des films minces (avant d'être nommée démouillage) servait initialement à mesurer des coefficients de diffusion. Nous allons développer ici les modifications qui ont été apportées au modèle initial de Brandon et Bradshaw~\cite{brandon1966mobility}, lorsque le démouillage est devenu un objet d'étude en soi.\par 
Brandon et Bradshaw~\cite{brandon1966mobility} emploient l'équation~\ref{eDiffFlux} pour décrire le moteur du démouillage, mais sur une configuration schématisée dans la figure~\ref{schemaBrandon}~: un bourrelet de rayon $r$ se propageant sur une couche d'épaisseur $h$. Cette description est très inspirée par les problématiques de démouillage de polymères, notoires à l'époque.\par 

\begin{figure}[!htb]
	\centering
	\includegraphics[width = 0.3\textwidth]{schemaBrandon}
	\caption{Forme du front de propagation telle que décrite par Brandon et Bradshaw~\cite{brandon1966mobility}.}
	\label{schemaBrandon}
\end{figure}

En considérant que la courbure dans le film plat est nulle (le potentiel chimique est alors celui de référence), on obtient~:
\begin{equation}
\Delta \mu = \Omega\gamma\kappa = \Omega\gamma\left(\dfrac{1}{r}-\dfrac{1}{r'}\right),
\end{equation}
avec $r'$ le rayon de courbure parallèle au substrat, qui est considéré comme négligeable par Brandon et Bradshaw~\cite{brandon1966mobility}. Ce modèle implique deux choses~:
\begin{itemize}
\item l'existence d'un bourrelet qui se propage ;
\item l'épaississement de ce bourrelet au cours de la propagation.
\end{itemize}
Notons de plus que si le bourrelet s'épaissit, le moteur de sa propagation (sa courbure) diminue. Ainsi, plus il se propage, plus il est ralenti.\par 
Une modification notable de ce modèle a été proposée par Jiran et Thompson~\cite{jiran1990capillary, jiran1992capillary}. Leur étude porte sur le démouillage de couches d'or sur un substrat amorphe (silice ou nitrure de silicium). Jiran et Thompson observent que la vitesse de propagation des trous se fait à vitesse constante, contrairement à ce que nous avons conclu du modèle précédent. Selon eux, les trous forment des doigts (ou dendrites) qui se propagent (voir la figure~\ref{schemaDoigts})~; l'accumulation de matière se fait en réalité sur les bords des doigts. Ils adaptent donc l'équation de Brandon et Bradshaw pour tenir compte de cette modification~:
\begin{equation}
\Delta \mu = \Omega\gamma\left(\dfrac{1}{h}-\dfrac{1}{r'}\right).
\end{equation}
Et à nouveau, ils supposent que $r' \gg h$, donc $\kappa_\parallel \ll \kappa_\perp$ (avec $\kappa_\parallel$ la courbure dans le plan du substrat et $\kappa_\perp$ la courbure perpendiculaire au substrat de la surface du front de démouillage). Ainsi, au bout du doigt, le front n'est pas épaissi, le potentiel chimique est donc constant, impliquant que la vitesse est constante.\par 

\begin{figure}[!htb]
	\centering
	\includegraphics[width = 0.9\textwidth]{schemaDoigts}
	\caption{Forme du front de démouillage, selon Jiran et Thompson~\cite{jiran1990capillary,jiran1992capillary}.}
	\label{schemaDoigts}
\end{figure}

Ces modèles décrivent le matériau comme uniforme (absence de grains et de joints de grains) et isotrope (l'énergie de surface est identique en tout point). La structure cristalline du matériau n'est pas prise en compte, les modèles étant inspirés de la littérature de la matière molle. Finalement, plusieurs travaux ont proposé des modifications pour les cas qui s'écartaient de cette idéalité. Par exemple, l'anisotropie des énergies de surface a été prise en compte pour modifier la forme du front~\cite{klinger2011kinetics, zucker2013model}.\par 
Récemment, une nouvelle description de la propagation a émergé. Elle est centrée autour du rôle des grains \footnote{La littérature qui traite du rôle des grains est majoritairement issue d'un laboratoire israélien (Departement of Materials Engineering, Technion).}. Nous allons présenter trois modèles~; la figure~\ref{schemaGrainsIsraeli} permet de les comprendre.\par 

\begin{figure}[!htb]
	\centering
	\includegraphics[width = 0.9\textwidth]{schemaGrainsIsraeli}
	\caption{Forme du front de démouillage tenant compte des grains~\cite{atiya2014role, kosinova2014role, kovalenko2013solid}.}
	\label{schemaGrainsIsraeli}
\end{figure}

Atiya \textit{et al.}~\cite{atiya2014role} et Kosinova \textit{et al.}~\cite{kosinova2014role} étudient respectivement du platine et de l'or sur une alumine-$\alpha$. Les deux groupes d'auteurs proposent que le bourrelet est en réalité constitué de gros grains. Ils envisagent donc la propagation du bourrelet comme la croissance et la décroissance de grains successifs à proximité du front. La différence tient dans le fait que Kosinova \textit{et al.} introduisent la diffusion à l'interface comme mode de transport supplémentaire dans le modèle. Kovalenko \textit{et al.}~\cite{kovalenko2013solid} étudient le démouillage du fer, également sur l'alumine-$\alpha$. Pour leur part, ils n'observent pas de bourrelet, mais une croissance de grains au sein de la couche, loin du front. Ils proposent donc une croissance induite par la diffusion à longue portée, à l'interface, du matériau en train de démouiller. \par 
Dans ces études, la surface n'est pas décrite comme étant continue, du fait de la présence de grains. Le potentiel chimique est relié à la taille du grain, ainsi qu'à des énergies d'interface. La variation de potentiel lors de la disparition d'un grain est décrite par la grandeur~:
\begin{equation}
\Delta\gamma = \gamma_m+\gamma_i-\gamma_{sub},
\end{equation}
où les $\gamma_m$,$\gamma_i$ et $\gamma_{sub}$ sont respectivement l'énergie de surface du métal, l'énergie à l'interface substrat/métal et l'énergie de surface du substrat. $\Delta\gamma$ correspond à la diminution de la surface de métal exposée et de l'aire de l'interface métal/substrat mais aussi à l'augmentation de la surface de substrat exposée. Cette description permet de tenir compte de la nature discontinue de la couche métallique.\par 

\paragraph*{Remarque~:} Dans ces travaux, la diffusion à l'interface métal/substrat est considérée comme non négligeable. Ceci repose sur deux observations~: la forme des joints de grains, dont le profil ne correspond pas à ce qui attendu si on ne considère que la diffusion de surface~\cite{amram2014grain}, et la présence de grains de tailles très différentes placés côte à côte ou d'escarpements forts proches des joints de grains. Selon ces auteurs~\cite{kosinova2014role,kosinova2015mechanisms}, le profil devrait être lissé par une diffusion de surface rapide. Ils proposent des simulations numériques tenant compte de ces hypothèses pour étayer leur propos.\par 

	\subsection{Résumé}
Pour résumer les enseignements de la littérature, en ce qui concerne notre étude~:
\begin{itemize}
\item nous considérons le cas d'une couche polycristalline d'argent, sur un substrat amorphe ;
\item avant l'apparition et la propagation de trous, il y a une réorganisation de la structure cristalline et une apparition de buttes, cette étape est nommée \og temps d'induction \fg{} ;
\item les trous apparaissent le plus probablement par sillonnement des joints de grains ou annihilation des grains les plus petits ;
\item la rapidité du démouillage est fortement influencée par l'épaisseur initiale du film et la température de recuit (et d'un point de vue général, par la nature du métal)~;
\item la force motrice du démouillage ne fait pas l'unanimité~: il s'agit soit des variations de potentiel chimique dus à la courbure de la surface (modèle inspiré des polymères), soit d'un équilibre à l'échelle du grain entres les différentes énergies de surface et d'interface (qui tient compte de la microstructure).
\end{itemize}

\section{Le matériau~: l'argent}
	\subsection{Généralités}
L'argent, de numéro atomique 47 et de symbole Ag, a une température de fusion de 961,68~$^\circ$C.
Il cristallise dans un réseau de Bravais cubique face centrée (CFC) dont le paramètre de maille
est de 4,086 Å.\par 
Certains travaux concluent que la température de fusion d'un film mince varie en fonction de l'épaisseur. Cette variation a été étudiée par Gromov \textit{et al.}~\cite{gromov2011heterogeneous} (voir figure~\ref{gromovMelting}), qui estiment que pour un film d'argent de 20~nm, cette température est abaissée à 475~$^\circ$C.\par 
\begin{figure}[!htb]
	\centering
	\includegraphics[width = 0.6\textwidth]{gromovMelting}
	\caption{Évolution de la température de fusion de films minces métalliques en fonction de
l'épaisseur, selon~\cite{gromov2011heterogeneous}}
	\label{gromovMelting}
\end{figure}

Cependant, cette considération est très liée au démouillage~: la transition étudiée par Gromov \textit{et al.} correspond à la dispersion d'un film mince en gouttelettes. Comme il ne parle pas de démouillage dans ses travaux, nous supposerons que cet abaissement de température de fusion correspond plutôt à une température de démouillage. Dans le cas de particules supportées, cet abaissement est parfois appelé \og pré-fusion \fg{}. Il correspond à la contribution non négligeable des énergies de surface et d'interface dans la stabilité thermique du système, et a une dépendance en $1/a$ ($a$ le rayon de la particule). Il n'est concrètement significatif qu'à partir de tailles de l'ordre de 10~nm (variation supérieure à 5\%~\cite{takagi1954electron}).\par 

\conclusion{Nous considérons donc que l'argent reste toujours à l'état solide dans notre étude, ce qui est déterminant dans notre description~: l'argent reste cristallin et la diffusion de volume très lente par rapport à la diffusion de surface.\par}

	\subsection{Énergie de surface}
L'énergie de surface des solides cristallins n'est pas isotrope~: elle varie en fonction du plan cristallin exposé. Dans le cas de l'argent, plusieurs méthodes ont été employées pour déterminer les différentes énergies de surface\footnote{On limite en général cette étude aux facettes de bas indice, (111), (100) et (110), car les surfaces vicinales font intervenir le concept d'énergie de marche.}~\cite{stankic2013equilibrium,vitos1998surface}. Plusieurs valeurs extraites de la littérature sont représentées dans le tableau~\ref{tEnergies}. Ces valeurs sont situées entre 1,14 et 1,24 J.m$^{-2}$. La température peut modifier l'énergie de surface, mais nous supposerons une valeur constante.\par 
\begin{table}
\centering
\begin{tabular}{cccc}
\hline
facette & valeur (J.m$^{-2}$) & température ($^\circ$C) & source\\
\hline
$\cdot$ & 1,140 & 875-935 & exp.*~\cite{buttner1952adsorption}\\
$\cdot$ & 1,24 & -273 & exp.**~\cite{tyson1977surface}\\
$\cdot$ & 1,205 & -273 & exp.***~\cite{kumikov1983measurement}\\
(111) & 1,172 & -273 & DFT~\cite{vitos1998surface}\\
(100) & 1,200 & -273 & DFT~\cite{vitos1998surface}\\
(110) & 1,238 & -273 & DFT~\cite{vitos1998surface}\\
\hline
\end{tabular}
\caption{Valeurs des énergies de surface de l'argent solide de la littérature.*) les valeurs de~\cite{buttner1952adsorption} sont obtenues par mesure de tension de fils d'argent **) les valeurs de~\cite{tyson1977surface} sont extrapolées à 0~K à partir de mesures d'angle de contact à l'état liquide (962~$^\circ$C) ***) les valeurs de~\cite{kumikov1983measurement} sont des valeurs moyennes obtenues par des méthodes similaires à~\cite{buttner1952adsorption}.\og DFT \fg{} correspond à des calculs \textit{ab initio}. }
\label{tEnergies}
\end{table}
L'anisotropie des énergies est faible~: le ratio  $\gamma(111)/\gamma(110)$ calculé sous vide est de 1,056~\cite{vitos1998surface}, tandis que des mesures de rapports d'aspects sur des clusters d'argent équilibrés donnent une valeur de 1,08~\cite{stankic2013equilibrium}. Ces valeurs sont usuelles pour des matériaux CFC (cubique faces centrées)~\cite{sundquist1964direct}, dont l'or et le cuivre font également partie.\par
\paragraph*{Remarque~:} Le fait que l'énergie de surface ne soit pas isotrope est un élément de description très important pour les solides. Ici, nous n'avons considéré l'énergie des surfaces de bas indice. Entre ces surfaces, il existe autant de plans cristallographiques que d'orientations. D'un point de vue atomique et cristallographique, ces plans sont plus hauts en énergie. Ils ne seront pas ou peu représentés~; on passera donc, par exemple, d'un plan (111) à un plan (100).\par 
Ainsi, dire que l'anisotropie des énergies de surface est faible pour les surfaces de bas indice ne revient pas à dire que le métal n'est pas facetté. Cela implique seulement que les facettes (111), (100) et (110) auront tendance à toutes être représentées de manière équivalente. \textbf{À ce titre, l'argent est un excellent candidat pour tester les approches capillaires dans la problématique du démouillage, puisque la forme des particules n'est pas contrainte par une forte anisotropie}~\cite{brandon1966mobility}.\par 

	\subsection{Diffusion solide}
La diffusion a l'état solide peut emprunter plusieurs chemins~: la diffusion dans le volume, la diffusion aux joints de grains et la diffusion de surface (ou d'interface avec d'autres phases). Ces trois chemins sont associés à des coefficients de diffusion allant du plus petit au plus grand~: la diffusion de surface est généralement la plus rapide~\cite{heitjans2006diffusion}. On peut la désigner sous le terme d'autodiffusion lorsque l'espèce qui diffuse est de même nature que le solide au sein duquel elle diffuse. Les coefficients de diffusion suivent une loi d'Arrhénius~:
\begin{equation}
D = D_0e^{-Q/kT},
\label{eDiffusion}
\end{equation}
où $D_0$ est le coefficient d'auto-diffusion et $Q$ est l'énergie d'activation. Ce sont des paramètres à déterminer suivant les cas. Dans le cas de l'argent, ces coefficients dépendent de la facette considérée~\cite{agrawal2002predicting}. L'évolution des coefficients de diffusion avec la température est représentée sur la figure~\ref{diffusionCoeff}, pour les facettes de bas indices. Les coefficients sont calculés par simulation numérique (DFT). Ils augmentent avec la température, et la diffusion sur la face (111) est beaucoup plus rapide (trois ordres de grandeur à 200~$^\circ$C). Dans le cadre des couches minces, où les facettes exposées sont multiples et aléatoirement réparties, on considère généralement un coefficient de diffusion unique. Dans notre cas, nous le considérons compris entre les deux extrêmes de la figure~\ref{diffusionCoeff}.\par 
Nous pouvons estimer la distance caractéristique parcourue en une seconde en considérant $\sqrt{Dt}$. Cette distance est de 300~nm si D = 10$^{-9}$ cm$^{2}$.s$^{-1}$, et de 30 \micro\meter{} si D = 10$^{-5}$ cm$^{2}$.s$^{-1}$.\par 
\begin{figure}[!htb]
	\centering
	\includegraphics[width = 0.6\textwidth]{diffusionCoeff2}
	\caption{Évolution des valeurs des coefficients d'auto-diffusion de l'argent sur les facettes
principales en fonction de la température, selon~\cite{agrawal2002predicting}. Ces valeurs sont calculées par simulation \textit{ab initio} (DFT).}
	\label{diffusionCoeff}
\end{figure}



	\subsection{Influence de l'oxygène}
L'étude de l'interaction de l'oxygène avec l'argent a été principalement menée dans le domaine de la catalyse hétérogène. Cette interaction est complexe et très dépendante de la température et de l'activité de l'oxygène, mais nous verrons dans nos travaux (chapitre IV) qu'elle a un impact fondamental sur le démouillage. Nous allons l'aborder sous différents aspects~: formation de l'oxyde d'argent, adsorption et modifications des propriétés de surface.\par 

		\subsubsection{Oxydation de l'argent}
L'oxyde d'argent sous la forme la plus courante est Ag$_2$O. Sous forme massique et à pression standard, l'enthalpie libre de la réaction d'oxydation  $\Delta_rG^0$ suit la relation $\Delta_rG^0$~(kJ.mol$^{-1}$)$=-31,1+67.10^{-3}.T$, avec $T$ la température en Kelvin. L'oxyde n'est donc plus stable à partir de 200~$^\circ$C. En surface du matériau, la stabilité de l'oxyde en fonction de la température et de la pression partielle d'oxygène a été étudiée par calculs \textit{ab initio}~\cite{michaelides2003structures}. Un diagramme de phase est présenté sur la figure~\ref{diagrammeOxydationAg}. Ce diagramme concerne la face (111) de l'argent, mais nous considérerons qu'il est transposable à toutes les orientations. Nous observons que la température de décomposition est d'autant plus basse que la pression partielle d'oxygène est basse. Malgré les prévisions thermodynamiques, l'oxyde d'argent ne se forme pas spontanément dans l'air à température ambiante~\cite{rehren1991interaction}. Nous considérons donc qu'il n'y a pas d'oxyde d'argent massique qui se forme sur nos couches avant le début des expériences de démouillage. Pendant le démouillage, à plus haute température et sous des pressions partielles réduites en oxygène, l'oxydation est encore moins probable~: nous considérons qu'il ne se forme pas d'oxyde non plus.\par  
\begin{figure}[!htb]
	\centering
	\includegraphics[width = 0.4\textwidth]{diagrammeOxydationAg}
	\caption{Diagramme de phase Ag/Ag2O calculé dans~\cite{michaelides2003structures}.}
	\label{diagrammeOxydationAg}
\end{figure}
\paragraph*{Remarque :} Le fait que l'argent soit sous forme de particules diminue l'enthalpie libre de la réaction d'oxydation, $\Delta_rG^0$, mais cet effet ne devient vraiment perceptible que pour des particules dont la taille est de l'ordre de 10~nm~\cite{levard2012environmental, campbell2006transition}. Pour écarter tout risque, nous avons adopté un protocole rigoureux de production et de stockage des échantillons. Ce protocole sera décrit dans le chapitre 2.\par 




		\subsubsection{Adsorption de l'oxygène sur l'argent}
L'absence d'oxyde n'interdit pas l'interaction de l'oxygène avec l'argent~: il peut s'adsorber. On distingue plusieurs espèces d'oxygène adsorbé à la surface de l'argent~\cite{nagy1999correlation}~; ces différentes espèces sont représentés sur le schéma de la figure~\ref{schemaOxygeneEspeces}. O$\alpha$ correspond à une chimisorption dissociative, où les atomes d'oxygène restent en surface. Cette espèce se concentre sur les facettes orientées (110). S'il diffusent dans l'argent, on parle alors de l'espèce O$\beta$. Il existe une autre espèce, appelée O$\gamma$, qui est semblable à O$\alpha$, mais qui possède une énergie de liaison beaucoup plus forte~\cite{bao1996interaction}. De plus, elle ne se présente que sur les facettes orientées (111).\par 
\begin{figure}[!htb]
	\centering
	\includegraphics[width = 0.2\textwidth]{schemaOxygeneEspeces}
	\caption{Schéma représentant les différentes espèces d'oxygène interagissant avec l'argent. Inspiré de~\cite{nagy1999correlation}.}
	\label{schemaOxygeneEspeces}
\end{figure}
Les stabilités de ces espèces sont différentes. O$\alpha$ se désorbe à une température proche de 300~$^\circ$C~\cite{engelhardt1976adsorption}, tandis que O$\beta$ se désorbe à 375~$^\circ$C et O$\gamma$ à 600~$^\circ$C~\cite{nagy1999correlation}.\par 
L'adsorption d'oxygène n'est pas neutre pour le réseau cristallin~: le paramètre de maille peut augmenter en présence d'oxygène (de l'ordre de 0,2\% à 500~$^\circ$C)~\cite{nagy1999correlation}. Une autre modification très régulièrement observée est la modification de la topographie de la surface par création de nouvelles facettes~\cite{buttner1952adsorption, nagy1999correlation, engelhardt1976adsorption, yoshihara1979effect}. Une illustration de ce phénomène est présenté sur la figure~\ref{nagyFacetting}~: une feuille d'argent de 100 \micro\meter{} d'épaisseur (non supportée) a été recuite pendant quatre jours dans une atmosphère neutre (He) ou contenant de l'oxygène (10\%). Aucune modification n'est visible sur la première feuille, tandis que la morphologie de la surface de la feuille recuite sous oxygène a énormément changé. De nombreuses facettes sont visibles. Dans leur étude, Nagy \textit{et al.}~\cite{nagy1999correlation} désignent l'adsorption sélective de l'oxygène comme étant responsable d'un déplacement des atomes d'argent, menant à ce genre de structures.\par 
Notons que cet effet peut avoir lieu à des pressions très faibles d'oxygène~: Nagy \textit{et al.}~\cite{nagy1999correlation} observent cet effet alors que la pression d'oxygène n'excède pas 10$^{-5}$~atm.\par 
\begin{figure}[!htb]
	\centering
	\includegraphics[width = 0.8\textwidth]{nagyFacetting}
	\caption{Images MEB d'une feuille d'argent (100~\micro\meter) recuite à 750~$^\circ$C pendant quatre jours a) dans une atmosphère neutre (He) b) dans une atmosphère contenant 10\% d'oxygène. Images issues de~\cite{nagy1999correlation}.}
	\label{nagyFacetting}
\end{figure}

		\subsubsection{Abaissement de l'énergie de surface}
La mesure d'angle de contact d'une goutte d'argent liquide avec son substrat permet également de déterminer que l'oxygène abaisse l'énergie de surface du métal~\cite{chatain94}. Cette baisse a lieu dès une activité d'oxygène de 10$^{-3}$ (soit une pression de 10$^{-3}$~atm en considérant un gaz idéal). Cet abaissement est étayé par des simulations~\cite{muolo08}.\par 
L'effet observé pour de l'argent liquide a également lieu pour des solides. Buttner \textit{et al.}~\cite{buttner1952adsorption} effectuent des mesures de tension de surface en observant des fils d'argent lestés. Ils remarquent que l'oxygène abaisse l'énergie de surface de l'argent solide, dès une pression de l'ordre de 10$^{-5}$~atm (soit 1~Pa). La décroissance de l'énergie est linéaire avec le logarithme de la pression. Des simulations numériques effectuées par Molina \textit{et al.}~\cite{molina2011size} prédisent également une baisse de l'énergie de surface. Cette baisse est plus prononcée pour la face (100) que pour les autres faces de bas indice. La présence d'oxygène est donc susceptible de modifier l'anisotropie des énergies de surface.\par 

		\subsubsection{Influence de l'oxygène sur la diffusion}
Deux études font état d'une augmentation de l'autodiffusion de surface de l'argent en présence d'oxygène~\cite{rhead1965surface, yoshihara1979effect}. Dans les deux cas, la méthode employée pour mesurer les coefficients de diffusion est fondée sur une mesure de l'ouverture des joints de grains (cf. section~\ref{sHistoire}). Dans les deux cas, les mesures sont effectuées à une température comprise entre 650 et 960~$^\circ$C.\par 
Ces deux études estiment, grâce à la théorie de Mullins~\cite{mullins1957theory}, une augmentation du coefficient d'autodiffusion de deux ordres de grandeur. Cette augmentation pourrait être due à la fois à l'augmentation de $D_0$ (le coeffcient d'autodiffusion) et la diminution de $Q$ (la barrière d'activation, voir équation~\ref{eDiffusion}). Yoshihara~\cite{yoshihara1979effect} suggère que l'oxygène, en s'adsorbant à la surface de l'argent, occupe les sites où se fixent normalement les adatomes d'argent. Comme il leur est interdit de se fixer sur la surface, ils diffusent plus loin.\par 
Les valeurs prises à 700~$^\circ$C sont présentées dans le tableau~\ref{tDiffusion}. Remarquons que ces valeurs semblent en bon accord avec celles présentées sous vide dans la figure~\ref{diffusionCoeff}, mais prévoient une augmentation de un à deux ordres de grandeur du coefficient lorsque l'argent est exposé à l'air.\par 

\begin{table}
\centering
\begin{tabular}{ccccc}
\hline
source & \cite{rhead1963surface, rhead1965surface} & \cite{rhead1963surface, rhead1965surface} &\cite{yoshihara1979effect} & \cite{yoshihara1979effect}\\
\hline
condition ($P_{O_{2}}$) & 2.10$^4$~Pa & atm. réduc. & 10$^5$~Pa & 10$^{-11}$ Pa \\
valeur (cm$^2$.s$^{-1}$) & 5.10$^{-6}$ & 5.10$^{-8}$ & 8.10$^{-5}$ & 5.10$^{-6}$\\
\hline
\end{tabular}
\caption{Valeurs des coefficients d'autodiffusion de l'argent sur l'argent mesurés par la méthode de Mullins dans~\cite{rhead1963surface, rhead1965surface} et~\cite{yoshihara1979effect}.}
\label{tDiffusion}
\end{table}
Dans ce manuscrit, toutes les expériences présentées sont effectuées à des températures inférieures à 700~$^\circ$C, ce qui est hors de la gamme de température explorée dans la littérature mentionnée. En extrapolant les résultats de la littérature, une augmentation de la diffusion de deux ou trois ordres de grandeurs est attendue dans nos conditions de température.\par 

\conclusion{L'oxygène interagit avec l'argent solide par de nombreux biais. Même si nous excluons la formation d'oxyde d'argent dans notre système, l'adsorption de gaz a un effet sur la morphologie du métal. Cette adsorption est différente selon la facette considérée. On constate en parallèle que l'énergie de surface de l'argent est globalement réduite en présence d'oxygène, mais cette modification est plus prononcée pour la facette (100) que les autres. En outre, l'oxygène accélère l'autodiffusion de surface de l'argent. Il conviendra de tenir compte de ces phénomènes lors de la description de recuits effectués en présence d'oxygène.\par}



\section{Le contrôle du démouillage}
Le démouillage permet d'obtenir des particules à partir d'une couche initiale. Le démouillage en tant que procédé de synthèse a été de plus en plus étudié au cours de ces dernières années~\cite{thompson2012solid,leroy2016control}. Dans cette section, nous allons présenter plusieurs stratégies développées afin de contrôler les structures finales obtenues~:  contrôle au travers \textit{(i)} des paramètres de dépôt de la couche \textit{(ii)} des paramètres de recuit, \textit{(iii)} de la modification de la couche après dépôt, \textit{(iv)} de la nature du substrat et \textit{(v)} de la texturation du substrat.\par
	\subsection{Variation des paramètres expérimentaux}
La première manière de contrôler le démouillage, et la plus immédiate, consiste à faire varier les paramètres expérimentaux~: épaisseur initiale de la couche, température ou temps de recuit. Considérons en premier lieu l'épaisseur de la couche~: plus la couche est épaisse, plus les particules obtenues par démouillage sont grosses~\cite{schmidl2015formation, naffouti2016fabrication}. Ce paramètre a également été exploité par Pandey \textit{et al.}~\cite{pandey2017morphological} afin de faire varier les propriétés optiques des particules finales. Ruffino \textit{et al.}~\cite{ruffino2014self} observent ce phénomène sur des couches dont l'épaisseur varie à l'échelle locale.\par 
Comme nous l'avons évoqué au tout début de ce chapitre, la stabilité thermique des couches métalliques dépend de l'épaisseur. Aussi, pour une étude plus complète, est-il nécessaire de coupler l'étude de l'épaisseur de la couche à celle de la température et du temps de recuit~\cite{manuela2013dewetting}. Cet aspect est extrêmement important à prendre en compte~: il est nécessaire de chauffer à plus haute température lorsque l'épaisseur de la couche mince initiale augmente, afin de la faire démouiller. Ainsi, Tanyeli \textit{et al.}~\cite{tanyeli2013effect} remarquent qu'après le démouillage d'un film d'argent de 12~nm à 300~$^\circ$C coexistent des petites particules (assimilées à des demi-sphères de rayon 13~nm) avec des plus grosses (57~nm), tandis que l'on n'observe qu'une seule population de particules à 67~nm de rayon à 500~$^\circ$C. Morawiec \textit{et al.}~\cite{morawiec2013self} procèdent à des recuits de couches d'argent à des températures allant jusqu'à 500~$^\circ$C pour des couches d'épaisseur initiale comprise entre 5 et 25~nm. Ils observent (contrairement à Tanyeli \textit{et al.}) que la taille des particules dépend uniquement de l'épaisseur initiale de la couche, non de la température de recuit (cf. figure~\ref{morawiecSizeThickness} de la page~\pageref{morawiecSizeThickness}). Ils remarquent en revanche que la morphologie des particules obtenues peut varier~: des petites particules régulières, des particules en asticot ou un mélange de deux populations de particules de tailles très différentes (voir la figure~\ref{morawiecShapes} de la page~\pageref{morawiecShapes}).\par
\begin{figure}[!p]
	\centering
	\includegraphics[width = 0.5\textwidth]{morawiecSizeThickness}
	\caption{Évolution de la taille des particules après démouillage en fonction de l'épaisseur initiale de la couche (Ag/SiO$_2$). Les symboles vides représentent la taille des petites particules lorsque la population en taille est bimodale. Issu de~\cite{morawiec2013self}.}
	\label{morawiecSizeThickness}
\end{figure}
\begin{figure}[!p]
	\centering
	\includegraphics[width = 0.5\textwidth]{morawiecShapes}
	\caption{Diagramme montrant la forme des particules obtenues par démouillage, en fonction du temps de recuit et de l'épaisseur initiale (Ag/SiO$_2$). Type F~: démouillage partiel. Type I~: particules en asticots. Type S~: particules de taille et forme régulières. Type D~: deux populations de particules. Issu de~\cite{morawiec2013self}.}
	\label{morawiecShapes}
\end{figure} 
Morawiec \textit{et al.}~\cite{morawiec2013self} supposent qu'ils ont atteint l'état final après leur recuit, mais ne précisent pas le temps de recuit dans leur travaux. Cependant, de la même manière que la température de recuit voit son effet modifié par l'épaisseur initiale, le temps de recuit est également un paramètre important à prendre en compte. Finalement, il convient de considérer trois paramètres~: l'épaisseur initiale de la couche, la température de recuit et le temps de recuit. Une action sur n'importe lequel de ces paramètres constitue une forme de contrôle du démouillage.\par 

\paragraph*{Remarque~:} Un démouillage partiel (type F de la figure~\ref{morawiecShapes}) peut aussi avoir des propriétés optiques intéressantes~\cite{morawiec2013self, kunwar2016ag}. Pour une épaisseur donnée, on l'obtient plus facilement en limitant la température ou la durée du recuit.\par 

	\subsection{Mode de chauffage de la couche}
Un four ou une platine chauffante sont usuellement employés pour porter les échantillons à haute température. Kalfagiannis \textit{et al.}~\cite{kalfagiannis2016selective} proposent une synthèse de particules en illuminant la couche avec un laser. L'échauffement est alors issu de l'absorption d'énergie lumineuse. Le réglage de la fluence permet de contrôler la taille des particules et la distance entre elles. À plus forte puissance, le laser permet également de dépasser le point de fusion ; le démouillage liquide donne lieu à des états finaux différents~\cite{trice2007pulsed}. L'emploi du laser peut se faire même dans des empilements~\cite{siozios2015sub}. Makarov \textit{et al.}~\cite{makarov2016controllable} emploient un laser pour graver et faire démouiller simultanément une couche d'or, permettant l'obtention d'un réseau de particules.\par 
Le chauffage par micro-ondes a également été étudié~\cite{panagiotopoulos2015self}, ainsi qu'avec des rampes ultra-rapides~\cite{yu2016structural}. Cependant, ces modes de chauffage alternatifs n'ont pas été comparés aux modes usuels, et un seul jeu de paramètres a été employé dans les études citées~: il ne nous est pas possible de conclure quant à leur influence sur la morphologie.\par 

	\subsection{Modification après dépôt de la couche métallique}
Un autre stratégie consiste à modifier la couche métallique après son dépôt. La déformation par embossage de cette couche permet d'organiser la structure obtenue par démouillage~\cite{yoshino2012effects}. Mizsei et Lantto~\cite{mizsei2001situ} ont montré qu'il est possible de favoriser la création de trous à certains endroits précis en y exerçant une contrainte grâce à une pointe AFM. Un étude comparable a été réalisée par Kosinova \textit{et al.}~\cite{kosinova2017indentation} par indentation. Il semble que le démouillage ne soit pas seulement induit par une déformation locale de la couche, mais également par une réorganisation cristalline autour des indentations. Cette réorganisation est induite par des contraintes mécaniques.\par 
Notons enfin que l'ajout de couches au-dessus de la couche métallique rend son démouillage beaucoup plus lent ; il faut chauffer à plus haute température lorsque la couche est recouverte~\cite{thompson2012solid}. Recouvrir la couche métallique est donc en soi une manière de contrôler le démouillage.\par 

	\subsection{Modification de la nature du substrat}
	\label{sNatureSubstrat}
Nous avons souligné la multitude de systèmes étudiés dans la section 2). Pour une même couche métallique, la nature du substrat est déterminante pour déterminer la morphologie du démouillage, ceci constitue donc une forme de contrôle. Tanyeli \textit{et al.}~\cite{tanyeli2013effect} ont étudié le démouillage de l'argent sur Si, Si$_3$N$_4$, SiO$_2$, ITO (Indium Tin Oxide), et AZO (Aluminium Zinc Oxide, qui est du ZnO dopé Al). Même avec une couche initiale de même épaisseur (12~nm) et des conditions de recuit identiques, ils observent après démouillage des populations de particules différentes selon le substrat, comme par exemple sur ITO et AZO (voir figure~\ref{tanyeliSubstrateSize} de la page~\pageref{tanyeliSubstrateSize}). Sur AZO, la population finale de particules est centrée autour d'une valeur plus élevée et la distribution est plus large que sur ITO.\par 
L'épitaxie joue un rôle extrêmement important à cet égard~\cite{thompson1990grain, leroy2016control}. Outre la plus grande stabilité de la couche métallique lorsqu'elle est épitaxiée (comme Ag/ZnO), la présence d'orientations cristallines spécifiques permet d'obtenir des structures variées. Sur la figure~\ref{singleCrystalControl} de la page~\pageref{singleCrystalControl}, des monocristaux de Ni, épitaxiés sur un substrat de MgO monocristallin, démouillent et adoptent des morphologies différentes après le démouillage. Ceci est dû au fait que l'orientation des bords carrés est variable, alors que l'orientation cristalline ne l'est pas (car imposée par le substrat). Ainsi, modifier l'inclinaison des bords d'un monocristal par rapport à son orientation cristaline permet de changer la morphologie obtenue par démouillage.\par 
Dans une autre approche, Farzinpour \textit{et al.}~\cite{farzinpour2012altering} étudient le démouillage de l'or sur une couche sacrificielle d'antimoine, qui se sublime au fur et à mesure que l'or démouille. Plus cette couche d'antimoine est épaisse, plus les particules d'or finales sont grosses. Ceci est dû à deux phénomènes en compétition~: l'antimoine stabilise la couche d'or mais se sublime. Cette sublimation se fait uniquement lorsque la couche est percée~; la morphologie dépend davantage des points initiaux de sublimation que du démouillage de la couche d'or.\par 
\begin{figure}[!htb]
	\centering
	\includegraphics[width = 0.6\textwidth]{tanyeliSubstrateSize}
	\caption{Images MEB et analyse des populations en taille de particules d'argent obtenues par démouillage sur AZO ou ITO. Épaisseur initiale~: 12~nm. Issu de~\cite{tanyeli2013effect}.}
	\label{tanyeliSubstrateSize}
\end{figure} 
\begin{figure}[!htb]
	\centering
	\includegraphics[width = 0.6\textwidth]{singleCrystalControl}
	\caption{Démouillage de monocristaux de Ni de 120 nm d'épaisseur, en fonction de leur orientation cristalline. Image extraite de~\cite{ye2011templated}.}
	\label{singleCrystalControl}
\end{figure} 

	\subsection{Texturation du substrat}
	\label{sControleBiblio}
Un approche originale consiste à mettre en forme le substrat (texturation) avant le dépôt de la couche métallique. La texture du substrat contraint le démouillage pendant le recuit.\par  
La première étude proposant ce procédé est celle de Giermann et Thompson~\cite{giermann2005solid, giermann2011requirements}. Leur méthode consiste à procéder au démouillage d'une couche initiale continue sur un substrat de silicium dont la surface présente un réseau de trous en forme de pyramides inversées (obtenues par lithographie électronique). Le résultat de cette expérience est présenté sur la figure~\ref{giermannControl}.
\begin{figure}[!htb]
	\centering
	\includegraphics[width = 0.4\textwidth]{giermannControl}
	\caption{Démouillage d'une couche d'or sur un substrat texturé. Image issue de~\cite{giermann2005solid}.}
	\label{giermannControl}
\end{figure} 
\begin{figure}[!htb]
	\centering
	\includegraphics[width = 0.8\textwidth]{giermannTypes}
	\caption{Démouillage d'une couche d'or sur un substrat texturé par un réseau carré de pyramides inversées de 175 nm de période. a) couche de 21 nm, mésas étroites~: type \og dedans\fg, b) couche de 16 nm, mésas larges~: type \og mélangé\fg, c) couche de 21 nm, mésas larges~: type \og dehors\fg. Images issues de~\cite{giermann2011requirements}.}
	\label{giermannTypes}
\end{figure} 
\begin{figure}[!htb]
	\centering
	\includegraphics[width = 0.5\textwidth]{dimensionsCpyr}
	\caption{Paramètres géométriques employés dans nos travaux. $P$ désigne la période, $m$ la largeur de la mésa, $l$ la largeur de la base de la pyramide et $d$ la profondeur.}
	\label{schemaGeometrie}
\end{figure} 
Sur cette figure, nous pouvons observer que les particules d'or sont situées à l'intérieur des pyramides. Ceci assure une très grande organisation spatiale~: les particules suivent le réseau imposé par les pyramides. Le principe du démouillage sur une surface texturée a été employé dans différentes études postérieures. Par exemple, Oh \textit{et al.}~\cite{oh2009cobalt} reproduisent ces travaux avec du cobalt. La texture peut être modifiée, par exemple en considérant des lignes plutôt que des trous pyramidaux~\cite{petersen2008dewetting}, ou même des tranchées séparant des domaines plans~\cite{yoshino2012effects}. Altomare \textit{et al.}~\cite{altomare2016templated} utilisent ce principe pour produire des nanoparticules d'or de petite taille pour de la photocatalyse dans des trous tubulaires. L'utilisation de particules de polymère ou de silice organisées, soit comme masque pour texturer la sous-couche~\cite{yang2011template}, soit comme texture~\cite{wang2016thermal} relève également de la même approche.\par 

		\subsubsection{Modèle de Giermann}
Nous allons étudier plus en détail les possibilités de cette méthode, telles que définies par Giermann et Thompson.\par 
La première étude de démouillage sur une surface texturée a été publiée en 2005~\cite{giermann2005solid} et a été complétée par une seconde, parue en 2011~\cite{giermann2011requirements}. Dans cette seconde étude, une grande attention a été portée aux différentes configurations des structures démouillées obtenues par cette méthode. Trois cas typiques ont été identifiés (voir la figure~\ref{giermannTypes}). Le premier type, \og dedans \fg, est obtenu par démouillage d'une couche de 21~nm sur un réseau carré de pyramides de 175~nm de pas. Dans cette configuration, toutes les particules sont dans les trous. Le second type, \og mélangé \fg, est obtenu sur un substrat dont les mésas sont plus larges (c'est-à-dire le plateau séparant deux pyramides), avec une couche plus fine (16~nm). Certaines particules sont au fond des trous, mais d'autres sont simplement posées à la surface, sans interagir avec la texture. Enfin, le troisième cas, \og dehors \fg, est obtenu avec une couche de 21~nm, mais des mésas de pyramides beaucoup plus larges. Les particules sont beaucoup plus grosses et ne tiennent pas compte de la texture.\par 
Pour expliquer ces trois types, Giermann et Thompson ont proposé un modèle fondé sur les paramètres géométriques du système, définis sur la figure~\ref{schemaGeometrie}. $P$ désigne la période du réseau, $l$ désigne la largeur de la base, $d$ désigne la profondeur et $m$ la largeur de la mésa.\par 
Le volume de la pyramide est~:
\begin{equation}
\dfrac{1}{3}dl^2,
\end{equation}
et le volume de la couche contenu dans une période est~:
\begin{equation}
hP^2,
\end{equation}
avec $h$ son épaisseur. Pour délimiter les différents types de démouillage, Giermann définit des paramètres adimensionnés~: le rapport $h/d$ et le rapport $m/P$. Ces rapports sont tracés dans un repère, respectivement sur l'axe des ordonnées et l'axes des abscisses (voir figure~\ref{giermannDiagramme}). Giermann postule que pour être dans le régime \og dedans \fg, il faut que le volume de la couche n'excède pas le volume des pyramides. Sur le graphe, la ligne délimitant l'égalité des volumes est la courbe d'équation $3y = 1 - 2x + x^2$, tracée en bleu. Elle sépare les domaines correspondant aux structures démouillées de type \og dedans\fg{} (zone bleue) de celles de type \og mélangé\fg{} ou \og dehors\fg.\par 
\begin{figure}[!htb]
	\centering
	\includegraphics[width = 0.3\textwidth]{giermannShapePit}
	\caption{Schéma explicatif de l'orientation des particules d'or par rapport au substrat pour une particule localisée dans un trou en forme de pyramide inversée. Issu de~\cite{giermann2011requirements}.}
	\label{giermannShapePit}
\end{figure} 
La limite entre les cas \og mélangé \fg{} et \og dehors \fg{} a ensuite été estimée par analyse numérique~: l'évolution de la surface du métal a été modélisée par l'équation~\ref{eDiffFlux}. Si la surface du métal modélisée entre en contact avec le substrat au niveau du coin des mésas, la structure finale attendue est de type \og mélangé \fg. Elle prévoit qu'une augmentation du ratio $m/P$ tend à augmenter la probabilité de se trouver dans une configuration de type \og mélangé \fg.\par 

\paragraph*{Remarque~:} Cette structuration apportée par la texture de la surface concerne également l'orientation cristalline des particules. En effet, Giermann observe que l'orientation hors-plan dominante est (100), alors qu'elle est (111) sur une surface plane. Comme schématisé sur la figure~\ref{giermannShapePit}, cela provient du fait que la texture perpendiculaire à la surface correspond à des orientations (111) du substrat gravé de Si conduisant à une \og épitaxie géométrique\fg{} qui donne lieu à une orientation (100) dans le plan global (non incliné) du substrat.\par 
\begin{figure}[!htb]
	\centering
	\includegraphics[width = 0.6\textwidth]{giermannDiagramme}
	\caption{Diagramme de zone selon Giermann et Thompson~\cite{giermann2011requirements}. La limite représente l'égalité des volumes, au-delà de laquelle l'organisation des particules n'est pas possible.}
	\label{giermannDiagramme}
\end{figure} 
		\subsubsection{Travaux préliminaires du laboratoire SVI}
Lorsque l'on s'intéresse aux propriétés optiques des systèmes obtenus par cette méthode, la transparence du substrat est cruciale. Giermann et Thompson employaient du silicium pour obtenir par photolithographie les réseaux de pyramides inversées. Cependant, le silicium n'est pas transparent dans la gamme visible. La technologie de nano-impression permet de contourner cet obstacle en reproduisant la texture sur des substrats transparents.\par 
La nano-impression a été développée par Chou \textit{et al.}~\cite{chou1996nanoimprint} et a été adaptée par Le Bris \textit{et al.}~\cite{le2014self}. Il repose sur l'utilisation d'un polymère, le PDMS (polydiméthylesiloxane), qui est réticulé en contact avec une surface d'un moule déjà gravé (appelé \og master \fg), dont il conserve la forme en négatif. Ce PDMS sert ensuite de moule pour l'embossage de couches sol-gel. Ce procédé a été employé au laboratoire SVI pour texturer des couches de silice sur du verre~\cite{dubov2013superhydrophobic}. De plus amples détails seront fournis dans la partie expérimentale (chapitre 2).\par 
Le Bris \textit{et al.}~\cite{le2014self} ont procédé au démouillage de couches d'argent sur les textures de Giermann et Thompson reproduites par nano-impression. L'intérêt de cette approche est de disposer d'une organisation de nanoparticules métalliques sur un substrat transparent. La méthode permet non seulement de mesurer les propriétés optiques du système, mais aussi de copier le master un grand nombre de fois, à moindre coût. Un exemple de réseau de particules ainsi obtenu est présenté sur la figure~\ref{lebrisOrganisation}.\par 
\begin{figure}[!htb]
	\centering
	\includegraphics[width = 0.5\textwidth]{lebrisOrganisation}
	\caption{Image MEB d'une couche d'argent de 46 nm démouillée sur un substrat de SiO$_2$ texturé par la méthode de nano-imprint~\cite{le2014self}.}
	\label{lebrisOrganisation}
\end{figure}
Sur cette figure, on peut observer une organisation des particules d'argent au sein de la texture, qui correspond au type \og dedans \fg. Cependant, le ratio $V_{couche}/V_{pyr} = 1,3$, ce qui est donc supérieur à la limite imposée par le modèle de Giermann et Thomspon. Cela signifie que les particules s'organisent au sein des trous alors que leur volume est plus grand que les trous supposés les contenir.\par 

\subsection{conclusion}
En définitive, il existe plusieurs techniques pour contrôler le démouillage~:
\begin{itemize}
\item modifier les paramètres de démouillage~: épaisseur initiale de la couche, durée ou température
de recuit ;
\item changer de mode de chauffage ;
\item modifier la couche métallique après son dépôt ;
\item modifier le substrat, soit en changeant de nature chimique, soit en le texturant.
\end{itemize}
Nous avons sélectionné cette dernière approche~; c'est-à-dire procéder au démouillage sur une surface texturée. Nous avons cependant observé au cours d'expériences préliminaires que les modèles de la littérature ne décrivent pas les résultats obtenus dans notre système. Un des objectifs de ce travail de thèse a donc été d'en comprendre la raison.\par

\section{Réponse optique des structures métalliques}
La motivation qui nous pousse à vouloir contrôler le démouillage réside dans la réponse optique des systèmes obtenus, qui peut être modulée en fonction de l'organisation spatiale et la taille des objets formés. Cette réponse est d'intérêt dans de nombreuses applications, qui touchent notamment aux cellules photovoltaïques et à la fabrication d'éléctrodes transparentes (nanogrille d'argent). Dans le premier cas, on souhaite pour cela exalter l'absorption optique au travers de résonances plasmon, à une longueur d'onde choisie.\par 
La technique de NanoImprint laisse envisager la maîtrise à grande échelle de structures organisées répondant à ces besoins. 
Dans cette section, nous allons passer en revue de façon succincte les propriétés optiques d'objets métalliques.  La \og plasmonique\footnote{On peut définir la plasmonique comme étant le domaine d'étude et d'applications technologiques faisant intervenir l'interaction de la lumière avec des métaux, souvent à des échelles nanométriques.} \fg{} sera traitée dans les cas simples au début, puis dans des cas plus complexes. \par 
	\subsection{Fonction diélectrique des métaux}
La réponse optique d'un matériau se caractérise notamment par sa fonction diélectrique $\varepsilon(\omega)$~\cite{jacksonbook, palikbook}, où $\omega$ est la pulsation de l'onde électromagnétique. Pour un métal, cette fonction diélectrique est fortement impactée par la liberté qu'ont les électrons de se déplacer, interagissant de ce fait avec la lumière. Nous présentons dans un premier temps une description très basique du comportement des électrons.
		\subsubsection{Modèle de Drude}
Le modèle de Drude est une description dynamique du mouvement des électrons (issu de la mécanique newtonienne). Le mouvement d'un électron de masse $m$ et de position $x$ est décrit de la manière suivante~:
\begin{equation}
m\dfrac{d^2\textbf{x}}{dt^2} = \textbf{F}-\Gamma\dfrac{d\textbf{x}}{dt},
\label{eMouvement}
\end{equation}
sachant que~:
\begin{itemize}
\item $v = \dfrac{d\textbf{x}}{dt} $ est sa vitesse~;
\item $\textbf{F} = q\cdot(\textbf{E}+\textbf{v}\wedge \textbf{B})$ est la force électromagnétique qu'il subit, avec $q$ la charge, $\textbf{E}, \textbf{B}$ les champs électrique et magnétique~;
\item $\Gamma$ est un coefficient de frottement.
\end{itemize}
On écrit en formalisme complexe le champ $\textbf{E}$ et la position $\textbf{x}$ sous la forme $\textbf{E}=\textbf{E}_0e^{-i\omega t}$ et $\textbf{x}=\textbf{x}_0e^{-i\omega t}$. L'équation~\ref{eMouvement} devient donc~:
\begin{equation}
-m\omega^2 \textbf{x}-i\Gamma\omega\textbf{x} = -e\textbf{E},
\end{equation}
soit
\begin{equation}
\textbf{x} = \dfrac{e}{m\omega^2+i\omega\Gamma}\textbf{E}.
\end{equation}
En posant $\textbf{P} = -ne\textbf{x}$ ($\textbf{P}$ étant le champ de polarisation, $n$ la densité d'électrons), on obtient~:
\begin{equation}
P = \dfrac{-ne^2}{m\omega^2+i\omega\Gamma}\textbf{E}
\end{equation}
Pour faire intervenir la fonction diélectrique, il convient de considérer le champ de déplacement $\mathbf{D} = \epsilon_0 \mathbf{E} + \mathbf{P}	$~:
\begin{equation}
\textbf{D} = \varepsilon_0\left(1-\dfrac{\omega_p^2}{\omega^2+i\omega\gamma}\right)\textbf{E},
\end{equation}
où la fréquence plasma est définie comme $\omega_p^2 = ne^2/\varepsilon_0m$ et en posant $\gamma = \Gamma/m$. La fréquence plasma est caractéristique du métal, elle décrit la fréquence propre des oscillations du nuage d'électrons soumis au rappel exercé par les noyaux chargés positivement. Par définition, la constante diélectrique vaut~: $\textbf{D} = \varepsilon_0\varepsilon \textbf{E}$. On en déduit donc que~:
\begin{equation}
\varepsilon(\omega) = 1-\dfrac{\omega_p^2}{\omega^2+i\omega\gamma}.
\end{equation}
L'allure de la fonction diélectrique est représentée sur la figure~\ref{drudeDielectric}. La partie réelle $\varepsilon_1$ et imaginaire $\varepsilon_2$ sont représentées séparément. Il y a une divergence lorsque $\omega$ tend vers 0~: $\varepsilon_1$ tend vers $-\infty$ et $\varepsilon_2$ tend vers $+\infty$.\par 
Par ailleurs, $\gamma$ est rélié au libre parcours moyen des électrons $\Lambda = \hbar v_F/\gamma$ qui est la distance moyenne que l'électron peut parcourir sans collision. $v_F$ est la vitesse des électrons au niveau de Fermi, c'est-à-dire ceux concernés par le modèle de Drude. Pour indication $\Lambda$ est de l'ordre de quelques nanomètres dans les métaux à température ambiante.\par 
\begin{figure}[!htb]
	\centering
	\includegraphics[width = 0.75\textwidth]{drudeDielectric}
	\caption{Allure de la fonction diélectrique, selon le modèle de Drude.}
	\label{drudeDielectric}
\end{figure}
\paragraph*{Remarque~:} Lorsque $\omega\gg\omega_p$, les variations du champ sont trop rapides pour que les électrons puissent les suivre. On prend généralement en compte ce phénomène en écrivant~:
\begin{equation}
\varepsilon(\omega) = \varepsilon_\infty-\dfrac{\omega_p^2}{\omega^2+i\omega\gamma}.
\end{equation}
		\subsubsection{Transitions interbandes}
Aux hautes énergies, un partie du rayonnement pourra être absorbée au travers d'excitations électroniques. En théorie des bandes, on parlera de \og transitions interbandes \fg. Ces transitions peuvent être ajoutées relativement aisément dans le modèle de Drude en ajoutant à~\ref{eMouvement} un terme de rappel~:	
\begin{equation}
m\dfrac{d^2\textbf{x}}{dt^2}+\Gamma\dfrac{d\textbf{x}}{dt}+m\omega_i\textbf{x} = \textbf{F}.
\end{equation}
Il convient d'ajouter à l'équation autant de contributions $\omega_i$ que nécessaire. Ces contributions impactent la fonction diélectrique du gaz d'électrons libres sous la forme de composantes
lorentziennes du type~:
\begin{equation}
\varepsilon(\omega) = \dfrac{A_i}{\omega_i^2-\omega^2-i\gamma_i\omega}
\end{equation}
L'allure d'une contribution lorentzienne est représentée sur la figure~\ref{lorentzDielectric} pour deux valeurs de $\omega_i$. La première, $\omega_a$, correspond à une pulsation standard d'interbandes (\textit{e.g.} 3,9~eV pour l'argent~\cite{oates2005evolution}), tandis que la seconde, $\omega_a/100$ illustre ce qu'il se produit lorsqu'on fait tendre $\omega_i$ vers 0. La présence d'un pic dans la partie imaginaire $\varepsilon_2$ traduit physiquement une absorption. La partie réelle $\varepsilon_1$ marque un maximum suivi d'un minimum autour de la pulsation de résonance. Lorsque $\omega_i$ tend vers 0, l'allure de la lorentzienne est très semblable à un modèle de Drude. Nous exploiterons cette similarité dans nos travaux d'ellipsométrie (voir chapitre 3). \par
\begin{figure}[!htb]
	\centering
	\includegraphics[width = 0.75\textwidth]{lorentzDielectric}
	\caption{Allure de la fonction diélectrique d'une contribution lorentzienne. L'allure est
présentée pour deux valeurs de $\omega_i$ ($\omega_i=\omega_a$, choisi arbitrairement, et $\omega_i=\omega_a/100$).}
	\label{lorentzDielectric}
\end{figure}
Pour l'argent, la fonction diélectrique est représentée sur la figure~\ref{argentDielectric}. On observe bien un comportement de type Drude aux basses énergies, et des composantes de type interbandes à partir de 3,9~eV. Ces caractéristiques font de l'argent un très bon candidat pour les propriétés plasmoniques décrites ci-après.\par
\begin{figure}[!htb]
	\centering
	\includegraphics[width = 0.75\textwidth]{argentDielectric}
	\caption{Fonction diélectrique de l'argent, selon le données fournies par Woollam Co.~\cite{herzinger1998ellipsometric}.}
	\label{argentDielectric}
\end{figure}
	\subsection{Effets de confinement~: réponse d'une particule dans un milieu homogène}
Après avoir vu l'allure de la fonction diélectrique dont la partie imaginaire gouverne l'absorption volumique, il faut souligner que si l'object métallique a une taille réduite le déplacement des électrons est confiné. Cela conduit à des absorptions spécifiques liées à la forme de l'objet, appelées résonances plasmon~\cite{maier2007plasmonics,kreibigbook}.\par 
Pour commencer, nous allons traiter d'un cas dont la solution analytique est connue~: des petites particules sphériques dans un milieu homogène. Puis nous nous intéresserons à des systèmes qui s'écartent de ce cas idéal, décrit par un modèle \og simple \fg.\par 
		\subsubsection{Plasmons de surface localisés (LSP) dans des particules métalliques}
Lorsque la taille $D$ d'une particule est négligeable devant la longueur d'onde $\lambda$, le champ électrique qui y règne peut être considéré comme homogène, et le déplacement des électrons peut être considéré comme un dipôle, dans le champ lointain. C'est l'approximation quasi-statique. Nous allons voir le développement mathématique de cette hypothèse.\par 
Considérons le système représenté sur la figure~\ref{schemaPlasmons}~: une particule de rayon $a\ll \lambda$ illuminée
par un champ $\textbf{E}$, de fonction diélectrique $\varepsilon(\omega)$, entourée par un milieu extérieur de
fonction diélectrique constante $\varepsilon_m$.\par 
\begin{figure}[!htb]
	\centering
	\includegraphics[width = 0.5\textwidth]{schemaPlasmons}
	\caption{Notations employées pour décrire les plasmons de surface localisés.}
	\label{schemaPlasmons}
\end{figure}
Dans l'approximation quasi-statique, l'équation qui décrit ce système est l'équation de Laplace, pour le potentiel électrostatique $\Phi$~:
\begin{equation}
\Delta\Phi = 0,
\end{equation}
avec $E = -\nabla\Phi$. Les solutions de cette équation sont exprimées en fonction des variables $r$ et $\theta$, ainsi que des polynômes de Legendre. Les conditions aux limites donnent finalement la solution suivante~:
\begin{equation}
\Phi = - E_0r\cos \theta + \dfrac{\textbf{p}\cdot \textbf{r}}{4\pi\epsilon_0\epsilon_mr^3},
\end{equation}
avec
\begin{equation}
\textbf{p}= 4\pi \varepsilon_0 \varepsilon_m a^3 \dfrac{\varepsilon-\varepsilon_m}{\varepsilon+2\varepsilon_m}\textbf{E}.
\end{equation}
$\textbf{p}$ est le dipôle induit, qui permet d'introduire la polarisabilité $\alpha$ en posant $\textbf{p} = \varepsilon_0 \varepsilon_m\alpha \textbf{E}$.
\begin{equation}
\alpha(\omega) = 4\pi a^3\dfrac{\varepsilon(\omega)-\varepsilon_m}{\varepsilon(\omega)+2\varepsilon_m}.
\end{equation}
La polarisabilité est la quantité clé, dans l'approximation quasi-statique, qui permet de décrire la manière dont les particules métalliques interagissent avec la lumière. Remarquons qu'il y a une résonance lorsque $\varepsilon+2\varepsilon_m=0$, ce qui est plus usuellement écrite $\varepsilon_1(\omega) = -2\varepsilon_m$, en supposant que les parties imaginaires sont petites et $\varepsilon_m$ constante. Cette égalité est appelée la \textbf{condition de Fröhlich} pour la sphère, correspondant à l'excitation des résonances plasmon localisées (Localized Surface Plasmon Resonances, LSPR). Physiquement, le nuage électronique excité par l'onde électrique subit une force de rappel par le milieu continu.  L'absorption de la lumière est due au terme de déphasage.\par 
L'effet de l'environnement apparaît au travers de $\varepsilon_m$, qui déplace la résonance. Par exemple, une même particule d'argent dans l'air $(\varepsilon_m = 1)$ aura une résonance à plus haute énergie que dans la silice ($\varepsilon_m = 1,45$, mais cette valeur typique peut varier en fonction de la densité de la silice considérée).\par  
On peut définir $\sigma_{dif}$ et $\sigma_{abs}$, les sections efficaces de diffusion ou d'absorption. Elles sont définies comme le rapport entre la puissance diffusée  ou absorbée par la particule et l'irradiance de l'onde incidente. Ces sections efficaces sont parfois normalisées par la section de la particule ($\pi a^2$) pour obtenir les efficacités de diffusion et d'absorption, $Q_{dif}$ et $Q_{abs}$. Dans le cadre de l'approximation quasi-statique, on a~:
\begin{equation}
Q_{dif}=\dfrac{k^4}{6\pi}|\alpha|^2 \hspace{2cm} Q_{abs} = k \mathrm{Im}(\alpha),
\end{equation}
avec $k=2\pi/\lambda$ le nombre d'onde. Pour des particules plus petites que la longeur d'onde $ a \ll \lambda$, l'absorption domine car $\sigma_{abs} \sim a/\lambda  \ll \sigma_{dif} \sim (a/\lambda)^4$. Pour l'illustrer, $Q_{dif}$ et $Q_{abs}$ ont été calculés pour une particule de 10~nm de rayon, avec un modèle de Drude modélisant l'argent~\cite{yang2015optical} (voir figure~\ref{QvsSize}-a). L'évolution des intensités maximales en fonction de la taille de la particule est présentée sur la figure~\ref{QvsSize}-b. On observe bien une résonance pour les deux facteurs, mais également que $Q_{abs}$ domine pour les petites tailles, tandis que $Q_{dif}$ domine pour les grandes tailles. \par 
\begin{figure}[!htb]
	\centering
	\includegraphics[width = 0.5\textwidth]{Qabs_dif_10nm}\includegraphics[width = 0.5\textwidth]{Q_rayon}
	\caption{Gauche: $Q_{dif}$ et $Q_{abs}$ calculés pour une particule de rayon 10~nm dont la fonction diélectrique est un modèle de Drude adapté à l'argent~\cite{yang2015optical}. Droite~: évolution des intensité des résonances de $Q_{dif}$ et $Q_{abs}$ en fonction du rayon de la particule.}
	\label{QvsSize}
\end{figure}

\paragraph*{Remarque~:} Le cas des particules ellipsoïdales a également été traité dans la littérature~\cite{bohren1983surface}. Il en résulte que la polarisabilité prend une valeur différente suivant l'axe elliptique considéré, au travers du facteur de \og dépolarisation \fg. L'anisotropie de forme permet  ainsi d'observer des résonances à des fréquences différentes. De la même manière, facetter ou tronquer la particule et/ou la mettre en contact avec un substrat brise la symétrie du cas sphérique et donne lieu à des modes de résonances dits \og multipolaires \fg{} à des fréquences spécifiques~\cite{bedeauxbook,lazzari02d,lazzari14b}. Les fréquences dépendent de constantes diélectriques des matériaux mais la nature des modes uniquement de la géométrie de l'objet. Nous rediscuterons de ces effets par la suite.\par 
\conclusion{La réponse optique d'une particule métallique dépend uniquement de sa polarisabilité dans l'approximation quasi-statique. La section efficace d'absorption domine celle de diffusion.}

	\subsubsection{Au-delà de l'approximation quasi statique}
\paragraph*{Théorie de Mie~:}
Dans le paragraphe précédent, nous avons présenté les solutions analytiques existantes pour un cas simple~: une petite particule ($a\ll\lambda$), de forme sphérique ou ellipsoïdale plongée dans un milieu homogène. La taille de la particule permettait de se placer dans le cadre de l'approximation quasi-statique. Comme nous l'avons vu dans la section~C.\ref{sControleBiblio}, les tailles typiques que l'on obtient pour nos particules sont de l'ordre de plusieurs centaines de nanomètres, comparables aux longueurs d'onde dans le visible.\par 
Dans ces conditions, on peut observer un retard de polarisation (variations du champ au sein de la particule), ainsi que l'émergence de nouveaux modes qui ne sont pas des dipôles (quadrupolaire ou d'ordre supérieur). Mie~\cite{mie1908beitrage} développa une théorie qui implique un développement en série multipolaire de la solution exacte retardée pour une une sphère~\cite{kreibig2013optical, bohren1983surface}. Ce développement en série, pour être calculé, nécessite un traitement informatique.\par 
\paragraph*{Simulations numériques~:}
Lorsqu'on s'écarte de la forme sphérique/sphéroïdale, le traitement analytique de type Mie devient impossible. Il existe plusieurs méthodes pour calculer la réponse optique d'un système de géométrie donnée~; elles diffèrent par leur manière d'approcher le problème. Nous en donnons quelques exemples. La méthode DDA (Discret Dipole Approximation) repose sur la division de la structure métallique en une multitude de dipôles qui interagissent entre eux~\cite{draine1994discrete}. La méthode RCWA (Rigorous Coupled Wave Analysis) étudie l'interaction de la lumière avec une surface diffractante (donc périodique)~\cite{moharam1981rigorous}. Elle fait appel à la décomposition de l'espace réciproque en séries de Fourier. On peut également modéliser le système par éléments finis ; séparer l'espace en un maillage discret, et résoudre les équations de l'évolution de l'onde pour chaque boite en fonction de ses voisines~\cite{jin2015finite}.\par 
Toutes ces méthodes sont des manières d'approcher la solution par des calculs numériques dans un cadre d'hypothèse de travail. Elle impliquent une étude précise de la convergence de la solution, mais également une comparaison, si possible, avec des résultats expérimentaux.\par 

		\subsubsection{Intérêts des systèmes métalliques de taille finie}
Les particules métalliques peuvent permettre d'augmenter l'efficacité de cellules photovoltaïques~\cite{morawiec2013self, tan2013improved, tan2013micro, schuster2015plasmonic}. Le but est de modifier $\sigma_{dif}$ et $\sigma_{abs}$ dans la cellule~: augmenter la diffusion permet de prolonger le trajet optique dans la cellule et d'augmenter l'absorption par des LSP, qui produisent davantage de paires électrons-trous dans la phase active~\cite{atwater2010plasmonics}.\par 
Morawiec \textit{et al.}~\cite{morawiec2013self} étudient l'influence de la taille des particules d'argent obtenues par démouillage sur l'efficacité intégrée (sur tout le spectre visible) de diffusion. Les résultats sont présentés sur la figure~\ref{morawiecDiffusionSize}. On peut observer que plus la taille des particules est grande, plus l'efficacité de diffusion augmente.\par 
\begin{figure}[!htb]
	\centering
	\includegraphics[width = 0.5\textwidth]{morawiecDiffusionSize}
	\caption{Efficacité de diffusion intégrée sur le spectre optique de particules d'argent obtenues par démouillage, en fonction de leur taille moyenne. Issu de~\cite{morawiec2013self}.}
	\label{morawiecDiffusionSize}
\end{figure}
D'autres voies d'application sont envisageables, comme les vitrages bas-émissifs, qui demandent de l'absorption ou de la réflexion exclusivement dans la gamme infrarouge (IR), ou bien la modification de la coloration d'un vitrage, qui demanderait de l'absorption spécifique dans la gamme visible.\par 

\conclusion{L'interaction de la lumière avec un métal sous forme d'objets dont l'extension spatiale est limitée est différente de celle d'un métal massif. Cette réponse est connue de manière analytique pour des cas simples comme les particules sphériques dans l'approximation quasi-statique, dans un milieu homogène. En revanche, pour des particules de forme complexe, de grande taille ou dans un milieu non-homogène (comme sur un substrat), déterminer la réponse optique nécessite d'employer des méthodes spécifiques de résolution des équations de Maxwell. Ces mêmes paramètres sont également des moyens de contrôle en vue de possibles applications.\par}

	\subsection{Étude de systèmes complexes}
Nous allons étudier ici l'influence de certaines caractéristiques des particules sur leur réponse optique, lorsqu'elle ne rentrent pas dans le cadre de l'approximation quasi-statique. C'est bien le cas des particules obtenues par démouillage~: elles sont trop grandes pour être considérées comme ponctuelles, elles se situent à l'interface de deux milieux (substrat/air en général) et sont susceptibles d'interagir entre elles. Nous allons étudier ces différents aspects.\par 

		\subsubsection{Particules entre deux milieux diélectriques}
Royer \textit{et al.}~\cite{royer1987substrate} ont calculé la réponse théorique de particules ellipsoïdales aplaties dans différentes configurations~: dans le vide, dans du quartz ou sur un substrat de quartz. Pour le substrat de quartz, ils considèrent un indice effectif égal à la moyenne des deux indices des milieux homogène. Les résultats sont indiqués dans la figure~\ref{royerResonanceShift}. Le diamètre (ou le grand axe) des particules considéré par le calcul est de 40~nm.\par 
\begin{figure}[!htb]
	\centering
	\includegraphics[width = 0.5\textwidth]{royerResonanceShift}
	\caption{Position des pics de LSP de particules d'argent sphéroïdales, en fonction de leur rapport d'aspect R et du substrat. R=1 désigne une sphère et R=0 des disques plats. La ligne pointillée montre un calcul dans le vide, la ligne traitillée montre le calcul dans le quartz et la ligne solide correspond à des particules supportées. Les segments horizontaux montrent les variations expérimentales des positions des LSPR lorsqu'on recouvre des particules supportées par une couche de même indice que le substrat. Issu de~\cite{royer1987substrate}.}
	\label{royerResonanceShift}
\end{figure}
Considérons en premier lieu R = 1 (cas d'une particule sphérique). Dans l'air, le pic de résonance est unique et se situe vers 350~nm. Dans le quartz, le pic de résonance est unique également et se situe vers 400~nm. Lorsque la particule est supportée, même si elle est sphérique, donne lieu à deux résonances en raison d'effets images dans le substrat~\cite{yamaguchi74}.\par 
Si on déforme la particule en l'aplatissant, dans un milieu homogène, cela donne lieu à deux résonances (au lieu d'une). Un des résonances est décalée vers les faibles longueurs d'onde, tandis que l'autre est décalée vers les grandes longueurs d'onde. Quand la particule est supportée, elle a déjà deux résonances, même dans le cas sphérique (à cause des effets d'image). La déformation écarte encore davantage ces deux résonances.\par 
Lorsque le substrat est caractérisé par $\varepsilon_{sub}$ et le superstrat\footnote{le superstrat est le milieu se trouvant au dessus de la particule, par opposition au substrat.} par $\varepsilon_{sup}$, l'indice effectif est en général pris comme étant~\cite{morawiec2013self}~:
\begin{equation}
\varepsilon_{eff} = \dfrac{\varepsilon_{sub}+\varepsilon_{sup}}{2}
\end{equation}
Des calculs réalisés par une méthode d'éléments finis sur des objets plasmoniques indiquent que cette hypothèse donne de bons résultats~\cite{miljkovic2012simulating}.\par 
Si l'on se ramène au contexte du démouillage, le substrat joue donc un rôle important à deux titres~: à la fois dans l'état final de la structure métallique obtenue, mais aussi dans sa réponse optique. Ce double effet a été étudié par Tanyeli \textit{et al.}~\cite{tanyeli2013effect}. Ils ont procédé à la comparaison du démouillage de l'argent sur différents substrats (cf. section~\ref{sNatureSubstrat}), mais aussi de l'influence du substrat sur les propriétés optiques (voir figure~\ref{tanyeliSubstrateResonance}). Sur cette figure, nous observons que des particules de même taille ne présentent pas de LSPR à la même longueur d'onde lorsque le substrat varie. Notons que la LSPR pour des particules d'argent dont la taille est comprise entre 100 et 150~nm est située aux alentours de 450~nm. Ce pic correspond à celui décrit par Royer \textit{et al.}~\cite{royer1987substrate}, dont la position dépend de la forme de la particule~: à 400~nm lorsque le particule est sphérique, il se déplace vers le rouge lorsqu'elle est aplatie. Notons que ce dernier résultat a été obtenu par calcul pour des particules dont la taille était de 40~nm.\par  
\begin{figure}[!htb]
	\centering
	\includegraphics[width = 0.5\textwidth]{tanyeliSubstrateResonance}
	\caption{Position du pic de LSPR de particules d'argent obtenues par démouillage, en fonction de leur taille moyenne et du substrat. Issu de~\cite{tanyeli2013effect}.}
	\label{tanyeliSubstrateResonance}
\end{figure}
		\subsubsection{Variation de la forme des particules}
Nous venons de voir que la forme d'une particule influence la position des LSPR. Nous avons étudié le cas de particules éllipsoïdales, mais d'autres formes sont possible. Par exemple, le fait d'avoir des particules de type I (en asticot) ou de type S (sphériques, voir figure~\ref{morawiecShapes}) modifie la réponse optique~\cite{morawiec2013self, tan2013micro}. Murray \textit{et al.}~\cite{murray2007plasmonic} ont étudié des particules aux formes plus régulières, produites par lithographie par faisceau d'électrons (EBL). Ils ont en particulier comparé la réponse de particules en forme de disques, triangulaires ou en bâtonnet, comme représentés sur la figure~\ref{murrayShapeResonance}. Un bâtonnet présente deux pics~: à 450 et 875~nm. Un disque présente une seule résonance à 550~nm. Un triangle présente deux pics à 460 et 575~nm, qui sont décalés vers le rouge (475 et 650~nm) lorsque le triangle est plus gros.\par 
\begin{figure}[!htb]
	\centering
	\includegraphics[width = 0.5\textwidth]{murrayShapeResonance}
	\caption{Position du pic de LSP de particules d'argent obtenues par EBL (lithographie par faisceau d'électrons), en fonction de leur forme. Issu de~\cite{murray2007plasmonic}. La barre d'échelle indique 300~nm.}
	\label{murrayShapeResonance}
\end{figure}
Le nombre de pics est influencé par le degré de symétrie du système~: la sphère n'en a qu'un, tandis que le bâtonnet en présente deux (une résonance longitudinale et une résonance transverse). En d'autres termes, la polarisabilité (plus précisément~: le tenseur de polarisabilité) n'est pas isotrope. Si le cas des sphéroïdes a été analytiquement étudié, la recherche de solution pour des particules dont la forme est plus complexe nécessitent des études numériques (et/ou expérimentales).\par 
Chateau \textit{et al.}~\cite{chateau2015gold} étudient des particules d'or dont la forme est une bipyramide. En modifiant les conditions de synthèse de ces bipyramides, ils parviennent à faire varier leur rapport d'aspect. La position des pics en est modifiée~: plus la taille de l'objet est grande, plus la résonance est décalée vers le rouge. Ce principe peut s'exprimer, en termes généraux~: plus la taille du dipôle est grande, plus son énergie de résonance est faible. Par cette méthode, il parviennent à obtenir des résonances à grande longueur d'onde (1800~nm).\par 
Notons de plus que les endroits où la surface métallique présente des points anguleux  sont les endroits où le champ électromagnétique peut se trouver exalté~\cite{miljkovic2012simulating} (effet de pointe bien connu en électrostatique). Cet effet est particulièrement intéressant s'il a lieu dans une couche active d'une cellule photovoltaïque~: il permet plus de conversion de lumière en paires électrons-trous.\par 

		\subsubsection{Interaction entre particules et réseaux}
Jusqu'ici, nous avons implicitement mis de côté les interactions possibles entre particules, qui pourtant peuvent affecter la réponse des objets. Même dans l'approximation quasi-statique, le dipôle excité dans une particule peut modifier le champ local perçu par sa voisine conduisant au premier ordre à un simple décalage fréquentiel par rapport à l'objet isolé~\cite{yamaguchi74}. Nous allons étudier ici la possibilité de ces interactions, et leur effet sur la réponse optique.\par 
Maier~\cite{maier2007plasmonics} a réalisé des mesures de la position des LSPR sur des chaines de particules d'or de 50~nm de taille. Les résultats sont présentés sur la figure~\ref{maierGoldArrays}. En fonction de la polarisation de l'onde électrique (dans l'alignement des particules ou dans la direction orthogonale), le pic est décalé vers le bleu (mode transverse) ou vers le rouge (mode longitudinal). Cet effet est d'autant plus prononcé que la distance inter-particules est faible. Pour une distance de 100~nm, le décalage est de l'ordre de 5~nm. Si la lumière n'est pas polarisée, le décalage vers le rouge sera le plus conséquent~\cite{kinnan2010plasmon}.\par 
\begin{figure}[!htb]
	\centering
	\includegraphics[width = 0.85\textwidth]{maierGoldArrays}
	\caption{Position des LSPR de particules d'or sphériques de 50 nm de taille disposées en ligne, en fonction de la distance $d$ qui les sépare. a) Image de la structure, b) mesures expérimentales et ajustement par une dépendance en $d^{-3}$. Issu de~\cite{maier2007plasmonics}.}
	\label{maierGoldArrays}
\end{figure}
D'autres interactions plus prononcées ont été mises en évidence par Zou \textit{et al.}~\cite{zou2004narrow}. En étudiant des systèmes périodiques de particules d'argent, ils expriment la polarisabilité d'une particule de rayon $a$ dans un réseau comme étant~:
\begin{equation}
P = \dfrac{-A}{\omega - (\omega_p - \mathrm{Re}(S))+(i\gamma+\mathrm{Im}(S))}
\end{equation}
avec $\omega_p$ la pulsation de LSPR d'une particule isolée, $A = \dfrac{1}{2}\omega_pa^3$ et $S$ la \og somme des dipôles
retardés \fg, qui s'exprime~:
\begin{equation}
S = \sum_{i\neq j}\left(\dfrac{(1--ikr_{ij})(3\cos^2\theta_{ij}-1)e^{-ikr_{ij}}}{r_{ij}^3} +\dfrac{k^2\sin^2\theta e^{-ikr_{ij}}}{r_{ij}}\right)
\end{equation}
La partie réelle de S détermine le déplacement de la LSPR, tandis que sa partie imaginaire détermine sa largeur. En réalisant des calculs pour des sphères d'argent de 50~nm en réseau, Zou \textit{et al.} obtiennent les réponses représentées sur la figure~\ref{zouLatticeShift}. On observe que les LSPR sont décalées vers le rouge lorsque la distance inter-particules augmente. Ce décalage en longueur d'onde est beaucoup plus important que dans le cas précédent (jusqu'à 200~nm de différence) et malgré des distances qui dépassent largement 100~nm.\par
\begin{figure}[!htb]
	\centering
	\includegraphics[width = 0.5\textwidth]{zouLatticeShift}
	\caption{Position des LSPR de particules d'argent sphériques de 50~nm de taille disposées en réseau hexagonal, en fonction de la distance qui les sépare. L'onde est en incidence normale. Issu de~\cite{zou2004narrow}.}
	\label{zouLatticeShift}
\end{figure}
Enfin, une autre piste est étudiée par Kravets \textit{et al.}~\cite{kravets2008extremely}. Le fait que les particules soient en réseau donne lieu à un phénomène de diffraction. Que ce soit en réflexion ou en transmission, pour une longueur d'onde donnée, le réseau donnera lieu à des interférences constructives dans des directions spécifiques qui dépendent du réseau. Pour un réseau 1D illuminé en incidence normale, en considérant les ordres de diffraction dans le plan d'incidence, les directions et les longueurs d'onde sont reliées par l'équation~:
\begin{equation}
\sin\theta_m=\dfrac{m\lambda_m}{n_iP},
\end{equation}
avec $P$ la période du réseau, $n_i$ l'indice de l'air ou du substrat, $m$ un entier et $\theta_m$ l'angle auquel l'ordre de diffraction est observé. Les pics principaux sont attendus aux longueurs d'onde $\lambda_{m,air} = P sin \theta_m / m$ et $\lambda_{m,sub} = n_{sub}P sin\theta_m / m$.\par 
On remarque que si $\lambda_{m,sub} = n_{sub}mP$, alors $\sin \theta_m = 1$~: la direction est confondue avec la surface du substrat. Dans cette configuration où l'ordre diffracté est d'incidence rasante, la lumière peut interagir avec les particules sur un chemin très long et donner lieu à des résonances nouvelles. C'est ce qu'ont observé Kravets \textit{et al.}~\cite{kravets2008extremely} sur des réseaux de nanoparticules d'or synthétisés par EBL (lithographie électronique). Ces résonances dues au réseau lui-même n'apparaissent pas dans l'approximation quasi-statique~\cite{yamaguchi74}.\par 
 
\conclusion{Dans un réseau de particules, la position des LSPR peuvent être déplacées et de nouvelles résonances peuvent apparaître. La distance inter-particules (la période du réseau) est un paramètre déterminant pour savoir comment se déplacent ou se créent les résonances.}
	\subsection{Conclusion}
Un métal massif se caractérise par une fonction diélectrique. Lorsque l'on s'intéresse à des particules métalliques, de nouvelles résonances apparaissent, dues à un confinement spatial. Ces résonances sont dépendantes des caractéristiques des particules~:
\begin{itemize}
\item leur taille~: plus une particule est grande, plus ses résonances sont décalées vers le rouge. Lorsque la taille devient très importante, de nouveaux modes multipolaires peuvent apparaître~;
\item leur forme~: si la particule est anisotrope, cela induit également une anisotropie de la
réponse. Le champ électromagnétique est plus intense dans les zones anguleuses~;
\item leur environnement~: la réponse d'une particule est liée au milieu dans
lequel elle se trouve (un substrat, en particulier, induit une brisure de symétrie)~;
\item des interactions entre particules, particulièrement en réseau~: elles peuvent donner lieu à de nouvelles résonances.
\end{itemize}
Ces paramètres offrent autant de leviers afin de contrôler la réponse optique d'un système de particules. En contrepartie, ils rendent également plus complexe la compréhension et la prédiction de cette réponse. Les applications sont nombreuses (cellules photovoltaïques, vitrages).

\newpage
\bibliographystyle{ieeetr}
\bibliography{biblio}

%%%%%%%%%%%%%%%%%%%%%%%%%%%%%%%%%%%%%%%%%%%%%%%%%%%%%%
%%
%% Chapitre 2 : Méthodes
%%
%%%%%%%%%%%%%%%%%%%%%%%%%%%%%%%%%%%%%%%%%%%%%%%%%%%%%%
%\chapter{Méthodes}
%\begin{flushleft}
%\textit{L'expérience est donc le privilège de la raison.}
%\end{flushleft}
%\begin{flushright}
%\textbf{Claude Bernard}, \textit{Introduction à l'étude de la médecine expérimentale, 1865.}
%\end{flushright}
%\newpage\null\thispagestyle{empty}\newpage
\minitoc
\newpage

Ce chapitre présente les aspects techniques des synthèses des couches minces et des traitements effectués. Nous introduisons en premier lieu les méthodes de fabrication des différents échantillons, puis les méthodes de mesures, et enfin les outils d'analyse ou de simulation informatiques.\par 

\section{Préparation des échantillons}
Notre étude porte sur le démouillage de couches d'argent sur un substrat de silice amorphe. Plusieurs types d'échantillons ont été produits, selon les mesures expérimentales. Par exemple, si l'on s'intéresse aux mesures optiques, il convient de procéder au démouillage d'une couche d'argent sur un substrat transparent : nous travaillons donc sur du verre. En revanche, pour effectuer des mesures de microscopie \textit{in situ}, la transparence optique n'est pas requise, mais la finesse du substrat est recherchée, afin d'assurer un bon contact thermique : nous travaillons donc avec des wafers de silicium. Enfin, pour la microscopie à transmission (pour la cartographie ASTAR, voir section~\ref{sMethodeMesure}.\ref{sAstar}), nous avons utilisé des grilles dédiées en nitrure de silicium amorphe.\par 

	\subsection{Préparation du substrat}
\paragraph*{Wafers de silicium~:} Lorsque les propriétés optiques de nos échantillons ne sont pas en jeu, les substrat sont des wafers de silicium recouverts de leur oxyde natif ($\approx$2~~nm). Il s'agit de silicium (100) fourni par Neyco, d'une épaisseur de 500~\micro\meter{} et de diamètre 2”. Les wafers sont utilisés tels quels, sans nettoyage supplémentaire. Si aucune texture de substrat n'est requise, nous procédons au dépôt d'argent directement sur le wafer, mais il est également possible de procéder à la texturation de surface en déposant une couche de silice sol-gel (voir la section suivante).\par 

\paragraph*{Verre~:} Si le substrat doit être transparent, nous utilisons du verre (Planilux) fourni par Saint-Gobain. Il est nettoyé par lavage au Cerox (solution aqueuse de bille micrométriques d'oxyde de cérium, 20~\% en masse), puis rincé à l'eau désionisée. Le verre est séché par flux d'azote. Après ce nettoyage, il sera toujours recouvert d'une sous-couche sol-gel (voir la section suivante). Si aucune texture n'est requise, nous embossons la couche avec un moule plan. Cette précaution est prise car la nature chimique de la surface du verre, qui n'est pas maîtrisée, dépend fortement de l'état de vieillissement, des conditions de nettoyage et des conditions de fabrication.\par 

	\subsection{Texturation par nano-impression}
La technique de nano-impression peut être employée aussi bien sur le verre que sur les wafers de silicium. Elle consiste à imprimer sur une couche sol-gel un motif (la texture) répliqué à partir d'une surface initiale (le master) par le biais d'un moule. Elle est illustrée sur le schéma~\ref{schemaSolGel}, que nous allons détailler. Initialement proposée par Chou \textit{et al.}~\cite{chou1996nanoimprint}, elle a été modifiée par Peroz \textit{et al.}~\cite{peroz2009nanoimprint}, puis Dubov \textit{et al.}~\cite{dubov2013superhydrophobic}, du laboratoire SVI.\par 
\begin{figure}[!htb]
\centering
\includegraphics[width=0.9\textwidth]{schemaSolGel}
\caption{Schéma résumant la technique de nano-impression.}
\label{schemaSolGel}
\end{figure}
		\subsubsection{Fabrication des moules}
Les masters de silicium (étape 1 du schéma) sont fourni par la société Cemitec (Espagne), ce sont des masters gravés par la technique de lithographie électronique (EBL). La première étape consiste à les silaniser, c'est-à-dire à fonctionnaliser leur surface par greffage de silanes qui présentent des terminaisons hydrophobes. Cela permet de s'assurer que le moule que nous produirons par la suite n'y adhère pas.\par 
Après un traitement UV-ozone de 30 min, les masters sont placés dans un dessicateur sous vide pendant une nuit (13~h) en présence de quelques gouttes d'agent silanisant (1H,1H,2H,2Hperfluorodecyltrichlorosilane, ABCR). Ensuite, ils sont retirés, rincés à l'isopropanol, puis à l'eau distillée et séchés.\par 
Le PDMS (\textbf{P}oly\textbf{D}i\textbf{M}éthyle\textbf{S}iloxane, Sygard 184, Neyco) est mélangé à la spatule dans des proportions 10:1 avec un agent réticulant (fourni par Neyco), puis dégazé pour éliminer les bulles qui se sont formées pendant le mélange. Il est ensuite versé sur le master silanisé (étape~2), puis laissé à l'étuve à 80~$^\circ$C pendant une nuit.\par 
Le démoulage s'effectue par simple retrait du PDMS (étape~3). Le moule est recuit à 150~$^\circ$C pendant 2 h pour éviter une éventuelle réaction postérieure du moule. Par la suite, le master peut servir pour fabriquer de nouveaux moules.\par 

		\subsubsection{Embossage}
L'impression du motif du moule se fait sur une couche obtenue par le procédé sol-gel. Un procédé sol-gel (pour solution-gélification) permet de synthétiser un matériau vitreux (dans notre cas, la silice) sans fusion. Nous pouvons rester à des températures inférieures à 100~$^\circ$C. Il s'agit de faire polymériser un précurseur initialement en solution, ce qui donne un gel. Ce gel est ensuite densifié pour former une phase vitreuse.\par 
Les couches sol-gel sont préparées en mélangeant de l'éthanol, du MTEOS (\textbf{M}ethyl{-}\textbf{T}étra{-}\textbf{O}rtho{-}\textbf{S}ilicate) et une solution de HCl à pH~2 en proportions 2:1:1. Le mélange est chauffé à 60~$^\circ$C et agité pendant 30~min. Il est ensuite déposé par spin-coating sur le substrat (à une vitesse de 2000~tours.min$^{-1}$, et une accélération de 1000~tours.min$^{-1}$.s$^{-1}$, pendant une durée de 60~s). Pendant ce temps, les moules de PDMS sont nettoyés à l'aide de papier adhésif. Ils sont ensuite appliqués sur l'échantillon une fois sortis du spin-coater (étape~4). On y applique une force de 50~N tout en chauffant à 75~$^\circ$C pendant 1~h (presse ZWICK, étape~5). L'échantillon est refroidi avant de relâcher la force (étape~6 du schéma~\ref{schemaSolGel}).\par 

	\subsection{Dépôt des couches d'argent}
Le dépôt des couches d'argent s'effectue sur un bâti de pulvérisation cathodique magnétron en ligne (Lina350, Alcatel). La pulvérisation cathodique magnétron (parfois simplement appelée \og dépôt magnétron \fg) est fondée sur la création d'un plasma à partir du gaz, qui arrache des atomes ou des clusters d'atomes d'une source métallique, appelée \og cible \fg{}. Ces atomes et clusters se déposent sur la surface en regard. En général, le gaz est neutre (c'est notamment le cas pour l'argent), mais parfois il est choisi pour réagir avec les atomes dans le plasma, on parle alors de dépôt réactif. \par 
Pour nos dépôts d'argent, la pression de gaz neutre (Ar) est de 8.10$^{-3}$~mbar, la vitesse de dépôt est de 1,3~nm.s$^{-1}$. La puissance électrique fournie à la cible est de 210~W. Un temps de pré-pulvérisation de 5~min est systématiquement respecté pour nettoyer la cible. Les échantillons sont disposés sur un plateau horizontal de 40 cm de côté, déposés sur un contreverre de 2~mm. Ils défilent devant la cible~: le temps d'exposition est régulé par la vitesse du plateau.\par 
Le bâti permet d'effectuer des gradients de vitesse de défilement lors d'un dépôt~; en fonction de la position de l'échantillon sur le plateau, cela permet de déposer plusieurs épaisseurs avec le même plasma.\par 
Un trait de feutre est effectué sur le contreverre avant le dépôt. Après le dépôt, le contreverre est rincé à l'isopropanol, ce qui solvate le feutre et permet de décrocher la couche d'argent déposée par-dessus~: nous obtenons une marche, dont la hauteur est mesurée par AFM. Nous considérons que les mesures d'épaisseur ont une incertitude de $\pm$2 nm. Pour réduire cette incertitude, les dépôts sont régulièrement calibrés sur des couches plus épaisses (100~nm).\par 

\paragraph*{Remarque sur la reproductibilité des dépôts~:} Nous pouvons disposer plusieurs échantillons sur le plateau. Nous appellerons par la suite \og série d'échantillons \fg{} des échantillons dont la couche d'argent a été déposée en même temps. Si la vitesse de défilement est constante, tous les échantillons présentent la même épaisseur. Si la vitesse varie, l'épaisseur déposée n'est pas fixe, mais les échantillons font tout de même partie de la même série.\par 
Ces précautions sont prises car nous avons pu constater, au cours de la thèse, que des paramètres de dépôt \textit{a priori} identiques donnaient lieu à des couches de morphologies différentes. Deux états initiaux mesurés par AFM après dépôt pour une couche de 40 nm sont présentés sur la figure~\ref{AFMdepotReproductibilite}. Ces dépôts ont été obtenus à deux mois d'intervalle, le premier (a) a une rugosité de 4,9 nm, et le second (b) de 7,1 nm.\par 
Les raisons de ces variations tiennent sans doute au fait que l'emploi du bâti pour d'autres dépôts que celui de l'argent entraîne une pollution de la cible. Malgré l'étape de pré-pulvérisation, cette pollution peut influencer le dépôt. En termes de démouillage, cela peut influencer la cinétique (voir chapitre III), mais aussi légèrement la morphologie de la structure finale (densité et taille des particules). C'est pourquoi nous prenons la précaution de travailler avec une série d'échantillons lorsque nous souhaitons faire varier un paramètre de dépôt (épaisseur ou puissance).\par 

\begin{figure}[!htb]
\centering
\includegraphics[width=0.9\textwidth]{AFMdepotReproductibilite}
\caption{Cartographies AFM de couches d'argent de 40~nm mesurées après dépôt dans des conditions identiques, à deux mois d'intervalle. Sur l'image a), la rugosité mesurée est de 4,9 nm et sur l'image b), elle est de 7,1 nm.}
\label{AFMdepotReproductibilite}
\end{figure}

	\subsection{Conservation}
Les échantillons, s'ils ne sont pas utilisés dans la journée de leur production, sont conservés dans des poches scellées, sous gaz neutre (N$_2$, 0,7~atm). Ceci permet de s'assurer qu'il n'y a pas de pollution de l'argent par l'atmosphère et de limiter les risques de sulfuration (ou d'oxydation) de l'argent.\par 
Même ainsi conservés, les échantillons sont utilisés dans les deux semaines qui suivent leur production. Ceci nous permet de limiter d'éventuels effets de vieillissement.\par 

	\subsection{Chauffage}
Lorsque les échantillons sont chauffés \textit{ex situ} (c'est-à-dire préalablement à leur caractérisation), la méthode de chauffage dépend du recuit. Pour les recuits courts (jusqu'à 15~~min), et si le substrat est un wafer de silicium, nous utilisons une platine chauffante Linkam THMS600. La rampe de vitesse maximale est appliquée (150~$^\circ$C.min$^{-1}$), de sorte que nous considérons que le temps de recuit est le temps effectivement passé à la température de consigne.\par 
Pour les recuits plus longs ou sur des substrats plus épais, nous utilisons un four à moufle. Le four est préchauffé à la température de consigne, ainsi qu'une plaque de vitrocéramique qui sert de support. Le temps de recuit est pris comme le temps passé dans le four.\par 
Les recuits \textit{in situ} seront décrits en même temps que les méthodes de mesure associées.\par 

\section{Méthodes de mesure}
\label{sMethodeMesure}
La caractérisation de la couche d'argent est au centre de notre étude, ce qui nous a conduit à sonder sa morphologie, son orientation cristalline ou encore sa réponse optique. Nous allons ici présenter les différentes méthodes de caractérisation employées au cours de nos travaux.\par 
	\subsection{Microscopie à force atomique (AFM)}
L'AFM~\cite{binnig1986atomic} permet de sonder, à l'aide d'une pointe, la surface d'un échantillon afin d'obtenir une cartographie de la topographie à l'échelle du nanomètre.\par 
L'AFM employé est un Bruker Dimension Icon. Outre ses fonctions standard, il permet de réaliser une imagerie \textit{in situ} grâce à une platine chauffante (fournie par Bruker) qui peut monter jusqu'à 250~$^\circ$C. Un profil de chauffe typique est représenté sur la figure~\ref{AFMrampe}~: il a été obtenu pour une consigne de 150~$^\circ$C. La vitesse de chauffe est très rapide entre 20 et 110~$^\circ$C, puis ralentit ensuite. Ce profil est observé quelle que soit la température de consigne. La vitesse de refroidissement a la forme d'une exponentielle décroissante. Lorsque nous effectuons un chauffage, nous considérons que la température de consigne est atteinte au bout d'une minute, et que le refroidissement est instantané. Cette convention est valable quelle que soit la température de consigne (le profil de chauffe étant toujours similaire).\par 
La mesure de rugosité est implémentée directement par Bruker (logiciel NanoScope Analysis)~; elle est prise comme la moyenne arithmétique des écarts à la ligne moyenne ($R_a$).\par 

\begin{figure}[!htb]
\centering
\includegraphics[width=0.6\textwidth]{AFMrampe}
\caption{Calibration de la rampe de température de la platine chauffante de l'AFM, pour une température de consigne de 150~$^\circ$C.}
\label{AFMrampe}
\end{figure}

	\subsection{Microscopie électronique à balayage (MEB)}
Le MEB~\cite{oatley1966scanning} est une technique d'imagerie basée sur l'interaction d'un faisceau d'électrons avec une surface. Cette technique de microscopie nous permet de visualiser la morphologie de la couche d'argent. Son intérêt par rapport à l'AFM est un faible temps d'acquisition. Plusieurs types d'électrons sont détectables dans un MEB classique~; nous avons travaillé avec les électrons \og secondaires \fg, réputés sensibles à la morphologie de la surface.\par 
Le MEB employé au laboratoire pour les études \textit{post mortem} est un Zeiss équipé d'une pointe FEG (Field-Emission Gun). Les échantillons sont métallisés par un dépôt magnétron de platine d'une épaisseur équivalente de 0,7 nm. La métallisation n'est effectuée que pour des analyses \textit{post mortem}, et après les éventuelles mesures optiques.\par 
Les expériences de MEB \textit{in situ} ont été réalisées en collaboration avec Renaud Podor, de l'Institut de Chimie Séparative de Marcoule (ICSM). Une platine chauffante est introduite dans la chambre du microscope, elle permet d'atteindre des températures de l'ordre de 900~$^\circ$C~\footnote{L'emploi d'une autre platine permet d'atteindre des températures de l'ordre de 1300~$^\circ$C}. L'équipe de l'ICSM a implémenté un système de mesure plus fiable que le thermocouple constructeur, donnant une précision de l'ordre de deux degrés~\cite{podor2015development}.\par 
Outre la possibilité de travailler \textit{in situ} et la grande résolution offerte par ce microscope, son intérêt réside également dans la possibilité de faire de l'imagerie sous atmosphère gazeuse contrôlée. Les mélanges gazeux utilisés pour nos travaux sont : Ar+4\%H$_2$, ainsi que des dilutions de O$_2$ dans N$_2$ (1000 ppm ou 100 ppm, Air Liquide, impuretés et écarts $<$5~ppm). Les pressions admissibles dans la chambre sont comprises entre 10 et 400~Pa. La pression résiduelle de la chambre (conditions dites \og sous vide \fg) est de type vide secondaire (de l'ordre de 10$^{-3}$~Pa).\par 
Les conditions techniques d'acquisition (fréquence d'acquisition, tension d'accélération) ont été adaptées en fonction des expériences. Nous avons notamment constaté qu'un effet de faisceau pouvait apparaître, comme illustré sur la figure~\ref{MEBeffetFaisceau}. L'image a été prise au cours d'une session de MEB \textit{in situ} en observant le démouillage d'une couche d'argent de 40 nm sous vide. On voit distinctement que dans la zone délimitée, au centre, le démouillage est moins avancé que sur les bords. Cette zone centrale était précisément l'endroit ou la séquence d'images était enregistrée, à plus fort grossissement : le faisceau a ralenti le démouillage.\par 
\begin{figure}[!htb]
\centering
\includegraphics[width=0.6\textwidth]{MEBeffetFaisceau}
\caption{Effet de faisceau observé lors d'un session de MEB \textit{in situ}. La couche d'argent de 40~nm, en train de démouiller sous vide à 400~$^\circ$C, a été observée d'abord à fort grossissement, dans la zone délimitée.}
\label{MEBeffetFaisceau}
\end{figure}
Pour éviter cet effet, nous avons abaissé la tension et augmenté la vitesse de balayage lorsque nécessaire. En plus de ces précautions, le grossissement a été périodiquement réduit pendant l'expérience afin de vérifier que la zone analysée était semblable au reste de la couche (c'est ce contrôle qui est représenté sur la figure~\ref{MEBeffetFaisceau}).\par 
	\subsection{Cartographie ASTAR}
\label{sAstar}
La cartographie ASTAR~\cite{rauch2008automatic} couple une image TEM (Transmission Electronic Microscopy) avec une analyse des orientations cristallines de la couche d'argent. Cela permet d'avoir accès à ces orientations à l'échelle locale.\par 
Les cartographies ASTAR ont été réalisées en collaboration avec Sophie Bougon de l'ICMPE (Institut de Chimie et de Matériaux de Paris Est) sur un TEM JEOL 2000EX. L'ASTAR, aussi appelé ACOM (Automatic Cristal Orientation Mapping), est une technique où le faisceau d'électrons balaye la zone cartographiée~\cite{rauch2008automatic}. Pour chaque point de mesure, le cliché de diffraction en transmission est enregistré. Un logiciel compare le cliché à une banque de diffractogrammes théoriques obtenus en inclinant un cristal parfait (ici d'argent). Après optimisation, il en déduit l'orientation la plus probable du cristal de la zone analysée.\par 
Le TEM est équipé d'un porte-échantillon chauffant qui permet de réaliser des recuits sous vide. Les couches d'argent sont donc déposées sur des grilles TEM en Si$_3$N$_4$. Afin de limiter les risques de perte de transmission, l'épaisseur des couches a été limitée à 15 ou 20 nm. Une étude préliminaire de la dynamique de démouillage en MEB~\textit{in situ} a démontré qu'elle était similaire sur Si$_3$N$_4$ (amorphe) et SiO$_x$, prouvant que la changement d'un substrat amorphe à un autre est sans effet notable.\par 
Nous avons également observé des effets de faisceau, qui tendent à ralentir, voire inhiber le démouillage. Pour éviter cela, le faisceau était déplacé hors de la zone d'analyse pendant la chauffe.\par 

	
	\subsection{Diffraction des rayons X (DRX)}
La DRX~\cite{warren1969x} permet d'obtenir des informations sur l'orientation cristalline d'une couche à l'échelle globale. Il s'agit d'observer les pics de diffraction des rayons X engendrés par les plans atomiques des domaines cristallins du matériau.\par 
La DRX a été réalisée sur un diffractomètre Rigaku Smartlab en géométrie Bragg-Brentano ($\theta-2\theta$) normale au substrat. Les signaux obtenus ont été corrigés des effets de polarisation, du facteur de Lorentz et des effets de température. Les pics ont ensuite été modélisés par deux gaussiennes pour tenir compte du doublet $K_{\alpha 1}-K_{\alpha 2}$ du cuivre.
	\subsection{Mesures optiques}
La caractérisation optique des échantillons revêt une importance fondamentale dans nos travaux. Trois composantes principales nous intéressent : la transmission, la réflexion et l'absorption.\par 
Les mesures optiques sont réalisées sur un spectrophotomètre Lambda950 (Perkin-Elmer). La gamme mesurée s'étend de 250 à 2000 nm. Les mesures s'effectuent en sphère intégrante. La transmission T est mesurée en incidence normale, la réflexion R à 8$^\circ$ et l'absorption A est calculée comme 1-R-T. Pour mesurer les propriétés optiques de petits échantillons, le faisceau est réduit par un iris et focalisé grâce à une lentille (fournis par Perkin-Elmer et adaptés à la machine).\par 
Un changement de détecteur est opéré automatiquement à 840 nm, ce qui induit un bruit dans les spectres mesurés. Ce bruit n'a donc aucune origine physique.\par 

	\subsection{Ellipsométrie}
L'ellipsométrie est une méthode de caractérisation optique d'un échantillon. Elle repose sur l'analyse de la réflexion d'un faisceau optique polarisé. Le rapport des coefficients de réflexion en polarisations p (champ électrique parallèle au plan d'incidence de la lumière) et s (champ électrique perpendiculaire au plan d'incidence, et donc contenu dans la plan de la surface sur laquelle a lieu la réflexion) est obtenu et exprimé avec les paramètres $\Psi$ et $\Delta$, qui sont définis par $r_p/r_s = tan(\Psi)e^{i\Delta}$. L'analyse de ces paramètres permet de remonter jusqu'aux fonctions diélectriques des milieux rencontrés dans l'échantillon.\par 
Les études d'ellipsométrie \textit{in situ} ont été réalisées en collaboration avec Morten Kildemo, de la NTNU (Norwegian University of Science and Technology, Institute of Physics). Les mesures ont été obtenues sur une platine chauffante adaptée à l'ellipsomètre RC2 (J. A. Woolam). L'alignement de l'échantillon a été réalisé tout au long de l'expérience pour compenser les déformations induites par la température. Les spectres ont été enregistrés entre 0,7 et 5,9 eV (durée d'acquisition de 2 à 10 s), à l'aide des spectrogaphes InGaS et Si sur le bras de détection.\par 
La modélisation des réponses optiques a été effectuée grâce au logiciel CompleteEase (J. A. Woollam). Le substrat est considéré comme un cristal semi-infini de silicium (Si JAW~\cite{herzinger1998ellipsometric}) recouvert de 2~nm d'oxyde natif (SiO2 JAW~\cite{herzinger1998ellipsometric}). La fonction diélectrique du substrat a été calculée en tenant compte de la température. La couche supérieure d'argent, ajustée pour correspondre aux mesures, a une épaisseur laissée variable sans rugosité dans notre modèle. Bien que l'analyse soit effectuée sur toute la gamme d'énergie mesurée, nous concentrons la présentation des résultats en dessous de la première transition interbande de l'argent (3,9~eV~\cite{oates2005evolution}). Tous les modèles ont été ajustés sur les paramètres $\Psi$ et $\Delta$ mesurés. Nous présenterons par la suite les résultats par les fonctions diélectriques associées à ces ajustements. \par 

\section{Méthodes d'analyse et modélisation informatique}
Un grande quantité de données a été acquise par imagerie MEB, sous la forme d'images. Les séquences sont disponibles sous forme de vidéos (https://goo.gl/pRn6dh). Pour interpréter ces données, nous avons développé plusieurs algorithmes de traitement d'images que nous allons décrire dans cette section. Ces algorithmes ont été développés en langage Python (avec notamment les libraires \textit{numpy}, \textit{scipy}, \textit{skimage} et \textit{PIL}). La dernière section de cette partie sera dédiée aux conditions de simulations numériques qui interviennent dans la modélisation des propriétés optiques des structures d'argent.\par 
	\subsection{Segmentation des images}
La segmentation est l'étape qui nous permet de séparer une image en deux domaines distincts~: la zone du substrat découverte et la couche d'argent. Elle a été réalisée grâce à un code développé pendant la thèse. Un exemple de résultat est présenté sur la figure~\ref{ImageTraitement} de la page~\pageref{ImageTraitement}.\par 
\begin{figure}[!p]
\centering
\includegraphics[width=0.6\textwidth]{ImageTraitement}
\caption{Traitement appliqué à une image typique, au cours des différentes étapes de segmentation.}
\label{ImageTraitement}
\end{figure}
\begin{figure}[!p]
\centering
\includegraphics[width=0.45\textwidth]{histogrammeNDG}
\caption{Histogramme des niveaux de gris de l'image d'exemple de la figure~\ref{ImageTraitement}, avant et après application du filtre médian.}
\label{histogrammeNDG}
\end{figure}
\begin{figure}[!p]
\centering
\includegraphics[width=0.75\textwidth]{otsuFiltre}
\caption{Comparaison des résultats de segmentation d'une image par la méthode d'Otsu ou notre méthode.}
\label{otsuFiltre}
\end{figure}
En premier lieu, un filtre médian (2,2) est appliqué. Ce filtre prend, pour chaque pixel de coordonnées $(i,j)$, la valeur médiane des valeurs des pixels dont les coordonnées sont comprises entre $(i-2,i+2)$ et entre $(j-2,j+2)$. La différence n'est pas toujours visible à l'œil nu (cf. figure~\ref{ImageTraitement}), mais ceci permet de diminuer le bruit dans une image et d'affiner l'histogramme des niveaux de gris. Pour le constater, l'histogramme de l'image totale traitée en exemple est présenté sur la figure~\ref{histogrammeNDG}. Le paramètre de segmentation est pris dans l'histogramme des niveaux de gris (NdG) comme étant le minimum entre les deux plus grands maximums.\par 
Enfin, l'image segmentée est nettoyée de ses petits objets. La taille limite est fixée arbitrairement en fonction du grossissement~; un contrôle visuel est effectué pour chaque série de traitements.\par 

\paragraph*{Remarque~:} Notre méthode de segmentation n'est pas la plus courante. Usuellement, on emploie la méthode dite \og méthode d'Otsu \fg, qui cherche à minimiser la variance des valeurs des pixels au sein d'une même classe (pixels blancs ou pixels noirs). Cependant, cette méthode ne donne pas de bons résultats sur nos images MEB, comme nous pouvons le constater sur la figure~\ref{otsuFiltre}. Dans nos images, nous savons que le substrat est représenté par des pixels dont la variance est très faible (le substrat a une couleur uniforme), tandis que la valeur des pixels de la couche peut varier beaucoup plus. La méthode d'Otsu n'est pas la plus adaptée à ce cas-là.\par  

	\subsection{Mesure de la courbure locale}
	Dans le chapitre IV, nous accordons une grande importance à la mesure des propriétés locales du front de démouillage, notamment sa courbure. Nous décrivons ici la procédure pour accéder à ces données.\par 
	Une fois une image nettoyée à disposition, la fonction \textit{measure.find\_contours} du module \textit{skimage} de Python~\cite{van2014scikit} permet d'extraire les contours. Il est important de noter qu'il y a une convention d'orientation des contours qui permet de déterminer si les contours délimitent un trou ou une particule.\par 
	La première étape pour accéder à la courbure est la paramétrisation du contour. Cette paramétrisation correspond à la définition de l'abscisse curviligne $t$ permettant de déterminer des fonctions $x(t)$ et $y(t)$ pour décrire le contour. Il est important que la longueur d'arc entre $(x(t),y(t))$ et $(x(t+1),y(t+1)$ soit constante~; $t$ est interpolée grâce à la fonction $scipy.interpolate$ pour tenir compte de cette contrainte.\par 
	Une fois l'abcisse curviligne définie, les fonctions $x(t)$ et $y(t)$ sont obtenues grâce à la fonction $scipy.interpolate.UnivariateSplines$. L'avantage de ces fonctions tient de le fait que les dérivées sont facilement calculées. La courbure s'obtient selon l'équation~:
	\begin{equation}
	\dfrac{1}{r(t)}=\gamma(t)=\dfrac{x'(t)y''(t)-x''(t)y'(t)}{(x'^2(t)+y'^2(t))^{3/2}}
	\end{equation}

\paragraph*{Test sur des objets simples~:} Nous avons testé le code sur des objets simples de courbure connue~: un cercle, puis une ellipse. Pour le cercle, les résultats sont présentés sur la figure~\ref{courbureCercle} de la page~\pageref{courbureCercle}.\par 
Considérons d'abord le cercle, dont le rayon de courbure est constant et de valeur arbitraire 40. Sur la figure~\ref{courbureCercle}-a, nous pouvons observer que le rayon de courbure calculé est proche de 40, mais diverge sur les bords. Ceci provient du fait que les Splines sont optimisées pour approcher autant que possible un certain nombre de points. Quand elles arrivent au bord du contour, ce nombre de points diminue, puisqu'au-delà du contour, la Spline peut prendre n'importe quelle valeur : il
n'y a plus de points à approcher au-delà de l'extrémité.\par 
Pour éliminer cet effet, nous procédons à un doublage des contours. Considérons l'exemple du cercle pour l'expliquer. Le contour du cercle décrit en temps normal l'intervalle $[0,2\pi]$. Nous doublons ce contour, qui décrit alors les angles $[0,4\pi]$. Nous procédons au calcul des Splines sur ce grand intervalle, mais ne gardons pour le calcul de la courbure l'abscisse curviligne que sur l'intervalle $[\pi,3\pi]$. Les Splines ainsi calculées sont contraintes exactement de la même manière sur les extrémités (aux angles $\pi$ et $3\pi$) qu'au milieu. Cette précaution élimine les effets de bord (cf. figure~\ref{courbureCercle}-a). Elle est appliquée à tout contour fermé.\par 	
La définition des splines requiert deux paramètres : l'ordre du polynôme utilisé pour décrire la fonction en un point et une erreur admissible. L'ordre est fixé à 4 pour toutes nos mesures. Nous définissons l'erreur admissible comme étant égale à $n_{points} \cdot k_{err}$ où $k_{err}$ est le \og paramètre d'erreur \fg. La diminution de ce paramètre contraint davantage les Splines, ce qui améliore la précision du calcul, au détriment du temps de calcul. L'influence de ce paramètre est visible sur la figure~\ref{courbureCercle}-b. Pour $k_{err} = 10^{-4}$, l'erreur ne dépasse pas 0,3~\%~; elle est indétectable ($< 10^{-5}$~\%) à $k_{err} = 10^{-7}$.\par 
\begin{figure}[!p]
\centering
\includegraphics[width=\textwidth]{courbureCercle}
\caption{Rayon de courbure mesuré par notre code sur un cercle de 400 points et de rayon 40. a) Effet d'une introduction de périodicité dans le code, b) influence du paramètre \og erreur \fg{} sur le résultat.}
\label{courbureCercle}
\end{figure}
\begin{figure}[!p]
\centering
\includegraphics[width=0.5\textwidth]{courbureEllipse}
\caption{Rayon de courbure mesuré par notre code sur une ellipse de 400 points. Le demi grand axe est de 40, le demi petit axe est de 20.}
\label{courbureEllipse}
\end{figure}
Le résultat pour une ellipse est présenté sur la figure~\ref{courbureEllipse} de la page~\pageref{courbureEllipse}. L'approximation correspondant à $k_{err} = 10^{-4}$ conduit à une erreur de l'ordre de 2~\%, tandis que fixer $k_{err} = 10^{-7}$ réduit l'erreur à 0,5~\%. Le code développé fonctionne donc très bien sur des cas modèles
obtenus à partir de points théoriques.\par 

\paragraph*{Test sur des objets pixélisés~:} Nous avons procédé aux mêmes vérifications, mais cette fois-ci sur des images pixélisées. Sur des images de 200x200 pixels, nous avons étudié un cercle de rayon de 40~pixels, ainsi qu'une ellipse de demi grand axe de 40~pixels et de demi petit axe de 20~pixels. Les résultats sont présentés respectivement sur les figures~\ref{cerclePixel} et ~\ref{ellipsePixel} de la page~\pageref{cerclePixel}.\par 
Lorsque $k_{err} = 10^{-1}$, le comportement est très semblable à ce que l'on observait (pour la même valeur de $k_{err}$) dans les cas non pixélisés. En revanche, si on le diminue, le résultat devient aberrant, la courbure estimée est erratique (en bleu). On constate que les contours ajustés ne sont plus lisses, mais accidentés~; ils ont
tendance à passer par tous les points de l'image. C'est l'origine de l'erreur de mesure. Cette erreur n'excède pas les 12~\% dans le cas du cercle, et 20~\% dans le cas de l'ellipse, pour $k_{err} = 10^{-1}$. Les erreurs les plus importantes sont obtenues pour les courbures les plus faibles~: pour ces courbures, les pixels sont presque alignés, et plusieurs splines peuvent décrire leurs positions. Lorsque l'on augmente la résolution de l'image (en multipliant par 10 le nombre de pixels dans chaque
dimension), l'erreur commise est divisée par 2.\par 
\conclusion{La mesure de la courbure est limitée par la nature pixélisée de l'image~: une erreur incontournable est commise lors de la paramétrisation des Splines à partir des pixels. Pour que notre mesure soit valide, il faut~:
\begin{itemize}
\item une image dont la résolution soit raisonnable pour bien distinguer les contours ;
\item adapter le paramètre d'erreur pour qu'il ne soit pas trop contraignant (on le prend aussi petit que possible, sans que les contours ne deviennent accidentés) ;
\item des contours décrivant des objets suffisamment gros, c'est-à-dire des contours suffisamment
longs.
\end{itemize}
De cette manière, nous considérons que l'erreur maximale commise lors de la mesure de la courbure est de 10~\%. Cette méthode sera particulièrement employée dans le chapitre~IV. Nous considérons que la mesure des rayons de courbure approchant la résolution de l'image est peu fiable.}

\begin{figure}[!p]
\centering
\includegraphics[width=0.7\textwidth]{cerclePixel}
\caption{Détail du contour paramétré et rayon de courbure mesuré par notre code pour un cercle pixélisé de rayon 40~pixels.}
\label{cerclePixel}
\end{figure}
\begin{figure}[!p]
\centering
\includegraphics[width=0.7\textwidth]{ellipsePixel}
\caption{Détail du contour paramétré et rayon de courbure mesuré par notre code pour une ellipse pixélisée de demi grand axe 40 et demi petit axe 20~pixels.}
\label{ellipsePixel}
\end{figure}

	\subsection{Mesure de la qualité d'un réseau de particules}
À partir d'images MEB, nous avons développé un code pour quantifier la qualité de l'organisation des structures obtenues après démouillage sur des surfaces texturées, ce qui sera important dans le chapitre V. Il procède en deux temps. En premier lieu, les particules sont segmentées et leur centre de masse est identifié. En parallèle, le réseau théorique des centres des trous est renseigné manuellement sur l'image traitée. On donne pour cela la position de trois nœuds du réseau, à la fois sur l'image et dans le repère du réseau.\par 
Après cette première étape, pour chaque particule, on détermine sa taille, à quelle maille du réseau elle appartient et à quelle distance elle se trouve du centre de la maille. Les deux mesures (taille et distance) sont bornées : on exclut les particules ne respectant pas des limites fixées arbitrairement. La taille acceptée dépend de la période du réseau $P$, et la distance maximale au centre de la maille ne doit typiquement pas excéder $0,25P$.\par 
Un exemple de tri est présenté sur la figure~\ref{triExemple}. Il a été réalisé sur une surface de période 350 nm. Les particules vertes sont celles qui sont acceptées. Les particules bleues sont exclues en raison de leur position, les particules rouges en raison de leur taille. Enfin, les particules en contact avec les bords de l'image sont exclues (de même que les mailles qui ne sont pas entières sur l'image). Ces particules ne sont pas colorées par l'algorithme.\par 
Nous notons $n_{part}$ le nombre de particules correctes en taille et en position, $n_{tot}$ le nombre de particules totales et $n_{trou}$ le nombre de motifs présents sur l'image. Le code fournit en sortie les ratios $r_{tot} = n_{part}/n_{tot}$ et $r_{trou} = n_{part}/n_{trou}$. Cette méthode sera particulièrement employée dans le chapitre~V.\par 
\begin{figure}[!htb]
\centering
\includegraphics[width=0.3\textwidth]{triExemple}
\caption{Exemple de traitement d'image pour déterminer la qualité d'une structure obtenue par démouillage sur une surface texturée. Les particules vertes sont considérées comme conformes, le particules rouges sont exclues en raison de leur taille et les particules bleues sont exclues en raison de leur mauvais positionnement.}
\label{triExemple}
\end{figure}

	\subsection{Modélisation optique par éléments finis}
Nous avons également modélisé la réponse optique des structures d'argent obtenues par démouillage sur des surface texturées, afin d'en mieux comprendre l'origine. Les simulations ont été effectuées en collaboration avec Alexandre Baron du Crentre de Recherche Paul Pascal.\par 	
Les simulations électromagnétiques sont basées sur la résolution des équations de Maxwell par la méthode des éléments finis (implémentée dans le logiciel COMSOL)~\cite{kennedy2011analytical}. Le domaine de simulation, schématisé sur la figure~\ref{boiteSimul} est constitué d'une cellule élémentaire contenant une période du réseau. Il est constitué d'un parallélépipède d'air pour le superstrat et d'un parallélépipède de verre pour le substrat. La nanoparticule, constituée d'un hémisphère dans la partie "air" et d'un cône dans la partie "verre", est placée à l'interface air/verre. La pointe du cône est arrondie de façon à reproduire la morphologie de la particule réelle et pour réduire les erreurs dues aux défauts de maillage.\par 
La structure est illuminée par une onde plane en incidence normale à l'aide d'un port électromagnétique. Des conditions périodiques de Floquet\footnote{C'est à dire que la solution périodique est basée sur le Théorème de Floquet. Pour plus d'informations, on pourra se référer par exemple à la thèse : Samuel Nosal, « Modélisation électromagnétique de structures périodiques et matériaux artificiels : application à la conception d'un radôme passe-bande », \textit{École Centrale Paris}, 2009.} sont utilisées pour les frontières orthogonales au vecteur d'onde du réseau. Un port en réflexion ou transmission est défini pour chaque ordre de diffraction. Un étude en convergence est réalisée en fonction de la taille du maillage. La conservation de l'énergie est vérifiée en s'assurant que la somme des puissances transmises et réfléchies dans tous les modes et des puissances dissipées par effet Joule sont égales à la puissance incidente. L'absorption calculée représente les pertes ohmiques (effet Joule).\par 
\begin{figure}[!htb]
\centering
\includegraphics[width=0.7\textwidth]{boiteSimul}
\caption{Schéma de la cellule de simulation employée pour les calculs COMSOL.}
\label{boiteSimul}
\end{figure}

\newpage
\bibliographystyle{ieeetr}
\bibliography{biblio}

%%%%%%%%%%%%%%%%%%%%%%%%%%%%%%%%%%%%%%%%%%%%%%%%%%%%%%
%%
%% Chapitre 3 : Étude in situ
%%
%%%%%%%%%%%%%%%%%%%%%%%%%%%%%%%%%%%%%%%%%%%%%%%%%%%%%%
%\chapter{Les étapes du démouillage et le rôle des grains: apport du suivi \textit{in situ}}
%\begin{center}
%\includegraphics[width=0.4\textwidth]{nu-dans-les-escaliers}
%\end{center}
%\begin{flushright}
%\textbf{Marcel Duchamp}, \textit{Nu descendant un escalier, 1912.}
%\end{flushright}
%\newpage
%\newpage\null\thispagestyle{empty}\newpage
\minitoc
\newpage

Dans les produits verriers comprenant des couches pour le contrôle thermique (bas-émissif), la couche d'argent est incluse dans un empilement de plusieurs couches de diélectriques. La nature de ces couches peut varier, apportant d'autant plus de complexité à l'étude de la stabilité thermique de l'empilement et notamment celle du démouillage de la couche d'argent. Par souci de simplification, et en gardant à l'esprit que nous souhaitons ensuite travailler sur des surfaces texturées, nous considérons en premier lieu une couche d'argent sur un substrat de silice plane.\par
Ce chapitre porte sur les expérimentations menées sur le démouillage à l'état solide de couches minces d'argent dans l'air ou une atmosphère riche en oxygène. Dans ce chapitre, le rôle des grains\footnote{Nous appelons \og grains \fg{} les cristallites observés en MEB et en AFM.} et de leur orientation cristalline dans le mécanisme du démouillage est mis en lumière, notamment grâce à des techniques de suivi \textit{in situ}. Grâce au MEB \textit{in situ}, nous avons également étudié la cinétique du démouillage. Nous discutons également de l'apport d'autres techniques expérimentales, à la fois pour le suivi \textit{in situ} ou pour tenter de comprendre l'évolution de la couche en lien avec sa structure cristalline.\par 


\section{Suivi du démouillage par AFM}
Tout au long de nos travaux, nous allons devoir confronter des résultats obtenus \textit{post mortem} ou \textit{in situ}. Dans cette section, nous allons montrer que les résultats (notamment ceux obtenus au MEB) sont cohérents et permettent bien de tirer des conclusions générales sur le démouillage dans nos systèmes. 

\subsection{Diversité des cinétiques, unicité du mécanisme}
\label{sPostMortem}
Comme l'AFM permet de sonder les topologies de surface à une échelle comparable aux grains impliqués (quelques nanomètres), cette technique a été privilégiée dans un premier temps pour étudier l'évolution de couches d'argent lors de recuits. L'objectif est de comprendre comment, lors d'un recuit, une couche mince d'argent se transforme pour atteindre l'état présenté sur la figure~\ref{AFMfinal}, qui correspond à une couche d'argent de 30~nm recuite à 400~$^\circ$C pendant 15 minutes, puis cartographiée en AFM. Sur cette image, nous observons des îlots d'argent déconnectés les uns des autres. Ces îlots ont une forme régulière, semblant hémisphérique mais néanmoins facettée, comme on l'attendrait avec une construction de Wulff-Kaischew~\cite{wulff1901xxv, kaischew1951thermodynamique, winterbottom1967equilibrium}.\par 

\begin{figure}[htb!]
	\centering
	\includegraphics[width=.6\textwidth]{AFMfinal}
	\caption{Image AFM d'une couche d'argent de 30~nm recuite à 400~$^\circ$C pendant 15~min.}
	\label{AFMfinal}
\end{figure}

Cette morphologie est très différente de la couche mince initiale. L'enjeu est de comprendre quelles étapes permettent de passer de la couche aux îlots. Pour étudier cette transition de manière progressive, une série d'échantillons a été synthétisée en conservant les mêmes paramètres expérimentaux (une couche de 20~nm d'argent déposée dans les mêmes conditions de pulvérisation cathodique magnétron). Ces échantillons ont ensuite été recuits à une température relativement modérée (150~$^\circ$C) pendant des temps variables, afin d'observer le démouillage de manière progressive. Une séquence typique ainsi obtenue est visible sur la figure~\ref{AFMexSitu}. Précisons qu'il s'agit de six échantillons différents, recuits chacun séparément (deux images de l'échantillon recuit pendant 90~min sont présentées).\par

\begin{figure}[htb!]
	\centering
	\includegraphics[width = 1.0\textwidth]{sequenceAFMexSitu.jpg}
	\caption{Séquence d'images AFM obtenues sur des échantillons d'une couche d'argent de 20~nm, recuite à 150~$^\circ$C pendant des durées de 15 à 90 min. Un échantillon de référence est montré, ainsi qu'une seconde image prise sur le même échantillon recuit pendant 90 min.}
	\label{AFMexSitu}
\end{figure}

Sur la séquence, nous pouvons voir la couche dans son état initial, composée d'une multitude de grains. Lors de l'allongement du temps de recuit, des trous apparaissent (en noir) et se propagent, tandis que certaines zones de la couche voient leur épaisseur augmenter. Nous observons que les trous ont tendance à s'élargir avec l'accroissement du temps de recuit. L'AFM nous permet également d'extraire des profils de hauteur, pour évaluer l'épaisseur locale de la couche. Un profil typique obtenu de cette manière est représenté sur la figure~\ref{profil1}.

\begin{figure}[h!]
	\centering
	\includegraphics[width = 0.9\textwidth]{profil1}
	\caption{Image AFM d'une couche d'argent de 20 nm recuite à 150~$^\circ$C pendant 20 min, ainsi qu'un profil des hauteurs extrait.}
	\label{profil1}
\end{figure}

Cette séquence d'images AFM nous mène à faire plusieurs observations~:
\begin{itemize}
	\item il y a une apparition de trous, qui ont tendance à se propager avec la durée du recuit. Cette observation est conforme aux résultats de Presland~\cite{presland1972hillock}~;
	\item de manière concomitante, certains objets voient leur taille augmenter. Ces objets peuvent correspondre aux \og buttes \fg{}~observées par Presland~\cite{presland1972hillock, presland1972role}~;
	\item ces objets présentent des facettes régulières et bien définies, ce qui nous permet de les identifier comme des grains~;
	\item la couche qui subsiste sans être concernée par la croissance de grains ne change pas d'épaisseur, comme l'indique le profil de la figure~\ref{profil1}~;
	\item les particules obtenues à la fin du démouillage présentent des facettes et sont constituées d'un nombre réduit de grains, typiquement un ou deux (figure~\ref{AFMfinal}).
\end{itemize} 

Afin de quantifier la progression des trous, nous mesurons directement le taux de surface du substrat couverte par la couche d'argent (appelé par la suite \og taux de couverture \fg{}), en fonction du temps de recuit. D'autres séries d'échantillons on été fabriqués dans des conditions identiques (dépôt magnétron de 20~nm d'argent sur une série d'échantillons) pour tenter d'obtenir des données consolidées. Notons que ce que nous appelons \og série d'échantillons \fg{} désigne un ensemble d'échantillons dont les dépôts ont été réalisés au même moment et sur le même substrat (voir chapitre II). L'ensemble des résultats est présenté sur la figure~\ref{PostMortemAFM}, où chaque point représente un échantillon différent. À titre d'indication, la série AFM présentée plus haut correspond à la \og série 2\fg{}.\par

\begin{figure}[h!]
	\centering
	\includegraphics[width=.7\textwidth]{recouvrementPostMortem}
	\caption{Taux de couverture mesuré à l'AFM en fonction du temps de recuit à 150~$^\circ$C de couches d'argent de 20 nm.}
	\label{PostMortemAFM}
\end{figure}

Sur la figure~\ref{PostMortemAFM}, nous observons que le taux de couverture décroît en fonction du temps de recuit. Les valeurs mesurées par AFM indiquent que d'une série à l'autre, de très importantes variations peuvent survenir~: par exemple, après un recuit de 45~min, on peut observer un taux de couverture de 93~\% (série~1) ou de 38~\% (série~4). Ainsi, deux échantillons dont les conditions de synthèse, de manipulation et de recuit sont identiques (mais non concomitantes) peuvent avoir des taux de couverture très différents. De manière générale, nous avons constaté tout au long de nos travaux que la cinétique du démouillage est très variable d'un échantillon à l'autre. Pourtant, malgré ces différences cinétiques, les morphologies sont très similaires, si l'on compare les images AFM de deux échantillons ayant un taux de couverture semblable. Les grandes variations de cinétique auxquelles nous sommes confrontés, bien qu'inexpliquées, ne semblent donc pas indiquer une différence de mécanisme de démouillage.\par

\conclusion{ Ce constat implique deux choses. La première, c'est que ces différences ne seront pas rédhibitoires dans l'étude du démouillage, et la seconde, qu'il sera plus intéressant de considérer l'état d'avancement du démouillage d'un échantillon (par exemple par le taux de couverture) que le recuit qu'il a subi.\par} 

Ces observations sont en accord avec plusieurs éléments de la littérature, en premier lieu, avec les observations de Presland~\cite{presland1972hillock} ou Sharma~\cite{sharma1980agglomeration, sharma1980hillock, sharma1986hillock} sur la formation de \og buttes \fg{}. Nous observons une croissance de buttes, que nous pouvons identifier comme des grains du fait de leur facettes. Au vu de la différence de taille qu'un grain peut avoir avec ses voisins lorsqu'il croît, on parlera de \og croissance extraordinaire \fg{}~\cite{thompson1990grain}. Étant donné que le reste de la couche garde une épaisseur constante, nous pouvons affirmer que ces grains accumulent toute la matière qui provient du démouillage. Ce mouvement des trous vers certains grains centraux a déjà été observé par Kovalenko \textit{et al.}~\cite{kovalenko2013solid} qui étudiaient le démouillage de couches de fer sur saphir~\cite{kovalenko2013solid}. Notons de plus que la pression de vapeur saturante de l'argent à de telles températures (<250~$^\circ$C), qui est de l'ordre de $10^{-17}$~atm, permet d'exclure toute perte de matière par évaporation.\par

Les analyses AFM effectuées \textit{post mortem} nous ont permis d'identifier certaines caractéristiques du démouillage de l'argent sur la silice. Les grains semblent jouer le rôle d'attracteurs du matériau qui quitte l'interface. Cependant, nous n'avons que des données partielles sur la manière dont les grains croissent, notamment pour faire le lien entre la croissance des grains lors de la propagation des trous et les particules finales. De plus, nous avons montré que les mesures cinétiques ne sont pas reproductibles. Ces raisons nous ont donc incités à travailler \textit{in situ}, c'est-à-dire à considérer un même échantillon à plusieurs moments de son historique de démouillage.\par

\subsection{Suivi \textit{in situ} par AFM}
\label{sAFMinSitu}
L'AFM utilisé est équipé d'une platine chauffante pouvant atteindre une température de 250~$^\circ$C. Ceci permet de suivre l'évolution d'une même zone de l'échantillon après plusieurs recuits (voir le chapitre II pour plus de détails). Suivant la température de recuit et l'épaisseur de la couche, il est possible d'étudier une certaine partie du scénario de démouillage. En effet, si l'on privilégie des recuits à température modérée ($\approx 100^\circ$C) ou des couches relativement épaisses (jusqu'à 40~nm), le démouillage est lent et l'induction est aisée à observer. À l'inverse, avec des recuits à plus haute température ou avec des couches plus fines, nous pouvons plus facilement observer l'étape de propagation des trous.\par
Nous allons ici présenter trois séquences obtenues en AFM \textit{in situ}. La première balaye le démouillage entre l'état initial et un état avancé, la seconde se concentre sur l'étude de l'induction, tandis que la dernière se concentre sur la propagation.\par

\paragraph*{Étude générale}
\begin{figure}[!htb]
	\centering
	\includegraphics[width=1.0\textwidth]{AFMwholeDewet}
	\caption{Images AFM réalisées \textit{in situ} représentant le démouillage d'une couche de 20~nm d'argent, a) dans l'état initial puis chauffée pendant 5 min à b) 75~$^\circ$C, c) 100~$^\circ$C, d) 125~$^\circ$C, e) 150~$^\circ$C, f) 175~$^\circ$C, g) 200~$^\circ$C, h) 250~$^\circ$C. Les échelles (distances et hauteur) sont identiques pour toutes les images.}
	\label{AFMwhole}
\end{figure}
La séquence d'images de la figure~\ref{AFMwhole} permet d'observer le démouillage à différentes étapes. Les valeurs de taux de substrat couvert, mesurées à l'AFM, sont rapportées dans le tableau~\ref{tTaux}.\par 
Il est déjà intéressant de noter que dans l'état initial (image~\ref{AFMwhole}-a), la couche est très irrégulière. Elle présente des grains dont la taille latérale peut excéder 100~nm, tandis que d'autres sont difficilement discernables tant leur taille s'approche des limites de résolution de l'image (quelques nm). Un chauffage modéré à 75~$^\circ$C suffit à observer des trous (image~\ref{AFMwhole}-b), mais l'évolution de ceux-ci après le second chauffage à 100~$^\circ$C n'est pas très prononcée (image~\ref{AFMwhole}-c). Les trous commencent à se propager après un chauffage à 125~$^\circ$C (image~\ref{AFMwhole}-d). Sur les images~\ref{AFMwhole}-e, f et g (respectivement après chauffage à 150, 175 et 200~$^\circ$C), cette propagation est particulièrement visible, aussi bien que l'agglomération de l'argent dans certaines zones. Conformément à ce qui avait été observé \textit{post mortem}, les zones d'accumulation sont facettées, montrant à nouveau leur nature cristalline. Au-delà de 200~$^\circ$C, le taux de couverture se stabilise.\par

\begin{table}[!htb]
\centering
\begin{tabular}{ccccccccc}
\hline
échantillon & a & b & c & d & e & f & g & h\\
\hline
température de recuit ($^\circ$C) & - & 75 & 100 & 125 & 150 & 175 & 200 & 250\\
taux de couverture (\%) & 100 & 99 & 98 & 92 & 73 & 55 & 53 & 52 \\
\hline
\end{tabular}
\caption{Taux de substrat couvert en fonction de la température de recuit. Mesures effectuées sur la séquence de la figure~\ref{AFMwhole}.}
\label{tTaux}
\end{table}

\paragraph*{Étude de l'induction}
\begin{figure}[h]
	\centering
	\includegraphics[width=1.0\textwidth]{AFMinduction}
	\caption{Images AFM réalisées \textit{in situ} représentant le démouillage d'une couche d'argent de 40~nm, a) dans l'état initial, b) chauffée à 100~$^\circ$C pendant 5~min et c) à 150~$^\circ$C pendant 5~min. Les profils tracés au bas de la figure ont été relevés à l'endroit matérialisé par la ligne blanche sur l'image a). Les échelles (distances et hauteur) sont idéntiques pour toutes les images.}
	\label{AFMinduction}
\end{figure}

Nous avons également procédé au recuit d'une couche de 40~nm à des températures modérées, afin d'étudier plus en détail l'induction (c'est-à-dire ce qui survient avant la propagation de trous dans la couche, voir la section~\ref{sBiblioMeca}). Les résultats sont présentés sur la figure~\ref{AFMinduction}. En étudiant la séquence résultante, plusieurs observations claires s'offrent à nous. De nouveau, la taille latérale des grains dans la couche initiale est très variable, et elle augmente fortement après les recuits successifs. Les mesures AFM ne permettent toutefois pas d'extraire de manière convenable la taille des grains dans la couche, puisque la majorité des petits grains sont difficilement discernables. Nous pouvons tout de même constater~:
\begin{itemize}
\item une augmentation globale de la taille des grains, les plus gros pouvant atteindre jusqu'à 500~nm de diamètre~;
\item une augmentation de la rugosité~\footnote{Mesurée par AFM, voir chapitre II.} de la couche (respectivement 3.86, 4.67 and 7.15~nm)~;
\item une augmentation de l'amplitude crête à crête, visible sur les profils.
\end{itemize}
Ainsi, l'induction est le théâtre d'une croissance des grains dans la couche. Cette croissance a lieu dans le plan (taille latérale), mais aussi hors-plan (hauteur). De plus, les profils sont de plus en plus contrastés entre le sommet des grains de grande taille et des zones où la couche s'amincit, entre les grains.\par
Comme nous l'avons vu plus haut, à la fin de l'induction, des trous apparaissent et se propagent. Cette propagation s'accompagne de la croissance extraordinaire de certains grains. Nous avons cherché à étudier cette croissance dans la suite du démouillage.\par 

\paragraph*{Étude de la croissance des grains}
\begin{figure}[h]
	\centering
	\includegraphics[width=1.0\textwidth]{AFMgrainGrowth}
	\caption{Images AFM réalisées \textit{in situ} représentant le démouillage d'une couche d'argent de 20~nm chauffée à 150~$^\circ$C pendant un total de a) 3~min, b) 6~min et c) 9~min. Les profils tracés au bas de la figure ont été relevés à l'endroit matérialisé par la ligne blanche sur l'image a). Les échelles (distances et hauteur) sont identiques pour toutes les images.}
	\label{AFMgrainGrowth}
\end{figure}

Pour cela, nous nous sommes intéressés à la croissance d'un grain en particulier lors de la propagation des trous. La séquence d'images de la figure~\ref{AFMgrainGrowth} illustre cette croissance sur un grain arbitrairement sélectionné dans une couche de 20~nm recuite à 150~$^\circ$C, après 3, 6 ou 9 min. Ce grain a une taille beaucoup plus importante que ses voisins, à la fois latérale ($\approx$ 400~nm contre des tailles typiquement inférieures à 100~nm) et en hauteur ($\approx$ 50~nm, pour une couche de 20~nm). Il continue de croître pendant le démouillage, au détriment des ses voisins. Les profils mesurés le long de ce grain permettent d'identifier une direction dans laquelle le grain s'étend, tandis que le reste de sa structure est très peu impactée.\par 

\paragraph*{Discussion}
Tâchons maintenant d'interpréter ces données expérimentales. En premier lieu, il est important de noter que dans l'état initial, nous observons des grains de tailles très variables. Pendant l'induction, les petits grains disparaissent au profit des plus gros, ce qui augmente la taille moyenne des grains dans la couche (voir figure~\ref{AFMinduction}). Cependant, ceci a également pour effet d'augmenter la rugosité de la couche, notamment par l'amincissement de certaines zones, qui semblent être des joints de grains. Ceci est cohérent avec la théorie de sillonnement des joints de grains de Mullins~\cite{mullins1957theory}, qui prévoit un approfondissement des joints lors du chauffage. Il est cependant difficile d'observer en détail la morphologie des joints de grains, du fait des limitations de l'AFM, dont la largeur de pointe empêche de sonder les renfoncements trop abrupts. La série observée sur la figure~\ref{AFMwhole} semble toutefois indiquer que les trous se forment principalement aux endroits où la couche est plus fine, entre des grains. D'autres observations expérimentales~\cite{dannenberg2000situ} corroborent l'hypothèse du sillonnement comme source de trous. Cette discussion sera reprise dans le chapitre IV, nous la laissons en suspens pour le moment.\par 
L'AFM nous permet également d'étudier la croissance extraordinaire des grains qui accumulent le matériau déplacé par le démouillage. Nous pouvons souligner deux observations importantes. La première est que les grains les plus gros accumulent du matériau tout au long du démouillage. Sur la séquence de la figure~\ref{AFMwhole}, on peut observer cette croissance dès l'induction et tout au long du démouillage. La deuxième observation importante est que l'accumulation de matière par les grains se fait majoritairement par le dessus (voir figure~\ref{AFMgrainGrowth}), tandis que le reste de la structure du grain reste intacte. Ceci implique que le mécanisme de glissement invoqué dans les études de Kosinova \textit{et al}.~\cite{kosinova2014role}, couplé à de la diffusion à l'interface ou aux joints de grains, ne participerait pas à la croissance extraordinaire des grains dans notre système.\par
\conclusion{
Les modifications ayant lieu en surface, nous supposerons donc par la suite que c'est bel et bien la diffusion de surface qui permet le transport de matière. \par
}
A l'appui de cette hypothèse, les coefficients de diffusion de surface sont bien supérieurs aux coefficients de diffusions de joints de grains~\cite{heitjans2006diffusion}.

\subsection{Bilan}

Rappelons les conclusions tirées de l'étude AFM~:
\begin{itemize}
\item nous considérons un échantillon par son avancement dans le démouillage plutôt que par son recuit~;
\item dès l'état initial, les grains n'ont pas tous la même taille~;
\item pendant l'induction, il y a une réorganisation cristalline qui tend à augmenter globalement la taille des grains~;
\item certains grains sont le lieu d'agglomération de la matière mise en mouvement par le démouillage~; ils sont assimilables aux \og buttes \fg{} de Presland~\cite{presland1972hillock} ~;
\item le transport de matière semble principalement assuré par la diffusion de surface.
\end{itemize}
L'AFM a l'avantage de nous permettre d'accéder à la topographie de la couche. Cependant, il présente quelques désavantages~: la limitation de la platine chauffante (250~$^\circ$C) et une durée d'acquisition lente (compter entre 10 et 15 minutes par image de 5 $\mathrm{\mu m}$ de côté). Notons de plus que l'AFM repose sur le contact d'une pointe avec la surface, et que si celle-ci s'abime, il est difficile de retrouver la zone analysée~: le nombre d'images d'une séquence est donc limité par la durée de vie de la pointe. Un compromis est à trouver entre la fréquence d'acquisition et l'amplitude de progression du démouillage.\par 
Dans la partie suivante, nous allons nous intéresser à une autre technique de microscopie~: le MEB \textit{in situ}. 



\section{MEB \textit{in situ} et analyse d'images}
Les expériences de MEB \textit{in situ} ont été réalisées en collaboration avec Renaud Podor, de l'Institut de Chimie Séparative de Marcoule. Les séquences MEB peuvent contenir plusieurs centaines d'images. Pour des raisons pratiques, ces séquences sont disponibles sur internet (lien youtube https://goo.gl/pRn6dh). Nous insérerons dans la suite du manuscrit des images choisies pour représenter en quelques clichés les séquences. Par ailleurs, les détails expérimentaux relatifs à ces expériences sont disponibles dans le chapitre II. \par
Nous distinguerons par la suite deux types d'images enregistrées au MEB. Si le grossissement est faible (typiquement x3~000), les valeurs intensives telles que le taux de surface couverte ou la densité d'objets (trous et îlots) sont indépendantes de la zone d'analyse. Ces images permettent donc d'extraire des valeurs statistiques représentatives de l'état de la couche. D'autre part, le grossissement élevé (typiquement x20~000) offre en revanche la possibilité d'observer la structure locale de la couche. Nous allons discuter de ces deux grossissements séparément.

\subsection{Faible grossissement et données statistiques~: les étapes du démouillages}
\label{sLowMag}
La vidéo 1 est une séquence d'images enregistrées à faible grossissement (x3~000) lors d'un recuit à 390~$^\circ$C sous 400~Pa d'oxygène. Une série d'images représentatives est présentée sur la figure~\ref{MEBinSituLow}. Nous pouvons y observer la formation de trous (niveaux de gris sombres) et de buttes (niveaux de gris clairs), puis la propagation de trous. Après une segmentation de l'image (cf. le chapitre II), nous pouvons extraire de l'image plusieurs données statistiques~: le taux de couverture, la densité de trous et la densité d'îlots d'argent déconnectés. Ces données sont représentées sur la figure~\ref{MEBinSituLowGraphs}, ainsi que la longueur totale du front de démouillage sur l'image.\par

\begin{figure}[!htb]
	\centering
	\includegraphics[width=1.0\textwidth]{MEBinSituLow.jpg}
	\caption{Images représentatives de la séquence d'images de la vidéo 1. Couche d'argent de 40~nm chauffée à 390~$^\circ$C sous 400~Pa d'oxygène, observée en MEB \textit{in situ} à un grossissement de x3000. Relativement à l'image a), les images suivantes sont enregistrées avec un délai de b) 14~s, c) 28~s, d) 47~s et e) 1307~s.}
	\label{MEBinSituLow}
\end{figure}

\begin{figure}[!htb]
	\centering
	\includegraphics[width=.9\textwidth]{MEBinSituLowGraphs.png}
	\caption{Mesures par analyse d'images de l'évolution du taux de couverture, de la longueur du front et de la densité de trous et d'îlots d'argent dans la séquence de la vidéo 1, associée à la figure~\ref{MEBinSituLow}.}
	\label{MEBinSituLowGraphs}
\end{figure}

Sur nos courbes, l'origine des temps est fixée à la première image sur laquelle un trou est visible. On observe bien le temps d'induction, pendant lequel des trous sont présents mais ne se propagent pas (figure~\ref{MEBinSituLowGraphs}-haut). Le temps d'induction se termine vers 25~s, comme indiqué par la première ligne pointillée. Ensuite, le film se rétracte et découvre le substrat, avant d'atteindre un taux de recouvrement en apparence constant. L'évolution du taux de surface du substrat recouverte en fonction du temps est très semblable à celle observée par Presland~\cite{presland1972hillock}. L'avantage notable, directement lié à l'emploi du MEB\textit{ in situ}, est le grand nombre de points d'acquisition pendant le démouillage d'un échantillon donné.\par 
Nous pouvons de plus accéder facilement à la densité de trous et d'îlots d'argent, ainsi qu'à la longueur du front. L'étude de ces grandeurs est très instructive. Considérons d'évolution de la longueur du front en premier (figure~\ref{MEBinSituLowGraphs}-milieu). La longueur augmente dans un premier temps, le maximum est atteint autour de 75~s. Ensuite, la longueur décroît lentement. Intéressons-nous maintenant à la densité de trous dans la couche (figure~\ref{MEBinSituLowGraphs}-bas, courbe orange). Initialement presque nulle (rappelons que l'origine des temps est prise lorsqu'un trou est visible dans la couche), cette densité augmente avec le temps et devient maximale aux alentours de 48~s. Ensuite, elle diminue et tend vers 0, ce qui correspond au moment où tous les trous ont percolé\footnote{On parle d'un domaine percolé lorsqu'il est possible de le parcourir en entier sans en sortir. Cela peut désigner aussi bien les trous que la couche d'argent. Dans un système de deux phases en 2D, si un phase est percolée, l'autre ne l'est pas.}. Cette percolation a lieu vers 75~s. À l'opposé, si l'on étudie la densité de particules (ou îlots) isolées d'argent, ce nombre est nul jusqu'aux alentours de 50~s. Pendant cette période, il n'y a donc pas de particules isolées de la couche, tout le domaine d'argent est continu ou percolé. Puis la densité augmente rapidement avant de se stabiliser lentement, aux alentours de 130~s.\par
Soulignons bien les faits suivants~:
\begin{itemize}
\item la percolation des trous et le maximum de longueur du front de démouillage sont atteints au même moment (75~s), qui est repéré par la seconde ligne pointillée sur la figure~\ref{MEBinSituLowGraphs}~;
\item ce moment correspond également à la rupture de pente dans le taux de couverture en fonction du temps~;
\item le nombre total de particules isolées continue d'évoluer plus lentement et ne se stabilise que vers 130~s.
\end{itemize}
Considérons plus particulièrement le changement qui intervient à 75~s. Avant, les trous se propagent rapidement, diminuant le taux de couverture et augmentant la longueur du front. Mais lors de la percolation des trous, la propagation est très fortement ralentie. Dans la littérature~\cite{presland1972hillock}, ce ralentissement était interprété comme l'approche de l'état final. Nous avons continué d'étudier l'évolution à des temps plus longs.\par 
Pourtant, la morphologie de la couche d'argent continue de se modifier. Cela est notamment visible en étudiant l'évolution aux temps longs du taux de couverture, comme tracé sur la figure~\ref{MEBzoom}, dont les données sont également extraites des images de la vidéo~1.

\begin{figure}[h]
	\centering
	\includegraphics[width=1.0\textwidth]{MEBwithZoom}
	\caption{\'Evolution du taux de couverture en fonction du temps correspondant au film 2 (40~nm Ag, 390~$^\circ$C). La droite rouge est une régression linéaire effectuée pour les valeurs comprises entre 200 et 1600~s.}
	\label{MEBzoom}
\end{figure}

Nous pouvons observer l'évolution du taux de couverture entre 200 et 1600~s. Une régression linéaire (en rouge) a été ajoutée afin de faciliter le constat~: le taux de couverture continue de diminuer, très lentement. Ainsi, le changement brusque intervenant au moment où les trous cessent de se propager (deuxième ligne pointillée sur la figure~\ref{MEBinSituLowGraphs}) ne correspond pas à l'atteinte d'un état final. Il indique plutôt un changement de régime cinétique.
\conclusion{
Le démouillage se décompose en trois régimes distincts~: induction, propagation, et un troisième régime dont la nature reste à déterminer.\par 
}



\subsection{Fort grossissement et analyse locale~: la croissance extraordinaire de grains}
Dans cette section, nous allons analyser une séquence d'images obtenues à plus fort grossissement. Ceci nous permet de déterminer la nature du troisième régime que nous avons identifié.\par

\subsubsection{Rôle des grains}
La vidéo 2, représentée par la séquence d'images de la figure~\ref{MEBinSituIntro}, montre le démouillage d'une couche de 40 nm à 330~$^\circ$C sous une atmosphère riche en oxygène (100~Pa de O$_2$) avec un grossissement de x20~000. Elle est représentée par la séquence d'images de la figure~\ref{MEBinSituIntro}. Dans les toutes premières images, alors que le substrat est encore majoritairement recouvert par la couche, on observe une dispersion de la taille des grains d'argent. Si la plupart ont une taille inférieure à 100~nm et sont indiscernables, d'autres ont déjà une taille de l'ordre de 500~nm. De plus, on observe que ces grains les plus gros continuent à croître et donner naissance à des îlots, après le démouillage. Enfin, il existe une partie de la couche qui n'est pas modifiée tant qu'elle n'a pas été atteinte par la propagation des trous~; l'agglomération de l'argent n'a lieu que sur certains grains spécifiques.\par
Ces observations sont très semblables à celles effectuées en AFM, mais nous pouvons suivre l'évolution des grains pendant tout le démouillage, avec une grande fréquence d'acquisition. Ceci nous permet de superposer à nos données MEB les données topographiques (de hauteur des grains) obtenues en AFM.\par 


\begin{figure}[!htb]
	\centering
	\includegraphics[width=1.0\textwidth]{MEBinSituIntro}
	\caption{Séquence d'images réalisées en MEB \textit{in situ} d'une couche d'argent de 40~nm chauffée à 330~$^\circ$C sous une atmosphère de 100~Pa de O$_2$, grossissement x20~000. L'image a) est considérée à l'origine des temps. Les images suivantes sont enregistrées à b) 85~s, c) 214~s, d) 447~s, e) 584~s, f) 785~s.}
	\label{MEBinSituIntro}
\end{figure}

Le rôle de certains grains sélectionnés en tant que centres d'accumulation de la matière évacuée par le démouillage, déjà évoqué dans la section~\ref{sAFMinSitu}, est à nouveau mis en évidence. Pour le visualiser de manière certaine, considérons la figure~\ref{MEBzoom}. Sur cette figure, nous avons repéré en bleu les positions successives du front de démouillage à plusieurs instants (plus le front est clair, plus il est observé à des temps courts). Un grain spécifique a été repéré, et ses contours ont été superposés en vert (plus le contour est foncé, plus les temps mesurés sont courts).\par 

\begin{figure}[h]
	\centering
	\includegraphics[width=1.0\textwidth]{MEBzoomGrain}
	\caption{Superposition des contours successifs du front de la couche d'argent observés lors de la propagation. Analyse issue de la vidéo 2. Les contours bleus représentent les positions occupées par le front au cours du temps, du plus clair vers le plus sombre. Les contours verts représentent l'évolution temporelle des contours du grain central, du plus sombre au plus clair.}
	\label{MEBzoomGrain}
\end{figure}

Dans cette zone, nous voyons aisément les caractéristiques du démouillage~:
\begin{itemize}
\item la propagation des trous~;
\item la croissance extraordinaire d'un nombre restreint de grains~;
\item la stabilité de la couche qui n'est concernée ni par la propagation, ni par la croissance. 
\end{itemize}
Nous observons également que le grain central en train de croître (en vert) est séparé du front par une distance de l'ordre du micromètre. Comme le reste de la couche demeure inchangé pendant le démouillage, cela signifie que le transport de matière majoritaire a lieu du front vers les grains en croissance. Ce transport peut concerner des distances aussi longues que le micron. Ce mode de transport ne permet pas la formation d'un bourrelet sur le front de propagation, ce qui n'avait été observé jusque là que dans un système~\cite{kovalenko2013solid}.\par

Grâce au MEB \textit{in situ}, il est aisé d'étudier l'évolution de ces grains sélectionnés depuis l'induction jusqu'à un état de démouillage très avancé. Sur la figure~\ref{MEBposGrains}-a, nous avons superposé le contour des particules obtenues à la fin de la propagation (en vert) à l'image de la couche avant la propagation. Il est également possible de déterminer une \og zone d'influence \fg{} des grains (image~\ref{MEBposGrains}-b), matérialisée par les lignes blanches. Ces lignes sont les points du plan qui sont à égale distance des deux particules les plus proches. Tous les points contenus dans une zone sont plus proches de la particule centrale que de n'importe quelle autre particule.\par 

\begin{figure}[h]
	\centering
	\includegraphics[width=1.0\textwidth]{MEBposGrains}
	\caption{Deux images issues de la séquence de la vidéo 1 a) au début b) à la fin. Sur l'image a), la forme des particules de l'image b) sont superposées en vert. Sur l'image b), les lignes blanches marquent les points équidistants entre les particules les plus proches.}
	\label{MEBposGrains}
\end{figure}

Sur l'image~\ref{MEBposGrains}-a, nous observons que les grains les plus gros de la couche à l'état initial sont contenus dans les contours des particules de l'état final. Ceci confirme que dès l'induction, certains grains spécifiques connaissent une croissance extraordinaire et accumulent pendant la propagation l'essentiel du matériau qui démouille.\par

\conclusion{
Grâce à une analyse des images MEB \textit{in situ} à fort grossissement, nous pouvons conclure qu'un petit nombre de grains spécifiques pilotent la morphologie du démouillage.\par }

\subsubsection{Le troisième régime~: le frittage}
\label{sfrittage}
Nous avons démontré par l'analyse statistique des images MEB l'existence d'un troisième régime dans le démouillage, qui prend place après l'induction et la propagation. L'analyse locale de la morphologie nous permet de comprendre sa nature, dont nous allons discuter ici.\par
À la fin de la vidéo~2, les particules n'ont pas encore atteint leur forme d'équilibre~: on attendrait des particules dont la surface s'approche davantage d'une sphère tronquée, compte tenu de la faible anisotropie des énergies de surface de l'argent~\cite{stankic2013equilibrium}. La différence est illustrée par exemple sur la figure~\ref{MEBpostMortem}. Sur l'image~\ref{MEBpostMortem}-a, les particules sont bien définies et isolées mais présentent une forme irrégulière, on parle de forme \og en asticots \fg{} (traduction du terme \og worm-like \fg{} de la littérature). On s'attend à ce que les particules en asticots finissent par s'agglomérer jusqu'à leur forme finale, ce qu'on observe d'ailleurs à la fin de la vidéo~1. Un exemple de cette évolution est montré sur la figure~\ref{MEBpostMortem}-b.\par 
L'évolution attendue est représentée sur la figure~\ref{MEBfrittage}, où une particule est observée avant et après un intervalle de 5~min (images a et b). La différence d'image est calculée en c~: les zones noires montrent une différence négative (de la matière est partie), tandis que les zones claires indiquent une différence positive (de la matière a été apportée). On distingue clairement une protubérance de la particule (marquée 1 sur la figure~\ref{MEBfrittage}-c) qui disparaît, tandis que la zone 2) marque un léger affaissement et la zone 3) une croissance. Ceci indique clairement qu'il y a eu un déplacement de matière depuis la petite protubérance jusque dans le reste de la particule. La particule s'est réorganisée (2) pendant le transfert. \par

\begin{figure}[h]
	\centering
	\includegraphics[width=1.0\textwidth]{MEBpostMortem}
	\caption{Couche d'argent de 30~nm recuite à 400~$^\circ$C pendant 15~min. Deux avancements ont été observés a) état dit en asticots (\og worm-like \fg) et b) état final attendu.}
	\label{MEBpostMortem}
\end{figure}

\begin{figure}[h]
	\centering
	\includegraphics[width=1.0\textwidth]{MEBfrittage}
	\caption{Zoom sur une particule de la vidéo~1. L'image a) est une référence, l'image b) a été enregistrée 5~min plus tard, et l'image c) est une différence des deux précédentes.}
	\label{MEBfrittage}
\end{figure}

En résumé, nous observons que le mécanisme est une disparition, au sein d'une même particule, de petits domaines au profit du plus gros. Pour ses similitudes avec le frittage de particules dans le domaine des céramiques, nous avons baptisé ce troisième régime \og frittage \fg. Il est important de préciser que le nombre total de particules ne varie pas~: les particules distantes n'échangent pas de matière, ce n'est donc pas un processus de mûrissement.\par
Le frittage est la troisième étape, cinétiquement lente, remarquée dans la section~\ref{sLowMag}. Dans ce régime, nous observons que~:
\begin{itemize}
\item toute la couche a été transformée par la croissance ou la propagation des trous (il ne reste pas de zone \og intacte \fg{})~;
\item les particules tendent à passer d'une forme en asticot à une forme d'équilibre~;
\item se faisant, le périmètre et le taux de couverture décroissent lentement. 
\end{itemize}
Dans la littérature~\cite{presland1972hillock,morawiec2013self}, l'état atteint après la propagation des trous était considéré comme l'état final. Nous venons de démontrer qu'il existe en réalité un régime cinétique très lent, mais néanmoins existant, qui amène les îlots d'argent vers leur forme d'équilibre~: le frittage.\par



\subsection{Rôle des grains sur les surfaces texturées}
Nous avons vu dans le chapitre bibliographique que Gierman et Thompson~\cite{giermann2005solid, giermann2011requirements}, lors du démouillage d'une couche d'or sur une surface texturée, considéraient qu'une condition nécessaire à l'obtention de particules organisées était d'avoir un volume de métal inférieur au volume des trous dans la surface~\cite{giermann2005solid}. Pour nos textures en pyramides inversées, l'épaisseur équivalente au volume est donnée par la relation~:
\begin{equation}
e = \dfrac{bh}{3P^2}
\end{equation}
avec $b$ la base de la pyramide, $h$ sa hauteur et $P$ la période du réseau. En prenant nos valeurs ($b=450^2~nm^2$, $h=175~nm$ et $P=600~nm$), l'épaisseur maximale théorique pour obtenir une organisation est de 32~nm. Dans la vidéo~3, nous avons procédé au démouillage d'une couche de 40~nm d'épaisseur sur ces textures (la figure~\ref{MEBtexture} montre une image de la même expérience à plus faible grossissement). Nous y observons de la croissance extraordinaire de grains qui suit le réseau imposé par la texture, avant même la propagation des trous. Malgré un volume d'argent plus conséquent que le volume des trous de la texture, nous obtenons tout de même une organisation des particules.\par

\begin{figure}[h]
	\centering
	\includegraphics[width=.5\textwidth]{MEBtexture}
	\caption{Image MEB d'une couche d'argent de 40~nm sur une texture de période 600~nm, pendant le démouillage. La température de recuit est 350~$^\circ$C. Cette image est associée à la vidéo~3.}
	\label{MEBtexture}
\end{figure}

Grâce à l'étude réalisée sur des surfaces planes, nous pouvons comprendre cette contradiction. Plus précisément, l'élément central qui détermine l'organisation finale des particules est la croissance de certains grains (qui sont plus clairs que les autres sur la vidéo~3 et l'image~\ref{MEBtexture}), avant l'apparition des trous. Ces grains croissent en respectant la période imposée par le substrat. Comme nous savons que ce sont les grains qui déterminent ensuite la morphologie du démouillage, cela signifie que l'organisation n'est pas gouvernée par le volume de la couche, mais simplement par la présence, au sein d'une maille du réseau de la surface du substrat, d'un (et un seul) grain en croissance extraordinaire. Ce modèle permet d'expliquer pourquoi nous observons une organisation en contradiction avec le modèle de Gierman et Thompson.\par
Cette constatation permet également d'expliquer les défauts observés lors du démouillage de couches d'épaisseurs variables. Si la couche est trop fine, la densité surfacique de grains augmente, ce qui peut donner lieu à la croissance de plusieurs grains par trou. Inversement, si la couche est trop épaisse, la densité de grains initialement formés est faible et une particule donnée peut accumuler la matière provenant des mailles voisines. Nous discuterons plus en détail de cette question dans le chapitre~V, portant sur le contrôle du démouillage.\par
Notons de plus que la présence d'une texture sur la surface du substrat amène la couche d'argent à démouiller plus rapidement que sur une surface plane. En effet, nous observons un état \og final \fg{} dès 250~$^\circ$C, alors qu'une telle température sur une surface plane amène seulement à la fin de la propagation.

\subsection{Conclusion sur le mécanisme}
Grâce à toutes ces informations, nous pouvons proposer une nouvelle description du démouillage des couches polycristallines. Un schéma explicatif est présenté sur la figure~\ref{3etapes}. Une couche initiale (a) voit, pendant l'induction, certains des grains qui la composent croître de manière extraordinaire. En parallèle, des trous apparaissent dans la couche, sans se propager (b). Puis les trous se propagent, ce qui donne lieu à une accumulation de matière, mais uniquement sur des grains spécifiques (c). Le transport de matière peut avoir lieu sur des distances de l'ordre du micron. Il n'implique pas la présence de bourrelet, comme cela a pu être observé dans la littérature~\cite{brandon1966mobility, atiya2014solid}. Lorsque toute la couche a été affectée par la propagation des trous, les particules restantes n'ont pas nécessairement une forme d'équilibre (d), vers laquelle elles évoluent lentement pendant le frittage (e). 

\begin{figure}[h]
	\centering
	\includegraphics[width=1.0\textwidth]{3etapes}
	\caption{Schéma résumant le mécanisme du démouillage d'une couche d'argent sous air. Les pointillés rouges délimitent les différents régimes.}
	\label{3etapes}
\end{figure}

Pendant tout le processus, c'est la croissance extraordinaire d'un petit nombre de grains qui pilote la morphologie du démouillage. Dès l'induction, des grains plus gros prennent le pas sur leurs voisins et deviennent des centres d'accumulation de la matière provenant du front du démouillage. Pendant ce temps, les autres grains de la couche connaissent très peu de transformations. Ces grains donnent naissance à des particules dont la forme est irrégulière et évolue vers un profil d'équilibre pendant le régime de frittage.\par

\paragraph*{Remarque :}
Nous avons dissocié la propagation et le frittage. Cependant, il est possible d'observer des comportements de frittage avant la fin de la propagation (c'est-à-dire tant qu'il reste des régions de la couche qui ont gardé l'épaisseur initiale). Ceci dit, le frittage a une cinétique suffisamment lente pour que l'on puisse considérer qu'il ne joue pas un rôle significatif dans la morphologie du démouillage.\par \vspace{12pt}

Nous avons mis en évidence le rôle de certains grains spécifiques dans le démouillage. Cependant, une question demeure~: quelle est leur particularité~? Il semblerait que cette sélection ait lieu dès la phase d'induction, que nous avons tenté d'observer grâce à d'autres méthodes. Dans les deux sections suivantes, nous allons discuter des méthodes mises en œuvre et de leur apport dans la compréhension du démouillage.\par 


\section{Ellipsométrie \textit{in situ}}
\label{sEllipso}
Au cours des travaux de la thèse, nous avons été amenés à tenter de nouvelles approches pour caractériser le démouillage \textit{in situ}. Nous allons ici présenter les résultats obtenus avec de l'ellipsométrie \textit{in situ}. Ces travaux ont été réalisés en collaboration avec Morten Kildemo, de la NTNU (Norwegian University of Science and Technology, Institute of physics). Comme expliqué dans le chapitre II, le suivi par MEB donnait parfois lieu à des problèmes de faisceau qui contraignait nos analyses. Bien que nous ayons eu la possibilité de nous en affranchir dans la plupart des cas, l'ellipsométrie, en tant que méthode tout photon, ne donne pas lieu à ces effets.\par

\subsection{Étude préliminaire \textit{post mortem}}
De la même manière que nous avons commencé par étudier le mécanisme en \textit{post mortem} par AFM avant de nous intéresser aux mesures \textit{in situ} du MEB, nous avons procédé à une étude à température ambiante de spectres obtenus \textit{post mortem} sur des morphologies mesurées en AFM. Considérons tout d'abord les échantillons dont les surfaces sont représentées par les clichés AFM de la figure~\ref{AFMendPoint}. Ces échantillons représentent une couche a) dans l'état initial, b) pendant l'induction, c) pendant la propagation, d) à la fin de la propagation et e) pendant le frittage. Nous allons analyser les fonctions diélectriques extraites des mesures d'ellipsométrie réalisées sur ces échantillons.\par 

\begin{figure}[h]
	\centering
	\includegraphics[width=1.0\textwidth]{AFMendPoint.jpg}
	\caption{Clichés AFM de la couche d'argent de 20~nm étudiée en ellipsométrie a) dans l'état initial ou recuite 5~min à b) 100~$^\circ$C, c) 150~$^\circ$C, d) 300~$^\circ$C et e) 400~$^\circ$C. Les clichés proviennent d'échantillons issus d'une même série.}
	\label{AFMendPoint}
\end{figure}

La fonction diélectrique de la couche a été ajustée par une B-Spline\footnote{Cette méthode consiste à approximer de manière polynomiale la fonction diélectrique de la couche, afin de calculer la réponse théorique obtenue par le système en entier. Il s'agit ensuite d'optimiser le polynôme pour obtenir une solution satisfaisante.} isotrope sur un milieu semi-infini de silicium recouvert d'une couche de 2~nm d'oxide natif (voir chapitre~II). Les parties réelle ($\varepsilon_1$) et imaginaire ($\varepsilon_2$) des fonctions diélectriques extraites sont représentées sur la figure~\ref{endPoint}. On remarque qu'il n'y a pas de changement notable dans la fonction diélectrique après le premier recuit à 100~$^\circ$C. Des changements notables interviennent seulement après des recuits à plus haute température~: aux basses énergies de photons, la partie réelle $\varepsilon_1$ est très négative dans l'état initial mais bascule vers de valeurs positives à partir de 300~$^\circ$C. Ce changement est caractéristique d'une transition depuis un caractère conducteur vers un comportement isolant d'un matériau. Si l'on observe la partie imaginaire $\varepsilon_2$ des fonctions diélectriques aux basses énergies, on remarque qu'elle augmente dès un recuit à 150~$^\circ$C, est maximale après le recuit à 300~$^\circ$C et diminue après le dernier recuit à 400~$^\circ$C.\par

\begin{figure}[h]
	\centering
	\includegraphics[width=.5\textwidth]{EndPoint}
	\caption{Parties réelle et imaginaire des fonctions diélectriques extraites de l'analyse ellipsométrique des échantillons étudiés \textit{post-mortem}.}
	\label{endPoint}
\end{figure}


Cette transition semble correspondre à celle d'un modèle de Drude à un modèle d'oscillateur de Lorentz de basse énergie (voir chapitre I). En termes physiques, cela revient à dire que l'on passe d'une couche de métal conducteur à de larges particules où les électrons peuvent se mouvoir sur une distance large (plus grande que le libre parcours moyen) mais limitée, typiquement les particules d'argent d'une taille de plus de 1000~nm en asticots, observées sur l'image~\ref{AFMendPoint}-e.\par

\begin{figure}[h]
	\centering
	\includegraphics[width=.5\textwidth]{EndPointExplained}
	\caption{Partie imaginaire de la fonction diélectrique extraite de l'échantillon e).}
	\label{endPointExplained}
\end{figure}

Sur la figure~\ref{endPointExplained}, nous avons reproduit la dernière fonction diélectrique extraite (échantillon e, recuit à 400~$^\circ$C). La courbe pointillée représente un oscillateur de Lorentz, tandis que la zone verte indique une partie du signal qui n'est pas décrite par cet oscillateur. Nous interprétons cette forme plus complexe, entre 1.5 et 3~eV comme un signal dû à des objets de plus petite taille, sans doute les particules dont la taille varie entre 150 et 250~nm observées sur l'image~\ref{AFMendPoint}e. La forme de gaussienne proviendrait de leur dispersion en taille, en forme et en orientation~\cite{oates2005evolution}.\par

Nous avons identifié les principales caractéristiques du démouillage visibles en ellipsométrie~: 
\begin{itemize}
\item le moment où la phase d'argent n'est plus percolante~;
\item la formation des particules en asticots de grande taille ($\micro\meter$)~;
\item la présence de particules plus petites (<250~nm).
\end{itemize}
Forts de ces éléments d'interprétation, nous pouvons à présent aborder les données obtenues en ellipsométrie \textit{in situ}.\par

\subsection{Étude \textit{in situ}}
Nous avons mis en évidence le régime de frittage qui intervient après la propagation des trous (voir section~B.2.\ref{sfrittage}). En temps normal, il correspond à une évolution lente des particules d'argent vers leur forme d'équilibre. Cependant, nous avons pu remarquer que dans certains cas, le frittage était extrêmement rapide. Ceci est généralement observé sur les bords des échantillons ou autour de défauts ponctuels~; mais parfois sur toute la surface d'un échantillon. Sur la figure~\ref{MEBpostMortem} de la page~\pageref{MEBpostMortem}, les deux échantillons présentés sont préparés de manière identique~: une couche d'argent de 20~nm, recuite à 400~$^\circ$C pendant 15~min. L'un présente des particules en asticot, typique de ce qui est observé pendant le frittage (nous l'appellerons \og échantillon 1\fg), tandis que l'autre ne présente que des particules proches de l'état final attendu (nous l'appellerons \og échantillon 2 \fg). Dans le cadre du contrôle du démouillage, cette cinétique rapide nous intéresse, puisqu'elle permet l'obtention rapide de particules aux formes régulières. Nous aimerions donc connaître l'origine de cette différence. Précisons enfin que la différence entre les deux cinétiques est d'un autre ordre que les variations observées dans la section~\ref{sPostMortem}.\\[12pt]

\paragraph{Échantillon 1 (cas standard)~:}
Dans l'étude d'ellipsométrie, nous avons pu étudier \textit{in situ} les deux types de démouillage. La question est de savoir s'il s'agit d'une différence d'ordre cinétique uniquement, ou si les mécanismes impliqués diffèrent également. Dans un premier temps, nous allons nous intéresser à un échantillon dont le démouillage est le plus souvent observé, tel que visible sur la figure~\ref{MEBpostMortem}-a, (échantillon 1). Dans l'étude \textit{post mortem}, nous avions pu observer trois composantes dans la fonction diélectrique de la couche en train de démouiller~: une composante de Drude se transformant en une Lorentzienne à basse énergie et une Gaussienne entre 1.5 et 3~eV. Nous avons donc cherché à modéliser ces composantes tout au long du démouillage. Il nous est vite apparu que l'optimisation de la fonction diélectrique donnait des résultats aberrants, à cause d'un trop grand degré de liberté. Nous avons donc pris le parti de supprimer la composante de Drude, mais de laisser la Lorentzienne prendre une amplitude extrêmement grande et une position à très faible énergie, afin d'en imiter le comportement. Cette idée reste cohérente physiquement~: on donne la possibilité aux oscillateurs de Lorentz (les grosses particules de plus d'un micron) de devenir encore plus grosses, jusqu'à devenir de taille \og infinie \fg{}~(c'est-à-dire une couche continue).\par
Nous avons donc modélisé la réponse totale du système en faisant varier les paramètre suivants : position et amplitude d'un oscillateur Lorentzien, et amplitude d'un oscillateur gaussien situé entre 2 et 3~eV. Les résultats sont présentés sur la figure~\ref{EllipsoInSitu1}. Étudions dans un premier temps la partie imaginaire de la fonction diélectrique modélisée pendant le démouillage (figure~\ref{EllipsoInSitu1}-gauche). Aux basses énergies, on observe une croissance entre l'ambiante et 200~$^\circ$C. Passé cette température, elle diminue, tandis que la forme gaussienne apparaît entre 1.5 et 3.5~eV. Ce comportement est très semblable à celui de l'étude \textit{post mortem}~; nous pouvons donc facilement interpréter ces variations de manière similaire.\par
Pour plus de précisions, les variations des paramètres sont tracées sur la figure~\ref{EllipsoInSitu1}-droite. On observe la décroissance de la lorentzienne seulement à partir de 150~$^\circ$C~: elle imite un comportement de Drude à plus basse température. La décroissance se poursuit rapidement jusqu'à 250~$^\circ$C, où elle devient subitement plus lente. La gaussienne apparaît entre 225 et 250~$^\circ$C, puis évolue peu.\par
Grâce aux éléments précédents, nous pouvons interpréter ces paramètres de la manière suivante~:
\begin{itemize}
\item la couche est relativement peu modifiée jusqu'à 150~$^\circ$C, où la propagation des trous commence à perturber le comportement conducteur de la couche~;
\item entre 150 et 250~$^\circ$C a lieu la propagation des trous~;
\item à 250~$^\circ$C, l'évolution est fortement ralentie, c'est le début du frittage~;
\item des particules dont la taille est inférieure à 250~nm apparaissent à la fin de la propagation.
\end{itemize}

\begin{figure}[h]
	\centering
	\includegraphics[width=.45\textwidth]{ellipsoModel} \includegraphics[width=.45\textwidth]{EllipsoParamsFrench}
	\caption{Gauche~: partie imaginaire de la fonction diélectrique d'une couche d'argent de 20~nm modélisée pendant le suivi \textit{in situ} du démouillage, comme décrit dans le texte. Droite~: évolution des paramètres du modèle en fonction de la température.}
	\label{EllipsoInSitu1}
\end{figure}

Il est intéressant de remarquer que le signal relatif aux particules de taille relativement petite (< 250~nm) apparaît avant le début du frittage. Ceci signifie qu'à la fin de la propagation, les particules ne sont pas toutes en asticot~: certaines ont déjà une forme proche de l'équilibre.\par

\paragraph{Échantillon 2~:}
Considérons maintenant l'autre échantillon, appelé échantillon 2, dont le démouillage a donné un état final sous forme de particules hémisphériques très rapidement (cf. figure~\ref{MEBpostMortem}-b). Pour cet échantillon, un modèle isotrope ne permettait pas de décrire les mesures ellipsométriques. Nous avons donc utilisé un modèle de Bsplines uniaxial~: la fonction diélectrique perpendiculaire au substrat est différente de la fonction diélectrique parallèle. Très peu de changements ont été observés selon l'axe perpendiculaire, nous allons donc nous concentrer sur l'évolution de la fonction diélectrique parallèle. Son évolution est représentée sur la figure~\ref{ellipsoUniaxial}.

\begin{figure}[h]
	\centering
	\includegraphics[width=.5\textwidth]{ellipsoEps2Uni}
	\caption{Partie imaginaire de la fonction diélectrique parallèle de la couche d'argent de l'échantillon 2 modélisée à différentes températures.}
	\label{ellipsoUniaxial}
\end{figure}

%Sur la figure~\ref{ellipsoPlasmon}, nous pouvons observer les parties réelles et imaginaire de la fonction diélectrique parallèle obtenues après un recuit à 400~$^\circ$C. Nous observons une forte résonance à 3,26~eV (indiquée en pointillés) qui n'était pas présente sur l'échantillon~1.
%
%\begin{figure}[h]
%	\centering
%	\includegraphics[width=.6\textwidth]{ellipsoPlasmon}
%	\caption{Parties réelle et imaginaire de l'échantillon 2 après un recuit à 400~$^\circ$C.}
%	\label{ellipsoPlasmon}
%\end{figure}

Si l'aspect obtenu à 200~$^\circ$C est comparable à ce que l'on obtenait, à température égale, sur l'échantillon 1, son évolution à plus haute température diffère grandement. En effet, la gaussienne centrée autour de 2.5~eV n'apparaît pas seule à 250~$^\circ$C, on observe également un autre signal à plus haute énergie (3.25~eV). De plus, la lorentzienne à basse énergie a complètement disparu à 400~$^\circ$C.\par

La résonance à 3.26~eV indique la présence de particules d'argent plus petites que celles donnant lieu au signal gaussien. La disparition de la lorentzienne à basse énergie semble indiquer la disparation des particules en asticots. Ces observations correspondent bien à ce qui est visible sur la figure~\ref{MEBpostMortem}-b~: pas de particules en asticots, des particules de taille comprise entre 150 et 250~nm et des particules plus petites, de taille inférieure à 80~nm. Ce que l'ellipsométrie nous indique de plus, c'est que les deux populations de particules apparaissent au même moment.\par
\conclusion{
Nous avons bien affaire à deux états finaux différents~: dans le cas de l'échantillon 1, les particules en asticot disparaissent uniquement au profit de particules dont la taille est de l'ordre de 250~nm, tandis que dans le cas de l'échantillon~2, deux populations distinctes de particules se forment au même moment.\par 
}
Les raisons de ces différences sont inconnues, mais de nouveaux éléments de réponse seront apportés dans la partie~D.\ref{sModifSubstrat}.

\section{Sélection des grains lors du démouillage}
À plusieurs reprises dans les sections précédentes, nous avons évoqué la sélection de grains pendant le processus de démouillage. Nous allons ici discuter de l'origine de cette sélection. Presland~\cite{presland1972hillock}, lorsqu'il étudiait la croissance des buttes, attribuait leur formation à une relaxation des contraintes à l'interface. Selon lui, la croissance est régie par une diffusion de matière de l'interface à la surface des buttes, dont le flux s'écrit~:
\begin{equation}
j = \dfrac{D_s}{kTr_c}\left(\sigma-\dfrac{2\gamma}{r}\right),
\end{equation}
avec $D_s$ le coefficient d'autodiffusion solide, $kT$ le terme thermodynamique usuel, $r_c$ la distance entre les buttes, $\sigma$ la contrainte à l'interface, $\gamma$ l'énergie de surface et $r$ le rayon de la butte. Pour que le flux soit positif, il faut que la butte ait un rayon initial supérieur à un rayon critique~:
\begin{equation}
r > r_0 = 2\gamma/\sigma.
\end{equation}
Avec des valeurs typiques, ($\sigma = 10^{7}$~J.m$^{-2}$, $\gamma = 0.2$~J.m$^{-1}$), on obtient un rayon critique de l'ordre de 0,4~nm. Dans une couche de 20~nm d'épaisseur, les observations montrent que la taille des grains est du même ordre que l'épaisseur de la couche. Ce modèle ne permet donc pas de discriminer certains grains parmi d'autres puisqu'une grande majorité de grains a une taille supérieure à 0,4~nm~; nous devons donc trouver une explication ailleurs. Dans cette section, nous allons discuter de résultats expérimentaux qui permettent de progresser dans la compréhension de la sélectivité qui a été observée plus haut.\par


\subsection{Orientation cristalline et démouillage}
Nous avons deux méthodes à disposition pour étudier l'orientation cristalline des grains dans la couche d'argent. La première est courante, c'est la diffraction des rayons X (DRX), qui donne une information statistique à l'échelle de la taille du faisceau. La seconde est obtenue dans une chambre de TEM, où l'on fait diffracter le faisceau d'électrons~: l'ASTAR. L'ASTAR permet de cartographier des orientations cristallographiques dans une couche mince de manière locale. On perd donc l'aspect statistique des données au profit d'informations sur la structure locale.\par
\subsubsection{DRX}
Trois échantillons ont été comparés en diffraction. Il s'agissait d'une couche d'argent de 30~nm, dans son état initial, recuite à 125~$^\circ$C pendant 15~minutes ou recuite à 400~$^\circ$C pendant une heure. Les surfaces ainsi obtenues sont représentées sur la figure~\ref{AFMforDRX} de la page~\pageref{AFMforDRX}. Du point de vue du démouillage, elles correspondent à une couche a) dans l'état initial, b) pendant l'étape de propagation et c) pendant l'étape de frittage. Les diffractogrammes obtenus en DRX en mode $\theta-2\theta$ sont présentés sur la figure~\ref{DRXpostMortem}. Les aires sous les pics ont été mesurées, puis corrigées du facteur de structure de l'argent, du facteur de Lorentz, du facteur de polarisation et du facteur thermique, afin de déterminer le rapport statistique de prépondérance des familles de plan par rapport à la surface. Les résultats sont rassemblés dans le tableau~\ref{DRXratios}.\par

\begin{figure}[p]
	\centering
	\includegraphics[width=0.9\textwidth]{AFMforDRX}
	\caption{Clichés AFM réalisés sur des couches de 30~nm d'argent a) dans l'état initial b) recuite à 125~$^\circ$C pendant 15~minutes ou c) recuite à 400~$^\circ$C pendant une heure. Ces échantillons sont ceux caractérisés par la suite en DRX.}
	\label{AFMforDRX}
\end{figure}

\begin{figure}[p]
	\centering
	\includegraphics[width=0.9\textwidth]{DRXpostMortem.pdf}
	\caption{Diffractogrammes des échantillons présentés sur la figure~\ref{AFMforDRX}.}
	\label{DRXpostMortem}
\end{figure}

\begin{table}[p]
	\centering
	\begin{tabular}{cccc}
	\hline
	état de la couche & ratio (111) (\%) & ratio (200) (\%) & ratio (220) (\%)\\
	\hline
	état initial & 54 & 44 & 2\\
	pendant propagation & 56 & 40 & 4\\
	pendant frittage & 52 & 44 & 4\\
	\hline	
	\end{tabular}
	\caption{Pourcentage du signal corrigé du facteur de structures pour les principales familles de plans observées en DRX pendant le démouillage.}
	\label{DRXratios}
\end{table}

Considérons les trois principales orientations cristallines~: <111>, <100> et <110>. Les pics associés aux plans (111) et (200) sont bien observés sur le diffratogramme. En revanche, le signal associé aux plans (220) est très peu intense. Par ailleurs, on observe une tendance des pics à s'affiner, tout en conservant une constance dans la répartition des orientations cristallines tout au long du démouillage (voir le tableau~\ref{DRXratios}).\par 
Nous constatons que la direction <110> n'est statistiquement pas représentée orthogonalement au substrat. Pour les autres, l'affinement des pics indique une augmentation de taille du domaine cristallin sondé~; cela vient du fait que les grains croissent dans la direction orthogonale au plan. Le fait que l'intensité relative des pics se conserve signifie que l'orientation du cristallite d'argent par rapport au substrat n'a apparemment aucune influence sur sa croissance pendant le démouillage.\par



\subsubsection{ASTAR}
Une autre méthode permet de visualiser l'orientation hors plan des grains d'argent, il s'agit de l'ASTAR. Nous avons réalisé une cartographie des orientations sur une couche de 15~nm d'argent démouillée à 400~$^\circ$C pendant 15~minutes. Elle est représentée sur la figure~\ref{AstarOxy}. Sur cette cartographie, toutes les couleurs sont observées, indiquant que toutes les orientations sont représentées. Par ailleurs, les particules d'argent, même lorsqu'elles semblent avoir atteint une forme d'équilibre, comportent toujours des joints de grains.\par 
\conclusion{La diversité des orientations hors-plan renforce nos observations DRX~: ce facteur n'est pas déterminant dans la sélection des grains pendant le démouillage.\par}

\begin{figure}[h]
	\centering
	\includegraphics[width=.5\textwidth]{AstarOxy}
	\caption{Cartographie ASTAR des orientations cristallines hors-plan d'une couche d'argent de 15~nm recuite à 400~$^\circ$C pendant 15~minutes. }
	\label{AstarOxy}
\end{figure}

\subsubsection{De l'orientation cristalline et son effet sur la sélection des grains}
Nous avons démontré que c'est la sélection de certains grains qui détermine la morphologie du démouillage. En effet, ces grains croissent de manière extraordinaire en accumulant la matière qui est diffuse à cause du démouillage. Les résultats DRX et ASTAR pointent dans la même direction~: ce n'est pas l'orientation hors-plan de ces grains qui détermine s'ils vont croître.\par 
Cependant, ce résultat n'exclut pas pour autant que la croissance soit favorisée dans une direction cristallographique qui peut être inclinée par rapport au substrat. Cette distinction est importante, car on pourrait s'attendre au pilotage de la sélection au travers de l'anisotropie des énergies de surface des facettes d'un grain d'argent. En effet, Molina \textit{et al.}~\cite{molina2011size} ont calculé, en présence d'oxygène, un abaissement notable de l'énergie des plans (100) comparativement aux autres. Cet abaissement pourrait favoriser une sélection lors de la croissance des grains. Malheureusement, aucun effet ne contraint l'orientation <100> à se présenter de manière orthogonale au substrat~; même si une croissance extraordinaire était liée à cette orientation, la DRX ou l'ASTAR ne permettraient pas de l'observer.\par 


L'orientation cristalline peut intervenir d'une autre manière dans la sélection des grains. Un parallèle avait été fait entre la formation de buttes et celles des whiskers par Pennebaker~\cite{pennebaker1969hillock} dans l'étude de films minces d'or. Le principal mécanisme mis en jeu est la relaxation de contraintes présentes à l'interface, de la même manière que le présentaient Presland~\cite{presland1972hillock} ou Sharma~\cite{sharma1980agglomeration}. Pour relaxer ces contraintes, les joints de grains sont amenés à bouger. La vitesse de déplacement des joints est liée aux orientations relatives des grains impliqués de part et d'autre. Ceci pourrait expliquer, dans certains cas, la croissance plus rapide de certains grains au sein d'une couche. Ce mécanisme a été invoqué pour expliquer la formation des whiskers, très semblables aux grains si ce n'est qu'il croissent uniquement dans la direction hors-plan. La vérification expérimentale de cette hypothèse ne nous semble pas accessible.\par


\subsection{Modification du substrat lors du démouillage}
\label{sModifSubstrat}
Jusqu'à présent, nous avons supposé implicitement que le substrat est plan et qu'il n'est pas modifié pendant le démouillage. Dans cette partie, nous allons voir que ce n’est pas le cas.\par
Nous avons procédé à l'expérience suivante~: une couche d'argent de 40~nm a été déposée sur un wafer de silicium, puis recuite à 400~$^\circ$C pendant 5~minutes, puis 400~$^\circ$C pendant 6~heures. Sur la figure~\ref{AFMsiNu} sont représentées les images AFM prises a) avant le dépôt de la couche d'argent, b) après le premier recuit, c) après le premier recuit, mais sur les bords de l'échantillon d) après le second recuit, au même endroit que l'image b). Afin de repérer la zone sur l'échantillon, une rayure mécanique a été effectuée à côté de la zone mesurée, après le premier recuit. L'image e) a été prise sur la zone du substrat découverte par cette rayure.\par
Considérons d'abord la surface sur laquelle nous déposons les couches d'argent (image~\ref{AFMsiNu}-a)~: la surface présente une rugosité mesurée inférieure au nanomètre. Observons maintenant la surface après un recuit de 5~min à 400~$^\circ$C (images~\ref{AFMsiNu}-b et c). Dans la zone homogène de l'échantillon, les particules d'argent ont une forme d'asticot, tandis que sur les bords, les particules sont déjà agglomérées dans une forme proche de l'équilibre. Cette différence est celle que nous avions considérée dans la section~\ref{sEllipso}. D'autre part, nous observons des irrégularité sur la surface découverte du silicium, qui n'étaient pas présentes auparavant. Ces irrégularités prennent la forme de petits monticules mesurant jusqu'à 5~nm, répartis aléatoirement sur la surface visible entre les îlots d'argent.\par
Après un second recuit à 400~$^\circ$C, d'une durée beaucoup plus longue (6~h), nous avons imagé par AFM le même endroit que précédemment (image~\ref{AFMsiNu}-d, à comparer avec l'image b). Nous pouvons constater que les particules ont évolué, conformément à nos attentes, dans le sens du frittage. Cependant, si l'on regarde attentivement la surface, des traces sont visibles. Ces traces correspondent à la position du front de démouillage de l'image b), à quelques différences près. Ces traces sont constituées d'un renflement mesurant jusqu'à 6~nm, voisin d'une dépression de l'ordre du nanomètre. Sur l'image~\ref{AFMsiNu}-e, où l'argent a été retiré avant le second recuit, des traces semblables sont visibles sur la surface. Le renflement n'est pas visible, seule la dépression l'est.\par 

\begin{figure}[h]
	\centering
	\includegraphics[width=1.0\textwidth]{AFMsiNu}
	\caption{Images AFM a) d'un substrat de silicium (couvert d'oxyde natif) avant dépôt de couche d'argent, puis après démouillage à 400~$^\circ$C, 5~min, b) dans la zone homogène de l'échantillon c) sur les bords, et après un second recuit à 400~$^\circ$C, 6~heures d) au même endroit que l'image b); e) sur une rayure effectuée avant le second recuit. L'échelle couleur n'est valable que pour l'image a), elle n'est pas représentée pour les autres images.}
	\label{AFMsiNu}
\end{figure}

Comme les traces ont bien la forme du front de démouillage, elles sont bien apparues à cause du démouillage. Le fait qu'elles soient présentes sur l'image e) indique qu'elle se forment extrêmement rapidement, dès 5~min de recuit. Notons de plus que de telles traces sont aussi observables lorsque le substrat est une silice sol-gel de plusieurs centaines de nanomètres d'épaisseur~; elles ne sont pas reliées à la présence de silicium à quelques nanomètres de profondeur. Enfin, l'absence de ces traces sur les bords de l'échantillon, où le frittage a eu lieu dès le premier recuit, pose question. En effet, il n'est pas possible de déterminer si les traces sont apparues parce que le front a été bloqué, ou si le front a été bloqué à cause de la formation des ces traces.\par
La présence des monticules elle-même peut nous amener à nous interroger~: s'ils ont été formés de la même manière que les autres traces, par exemple aux joints triples, alors il est possible qu'ils influencent fortement la réorganisation des grains pendant l'induction par un effet d'épinglage de Zener~\cite{weygand1999zener}. Un tel piégage est également envisageable lorsqu'il s'agit de la progression du front de démouillage. Dans le cadre des techniques à disposition, il n'est pas possible d'étudier expérimentalement leur formation.\footnote{Des résultats prélimiaires en Auger à Balayage (réalisé en collaboration avec Muriel Bouttemy de l'institut Lavoisier, université Versaille Saint-Quentin) indiquent que les traces observées sont bien constituées de silice, et dépourvues d'argent.}\par
Ces traces sont sans doute dues à une déformation du substrat pour atteindre un équilibre au niveau de la ligne triple. Ce phénomène a été observé par Saiz \textit{et al.}~\cite{saiz04} dans le cas de gouttelettes métalliques liquides. Il s'agit, à notre connaissance, de la première mise en évidence de ce phénomène entre deux solides.\par  

\section[Conclusion]{Conclusion et comparaison avec les systèmes industriels}
Nous avons mise en oeuvre de nombreuse techniques dont la complémentarité nous a permis d'obtenir une meilleure compréhension du phénomène de démouillage. Pour chaque technique \textit{in situ}, nous avons développé des méthodes d'analyse en relation avec des résultats \textit{post mortem}. Cette approche a été fructueuse et nous a permis d'aboutir aux conclusions suivantes~:
\begin{itemize}
\item les méthodes expérimentales \textit{in situ} sont indispensables pour étudier le démouillage~;
\item le mouvement de matière est majoritairement dû à la diffusion de surface~;
\item un petit nombre de grains spécifiques accumulent la majorité de la matière mise en mouvement par le démouillage~;
\item il n'y a pas de propagation de bourrelet~: la couche garde une épaisseur constante avant de démouiller~;
\item il existe un troisième régime, après la propagation des trous~: le frittage.
\end{itemize}
Ces observations sont valables pour des couches d'argent polycristallines démouillant sur un substrat de silice dans une atmosphère contenant de l'oxygène. Elles sont indépendantes de la température (entre 100 et 400~$^\circ$C) ou de l'épaisseur (entre 15 et 80~nm). Dans cette partie, nous avons également étudié le rôle de l'orientation cristalline des grains sur le démouillage, et mis en évidence des déformations du substrat au cours du processus.\par

Notre stratégie dans l'étude du démouillage a été en premier lieu de simplifier le système. Une couche d'argent libre sur un substrat de silice est système est très simple par rapport aux empilements à l'argent des systèmes produits par Saint-Gobain. Même si certains aspects doivent pouvoir être transposés, certaines différences sont importantes. Par exemple, nous avons montré que la diffusion de surface était dominante, ce qui ne peut plus être le cas lorsque l'argent est recouvert d'une couche de diélectrique. D'autres chemins de diffusion, impliquant les autres couches de l'empilement, doivent alors prendre le relais, ce qui aura pour effet de ralentir le démouillage. En effet, il faut chauffer à plus haute température un empilement pour observer le démouillage ($\approx$600~$^\circ$C). Par ailleurs, l'argent n'est pas en contact avec l'air dans un empilement.\par

Pour cette dernière raison, nous avons souhaité faire varier notre système de manière simple. En l'occurrence, changer l'atmosphère de recuit et procéder au démouillage sous vide. Cette étude mérite à elle seule un chapitre entier du manuscrit.


%\begin{figure}[h]
%	\centering
%	\includegraphics[width=.7\textwidth]{}
%	\caption{}
%	\label{}
%\end{figure}
\newpage
\bibliographystyle{ieeetr}
\bibliography{biblio}


%%%%%%%%%%%%%%%%%%%%%%%%%%%%%%%%%%%%%%%%%%%%%%%%%%%%%%
%%
%% Chapitre 4 : Changement d'atmosphère
%%
%%%%%%%%%%%%%%%%%%%%%%%%%%%%%%%%%%%%%%%%%%%%%%%%%%%%%%
\setcounter{chapter}{3}
\chapter{L'influence de l'atmosphère sur le démouillage}
\begin{flushleft}
\textit{
Two roads diverged in a yellow wood,\\
And sorry I could not travel both\\
And be one traveler, long I stood\\
And looked down one as far as I could\\
To where it bent in the undergrowth;\\
\vspace{12pt}
Then took the other, as just as fair,\\
And having perhaps the better claim,\\
Because it was grassy and wanted wear;\\
Though as for that the passing there\\
Had worn them really about the same,\\
\vspace{12pt}
And both that morning equally lay\\
In leaves no step had trodden black.\\
Oh, I kept the first for another day!\\
Yet knowing how way leads on to way,\\
I doubted if I should ever come back.\\
\vspace{12pt}
I shall be telling this with a sigh\\
Somewhere ages and ages hence:\\
Two roads diverged in a wood, and I-\\
I took the one less traveled by,\\
And that has made all the difference.}
\end{flushleft}
\begin{flushright}
\textbf{Robert Frost}, \textit{The road not taken, 1916.}
\end{flushright}
\newpage
\minitoc
\newpage

Dans ce chapitre, nous allons discuter de l’influence de l’atmosphère de recuit sur le démouillage. L’intérêt de cette étude s’est imposé de lui-même dès les premières expériences réalisées, tant les observations diffèrent en fonction de l’atmosphère de recuit. Nous présenterons très globalement les changements observés en fonction de la présence d’oxygène lors du démouillage. Puis, pour plus de clarté, les observations et interprétations relatives à l’induction et à la propagation seront traitées dans deux parties successives. Ensuite, nous étudierons les effets d’un changement progressif d’atmosphère.\par 

\section{Changement d’atmosphère, observations cinétiques et statistiques}

		\subsubsection{Comparaison : vide et oxygène}
Les vidéos 1 et 4, représentent l’évolution d’une couche d’argent de 40~nm recuite respectivement dans une atmosphère d’oxygène ou sous vide, à un grossissement relativement faible (x3 000). Des images représentatives extraites de ces vidéos sont présentées sur les figures~\ref{MEBinSituLow} et \ref{MEBinSituLowVac} de la page~\ref{MEBinSituLow}. Le démouillage sous vide est très différent de celui que nous avons étudié dans le chapitre précédent. En termes de morphologie, les trous ont une forme irrégulière et présentent des dendrites. De la même manière que sous oxygène, nous pouvons également étudier l’évolution de quantités statistiques, tels que le taux de couverture et la densité de trous. Les résultats sont représentés sur la figure~\ref{MEBcomparaison} de la page~\pageref{MEBcomparaison}.\par 
\begin{figure}[!p]
\centering
\includegraphics[width = 0.8\textwidth]{MEBinSituLow}
\caption{Séquence d’images représentatives de la vidéo 1 (déjà utilisée dans le chapitre III). Couche d’argent de 40~nm chauffée à 390~$^\circ$C sous 400~Pa d’oxygène, observée en MEB \textit{in situ} à un grossissement de x3000. Par rapport à l’image a), les images suivantes sont enregistrées avec un délai de b) 14~s, c) 28~s, d) 47~s et e) 1307~s.}
\label{MEBinSituLow}
\end{figure}
\begin{figure}[!p]
\centering
\includegraphics[width = 0.8\textwidth]{MEBinSituLowVac}
\caption{Séquence d’images représentatives de la vidéo 4. Couche d’argent de 40~nm chauffée à 390~$^\circ$C sous vide, observée en MEB \textit{in situ} à un grossissement de x3000. Par rapport à l’image a), les images suivantes sont enregistrées avec un délai de b) 39~s, c) 89~s, d) 142~s et e) 440~s.}
\label{MEBinSituLowVac}
\end{figure}
\begin{figure}[!p]
\centering
\includegraphics[width = 0.7\textwidth]{MEBcomparaison}
\caption{Évolution du taux de couverture et de la densité de trous en fonction du temps lors du démouillage d’une couche d’argent de 40~nm recuite à 350~$^\circ$C.}
\label{MEBcomparaison}
\end{figure}
Dans les deux cas, l’origine de l’axe des temps est pris au début de la propagation de trous. En comparant les données obtenues lors d’un recuit sous oxygène et sous vide, bien que l’allure soit semblable, plusieurs différences apparaissent. Le taux de couverture atteint à la fin de la propagation diffère légèrement (35~\% sous vide contre 32~\% sous oxygène). La densité de trous maximale sous vide est beaucoup plus faible que sous oxygène (0,10~\micro\meter$^{-2}$ contre 0,52~\micro\meter$^{-2}$). Enfin, la percolation des trous ainsi que le ralentissement des variations du taux de couverture arrivent beaucoup plus tard et de manière beaucoup plus progressive sous vide que sous oxygène.\par 
Cette différence dans la morphologie du démouillage n’avait jamais été observée pour un même système. Des comportement semblables ont été décrits par Kwon \textit{et al.}~\cite{kwon2003comparison} qui comparaient le démouillage de deux métaux différents, le cuivre et l’or, sur de la silice amorphe. Nous l’observons uniquement en changeant l’atmosphère de recuit. À cet égard, notre système est particulièrement intéressant, puisqu’il permet de faire le lien entre deux morphologies de démouillage très différentes en variant uniquement l’atmosphère de recuit.\par 
Notre objectif est donc d’expliquer, par le seul changement d’atmosphère, ce changement de la densité de trous, de leur forme et de leur propagation.\par 

		\subsubsection{Autres gaz dans l’atmosphère}
Avant d’aller plus loin, il convient de démontrer que l’oxygène est effectivement à l’origine de ces changements, et pas la simple présence de gaz. Pour cela, nous avons également procédé à un recuit sous atmosphère d’argon hydrogéné d’une couche d’argent. Une image MEB d’une couche de 80~nm recuit à 500~$^\circ$C (image prise dès la température atteinte) sous une atmosphère réductrice d’argon hydrogéné (Ar:H2 4~\%, 400~Pa) est présentée sur la figure~\ref{MEBargon} de la page~\pageref{MEBargon}. Le démouillage présente les mêmes caractéristiques que sous vide~: peu de trous, démouillage lent. Nous en déduisons que c’est bien l’oxygène qui est responsable des changements observés (pas la simple présence de gaz), ni les potentielles impuretés présentes dans les mélanges gazeux employés dans la chambre du microscope. Cette affirmation sera confirmée lors de la variation progressive de la pression partielle d'oxygène (section~D de ce chapitre).\par 
\begin{figure}[!p]
\centering
\includegraphics[width = 0.7\textwidth]{MEBargon}
\caption{Image MEB d’une couche d’argent de 80~nm recuite à 500~$^\circ$C dans une atmosphère d’argon hydrogéné (4~\%, 400~Pa).}
\label{MEBargon}
\end{figure}

\section{Influence de l'oxygène sur l'induction}

		\subsection{Observations expérimentales~: croissance cristalline}
Dans le chapitre III, nous avons présenté les mesures AFM \textit{in situ} réalisées sur une couche durant l’induction lors d’un recuit à l’air. Nous avions alors observé un grossissement de la taille moyenne des grains, ainsi qu’une augmentation de rugosité. Pour étudier l’induction sous vide, nous avons employé la technique ASTAR (voir le chapitre II pour plus d’explications). Une couche de 15~nm d’argent a été recuite dans la chambre du TEM à 75 puis 125~$^\circ$C. Nous avons réalisé in situ trois cartographies des orientations : dans l’état initial, après le premier recuit et après le second recuit. Ces cartographies sont représentées sur la figure~\ref{astarVac}. Bien que l’image C soit très différentes des deux autres, un repérage soigneux en imagerie directe nous a permis de nous assurer qu’il s’agissait bien de la même zone, malgré de possibles effets de dérive thermique. Nous pouvons faire plusieurs observations :
\begin{itemize}
\item la couche présente une multitude de grains dont la taille peut varier de quelques nanomètres à plusieurs centaines de nanomètres ;
\item à l'échelle de la statistique des images, il n’y a pas d’orientation cristalline privilégiée en apparence (avant ou pendant le recuit) ;
\item après le premier recuit, certains grains ont grossi (voir le grain pointé par la flèche rouge sur les images A et B), tandis que d’autres se sont scindés (voir grain pointé par la flèche bleue sur les images A et B) ;
\item après le second recuit (image C), les orientations de beaucoup de grains, observées localement, ont varié.
\end{itemize}
En outre, la taille mesurée moyenne des domaines cristallins a cru après les deux recuits, évoluant de 12~nm (état initial) à 17~nm (après le second recuit). Cette croissance moyenne reste conforme à ce qui est attendu, mais les valeurs numériques sont à considérer avec précaution : la validité statistique n’est pas vérifiable à cette échelle.\par 
\begin{figure}[!htb]
\centering
\includegraphics[width = 0.5\textwidth]{astarVac}
\caption{Cartographies ASTAR réalisées sur une couche d’argent de 15~nm, a) dans l’état initial, b) chauffée à 75~$^\circ$C et c) chauffée à 125~$^\circ$C.}
\label{astarVac}
\end{figure}
Considérons les représentativités relatives des orientations dans la couche. Sur les cartographies, malgré la faible anisotropie, l’orientation (110) est moins représentée que les autres, mais quelques grains présentant cette orientation sont observables. Cela peut s’expliquer par la hiérarchie des énergies de surface sur les différentes facettes de l’argent, qui est la suivante : $\gamma(111) < \gamma(100) < \gamma(110)$ avec une faible anisotropie, de l’ordre de 1,08 entre les deux extrêmes~\cite{stankic2013equilibrium, molina2011size}. Cette observation est également cohérente avec les diffractogrammes DRX du chapitre précédent, qui indiquent que la fréquence d'apparition de cette orientation est extrêmement faible.\par 
Les cartographies ASTAR montrent qu’il y a une très importante réorganisation de la structure cristalline de la couche pendant l’induction. Cette réorganisation a lieu à très basse température : dès 75~$^\circ$C. Nous observons une croissance modérée de la taille moyenne des domaines cristallins, mais surtout une modification des orientations cristallines. Il est également surprenant de constater que si certains grains peuvent croître, d’autres peuvent se scindér.\par 
Par ailleurs, les mesures effectuées sur les séquences d’images (cf. figure~\ref{MEBcomparaison}) indiquent une grande différence dans la densité de trous formés, en fonction de l’atmosphère. Afin d’étudier la topographie de la couche, nous avons de plus réalisé des images AFM \textit{post mortem} d’une couche d’argent de 40~nm recuite à 150~$^\circ$C pendant 15~min sous air ou 300~$^\circ$C pendant 30~min sous vide. Les images AFM sont présentées sur la figure~\ref{AFMcomparaison}. Les densités de trous mesurées sont respectivement de 1,1~\micro\meter$^{-2}$ et 4,9.10$^{-3}$~\micro\meter$^{-2}$ (la densité de trous sous vide a été estimée sur une image MEB du même échantillon à plus faible grossissement). Il est intéressant de remarquer que la taille des grains, dans la couche non démouillée recuite sous vide, est très imposante : encore plus que sous oxygène. Ceci signifie que la réorganisation cristalline de la couche, que nous avons observée en ASTAR, continue d’avoir lieu. En effet, elle n'est pas perturbée par la formation et la propagation de trous dans la couche.
\begin{figure}[!htb]
\centering
\includegraphics[width = 0.7\textwidth]{AFMcomparaison}
\caption{Images AFM obtenues sur une couche d’argent de 40~nm recuite a) à 150~$^\circ$C pendant 15~min sous air et b) à 300~$^\circ$C pendant 30~min sous vide.}
\label{AFMcomparaison}
\end{figure}
\paragraph*{Remarque~:} Les densités de trous mesurées ici différent de celles mesurées en MEB \textit{in situ}. Ceci provient de la différence des conditions de recuit (température et durée), et du fait que l’étude \textit{post mortem} a été réalisée sur une série différente d’échantillons (voir chapitre III, section~A.\ref{sPostMortem}). Cette remarque ne modifie en rien l’observation d’une très forte réorganisation de la structure cristalline, quelle que soit l’atmosphère, ni la comparaison de la taille des grains, qui continue d’augmenter sous vide.\par 

		\subsection{Discussion}
Le concept d'induction a été proposé par Presland~\cite{presland1972hillock}, qui observait la croissance de buttes. Nous avons démontré dans le chapitre précédent que ces buttes étaient en réalité des grains en croissance extraordinaire. Les grains croissent par une réorganisation de la structure cristalline de la couche, que nous avons observée par AFM lors d’un recuit sous air, ou par ASTAR lors d’un recuit sous vide. Quelle que soit l’atmosphère, cette réorganisation a bien lieu.\par 
Cependant, nous avons observé une grande différence dans la densité de trous apparaissant lors du démouillage suivant l’atmosphère de recuit. Dans cette mesure, la réorganisation de la couche mène à des structures très différentes. Pour expliquer cela, nous pouvons nous référer à la théorie de Mullins~\cite{mullins1957theory} sur le sillonnement des joints de grains. À la ligne triple du joint de grains (contact entre les deux grains et le milieu ambiant), l’angle de contact $\theta_{GB}$ (voir figure~\ref{schemaGrainBoundary}) est déterminé par un rapport entre l’énergie de l’interface du joint de grain $\gamma_{GB}$ et l’énergie de surface du métal $\gamma_m$ par la loi de Young :
\begin{equation}
\gamma_{GB}/\gamma_M = 2\cos(\theta_{GB}/2).
\end{equation}
Ainsi, l’angle de contact dépend de l’énergie de surface du métal. Or, plusieurs travaux expérimentaux~\cite{buttner1952adsorption} ou théoriques~\cite{molina2011size} montrent que l’oxygène réduit l’énergie de surface de l’argent. Ainsi, la présence d’oxygène modifie l’angle de contact aux joints de grains. Si $\gamma_M$ diminue, alors le cosinus augmente, amenant $\theta_{GB}$ à diminuer. En d’autres termes, si on diminue l’énergie de surface $\gamma_M$, on privilégie la création de surface de métal libre entre deux grains, plutôt que de l’interface du joint de grains.\par 
\begin{figure}[!htb]
\centering
\includegraphics[width = 0.4\textwidth]{schemaGrainBoundary}
\caption{Schéma d’un joint de grain.}
\label{schemaGrainBoundary}
\end{figure}
Mais il faut souligner que la théorie de Mullins prévoit une évolution progressive du sillonnement des joints de grains. Dans notre cas, ce n’est pas ce que nous observons. Les réorganisations de la structure cristalline sont importantes, brusques et nombreuses. Ces réorganisations répondant à une minimisation de l’énergie, il est logique que les angles de contact apparaissant lors de l’induction tendent à s’approcher de l’angle d’équilibre théorique.\par 
Cette différence est schématisée sur la figure~\ref{schemaAngle}. Si l’on considère une couche constituée de grains, et qu’une réorganisation a lieu sous la forme de la fusion de deux grains, alors elle ne donnera pas le même résultat suivant l’atmosphère de recuit. Sous vide, l’angle de contact serait large, et donc les sillons des joints de grains peu profonds. Sous atmosphère d’oxygène, au contraire, cet angle serait petit et favoriserait le sillonnement.\par 
\begin{figure}[!htb]
\centering
\includegraphics[width = 0.9\textwidth]{schemaAngle}
\caption{Schéma de la reconstruction des grains dans la couche d’argent pendant l’induction, sous vide ou sous atmosphère d’oxygène.}
\label{schemaAngle}
\end{figure}
Cette conception tient bien compte des deux éléments mis en avant par les résultats expérimentaux : une très forte réorganisation de la structure, quelle que soit l’atmosphère, mais une tendance bien plus forte à former des trous sous oxygène. Nous avons cherché à quantifier cet effet. Pour cela, nous avons besoin des valeurs de l’énergie de surface et de l’énergie à l’interface des joints de grains. Plusieurs travaux de la littérature traitent de ce sujet, mais les valeurs peuvent varier énormément. Les travaux qui cherchent à quantifier ces valeurs mesurent en réalité les angles aux joints de grains, en déduisent les ratios $\gamma_{GB}/\gamma_M$, et supposent la valeur de $\gamma_M$. Sur la figure~\ref{energiesFr} sont représentées les valeurs prises dans les références~\cite{buttner1952adsorption} pour la surface libre et~\cite{inman1963interfacial, fiala1975surface, kudrman1969relative} pour les ratios  $\gamma_{GB}/\gamma_M$, pour l’argent. Nous avons sélectionné l’énergie de surface de Buttner comme référence, bien que celui-ci supposait déjà une valeur de $\gamma_{GB}$, car c’est le seul qui décrivait une évolution dans l’oxygène. Nous avons calculé les valeurs théoriques des angles de contact aux joints de grains en fonction de l’activité d’oxygène dans l’atmosphère, avec les différents ratios $\gamma_{GB}/\gamma_M$ de la littérature (voir figure~\ref{energiesFr}.\par 
\begin{figure}[!htb]
\centering
\includegraphics[width = 0.7\textwidth]{energiesFr}
\caption{Évolution des énergies de surfaces et de l’interface aux joints de grains en fonction de l’activité d’oxygène dans l’atmosphère, ainsi que l’angle d’équilibre prévu aux joints de grains. Les valeurs sont tirées de \cite{buttner1952adsorption, inman1963interfacial, fiala1975surface, kudrman1969relative} et concernent l’argent. L'oxygène est considéré comme un gaz idéal et sont activité est égale à sa pression partielle normalisée par $P^0$.}
\label{energiesFr}
\end{figure}
Buttner \textit{et al.}~\cite{buttner1952adsorption} prévoient une diminution de l’énergie de surface à partir d’une activité d’oxygène de $10^{-5}$. Comme eux, nous supposons que les énergies aux joints de grains ne sont pas soumises à cette évolution. Dans les trois travaux cités~\cite{inman1963interfacial, fiala1975surface, kudrman1969relative}, un facteur~4 est observé entre l’énergie la plus haute et l’énergie la plus basse à l'interface des joints de grains. Un calcul en tenant compte de ces différentes valeurs donne des angles de joints de grains qui ont effectivement tendance à se réduire (et donc le joint à s’approfondir). La valeur la plus extrême (par Inman \textit{et al.}~\cite{inman1963interfacial}) prédirait même un angle de 0$^\circ$ sous air : une dissociation totale des grains.\par 
Les valeurs de la littérature ne permettent pas facilement de trancher sur une valeur numérique, tant elles sont diverses. Dans tous les cas, augmenter l’activité de l’oxygène tend à favoriser l’apparition de trous. Notons de plus que jusqu’ici, nous considérions que les énergies étaient isotropes et homogènes, ce que la présence de facettes dément~\cite{wulff1901xxv}. Le fait de travailler sur un substrat amorphe contribue à augmenter les degrés de liberté quant à l’orientation cristalline des grains les uns par rapport aux autres.\par 
Par exemple, l’énergie de surface des facettes est relativement isotrope sous vide, mais connaît une diminution plus prononcée pour la surface (100)~\cite{molina2011size}. Si c’est la surface exposée lors du sillonnement d’un joint de grain, le phénomène sera d’autant plus amplifié. Mais comme nous l’avions expliqué dans le chapitre précédent, il ne nous est pas expérimentalement possible de sonder les orientations relatives de grains au cours du démouillage. Une mise en oeuvre de modélisation pourrait, à cet égard, s’avérer fructueuse pour aller plus loin.\par 
\conclusion{Pendant l’induction, la structure cristalline de la couche se réorganise. Ces réorganisations sont des changements drastiques des orientations cristallines des grains, pas un simple sillonnement progressif. La présence d’oxygène dans l’atmosphère favorise les réorganisations qui donnent naissance à des trous. Notre modèle est cohérent avec les valeurs énergétiques de la littérature. Cependant il n’est pas possible de rentrer dans des considérations quantitatives.\par}
Notons enfin que Kwon \textit{et al.}~\cite{kwon2003comparison} observaient cette différence de densité de trous, qu’ils assignaient à un changement de mécanisme de formation. Si la densité de trous était importante (cas du cuivre), alors les trous provenaient du sillonnement des joints de grains, tandis que dans l’autre configuration, ils semblaient provenir d’accumulation de lacunes à un joint de grains (cas de l’or). Suite à nos travaux, nous sommes en mesure de proposer un unique mécanisme dépendant de l’atmosphère de recuit, par la variation de l'énergie de surface du métal. Dans cette mesure, le choix du système Ag/SiO$_2$ peut permettre d'appréhender sous un nouvel angle l’étude du démouillage d’autres couches minces polycristallines.\par 
\section{Influence de l'oxygène sur la propagation}
		\subsection{Description de notre approche}
L’une des principales différences induite par la modification de l’atmosphère est la forme des trous : nous observons des dendrites sous vide, mais pas sous oxygène. Au delà de cette observation qualitative, est-il possible de différencier de manière explicite la propagation des trous en fonction de l’atmosphère ? Pour répondre à cette question, nous avons développé un méthode d’analyse spécifique que nous allons décrire maintenant.\par 
Les images obtenues à un grossissement de x3~000 ne permettent pas l'évolution de la forme du front de démouillage de manière statisfaisante, car la délimitation  obtenue par segmentation n'est pas repérée de façon suffisamment précise. Nous travaillons donc avec des séquences d'images réalisées à un grossissement x10~000 ou x20~000, suivant les cas.\par 
Notre étude s’est portée en particulier sur la vitesse de propagation locale du front en fonction de sa courbure locale. En effet, la courbure semble varier à première vue suivant l’atmosphère de recuit, notamment à cause de la présence des digitations. Nous allons ici aborder le développement d’une méthode d’analyse construite spécifiquement pour étudier spécifiquement le démouillage sous cet angle.\par 

			\subsubsection{Définition des objets étudiés}
\paragraph*{La courbure :} La courbure $\kappa$ d’un arc $\mathcal{C}$ dans le plan se définit de plusieurs manières équivalentes~; nous retiendrons pour plus de facilité qu’il s’agit de l’inverse du rayon de courbure $r$, rayon du cercle tangent à l’arc au point considéré. Dans le plan, il est possible de donner un sens à la courbure en fonction de la direction de la concavité de l’arc.\par 
La courbure d’une surface $\mathcal{S}$ dans l’espace peut être définie de plusieurs manières. Dans notre cas, nous référons aux travaux de Taylor~\cite{taylor1992ii}. Nous considérons la courbure moyenne, qui se définit par rapport aux \og courbures principales \fg~:
\begin{equation}
\kappa = \dfrac{\kappa_1+\kappa_2}{2}.
\label{eCourb}
\end{equation}
Le schéma présenté sur la figure~\ref{schemaPlans} permet d’illustrer la manière de trouver les courbures principales d’une surface. On définit d’abord le plan tangent à la surface au point considéré. Perpendiculairement à ce plan tangent, tout plan interceptera la surface $\mathcal{S}$ par un arc, dont la courbure est facilement accessible. Les courbures principales sont les courbures maximale et minimale, obtenues dans des plans perpendiculaires entre eux (et perpendiculaires au plan tangent).\par
Il existe de plus un théorème stipulant que la courbure moyenne ne dépend pas du choix des plans perpendiculaires, tant qu’il sont orthogonaux entre eux ~\cite{taylor1992ii}. Le schéma~\ref{schemaPlansApplique} illustre l’application de ce théorème à la surface d'un domaine d’argent, au niveau du front de démouillage. Nous considérons la courbure dans le plan du substrat, $\kappa_\parallel$ et la courbure perpendiculaire à ce plan, $\kappa_\perp$. Cette hypothèse implique qu’au niveau du front de démouillage, l’angle de contact entre la couche et le substrat est de 90$^\circ$, pour que le plan tangent soit perpendiculaire au substrat. Il est difficile de mesurer cet angle expérimentalement, ce qui nous oblige à simplement accepter cette approximation. C’est par ailleurs ce qui a été implicitement supposé dans les modèles de la littérature~\cite{brandon1966mobility, jiran1990capillary}.\par 
lorsqu’on observe la couche en vue de dessus, nous accédons uniquement à la courbure $\kappa_\parallel$. Par convention, la courbure sera considérée comme positive si la concavité de l’arc est orientée à l’intérieur du métal. Une particule d’argent aura une courbure positive, tandis qu’un trou dans un milieu continu aura une courbure négative. Sur la figure~\ref{schemaPlansApplique}, les deux courbures sont positives. Sur le schéma~\ref{schemaPlans}, elles sont de signes opposés.\par 

\begin{figure}[!htb]
\centering
\includegraphics[width = 0.7\textwidth]{schemaPlans}
\caption{Schéma montrant pour une surface les plans tangent et perpendiculaires. L’interception de la surface par les plans perpendiculaires est représentée par les arcs pointillés, dont les courbures au point considéré sont les courbures principales.}
\label{schemaPlans}
\end{figure}
\begin{figure}[!htb]
\centering
\includegraphics[width = 0.7\textwidth]{schemaPlansApplique}
\caption{Schéma de l'application de la définition de la courbure à une surface d’un domaine d’argent.}
\label{schemaPlansApplique}
\end{figure}

\paragraph*{Remarque~:} La définition de la courbure a été discutée par Taylor~\cite{taylor1992ii}, qui néglige pour des raisons de traitement mathématique le facteur 1/2 dans l’expression~\ref{eCourb}, et qui introduit le concept de \og courbure moyenne pondérée \fg. Cette courbure concerne les surfaces présentant une anisotropie d’énergie, et repose sur des questions énergétiques. La courbure devient alors~:
\begin{equation}
\kappa = a_1\kappa_1+a_2\kappa_2
\end{equation}
où les facteurs $a_i$ dépendent de l’énergie de surface dans la direction considérée, ainsi que sa variation dans la direction orthogonale. Dans un repère polaire d'un arc de courbure principale, pour un point de la surface de coordonnées $(r,\theta)$, $a=\gamma+\partial\gamma/\partial\theta$. Précisons que par \og direction  \fg, Taylor entend toutes les directions mathématiques possibles, pas seulement les directions correspondant aux faibles indices de Miller. Il ne nous est pas possible d'accéder expérimentalement aux coefficients $a_i$ car l'orientation cristalline de la surface à cette échelle nous est inaccessible. Nous mettons de côté ces raffinements, et nous rediscuterons de cette hypothèse par la suite.\par 

\paragraph*{La vitesse locale~:} Considérons deux images successives, $n$ et $n + 1$ d'une séquence de démouillage. Entre ces deux images, il y a eu une petite évolution du front. Sur chaque point du front l’image $n$, nous pouvons calculer le vecteur normal au front. En propageant ce vecteur jusqu’au front de l’image $n+1$, nous obtenons une distance, qui divisée par la temps d'acquisition, donne la vitesse de propagation. La figure~\ref{schemaCourbure} illustre le principe de cette mesure.\par 
Ce calcul simple est valable dans le cas où les images sont suffisamment proches. S’il y a trop d’évolution du front entre deux images, les vecteurs normaux de plusieurs points du contour de l’image $n$ risquent de se croiser avant d’atteindre le contour de l’image $n + 1$ ; cela donne des distances surévaluées. Nous veillons donc à prendre une fréquence d’acquisition d’images relativement élevée afin de garder une distance petite entre deux fronts successifs.\par 

\begin{figure}[!htb]
\centering
\includegraphics[width = 0.9\textwidth]{schemaCourbure}
\caption{Schéma présentant le principe de la mesure de la courbure locale et de la vitesse de propagation en un point du contour d’une image $n$.}
\label{schemaCourbure}
\end{figure}

			\subsubsection{Représentations des données et cas modèles}
Pour chaque point du front sur chaque image, nous obtenons à partir de l'analyse d'images MEB deux valeurs : la courbure locale $\kappa_\parallel$ et la vitesse de propagation $v$. Pour une séquence d’images, nous comptons le nombre de points obtenus par couple de coordonnées $(\kappa_\parallel,v)$. Ceci nous donne un histogramme, que nous choisissons de projeter dans le plan des coordonnées, en attribuant un code couleur à l’intensité.\par 
Enfin, il est nécessaire de normaliser les données obtenues. En effet, les courbures ne sont pas isoprobables sur le front, et ce déséquilibre peut entraîner une mauvaise interprétation. Deux expériences de pensée permettent d’illustrer ce phénomène.\par 
Premièrement, considérons un front circulaire se propageant à vitesse constante $v_0$ (c’est-à-dire que la vitesse est indépendante de la courbure). La longueur du front va augmenter au fur et à mesure du temps, ce qui va augmenter le nombre de points considérés sur le front (la distance $d$ entre ces points reste constante dans notre analyse). Ainsi, plus le rayon $r$ du cercle sera grand, plus la quantité de points de coordonnées $(1/r,v_0)$ sera grande. Si l’on ne normalise pas l’histogramme, on pourrait donc croire que la vitesse augmente avec le rayon. Il est bien important de remarquer que la surface du disque va bien croître de manière quadratique avec le temps, mais que nous ne considérons ici que des données locales : un point individuel sur le front, lorsqu’il se déplace, balaye toujours la même surface $d \cdot v_0 \cdot t$ avec $d$ la distance entre points, et $t$ l’intervalle de temps entre deux images successives.\par
Deuxièmement, considérons un front dont l’allure est représentée sur la figure~\ref{schemaNormalisation}. Il s'agit de quatre digitations, trois ont une courbure égale $\kappa_b$ et la dernière a une courbure différente $\kappa_a$. L'image a) est une situation initiale. En b), c) et d), la digitation de courbure $\kappa_a$ se propage à une vitesse arbitraire $v_0$. On aurait donc $n(\kappa_a,v_0)=3$. Par ailleurs, les digitations de courbure $\kappa_b$ se propagent une fois chacune. On a donc $n(\kappa_b,v_0)=3$. Cependant, on constate que la digitation de courbure $\kappa_a$ se propage plus rapidement (car plus fréquemment). Il ne faut donc pas considérer le compte total, mais normaliser ce compte par la fréquence d'apparition des courbures considérées.\par 
\begin{figure}[!htb]
\centering
\includegraphics[width = 0.6\textwidth]{schemaNormalisation}
\caption{Schéma d'un front constitué de quatre digitations se propageant à des vitesse différentes.}
\label{schemaNormalisation}
\end{figure}
Ainsi, nous consiédrons la quantité~:
\begin{equation}
P(v|\kappa) = \dfrac{n(\kappa,v)}{n(\kappa)},
\end{equation}
qui n'est autre que l'expression mathématique de la probabilité conditionnelle, sachant la courbure, d'observer une vitesse donnée.\par \vspace{12pt}
Pour plus de facilité pour la lecture des résultats expérimentaux, nous  présentons d'abord des cas modèles. Ces cas font référence à la figure~\ref{modelesFr}, où sont représentés des front de démouillage théoriques en cours de propagation et les histogrammes associés. Dans le cas a), le front est linéaire et se propage à une vitesse $v_0$. la courbure dans le plan $\kappa_\parallel = 0$. Ses coordonnées sont donc dans un unique point : $(0, v_0)$. Dans le cas b), le front est circulaire (correspondant à un trou se propageant), de courbure $\kappa =1/r(t)$ et se déplace à une vitesse $v_0$. À $t$, il est donc représenté en un unique point $(1/r(t), v_0)$. Lorsque $r(t)$ augmente, la courbure diminue, le point se rapproche de l’axe $\kappa = 0$. Dans le cas c), on s’intéresse à la propagation d’un trou en forme de doigt à une vitesse $v_0$. Le front est de courbure nulle sur les côtés et circulaire au bout, avec un rayon $r$. La courbure est donc soit nulle, soit de $1/r$. Les points où la propagation a lieu se situent à l’extrémité seulement, où ils ont tous une courbure $1/r$. La vitesse dépend cependant de la position des points sur la partie circulaire du doigt; la propagation du front est donc représentée sur l'histogramme par un ensemble de points tels que $\kappa = 1/r$ et $0 \geq v \geq v_0$.\par 

\begin{figure}[!htb]
\centering
\includegraphics[width = 0.6\textwidth]{modelesFr}
\caption{Cas modèles de propagation de front et les histogrammes de $P(v|Ç\kappa)$ associés.}
\label{modelesFr}
\end{figure}

		\subsection{Observations expérimentales~: courbure et vitesse de propagation}
Comme nous l’avions vu dans le chapitre précédent, la morphologie du démouillage est indépendante de la température de recuit dans une large gamme. Nous considérons qu’elle ne varie pas tant que l’argent n’est pas sublimé. Or, ce phénomène n’est significatif qu’à partir d’une température de 700~$^\circ$C. De plus, dans notre étude, il est difficile expérimentalement de se placer dans des conditions de recuit identiques pour comparer le démouillage sous vide ou sous oxygène. En effet, si l’on voulait observer un temps de recuit raisonnable sous vide, alors le démouillage sous oxygène est trop rapide.\par 
Nous n’avons donc pas cherché à nous placer dans des conditions de recuit identiques entre les deux atmosphères, mais seulement à favoriser la progression du démouillage, quitte à augmenter la température sous vide. Des couches d’argent de 60~nm (respectivement 80~nm) ont été observées au MEB lors d’un recuit à 230~$^\circ$C (respectivement 450~$^\circ$C) sous oxygène (respectivement vidéos 5 et 7)  ou à 480~$^\circ$C (respectivement 570~$^\circ$C) sous vide (respectivement vidéos 6 et 8). Les résultats sont représentés sur la figure~\ref{proba60} (respectivement \ref{proba80}) de la page~\pageref{proba60}.
Des séquences d'images des vidéos~7 et 8 (couches de 80~nm) sont présentées sur les figures~\ref{MEBgrainsOx} et \ref{MEBgrainsVac} de la page~\pageref{MEBgrainsOx}.\par 
\begin{figure}[!p]
\centering
\includegraphics[width = 0.5\textwidth]{probaOx60fr}\includegraphics[width = 0.5\textwidth]{probaVac60fr}
\caption{Probabilité sachant la courbure d’observer une vitesse de propagation locale donnée $P(v|\kappa)$ observée lors de la propagation d’un front de démouillage sur une couche d’argent de 60~nm. Gauche : sous oxygène, à 230~$^\circ$C. Droite : sous vide, à 430~$^\circ$C.}
\label{proba60}
\end{figure}

\begin{figure}[!p]
\centering
\includegraphics[width = 0.5\textwidth]{probaOx80fr}\includegraphics[width = 0.5\textwidth]{probaVac80fr}
\caption{Probabilité sachant la courbure d’observer une vitesse de propagation locale donnée $P(v|\kappa)$ observée lors de la propagation d’un front de démouillage sur une couche d’argent de 80~nm. Gauche : sous oxygène, à 450~$^\circ$C. Droite : sous vide, à 570~$^\circ$C.}
\label{proba80}
\end{figure}

\begin{figure}[!p]
\centering
\includegraphics[width = 0.7\textwidth]{MEBgrainsOx}
\caption{Images extraites de la vidéo 7 : couche d’argent de 80~nm démouillant sous atmosphère de 100~Pa d’oxygène à 450~$^\circ$C. Sur les images du bas, des grains ont été surlignés en bleu, les flèches rouges indiquent les directions possibles de propagation. L'intervalle de temps d'acquisition est de 40~s entre les deux premières images et de 1~min entre les deux dernières.}
\label{MEBgrainsOx}
\end{figure}

\begin{figure}[!p]
\centering
\includegraphics[width = 0.7\textwidth]{MEBgrainsVac}
\caption{Images extraites de la vidéo 8 : couche d’argent de 80~nm démouillant sous vide à 570~$^\circ$C. Sur les images du bas, des grains ont été surlignés en bleu, les flèches rouges indiquent les directions possibles de propagation. Les images ont été acquises avec un intervalle de 20~s entre chacune.}
\label{MEBgrainsVac}
\end{figure}
Considérons tout d'abord les histogrammes (figures~\ref{proba60} et \ref{proba80}). Les deux épaisseurs étudiées montrent des tendances similaires, dans les deux atmosphères de recuit. Sous vide, plus la courbure $\kappa$ tend vers des valeurs très négatives, plus la vitesse de propagation et la probabilité d’évolution augmentent. Cette augmentation est valable jusqu’à des valeurs de l’ordre de -18~\micro\meter$^{-1}$, soit un rayon de courbure de 55~nm. Sous oxygène, la probabilité semble être relativement constante, indépendamment de la courbure $\kappa$.\par 
Ceci montre que la forme du front est bien dépendante de l’atmosphère. Sous vide, la propagation au fortes courbures négatives est interprétée comme la propagation de doigts (Cf.les cas modèles de la figure~\ref{modelesFr}), tandis qu’elle semble isotrope sous oxygène.\par 
Considérons maintenant les séquences d'images (figures~\ref{MEBgrainsOx} et \ref{MEBgrainsVac}). Ces séquences mettent en lumière une autre différence est remarquable si l’on compare les démouillages sous vide ou sous oxygène. Nous avons discuté, dans le chapitre précédent, de la croissance extraordinaire d’un nombre restreint de grains qui pilotaient la morphologie du démouillage. Cette croissance, observée sous oxygène, n’est plus observée sous vide.\par
Afin de mieux visualiser ce comportement, sur les images des figures~\ref{MEBgrainsOx} et \ref{MEBgrainsVac}, nous avons représenté des images successives en surlignant en bleu les limites des plus gros grains. Les flèches rouges indiquent les directions possibles de propagation, c’est-à-dire les parties du front en contact avec les grains dont la hauteur n’a pas augmenté. Si un grain grossit, il devient un obstacle, et le front doit le contourner pour progresser. Sous vide, ces grains sont nombreux et sont à une grande proximité du front ; celui-ci ne peut alors progresser qu’entre des grains, ce qui donne lieu à la digitation. En revanche, sous oxygène, du fait de la sélection d’un nombre très restreint de grains, le front n’est pas bloqué et peut se propager beaucoup plus librement.\par 
Ces observations expérimentales montrent bien que d’importantes modifications sont apportées par l’oxygène pendant la propagation : la forme des trous est changée. Cela rappelle beaucoup les travaux effectuées par Kwon\textit{ et al.}~\cite{kwon2003comparison} qui comparaient de l’or et du cuivre, mais dans notre cas, un seul paramètre change : la présence d’oxygène. Dans la partie suivante, nous allons chercher à expliquer ces différences avec un modèle simple.\par 

\paragraph*{Remarque (relative à la définition de la courbure)~:}
Nous avions évoqué l’approximation que constituait notre définition de la courbure vis-à-vis d’un modèle tenant compte des anisotropies, tel que défini par Taylor~\cite{taylor1992ii}. Il faudrait, en toute rigueur, tenir compte des variations de l’énergie de surface ainsi que sa dérivée dans les différentes directions (le coefficient $a=\gamma+\partial\gamma/\partial\theta$). Cependant, une cartographie ASTAR réalisée sur une couche d’argent de 20~nm recuite à 250~$^\circ$C pendant 12~min sous vide, présentée sur la figure~\ref{astarVacDem} nous permet de constater que l’orientation cristalline est toujours aussi aléatoire au bout des doigts que sur les bords. Nous pouvons donc penser que l’orientation cristalline reste suffisamment aléatoire pour que la légère anisotropie des énergies n’influence pas le démouillage (c'est-à-dire que les coefficients $a_i$ soient semblables en tout point), et donc que notre définition de la courbure reste valable.\par 

\begin{figure}[!htb]
\centering
\includegraphics[width=0.75\textwidth]{AstarVacDem}
\caption{Cartographie Astar d’une couche d’argent de 20~nm recuite à 250~$^\circ$C pendant 12~min sous vide.}
\label{astarVacDem}
\end{figure}

		\subsection{Discussion}

			\subsubsection{Détermination de la force motrice du démouillage}
Nous avons montré que la propagation des trous était différente selon l’atmosphère de recuit. Dans beaucoup de travaux~\cite{presland1972hillock, jiran1990capillary, zucker2013model}, la courbure de la surface au niveau du front est considérée comme le moteur du démouillage. La diffusion solide est reliée à une différence de potentiel chimique, le potentiel s’exprimant par~:
\begin{equation}
\mu = \mu^0 +\gamma \Omega \kappa
\end{equation}
avec $\gamma$ l’énergie de surface et $\Omega$ le volume atomique. Jiran et Thompson~\cite{jiran1990capillary} avaient notamment proposé un modèle de démouillage où le front se propageait aux endroits où il était le plus mince, car la courbure $\kappa_\perp$ y était maximale. Ils supposaient alors que $\kappa_\perp \gg \kappa_\parallel$. Nos observations montrent d’une part que cette hypothèse n’est pas vraie, mais surtout que le modèle de courbure n’est pas applicable. En effet, si l’on regarde la vitesse de propagation du front sous vide, on s’aperçoit qu’elle est maximale précisément lorsque $\kappa\parallel = 1/h$, avec $h$ l’épaisseur du film. Or, $h$ est le rayon de courbure minimal dans la direction orthogonale, donc la courbure $\kappa_\perp\leq 1/h$. Si l’on regarde maintenant la courbure moyenne, ceci implique :
\begin{equation}
\kappa = \dfrac{\kappa_\perp+\kappa_\parallel}{2}\leq 0.
\end{equation}
Ainsi, un modèle basé sur la variation du potentiel chimique lié à la courbure prévoit que l’extrémité des doigts sous vide serait l’endroit où le potentiel est le plus bas, donc une zone d’agglomération de la matière. Nous observons expérimentalement l’inverse : c’est l’endroit où le démouillage est le plus rapide.\par 
\conclusion{Nous concluons donc que la courbure « macroscopique » (dont l’inverse est comparable à l’épaisseur de la couche) n’est pas responsable de la morphologie du démouillage. Ceci implique deux choses : le modèle supposant que c’est la force motrice de la propagation n’est pas valable, et l’explication des différences en fonction de l’atmosphère est à trouver ailleurs.\par}
Or, nous observons que la croissance de grains est radicalement différente en fonction de l’atmosphère de recuit (voir les figures page~\pageref{MEBgrainsOx}) : sous oxygène, peu de grains concentrent la matière provenant du démouillage, tandis que sous vide, l’agglomération est beaucoup plus répartie entre des grains proches du front. Nous considérons que la propagation se fait exclusivement sur les zones de la couche qui ont gardé l’épaisseur initiale, ce qui est équivalent à dire que les grains, une fois qu’ils ont grossi, bloquent la propagation du front.\par 
Cette simple considération permet d’expliquer pourquoi, sous oxygène, la vitesse est indépendante de la courbure : le front se propage dans toutes les directions, il est rarement contraint par des gros grains. Sous vide, la raison pour laquelle la propagation se fait là où la courbure est la plus négative est moins immédiate, mais néanmoins justifiée : c’est au bout des doigts que l’on trouve plus facilement un couche encore intacte, tandis que l’argent s’accumule majoritairement sur les bords.\par 
Considérons à nouveau les modèles de la littérature, résumés sur la figure~\ref{schemaPropagation}. Dans un premier temps, Jiran et Thompson~\cite{jiran1990capillary} considéraient que la propagation se faisait au niveau des zones d’amincissement du bourrelet du front, qui étaient situées au bout des doigts (figure~\ref{schemaPropagation}-a). Kosinova \textit{et al.}~\cite{kosinova2014role} et Atiya \textit{et al.}~\cite{atiya2014role}, plus récemment, ont mis en avant le rôle des grains dans le démouillage. Ce faisant, ils gardent l’idée d’un bourrelet qui progresse dans la couche (figure~\ref{schemaPropagation}-b). La progression serait assurée par une succession de croissance puis réduction des grains au niveau du front.\par 
Notre description du démouillage ne correspondent à aucun de ces modèles. Sous vide, si nos observations sont en accord avec celles de Jiran, nous avons montré en revanche que le modèle de courbure n’était pas applicable. Parallèlement, nous sommes en accord sur l’importance des grains dans le modèle employé par Kosinova \textit{et al.} ou Atiya \textit{et al.}, mais leur manière de décrire la progression du front n’est pas en accord avec nos observations : il n’y a pas de bourrelet se déplaçant au niveau du front. Notre conception du démouillage se démarque donc des deux modèles précédents.\par 
\conclusion{Le démouillage n’est pas induit par la courbure locale. Il correspond à la disparition des grains les plus petits, dont la matière s’accumule dans d’autres grains. Les grains ayant grossi bloquent la propagation du front, qui se fait donc là où la couche n’a pas gagné en épaisseur : sous vide, la propagation a majoritairement lieu au bout des doigts (figure~\ref{schemaPropagation}-c). Sous oxygène, elle est très peu contrainte (figure~\ref{schemaPropagation}-d).\par }
\begin{figure}[!htb]
\centering
\includegraphics[width=0.6\textwidth]{schemaPropagation}
\caption{Modèles pour décrire la propagation du front lors du démouillage a) selon Jiran et Thompson~\cite{jiran1990capillary}, b) selon Kosinova \textit{et al.} ou Atiya \textit{et al.}~\cite{kosinova2014role, atiya2014role}, c) selon nos travaux, sous vide et d) selon nos travaux, sous oxygène.}
\label{schemaPropagation}
\end{figure}

\paragraph*{Remarque~:} Nous avons conclu que la courbure n’est pas la force motrice de la transformation. Précisons que cette assertion est vraie pour la propagation, mais pas pour le frittage. En effet, le frittage, tel qu’observé dans le chapitre III et la section B.2.\ref{sfrittage}, au sein d’un îlot d’argent, amène les zones de fortes courbures (petites protubérances) à alimenter les zones de faibles courbures. Cette transformation est illustrée sur la figure~\ref{MEBfrittage} de la page~\pageref{MEBfrittage}. Or, nous avons vu que le frittage était très lent par rapport à la propagation. Cette observation corrobore la présence, pendant la propagation, d’une force motrice d’une autre nature, qui accélère le processus.\par \vspace{12pt}

La question de la nature de cette force motrice pendant la propagation demeure donc. Les travaux de Kosinova \textit{et al.} et Atiya \textit{et al.}~\cite{kosinova2014role, atiya2014role} définissent une force motrice liée aux différentes énergies de surface et d'interface du système. Pour des grains cubiques de taille identique, elle s’écrit $\Delta\gamma = \gamma_M+\gamma_i-\gamma_S$, avec les énergies respectivement de surface du métal, de l’interface et de surface du substrat. Cette description est en contradiction avec nos observations : une diminution de $\gamma_M$ entraînerait une diminution de la force motrice, or la propagation est plus rapide sous oxygène, lorsque l’énergie de surface est la plus faible (notons que nous considérons $\gamma_i$ indépendante de l’atmosphère comme Buttner~\cite{buttner1952adsorption}, et Parikh~\cite{parikh1958effect} a observé par mesures mécaniques que $\gamma_s$ ne varie pas en fonction de la présence d’oxygène). Ainsi, la diminution d’énergie de surface du système ne semble pas suffisante pour expliquer nos observations.\par 
Pendant le démouillage, un autre processus n’est pas gouverné par la diminution de la surface libre : c’est l’induction. Pendant cette étape, la réorganisation des grains est certes influencée par l’énergie de surface, mais beaucoup plus par les énergies aux joints de grains, ainsi qu’à l’interface. De ce point de vue, nous considérons la propagation comme la prolongation de l’induction, mais avec un phénomène supplémentaire : la propagation des trous.\par 
\conclusion{Ceci implique que la réduction au cours du temps de l’énergie aux joints de grains et à l’interface est le
véritable moteur de la propagation. La manière dont cette réduction s'opère dépend de l'énergie de surface du métal, et donc de l'atmosphère de recuit dans le case de l'argent.\par }

			\subsubsection{La diffusion, un facteur non limitant}
Durant toute la discussion, nous n’avons pas abordé la question de la diffusion. Plusieurs travaux suggèrent que l’oxygène augmente l’auto-diffusion de l’argent sur l’argent~\cite{rhead1963surface, yoshihara1979effect}, comme étudié par exemple dans le cas du cuivre~\cite{bradshaw1964surface}. Pourtant, nous n’avons pas retenu cette hypothèse comme déterminante dans notre modèle, et cela pour deux raisons.\par 
La première est liée à l’affirmation de l’augmentation de la diffusion ne présence d'oxygène elle-même : dans les publications traitantdu sujet, la méthode employée pour mesurer la variation de diffusion est la mesure cinétique du sillonnement des joints de grains. Ce phénomène, rationalisé par Mullins~\cite{mullins1957theory} (cf. chapitre I), indique que le sillonnement suit une loi :
\begin{equation}
w = 4,6(Bt)^{1/4},
\end{equation}
avec $w$ l’écartement entre les deux maxima avoisinant le joint, et $B = D\gamma_M \Omega^2\nu/kT$, avec $D$ le
coefficient d’auto-diffusion, $\gamma_M $ l’énergie de surface, $\Omega$ le volume atomique, $\nu$ la densité surfacique
d’adatomes et $kT$ dans son acception usuelle en thermodynamique. Or, dans les différents travaux, une amplification de l’évolution de $w$ sous gaz est interprétée comme une augmentation de $D$, en supposant $\gamma_M$ constante. Au vu de la discussion dans les travaux de Buttner~\cite{buttner1952adsorption}, cette hypothèse n’est pas vérifiée, et il est très difficile de conclure sur un réelle augmentation de $D$. Notons que Rhead~\cite{rhead1963surface} évoque dans sa publication le problème de l’apparition de nouvelles facettes lors de recuits sous oxygène, liée à la modification des énergies de surface. Dans son étude, il ne caractérise donc que les joints de grains où il n’observe pas, en apparence, ce phénomène de facettage, mais rien n'indique que l'augmentation de $w$ ne soit pas liée au même phénomène. Nous en concluons que l’augmentation de la diffusion, en plus d’être mal quantifiée, n’est pas absolument certaine. Nous préférons à ce titre retenir la diminution des énergies de surface comme le facteur déterminent.\par 
La seconde raison nous appelle à comparer notre système à celui de Kwon \textit{et al.}~\cite{kwon2003comparison} Par comparaison des morphologies, nous mettons en parallèle le démouillage de l’argent sous vide avec celui de l’or d’une part (on parlera du démouillage de type \og vide \fg), et le démouillage de l’argent sous oxygène avec celui du cuivre d’autre part (on parlera du type \og oxygène \fg). Remarquons que cette comparaison est légitime dans la mesure ou il s’agit de systèmes semblables : couches polycristallines, métaux cfc, substrat de silice amorphe. Dans les différents systèmes étudiés, nous pouvons comparer les morphologies et les coefficients de diffusion : les résultats sont présentés dans le tableau~\ref{tDiffusion}. On remarque rapidement que les coefficients de diffusion jouent en sens contraire : le coefficient de diffusion de l’or à 700~$^\circ$C (type \og vide \fg) est plus important que celui du cuivre à 300~$^\circ$C (type \og oxygène \fg). Or, si on garde l’hypothèse d’une augmentation du coefficient de l’auto-diffusion de l’argent sous oxygène, il devient donc plus grand dans le type \og oxygène \fg{} que dans le type \og vide \fg. L’effet est donc opposé ; la diffusion n’est pas la raison principale du changement de morphologie.\par 
\conclusion{Pour ces deux raisons, nous considérons que la diffusion n’est pas un facteur limitant,
quelle que soit l’atmosphère.\par}
\begin{table}
\centering
\begin{tabular}{ccccc}
\hline
système & Ag (vide, & Au (atm. réduc.,) & Ag (air ambiant, & Cu (atm. réduc.,\\
\hspace{1pt} &  200~$^\circ$C)&  700~$^\circ$C &  200~$^\circ$C) &  300~$^\circ$C)\\
\hline
morphologie & type vide & type vide & type oxygène & type oxygène\\
D (cm2.s$^{-1}$) &  $\approx$ 10$^{-8}$ & 4,6.10$^{-6}$ & $\approx$ 10$^{-6}$* & 7,8.10$^{-7}$\\
\end{tabular}
\caption{Morphologies observées pendant le démouillage et valeurs des coefficients d’autodiffusion des métaux obtenues par calcul dans la publication~\cite{agrawal2002predicting}. *) Si l’on conserve l’hypothèse d’un coefficient plus important sous oxygène, en extrapolant les travaux de Rhead~\cite{rhead1963surface}, comme expliqué dans le chapitre I, section C.}
\label{tDiffusion}
\end{table}
Ainsi, lorsqu’on observe une sélection des grains sous oxygène, dont la croissance a lieu à un micron du front, et pas sous vide, nous pensons que la différence n’est pas expliquée par la diffusion. D’autres pistes sont à envisager :
\begin{itemize}
\item une limitation par l’adsorption d’oxygène : certains travaux estiment que l’oxygène ne s’adsorbe pas sur les facettes (111)~\cite{engelhardt1976adsorption}. L’adsorption d’oxygène pourrait limiter l’agglomération de l’argent partout ailleurs que sur ces faces \og nues \fg ;
\item une limitation par l’énergie de surface : selon d’autres travaux~\cite{molina2011size}, la présence d'oxygène abaisserait beaucoup plus l'énergie de la surface (100) que celle des autres orientations, ce qui pourrait expliquer une croissance préférentielle dans cette orientation.
\end{itemize}
Toute la question demeure dans la facilité qu’a un atome d’argent de s’extraire d’une surface pour devenir un adatome libre, diffusant vers d’autres sites plus favorables~\cite{combe00}. Si cette étape est facile, alors la sélectivité de la croissance correspondrait à un équilibre gouverné par les rapports d’énergie uniquement. Si au contraire l’extraction d’adatomes est limitante, alors le processus ne serait pas équilibré et la sélection serait uniquement liée au premier endroit où les adatomes peuvent se fixer : là où il n’y a pas d’adsorbat. Il ne nous est pas possible de trancher entre ces deux possibilités, il est également possible que les deux coexistent.\par
Notons que la morphologie observée du démouillage de l’argent sous oxygène est très proche de celle du fer sous hydrogène~\cite{kovalenko2013solid} . Dans cette étude, Kovalenko \textit{et al.} attribuaient la sélection à un résultat de la diffusion à l’interface ou aux joints de grains, arguant que la diffusion de surface aurait tendance à gommer les trop grandes différences de tailles entre grains voisins si elle avait lieu. Nous avons montré dans le chapitre III que la diffusion de surface restait prépondérante ; nous n’adhérons pas à leurs arguments. En revanche, il est intéressant de noter que Grenga \textit{et al.}~\cite{grenga1976surface}, en observant les modifications de particules de Fe en présence de $H_2$, on noté deux effets : à la fois une adsorption différente suivant les facettes et un changement de l’anisotropie de l’énergie de surface, observations cohérentes avec nos hypothèses dans le cas de l'oxygène sur l'argent.\par 

\section{Variations progressives d'atmosphère}
Jusqu’ici, nous avons considéré deux cas extrêmes d’atmosphère de recuit, entre le vide secondaire et une atmosphère « riche » en oxygène (pression partielle >10~Pa). Nous allons ici présenter ce qu’il advient sous des pressions partielles d’oxygène intermédiaires, ou dans des empilements enrichis en oxygène.\par 
	\subsection{Densité de trous en fonction de la pression partielle d'oxygène}
Nous avons étudié en MEB in situ le démouillage sous plusieurs pressions partielles d’oxygène. Étant donné que le démouillage connaît de grandes variations de cinétique en fonction de l’atmosphère de recuit, nous n’avons pas pris le parti d’étudier le démouillage en intégralité pour chaque atmosphère. À la place, nous avons fixé arbitrairement une température (350~$^\circ$C) et observé l’évolution d’une couche de 40~nm sous différentes atmosphères. Ces atmosphères sont obtenues en introduisant dans la chambre du MEB de l'air, de l'oxygène pur, un mélange $N_2:O_2$ (1000~ppm) ou un mélange $N_2:O_2$ (100~ppm), ainsi qu'en variant la pression entre 400 et 10~Pa. Lors du démouillage, des trous apparaissent. Nous avons simplement mesuré la densité maximale de trous au court d’une rampe de température de 10~$^\circ$C/min jusqu’à la température finale. Cette densité maximale était soit celle mesurée à 350~$^\circ$C (si le démouillage est lent), soit celle mesurée au cours du recuit (si le démouillage est rapide, les trous percolent avant 350~$^\circ$C). La vitesse de la rampe a été fixée, car nous observons que la densité maximale dépend de la rampe. Une rampe lente favorise davantage la propagation des premiers trous formés face à la création de nouveaux trous ; nous avons donc travaillé avec une rampe aussi rapide que les conditions techniques le permettent. Les résultats sont présentés dans la figure~\ref{pressionPartielle}.\par 
\begin{figure}[!htb]
\centering
\includegraphics[width=0.7\textwidth]{pressionPartielle}
\caption{Densité  maximale de trous maximale mesurée lors du démouillage de couches d’argent de 40~nm lors d’une rampe jusqu’à 350~$^\circ$C (10~$^\circ$C/min) en fonction de l’activité d’oxygène dans l’atmosphère de recuit. Deux séries ont été réalisées (symboles verts et bleus).}
\label{pressionPartielle}
\end{figure}
Deux séries différentes ont subi ce traitement (points bleus et points verts)~; nous pouvons en observer la cohérence sur la figure~\ref{pressionPartielle}. Aux très basses pressions d’oxygène (activité <5.10$^{-7}$), la densité maximale de trous ne varie pas. Elle augmente de manière linéaire à partir d’un seuil aux alentours de 5.10$^{-7}$-10$^{-6}$, jusqu’à la pression d’oxygène dans l’air.\par 
Comme nous l’avions calculé auparavant (voir figure~\ref{energiesFr}), l’énergie de surface a une influence sur l’angle de contact au niveau des joints de grains. Une diminution de l’énergie mène à un plus grand nombre de trous dans la couche. Ainsi, les variations de densité de trous que nous mesurons sont révélatrices de l’abaissement de l’énergie de surface de l’argent en présence d'une pression partielle d'oxygène qui conduit à un taux de recouvrement significatif des surfaces/interfaces d'argent. Cet effet est visible pour une activité aussi faible que 5.10$^{-7}$, ce qui est un seuil plus bas de deux décades que celui proposé par Buttner~\cite{buttner1952adsorption}, cf. figure~\ref{energiesFr}. Par ailleurs, des mesures d'angles de contact de l'argent liquide sur un substrat de sapphire pouvaient permettaient de constater une baisse de cette énergie pour une activité comprise entre 10$^{-3}$~\cite{chatain94} ou 1010$^{-6}$~\cite{muolo08}.  À notre connaissance, une modification à si faible activité n’avais pas été observée dans la littérature dans cette gamme de température (c'est-à-dire pour de l'argent solide).\par 
Il serait trop hasardeux d’estimer numériquement la variation d’énergie : cela nécessiterait de construire un modèle à multiples paramètres (distribution d’épaisseur des joints de grains, densité des joints de grains, configuration locale des orientations cristallines, croissance en fonction de la température, etc.) dont la plupart sont inconnus.\par 
\conclusion{L’influence de l’oxygène sur le démouillage est très progressive et a lieu dès de très
faibles activités.\par }
	\subsubsection{Présence d'oxygène dans les empilements}
Dans le cas des empilements, certains paramètres sont différents : il n’y a plus de surface libre du métal, donc pas d’atmosphère ni de diffusion de surface. Pourtant, suivant les empilements, l’aspect des défauts peut varier. Dans certains cas, on observe plutôt des \og dendrites \fg, tandis que dans d’autres, ce sont plutôt des « dômes ». Comme nous l’avons expliqué plus haut, la présence de croissance extraordinaire de grains est liée à la présence d’oxygène, induisant une forte sélectivité des facettes qui peuvent croître. La question est alors de savoir si la présence d’oxygène dans l’empilement est envisageable.\par 
Nous avons donc procédé à une expérience en faisant varier la nature du substrat. C’est une couche de SnZnO$_x$, qui dans un cas contient de l’oxygène en proportions stœchiométriques, et dans l’autre contient de l’oxygène surnuméraire. Cette couche s’obtient aisément lors d’un dépôt magnétron réactif : en fonction de la proportion de dioxygène dans le plasma de dépôt, il est possible de contrôler les proportions dans la couche finale (voir partie expérimentale pour les détails du dépôt).\par 
Nous avons procédé à un recuit sous un vide secondaire à 300~$^\circ$C dans les deux cas et comparons l’état de la couche. La figure~\ref{MEBempilOx} présente les états de la couche obtenus en fonction du substrat. Sur la couche d’oxyde en proportions stoechiométriques (image a), la morphologie est très semblable à ce que l’on observe sous vide avec de la silice amorphe. En revanche, lorsque la couche d’argent démouille sur la couche d’oxyde enrichie en oxygène, on observe une morphologie proche de celle observée après un recuit sous oxygène. Notons que la couche a atteint cet état très vite (avant même de réaliser la première mise au point pendant la rampe), mais qu’elle a cessé d’évoluer. Ce comportement semble indiquer deux choses : le substrat est capable de fournir de l’oxygène à la couche d’argent lors du démouillage, modifiant son comportement, mais cet apport est limité.\par 
\conclusion{Ainsi, même dans les empilements, de l’oxygène peut être mis en contact avec la couche d’argent, influençant sa stabilité et son démouillage. La nature des couches directement en contact avec la couche d’argent, mais aussi la capactité de l'empilement à relâcher de l'oxygène au travers de l'argent est donc un facteur important pour comprendre le démouillage.\par}
\begin{figure}[!htb]
\centering
\includegraphics[width=0.7\textwidth]{MEBempilOx}
\caption{Couche d’argent de 40~nm recuite à 300~$^\circ$C sous vide sur une couche de SnZnO$_x$ amorphe a) stroechiométrique en oxygène b) avec un excès d'oxygène.}
\label{MEBempilOx}
\end{figure}
\section{Conclusion}
Le démouillage se décompose en trois étapes : l’induction, la propagation et le frittage.\par 
\textbf{Pendant l’induction}, la structure cristalline connaît de grands changements : augmentation de la taille moyenne des grains, de la rugosité et de façon surprenante, changement des orientations cristallines par rapport au substrat. Selon la littérature, Ces changements sont gouvernés par la réduction de l’énergie de surface, des contraintes à l’interface et très localement de l’énergie à l’interface des joints de grains.\par 
L’oxygène favorise à la fois une croissance extraordinaire de certains grains spécifiques, auparavant appelés \og buttes \fg, mais aussi l’apparition de trous dans la couche. En absence d’oxygène, les équilibres locaux ont tendance à ne pas créer de trous, mais la transformation durant l’induction a bel et bien lieu.\par 
Pendant la propagation, les trous se progressent dans la couche, diminuant le taux de couverture. La propagation est assurée par la disparition des grains les plus petits, qui alimentent la croissance d’autres grains. Si un grain a cru, il est beaucoup moins susceptible d’être modifié par le démouillage. Sous oxygène, la croissance ne concerne qu’un petit nombre de grains sélectionnés durant l’induction. Elle peut avoir lieu à une distance de l’ordre du micron du front de démouillage. Ceci assure au trou la possibilité de se propager quasiment librement au sein de la couche. Sous vide, l’agglomération est beaucoup plus répartie entre des grains proches du front, contraignant la progression du front à s’adapter sans cesse à l’évolution locale qu’elle engendre.\par 
Aucun bourrelet de propagation n’est visible lors du démouillage, quelle que soit l’atmosphère de recuit. Nous avons montré que la courbure locale n’est pas le la force motrice de la propagation. La propagation est liée aux mêmes forces motrices que l’induction : c’est une réorganisation de la structure cristalline qui a lieu à une échelle plus grande. La présence de trous accélère cette réorganisation.\par 
Lorsque toute la couche a été réorganisée par la propagation, la dernière étape prend place : le frittage. Cette étape est gouvernée par l’énergie de surface des îlots d'argent restants, elle a une cinétique beaucoup plus lente que les précédentes. Le frittage tend à amener les îlots vers leur forme d'équilibre, sans que ceux-ci n'échangent de matière entre eux : à l'échelle de nos mesures, le frittage ne conduit pas à une réduction de la densité des particules. Il est important de noter que la différence entre frittage et propagation n'est pas uniquement cinétique, mais résulte bien d'un changement de force motrice. Enfin, notons que le frittage n'est pas exclus pendant la propagation, il est simplement trop lent pour être significatif à ce moment-là. Notons enfin que le frittage n'a pas (ou peu) été observé sous vide car sa cinétique est extrêmement lente.\par \vspace{12pt}
Toutes ces conclusions ont pu être formulées grâce à la coïncidence de plusieurs facteurs. Le premier est la possibilité d'avoir pu étudier en MEB le démouillage \textit{in situ}, en temps réel et en atmosphère contrôlée. À ce titre, le MEB \textit{in situ} est un outil précieux pour l'étude du démouillage. Le second est le choix du système (argent polycristallin sur substrat amorphe), qui permet de s'affranchir des effets d'épitaxie qui interviennent sur les substrats cristallins. Ce choix correspond à une volonté de s'approcher de la thématique industrielle développée par Saint-Gobain. Le fait de s'intéresser à ce système original nous a permis d'observer un comportement inédit, et d'approfondir la compréhension fondamentale du démouillage. La démarche d'application de recherche fondamentale dans le cadre d'un système industriel s'est révélée extrêmement riche. Cette dernière considération s'applique bien au-delà de la seule thématique du démouillage.\par 

\newpage
\bibliographystyle{ieeetr}
\bibliography{biblio}



%%%%%%%%%%%%%%%%%%%%%%%%%%%%%%%%%%%%%%%%%%%%%%%%%%%%%%
%%
%% Chapitre 5 : Réponse optique
%%
%%%%%%%%%%%%%%%%%%%%%%%%%%%%%%%%%%%%%%%%%%%%%%%%%%%%%%
%\setcounter{chapter}{4}
\chapter{Réseaux de particules organisées~: optimisation et réponse optique}
\begin{flushleft}
\textit{
\og Que pour toi ce trésor s’arrange\\	
En une mosaïque étrange	\\
\hspace{12pt}Aux tons divers,	\\
Que ces belles choses sans nombre	\\
De leurs feux illuminent l’ombre.\\	
\hspace{12pt}De tous tes vers,\\
\vspace{12pt}
Combine d’une main savante,	\\
Imagine, compose, invente,	\\
\hspace{12pt}Refais, refonds,	\\
Sers-toi des poinçons et des limes,	\\
Et que tes dessins soient sublimes	\\
\hspace{12pt}Et soient profonds,\\
\vspace{12pt}
Quand ton œuvre sera finie,	\\
Malgré l’effort de ton génie,	\\
\hspace{12pt}Tous tes cadeaux	\\
Ne pourront remplacer encore	\\
Ceux dont la nature décore	\\
\hspace{12pt}Mon petit dos.\fg [...]\\
\vspace{12pt}
Et je vis, sur ma main, bourdonnant de colère,\\	
Un être merveilleux et pourtant tout petit.	\\
Ce rien du tout luisait comme un spectre solaire.\\	
C’était un scarabée. Il eut peur et partit.}
\end{flushleft}
\begin{flushright}
\textbf{Jean Richepin}, \textit{À la gloire des insectes, 1881.}
\end{flushright}
\minitoc
\newpage

\textit{La mise en place des mesures optiques ainsi que de nombreux résultats de ce chapitre sont le fruit du stage de Barbara Bouteille (de l'ESCPI) au laboratoire SVI.}\par 
Dans ce chapitre, nous allons étudier le contrôle du démouillage. Notre stratégie est fondée sur la texturation du substrat sur lequel les couches d'argent démouillent. Nous allons, dans un premier temps, nous intéresser aux aspects de synthèse permettant l'optimisation de l'organisation spatiale des particules en réseaux. Ensuite, nous présenterons les résultats expérimentaux concernant l'optique de ces réseaux et nous discuterons de l'apport de la modélisation numérique pour comprendre les phénomènes optiques à l'oeuvre.

\section{Contrôle du démouillage}
Comme nous l'avons remarqué dans le chapitre I, il existe plusieurs stratégies pour contrôler le démouillage. Nous avons retenu celle qui consiste à procéder au démouillage sur des surfaces texturées. Cette stratégie, aussi étudiée par Giermann et Thompson~\cite{giermann2005solid, giermann2011requirements}, avait été appliquée avec succès dans notre laboratoire à des couches d'argent~\cite{le2014self}. Dans cette partie, nous allons discuter de la texturation de surface, de son influence sur le démouillage, mais aussi des relations entre paramètres de dépôt de la couche et structures démouillées.\par 

	\subsection{Fabrication des surfaces texturées}
	\subsubsection{Principe de fabrication}
Les détails expérimentaux de la procédure sont exposés dans le chapitre II. Nous obtenons des surfaces texturées grâce à un procédé sol-gel appelé nano-impression~\cite{le2014self,chou1996nanoimprint}. Ce procédé vise à répliquer sur des couches sol-gel la texture d'un échantillon initial, appelé \og master \fg. Comme schématisé sur la figure~\ref{schemaSolGelMini}, on peut décrire ce procédé en trois étapes~: 1) le dépôt de la couche sol-gel, 2) l'embossage~: la condensation de la couche en présence d'un moule qui contraint la topographie de la surface, pendant un recuit et sous pression, et 3) un dernier recuit pour terminer la condensation et éliminer les traces de solvant.
\begin{figure}[!htb]
\centering
\includegraphics[width=0.9\textwidth]{schemaSolGelMini}
\caption{Schéma représentant le procédé sol-gel utilisé pour texturer les substrats. 1) dépôt d'une couche sol-gel sur le substrat 2) embossage (50~N, 75~$^\circ$C, 45 min) 3) recuit à 500~$^\circ$C, 2~h.}
\label{schemaSolGelMini}
\end{figure}

	\subsubsection{Géométrie des masters}
Nous avons à disposition 7 masters achetés auprès d'un fournisseur (Cemitec). Ce sont des réseaux de pyramides inversées~: six réseaux carrés et un réseau hexagonal. Les dimensions de ces masters sont caractérisées par des paramètres géométriques (voir figure~\ref{dimensionsCpyrChapter3}). $P$ est la période, $l$ est la largeur de la base de la pyramide, $d$ est la profondeur de la pyramide et $m$ est la largeur de la mésa (plateau qui sépare les pyramides). Les pyramides inversées sont produites sur le master par lithographie électronique (EBL), qui grave un masque à la surface. Une attaque chimique est réalisée sur toute la surface qui n'est pas recouverte par le masque. Cette attaque suit les plans cristallins du silicium (111), la géométrie des pyramides est donc imposée. Comme les plans inclinés (111) forment un angle de 54,7$^\circ$ avec la base (100), la profondeur $d$ vaut donc~:
\begin{equation}
d = \dfrac{l}{2}\tan(54,7^\circ).
\end{equation}
Ceci nous permet d'estimer la profondeur des pyramides, le fournisseur nous ayant fourni la largeur. Les valeurs des différents réseaux sont présentées dans le tableau~\ref{tGeometrie}. Le réseau hexagonal possède les mêmes paramètres géométriques que le réseau de période 350~nm. Par la suite, nous désignerons les réseaux par leur période.\par 

\begin{figure}[!htb]
\centering
\includegraphics[width=0.4\textwidth]{dimensionsCpyr}
\caption{Schéma représentant les paramètres géométriques des masters, avec $P$ la période, $l$ la largeur de la base de la pyramide, $d$la profondeur de la pyramide et $m$ la largeur de la mésa (plateau qui sépare les pyramides).}
\label{dimensionsCpyrChapter3}
\end{figure}

\begin{table}[!htb]
\centering
\begin{tabular}{ccccccc}
\hline
$P$ (nm) & 200 & 350 & 500 & 600 & 750 & 1000\\
\hline
$d$ (nm, calculée) & 90 & 150 & 260 & 315 & 394 & 415\\
$l$ (nm) & 125 & 215 & 370 & 450 & 560 & 590\\
$m$ (nm) & 75 & 135 & 130 & 150 & 140 & 410\\
\hline
\end{tabular}
\caption{Géométrie des masters qui ont servi pendant nos travaux, avec $P$ la période, $d$ la profondeur de la pyramide, $l$ la largeur de la base et $m$ la largeur de la mésa.}
\label{tGeometrie}
\end{table}

	\subsubsection{Texture obtenue par nano-impression}
Nous avons contrôlé par AFM la morphologie des textures obtenues par cette méthode, avant et après le dernier recuit, pour le réseau de 600~nm. Les profils AFM sont présentés sur la figure~\ref{profilPyramides}. On observe que la période du réseau et la largeur des mésas varient peu. En revanche, la profondeur de la pyramide est moindre après le recuit~: elle passe d'une moyenne de 215~nm à 170~nm.\par 
Nous avons mené cette étude sur les différentes textures et les résultats sont présentés dans le tableau~\ref{tPerteRecuit}. $d_{master}$ désigne la profondeur du master, $d_{embossage}$, la profondeur mesurée après embossage sur la couche sol-gel et $d_{recuit}$ désigne la profondeur mesurée sur la couche sol-gel, après le dernier recuit. Nous pouvons constater une diminution systématique de la profondeur des pyramides~: une première fois après la réplication par embossage, une seconde fois lors du dernier recuit. La limitation de la mesure par AFM, c'est-à-dire la difficulté de sonder des cavités trop étroites, ne permet pas d'expliquer une diminution aussi importante pour ces profils peu accidentés. Nous pouvons affirmer que le dernier recuit a tendance à diminuer fortement le rapport d'aspect des pyramides. Cette modification n'affecte pas la période, mais elle a tendance à adoucir les profils et arrondir les mésas. Elle est particulièrement marquée pour la texture de 250~nm.\par 
\begin{figure}[!htb]
\centering
\includegraphics[width=0.6\textwidth]{profilPyramides}
\caption{Profils AFM extraits de cartographies d'un échantillon texturé par le réseau de 600~nm, avant et après le dernier recuit (étape~3 de la figure~\ref{schemaSolGelMini}).}
\label{profilPyramides}
\end{figure}
\begin{table}[!htb]
\centering
\begin{tabular}{ccccccc}
\hline
$P$ (nm) & 200 & 350 & 500 & 600 & 750 & 1000\\
\hline
$d_{master}$ (nm) & 90 & 150 & 260 & 315 & 394 & 415\\
$d_{embossage}$ (nm) & 50 & 150 & 195 & 215 & 325 & 300\\
$d_{recuit}$ (nm) & 25 & 80 & 170 & 170 & 295 & 270\\
\hline
\end{tabular}
\caption{Géométrie des masters qui ont servi pendant nos travaux. Avec $P$ la période, $d$ la profondeur de la pyramide, $l$ la largeur de la base et $m$ la largeur de la mésa.}
\label{tPerteRecuit}
\end{table}
\paragraph*{Remarque~:} Cette réduction du rapport d'aspect est simplement constatée ici, mais les modifications des paramètres de la couche (notamment la chimie de la synthèse sol-gel) pourraient permettre de contrôler davantage ce phénomène et varier les rapports d'aspect des textures obtenues.\par 

	\subsection{Structures métalliques obtenues par démouillage}
Une fois les substrats texturés, nous déposons une couche d'argent et procédons à son démouillage (un nouveau recuit à 400~$^\circ$C, pendant 2~h). Nous nous intéressons maintenant aux particules métalliques obtenues après démouillage.\par 
Sur la figure~\ref{MEBarraysAllText}, nous pouvons observer les structures métalliques obtenues pour chaque texture. Les épaisseurs initiales ont été variées, nous discuterons de ceci dans la section suivante. On remarque une bonne organisation des particules obtenues par démouillage en réseau, conformément à la texture imposée par le substrat. Nous observons également des défauts typiques~:
\begin{itemize}
\item des particules manquantes ;
\item des particules interstitielles (sur les mésas, entre deux trous) ;
\item plusieurs particules par trou ;
\item des particules pontantes, qui font le lien entre plusieurs trous.
\end{itemize}
Comme nous l'avons démontré dans le chapitre III, les grains sont les entités qui gouvernent la morphologie du démouillage. Sur les surfaces texturées, ils permettent de s'affranchir des questions de volume quant à l'organisation des particules au sein de la texture. Nous allons donc discuter de l'optimisation des réseaux de particules obtenus, non en fonction du volume de métal déposé, mais en fonction de la taille des grains.\par 
\begin{figure}[!htb]
\centering
\includegraphics[width=\textwidth]{MEBarraysAllText}
\caption{Particules métalliques obtenues après démouillage de couches d'argent sur les surfaces texturées~: toutes les périodes à disposition sont présentées, ainsi que sur le réseau hexagonal. Les épaisseurs respectives des couches initiales sont 24, 30, 30, 37, 54, 70 et 90~nm.}
\label{MEBarraysAllText}
\end{figure}

	\subsection{Optimisation du dépôt d'argent}
	\label{sQuality}
	\subsubsection{Mesure de la qualité}
Bien que la qualité du réseau soit appréciable de manière visuelle, il est plus aisé de comparer des valeurs numériques. Nous avons donc développé un code de traitement d'images pour discerner, sur une surface, quelles étaient les particules de taille correcte et correctement positionnées. Les détails de la procédure sont exposés dans le chapitre II. Nous obtenons en sortie une image traitée, dont un exemple typique est représenté sur la figure~\ref{triExempleChpter3}, ainsi qu'une évaluation numérique de la qualité de l'organisation. Cette évaluation est faite sous la forme de deux rapports~: $r_{tot}$ et $r_{trou}$. $r_{tot}$ évalue le nombre de particules correctes en taille et position sur le nombre de particules total et $r_{trou}$ évalue le nombre de particules correctes en taille et position sur le nombre de trous sur l'image (c'est-à-dire le nombre idéal de particules). \par 
\begin{figure}[!htb]
\centering
\includegraphics[width=0.25\textwidth]{triExemple}
\caption{Exemple d'image obtenue après la procédure de quantification de la qualité d'un réseau. Les particules vertes sont considérées de taille correcte et correctement positionnées. Les particules rouges sont trop grandes ou trop petites, les particules bleues sont de taille correcte, mais mal positionnées par rapport au réseau.}
\label{triExempleChpter3}
\end{figure}
	\subsubsection{Influence de l'épaisseur}
Grâce à cet algorithme, nous avons étudié l'évolution de la qualité de dépôt en fonction de l'épaisseur initiale de la couche. En effet, la taille des grains augmente avec l'épaisseur de la couche d'argent native déposée~\cite{hillert1965theory, thompson1990grain}. Nous prenons ici l'exemple de la texture en réseau carré de 350~nm de période.\par 
Des couches d'argent d'épaisseurs comprises entre 20 et 75~nm ont été déposées sur la surface texturée. Après un recuit de 2~h à 400~$^\circ$C, la surface a été caractérisée par MEB, puis les images ont été traitées par notre algorithme (voir figure~\ref{triMEB350} de la page~\pageref{triMEB350}). Pour les épaisseurs les plus faibles (20 et 25~nm, images~a et b), nous observons beaucoup de particules. Beaucoup sont mal placées (bleues) ou de trop petite taille (rouges). Pour ces images, on trouve tout de même un nombre de particules vertes proche du nombre de trous, aussi considérons nous que le rapport $r_{tot}$ est plus adapté pour décrire la qualité du réseau. En effet, le rapport $r_{trou}$ ne tient pas compte de la présence de beaucoup de particules supplémentaires sur le réseau. Pour les épaisseurs de 30 à 40~nm, on remarque que le réseau de particules est plus conforme au réseau du substrat. Pour 30~nm d'épaisseur initiale (image~c), on observe davantage de particules bleues que de particules rouges, mais c'est l'opposé pour une épaisseur initiale de 40~nm (image~e). Cela signifie qu'une épaisseur faible favorise les particules surnuméraires interstitielles, tandis qu'une épaisseur plus grande favorise les particules pontantes ou manquantes. Pour les épaisseurs initiales encore plus grandes (de 45 à 75~nm, images~f, g et h), le nombre de particules vertes diminue fortement, et les particules sont de moins en moins déconnectées. Le cas extrême de l'image~h montre que le domaine d'argent est toujours percolé. Pour ces épaisseurs-là, le rapport $r_{trou}$ sera privilégié~: en effet, le nombre total de particules diminue fortement, ce qui a tendance à augmenter le rapport $r_{tot}$ de manière artificielle.\par 
Nous pouvons discuter ces observations en termes de densité de grains. Pour une épaisseur initiale faible, la taille des grains est petite, donc leur densité grande. Si l'on suppose que le nombre de grains qui croissent pendant le démouillage est directement proportionnel au nombre de grains dans la couche initiale, cela implique que l'on observe une plus grande densité de particules. C'est effectivement le cas (images~a, b et c), et cette trop grande densité de particules donne lieu à une grande population de particules interstitielles.\par 
\begin{figure}[!p]
\centering
\includegraphics[width=0.75\textwidth]{triMEBlow}
\includegraphics[width=0.75\textwidth]{triMEBhigh}
\caption{Images MEB de structures obtenues après un recuit de 2~h à 400~$^\circ$C sur un substrat texturé par le réseau de période 350~nm, traitées par notre algorithme. Différentes épaisseurs de couches initiales sont comparées~: a) 20~nm, b) 25~nm, c) 30~nm, d) 35~nm, e) 40~~nm, f) 45~nm, g) 55~nm, h) 75~~nm. Les particules vertes sont de taille correcte et bien positionnées, les particules rouges sont de mauvaise taille, les particules bleues sont mal positionnées et les particules grises sont exclues du traitement car elles touchent les bords de l'image.}
\label{triMEB350}
\end{figure}

Les rapports $r_{tot}$ et $r_{trou}$ mesurés sont présentés sur la figure~\ref{qualiteStructures}. Les deux rapports sont faibles lorsque l'épaisseur initiale de la couche est faible, ils augmentent, passent par un maximum pour une épaisseur \og optimale \fg{}  située entre 30 et 40~nm, puis diminuent. On remarque également que $r_{tot}<r_{trou}$ pour une épaisseur plus petite que l'épaisseur optimale, puis $r_{tot}>r_{trou}$ pour une épaisseur plus grande.\par 
Ces rapports décrivent de manière numérique nos observations qualitatives réalisées sur les images. Ils permettent de proposer une épaisseur optimale~: on prend le point où 
$r_{tot}=r_{trou}$, soit une épaisseur initiale de la couche de 33~nm.\par 
\begin{figure}[!htb]
\centering
\includegraphics[width=0.6\textwidth]{qualiteStructures}
\caption{Mesure de la qualité des structures obtenues, mesurées par notre algorithme~: $r_{tot}$ (bleu) et $r_{trou}$ exprimés en pourcentage, en fonction de l'épaisseur initiale. Les courbes sont des polynômes de degré 4 ajustés sur les données pour guider l'œil.}
\label{qualiteStructures}
\end{figure}

Un procédure identique a été appliquée aux autres textures. Les épaisseurs optimales déduites sont présentées dans le tableau~\ref{tOptimaux}. Plus la période est grande, plus l'épaisseur optimale est forte. Cette observation est conforme à notre raisonnement en termes de taille de grains.\par 

\begin{table}[!htb]
\centering
\begin{tabular}{cccccc}
\hline
$P$ (nm) & 350 & 350 (hexagonal) & 500 & 600 & 750 \\
\hline
$h_{opti}$ (nm)&  33 & 30 & 44 & 52 & 68\\
\hline
\end{tabular}
\caption{Valeurs des épaisseurs initiales de la couche métallique donnant les meilleurs réseaux de particules après démouillage, en fonction de la texture du substrat.}
\label{tOptimaux}
\end{table}

\paragraph*{Remarque~:} Notre avons supposé que le nombre de grains qui connaissent une croissance extraordinaire est directement proportionnel au nombre de grains total. En termes statistiques, ceci implique que la probabilité pour un grain donné de connaître une croissance extraordinaire est indépendante de la densité initiale de grains.\par  


Si l'on compare nos observations au modèle de Giermann et Thompson~\cite{giermann2005solid, giermann2011requirements}, l'effet apparent de l'épaisseur n'est donc pas si éloigné. En effet, augmenter l'épaisseur initiale de la couche augmente le volume de métal. La quantité de défauts augmente lorsque l'épaisseur initiale dépasse une certaine limite. Cette limite n'est cependant pas liée au volume.\par

	\subsubsection{Influence de la puissance lors du dépôt magnétron}
La puissance électrique fournie à la cathode pendant le dépôt magnétron est un paramètre que l'on peut faire varier (voir chapitre II). Pour toutes les expériences menées jusqu'ici, nous avons employé une puissance de 210~W. Comme la vitesse de dépôt est directement proportionnelle à la puissance, nous avons modifié la vitesse de défilement des échantillons en conséquence pour conserver des épaisseurs initiales constantes. Dans le bâti que nous employons, cette vitesse de défilement est bornée, à la fois par un minimum et par un maximum. La puissance électrique accessible est donc comprise entre 100~W et 600~W.\par 
Des images AFM ont été réalisées sur des couches d'argent de 30~nm d'épaisseur déposées à différentes puissances, sur des surfaces planes. Les images sont présentées sur la figure~\ref{AFMpower}. Nous observons que l'augmentation de la puissance favorise la présence de grains de grande taille. Pour la puissance de 600~W, il est même possible d'observer un domaine cristallin dont la taille avoisine le micron (image~d). La rugosité a été mesurée par AFM, elle est présentée dans le tableau~\ref{tRguoPower}.\par 
%Cette augmentation de taille est liée à la densité de germes lors du dépôt. Lorsque la puissance augmente, les atomes et clusters qui sont éjectés de la cible métallique ont une énergie plus grande et donc une probabilité plus faible de se fixer à la surface de l'échantillon. La densité de germes diminue, le nombre de grains qui en résultent également, ces grains seront donc plus gros. Bien qu'en apparente contradiction avec l'augmentation de la vitesse de dépôt, ce mécanisme a été \par 
\begin{figure}[!htb]
\centering
\includegraphics[width=\textwidth]{AFMpower}
\caption{Cartographies AFM d'une couche d'argent de 30~nm déposée sur une couche de silice plane à différentes puissances a) 100~W, b) 200~W, c) 400~W et d) 600~W.}
\label{AFMpower}
\end{figure}

\begin{table}[!htb]
\centering
\begin{tabular}{ccccc}
\hline
P (W) & 100 & 200 & 400 & 600 \\
\hline
rugosité (nm) & 4,2 & 5,2 & 5,5 & 5,0\\
\hline
\end{tabular}
\caption{Rugosité mesurée sur des couches d'argent de 30~nm, en fonction de la puissance de dépôt.}
\label{tRguoPower}
\end{table}
Nous avons procédé au dépôt de couches de 30~nm sur la texture de période 350~nm, en réseau carré. Les structures obtenues après démouillage (400~$^\circ$C, 2~h) ont été analysées de manière identique que dans la section précédente. Les rapports obtenus sont présentés sur la figure~\ref{qualitePower}. Le rapport $r_{trou}$ est maximum pour 400~W de puissance de dépôt, tandis que le rapport $r_{tot}$ est maximal pour une puissance de 600~W. Par un raisonnement identique à la section précédente, nous déduisons que 400~W (où $r_{tot}=r_{trou}$) est la puissance optimale de dépôt, pour une épaisseur de 30~nm et une texture de réseau carré de période 350~nm.\par 
\begin{figure}[!htb]
\centering
\includegraphics[width=0.6\textwidth]{qualitePower}
\caption{Évolution des rapports $r_{tot}$ et $r_{trou}$ en fonction de la puissance de dépôt, pour une couche d'argent de 30~nm démouillée sur une surface texturée par le réseau carré de 350~nm de période.}
\label{qualitePower}
\end{figure}

\subsubsection{Conclusion sur l'optimisation}
Nous avons expérimenté plusieurs paramètres de dépôt et mesuré leur influence sur la qualité des réseaux de particules obtenus. Les variations de qualité sont plus importantes lorsque l'on fait varier l'épaisseur initiale de la couche que lorsque l'on fait varier la puissance de dépôt. En termes d'optimisation du dépôt, le paramètre le plus déterminant reste donc  cette épaisseur.\par 
Pour être exhaustif, un autre paramètre qui pourrait être modifié est la pression de gaz neutre du plasma du dépôt. Nous n'avons pas procédé à cette étude, mais pour une texture donnée, on pourrait définir un jeu de trois paramètres optimaux~: épaisseur, pression et puissance. Un changement de texture entraîne un changement de ce jeu de paramètres.\par 

	\subsection{Propriétés des particules}
Nous avons caractérisé en détail la qualité des réseaux. Nous étudions maintenant les modifications apportées aux particules individuelles, c'est-à-dire en dehors de leur organisation spatiale.\par 

\subsubsection{Modification de la forme des particules}
Une caractéristique des particules susceptible d'affecter leur réponse optique est leur forme. Nous avons réalisé des clichés MEB \og en tranche \fg{} de particules sur une surface plane et sur une surface texturée. Les images sont présentées dans la figure~\ref{MEBforme}. Sur une surface plane (image a), les particules sont assimilables à des hémisphères facettées. Cette forme est attendue, compte-tenu de la faible anisotropie des facettes de bas indice de l'argent~\cite{stankic2013equilibrium, winterbottom1967equilibrium}. Sur une surface texturée, une particule a été observée dans un trou pyramidal de la texture (image~b). Cette particule présente un sommet hémisphérique qui émerge de la surface et une base pyramidale (ou conique, la projection de profil ne permettant pas de trancher) qui s'insère dans le substrat. Sur l'image, l'angle formé par les pans inclinés de la particule et le plan du horizontal du substrat est de 38$^\circ$. Cet angle correspond à celui de la pyramide inversée obtenue par nano-impression (37$^\circ$). La particule adopte donc la forme du trou dans lequel elle s'est formée.\par
\begin{figure}[!htb]
\centering
\includegraphics[width=0.7\textwidth]{MEBforme}
\caption{Images MEB réalisées en tranche sur a) une couche d'argent de 20~nm démouillée sur une surface plane b) une couche d'argent de 35~nm démouillée sur un réseau de 350~nm de période.}
\label{MEBforme}
\end{figure}
Ce constat diffère des observations de Giermann et Thompson, qui travaillaient avec de l'or. Ils observaient en effet des particules dont la forme était davantage contrainte par des relations d'épitaxie avec les pans inclinés des pyramides que la forme des pyramides elle-même. Notamment, les particules n'épousaient pas le fond de la pyramide (voir chapitre~I). Dans notre cas, l'argent épouse bien toute la structure du trou.\par  

\subsubsection{Modification de la taille des particules}
Une autre caractéristique des particules pouvant modifier leur réponse optique est leur taille. Par la suite, nous assimilons la \og taille \fg{} au diamètre effectif des particules, c'est à dire le diamètre d'un disque présentant la même superficie. Sur une surface plane, Morawiec \textit{et al.}~\cite{morawiec2013self} ont observé que la taille des particules obtenues par démouillage était proportionnelle à l'épaisseur initiale de la couche. En d'autres termes, pour une épaisseur donnée correspond une taille de particules.\par 
Nous avons mesuré les distributions en taille des particules obtenues, pour une même épaisseur initiale, sur une surface plane et sur les surfaces texturées par les réseaux de période 350~nm (carré et hexagonal). Elles sont présentées sur la figure~\ref{taillePartComp}. Nous observons que la distribution des particules obtenues sur une surface plane est large et centrée autour d'une valeur moyenne de 314~nm, avec une largeur à mi-hauteur de 118~nm. Sur les surfaces texturées (respectivement réseau carré et hexagonal), les valeurs moyennes des tailles de particules sont abaissées à 250 et 235~nm, tandis que les largeurs à mi-hauteur sont de 50 et 42~nm. La présence du réseau a un impact fort sur la distribution en taille des particules, en diminuant à la fois la valeur moyenne mais aussi la largeur de la distribution.\par 
Comparée à une surface plane, la diminution de la taille moyenne s'explique par l'augmentation de la densité de particules, imposée par le substrat. En effet, à volume total constant, augmenter le nombre de particules équivaut à diminuer leur taille. C'est pour cette raison que lorsque l'on considère le réseau hexagonal, plus dense, la taille moyenne est légèrement diminuée par rapport au réseau carré. La diminution de la largeur de la distribution est quant à elle reliée au fait que les particules, étant régulièrement espacées, ont toutes une zone d'influence (définie dans le chapitre~III) comparable, c'est-à-dire qu'elles accumulent toutes une quantité de matière équivalente. Ceci limite les variations de taille des particules observées. \par 
\begin{figure}[!htb]
\centering
\includegraphics[width=0.6\textwidth]{taillePartComp}
\caption{Distribution en taille des particules pour une couche d'argent de 30~nm démouillée sur une surface plane, texturée par un réseau carré (période 350~nm) ou un réseau hexagonal (période 350~nm).}
\label{taillePartComp}
\end{figure}

\subsection{Conclusion}
Le procédé de nano-impression est compatible avec le démouillage sur surfaces texturées proposé par Giermann et Thompson~\cite{giermann2005solid}. Ceci est vrai pour des réseaux de pyramides inversées dont la période varie entre 200 et 1000~nm. Pour chaque texture, il est possible d'optimiser le dépôt initial de la couche métallique afin d'obtenir, après démouillage, des réseaux de particules de bonne qualité. Cette optimisation est principalement fondée sur l'épaisseur initiale de la couche, mais aussi sur la puissance électrique fournie à la cathode pendant la pulvérisation magnétron, ainsi que la pression de gaz employée pour créer le plasma de dépôt. Enfin, comparées à des particules obtenues sur une surface plane, les particules obtenues sur surfaces texturées ont de nouvelles caractéristiques~: leur morphologie adopte celle des trous dans lesquels elles se sont formées, elles sont pour partie immergées dans le substrat, leur taille est réduite et plus régulière et elles sont maintenant organisées en réseau.\par 

\section[Propriétés optiques des réseaux de particules ]{Propriétés optiques des réseaux de particules obtenues par démouillage}
L'intérêt de la nano-impression est la possibilité de procéder au démouillage sur des surfaces texturées avec des substrats optiquement transparents. Nous avons réalisé nos mesures sur une verre Planilux Saint-Gobain, transparent dans une large gamme ($>$350~nm). Nous pouvons donc mesurer les propriétés optiques de réflexion (R), de transmission (T) et d'absorption (A). Nous présentons dans cette section les réponses optiques macroscopiques des structures métalliques obtenues par démouillage.\par 

\subsection{Mesures expérimentales}
\paragraph*{Remarque préliminaire~:} L'absorption (A) que nous présentons dans les résultats expérimentaux est déduite des mesures de transmission (T) et de réflexion (R), voir chapitre II. Ce n'est pas une mesure directe. Par ailleurs, la dispositif employé introduit un artéfact de mesure systématiques à une longueur d'onde de 840~nm~; les grandes oscillations brusques autour de cette longueur d'onde n'ont pas de signification physique.\par 

	\subsubsection{Substrat texturé, sans argent}
Avant de caractériser les structures obtenues par démouillage (contenant de l'argent), nous avons contrôlé l'impact sur la réponse optique de la texture seule. Nous avons mesuré la transmission et la réflextion pour une couche sol-gel plane et pour une couche sol-gel texturée par le réseau de période 600~nm. Les spectres de T et R, ainsi que de l'absorption déduite (A = 1-R-T) pour ces deux substrat sont présentés sur la figure~\ref{optiqueSolGelNu}.\par 
\begin{figure}[!htb]
\centering
\includegraphics[width=\textwidth]{optiqueSolGelNu}
\caption{Spectres de transmission (T), de réflexion (R) mesurés pour une surface sol-gel plane ou texturée par un réseau de période 600~nm. Spectre d'absorption déduite (A = 1-R-T) pour ces mêmes surfaces. La ligne pointillée indique la limite d'absorption du substrat de verre dans l'UV.}
\label{optiqueSolGelNu}
\end{figure}
À part pour des longueurs d'onde inférieures à 350~nm, T et R sont relativement constantes. T a une valeur de 90~\%, et R une valeur de 8~\%. On ne remarque qu'une différence faible entre les deux échantillons sur T entre 400 et 800~nm~: un abaissement de quelques points pour la surface texturée. Cet abaissement se traduit par une hausse équivalente de A au même endroit.\par 
Le verre et la silice ayant un indice optique proche de 1,44 (ceci peut varier pour une couche sol-gel, en fonction de sa densité), il existe une réflexion à l'interface air/verre. C'est cette réflexion que l'on observe, elle correspond à une valeur de R de 8~\%. Cet effet est observable sur les deux échantillons. En revanche, la différence de transmission observée entre 400 et 800~nm est inattendue. En effet, même si un réseau de 600~nm diffracte la lumière dans le visible, le fait de travailler en sphère intégrante devrait nous affranchir d'éventuels effets liés à ce phénomène. Il existe cependant un petit interstice entre l'échantillon et l'entrée de la sphère intégrante lors des mesures de T~; il possible qu'en transmission, un ordre de diffraction en émergence rasante avec la surface de l'échantillon soit partiellement perdu avant d'entrer dans la sphère. Ce mode serait attendu pour $\lambda = n_{silice}P$ (et pour des longueurs d'onde légèrement inférieures, suivant la largeur de l'interstice).\par 
Cette différence ne représentant qu'une perte de 4~points pour une transmission de 90~\%, nous considérons qu'elle est négligeable. Nous pouvons en conclure que la texturation n'a pas d'incidence sur les mesures optiques.\par 

	\subsubsection{Comparaison : particules d'argent obtenues sur surfaces planes et texturées}
Observons maintenant les structures d'argent démouillées. Nous comparons les réponses optiques d'une couche de 30~nm démouillée sur une surface plane ou une surface texturée à 350~nm. Les résultats sont présentés dans la figure~\ref{optiqueCompareTextFlat} de la page~\pageref{optiqueCompareTextFlat}. La transmission mesurée de la couche démouillée sur une surface plane présente deux pics à des longueurs d'onde de 360 et 520~nm. La transmission relative à la surface texturée a une allure très semblable, mais avec une valeur plus petite. La réflexion relative à la surface plane présente un pic large à 775~nm et un épaulement aux alentours de 400~nm. La réflexion relative à la surface texturée est globalement plus importante, mis à part une forte diminution centrée sur une longueur d'onde de 530 nm.\par 
Pour les particules obtenues par démouillage sur une surface plane, l'absorption optique présente un unique pic centré à une longueur d'onde de 420~nm. Pour les particules obtenues sur la surface texturée, on retrouve le même pic, plus un nouveau pic centré autour de 530~nm. Ce pic provient directement de l'abaissement de la réflexion observé un peu plus tôt.\par 

\begin{figure}[!p]
\centering
\includegraphics[width=0.8\textwidth]{optiqueCompareTextFlat}
\caption{Spectres de transmission (T), de réflexion (R) mesurés pour une couche d'argent de 30~nm démouillée sur une surface sol-gel plane ou texturée par un réseau carré de période 350~nm. Spectre d'absorption déduit (A = 1-R-T) pour ces mêmes surfaces. La ligne pointillée indique la limite d'absorption du substrat de verre dans l'UV.}
\label{optiqueCompareTextFlat}
\end{figure}
Nous avons déterminé en étudiant une couche sol-gel seule que l'introduction d'un réseau périodique ne perturbe pas la mesure, \textit{via} la diffraction. Cette hausse d'absorption très nette autour de 530~nm a donc une origine physique et ne peut pas être imputé à un artéfact de mesure. De manière générique, nous appelons \og premier pic \fg{} le pic situé à 420~nm et \og second pic \fg{} selui situé à 530~nm (nous proposerons plus tard dans ce manuscrit une dénomination plus adéquate). Le premier pic, étant présent de la même manière pour des particules désorganisées (sur surfaces planes) que pour des particules organisées (sur surfaces texturées), n'est donc pas spécifiquement dû à l'organisation spatiale des particules. Le second pic, en revanche peut provenir de l'organisation, mais aussi des autres changements apportés dans le système, comme la forme des particules ou leur taille. Par la suite, nous allons étudier principalement la réponse en absorption des réseaux de particules obtenus. \par 
   
	\subsubsection{Influence de la qualité du réseau}
Nous considérons toujours les réseaux de particules de période 350~nm. Nous avons montré dans la section~\ref{sQuality} comment optimiser le dépôt de la couche initiale afin d'obtenir par démouillage des réseaux de bonne qualité. Des mesures optiques ont été réalisées pour ces mêmes échantillons. Les mesures concernant la variation en épaisseur sont présentées sur la figure~\ref{optiqueCompareQuality} de la page~\pageref{optiqueCompareQuality}. Considérons tout d'abord les épaisseurs initiales inférieures à 35~nm (spectres de gauche). Nous observons très peu de variations dans le premier pic d'absorption (centré à 420~nm). En revanche, l'allure du second pic est dépendante de l'épaisseur initiale~: ce pic est plus fin et plus intense à mesure que l'épaisseur initiale approche de 35~nm. Considérons maintenant les épaisseurs plus grandes (supérieures à 40~nm, spectres de droite). Lorsque l'épaisseur initiale est de 40~nm, le spectre est très semblable à celui obtenu pour une épaisseur de 35~nm. En revanche, si l'on porte l'épaisseur à 45~nm, le premier pic perd en intensité. À 55~nm, les deux pics perdent en intensité. À 75~nm, ils sont encore discernables, mais larges et peu intenses.\par  
Nous pouvons faire un lien direct entre l'allure du second pic et la qualité du réseau. En effet, les épaisseurs initiales comprises entre 30 et 40~nm sont celles qui donnent les réseaux de meilleure qualité après démouillage. C'est sur ces réseaux que le second pic est le plus fin et le plus intense. À l'opposé, si la qualité du réseau baisse, le second pic perd en intensité et en finesse. Le premier pic subit également une perte en intensité lorsque l'épaisseur initiale est trop importante (à partir de 45~nm). Nous avions observé que les particules avaient de plus en plus de mal à se déconnecter pour de telles épaisseurs initiales. Ces observations sont cohérentes avec notre hypothèse~: le second pic dépend de l'adéquation des particules avec la texture (organisation, taille ou forme), tandis que le premier pic dépend des particules indépendamment les unes des autres.\par 
\begin{figure}[!p]
\centering
\includegraphics[width=\textwidth]{optiqueCompareQuality}
\caption{Spectres d'absorption déduits pour des réseaux de particules obtenus par démouillage sur une surface texturée par un réseau carré de période 350~nm, pour différentes épaisseurs initiales de couche. Gauche~: de 20 à 35~nm. Droite~: de 40 à 75~nm.}
\label{optiqueCompareQuality}
\end{figure}
Dans la section~\ref{sQuality}, nous avons également étudié l'influence de la puissance électrique fournie pendant le dépôt sur la qualité du réseau de particules obtenus après démouillage. Les spectres d'absorption associés sont présentés sur la figure~\ref{optiqueComparePower}. Le premier pic connaît de petites variations en intensité suivant la puissance employée pendant le dépôt. Le second pic également, mais nous notons une augmentation de son intensité et une réduction de sa largeur pour les puissances de 400 et 600~W. Ces puissances sont précisément celles qui donnaient lieu à des réseaux de meilleure qualité, après démouillage. Ces observations sont tout à fait conformes aà nos conclusions précédentes~: la qualité du réseau a une grande influence sur la forme du second pic d'absorption.\par  
\begin{figure}[!htb]
\centering
\includegraphics[width=0.6\textwidth]{optiqueComparePower}
\caption{Spectres d'absorption déduits pour des réseaux de particules obtenues par démouillage sur une surface texturée par un réseau carré de période 350~nm, pour différentes puissances fournies pendant le dépôt d'une couche de 30~nm.}
\label{optiqueComparePower}
\end{figure}
En termes de contrôle du démouillage, il apparaît donc que l'optimisation du dépôt en amont du démouillage a une forte influence sur la réponse optique du réseau de particules obtenu. Il y a une très forte corrélation entre la qualité mesurée sur les images MEB et l'intensité et la finesse du second pic observé dans les spectres d'absorption.\par 
	\subsubsection{Changement de texture}
Nous avons jusqu'ici étudié la réponse optique de réseaux de particules obtenus par démouillage sur les surfaces texturées par réseaux carrés de période 350~nm. Nous disposons de réseaux d'autres périodes, dont nous avons mesuré la réponse optique. Les spectres d'absorption associés sont présentés sur la figure~\ref{optiqueComparePeriods}. Les épaisseurs déposées pour les différentes textures sont (par ordre croissant de périodes)~: 20, 35, 45, 50 et 70~nm. Ces épaisseurs correspondent, à l'incertitude près, à l'épaisseur optimale pour chaque texture. Concernant le premier pic d'absorption, il est toujours positionné au même endroit, mais son intensité varie d'un réseau à l'autre. Le second pic d'absorption se déplace, sa position est marquée sur les spectres par la ligne en pointillés. Lorsqu'il se déplace vers les grandes longueurs d'onde, l'absorption entre les deux pics reste à une valeur comprise entre 20 et 30~\%, elle est relativement constante.\par
Nous avons tracé sur la figure~\ref{posVSperiod} les positions du second pic en fonction de la période du réseau. Une régression linéaire effectuée sur ces points donne une pente de 1,44.\par 
\begin{figure}[!htb]
\centering
\includegraphics[width=\textwidth]{optiqueComparePeriods}
\caption{Spectres d'absorption mesurés sur des réseaux de particules obtenus par démouillage sur des surfaces texturées pour différentes périodes. Les épaisseurs initiales des couches sont dépendantes de la période du réseau (voir texte).}
\label{optiqueComparePeriods}
\end{figure}
\begin{figure}[!htb]
\centering
\includegraphics[width=0.6\textwidth]{posVSperiod}
\caption{Position du second pic d'absorption (longueur d'onde en~nm) en fonction de la période du réseau de particules d'argent. La droite est une régression linéaire passant par l'origine.}
\label{posVSperiod}
\end{figure}
Cette dépendance linéaire entre la position du second pic d'absorption et la période du réseau de particules d'argent laisse supposer que c'est bien la période du réseau qui détermine la position du pic et la qualité du réseau qui détermine son intensité et sa finesse. Cependant, tous nos réseaux sont obtenus par démouillage, ce qui lie intrinsèquement la période du réseau aux autres paramètres~: il n'est pas possible d'obtenir, par exemple, un réseau de grande période avec de petites particules. Nous avions constaté dans la chapitre I que ces paramètres (taille et forme) pouvaient modifier la réponse optique des particules. En l'état actuel, les expériences de démouillage ne nous permettent pas de trancher.\par 

Enfin, nous avons réalisé les mesures optiques pour le réseau hexagonal de période 350~nm. Le spectre d'absorption est présenté sur la figure~\ref{optiqueCompareGeometry}, ainsi que les spectres relatifs au réseau carré de même période et aux particules obtenues sur une surface plane, pour les mêmes épaisseurs initiales (30~nm). Nous observons que le premier pic est toujours à la même position (420~nm), comme attendu. En revanche, le second pic s'est déplacé à 490~nm. Ce déplacement, bien que petit, a été observé systématiquement sur des expériences de contrôle~; il s'agit bien d'un différence d'absorption.\par 
Les réseaux carrés et hexagonaux ont des paramètres géométriques très proches~: la période, la taille et la forme des particules. Ceci semble indiquer que l'organisation est effectivement responsable de ce deuxième pic d'abosrption. La raison physique expliquant ce phénomène reste encore à identifier.\par 

\begin{figure}[!htb]
\centering
\includegraphics[width=0.6\textwidth]{optiqueCompareGeometry}
\caption{Spectres d'absorption déduits pour des particules d'argent obtenues par démouillage sur une surface plane, une surface texturée en réseau carré (P = 350~nm) ou un réseau hexagonal (P = 350~nm). L'épaisseur initiale est de 30~nm.}
\label{optiqueCompareGeometry}
\end{figure}

\subsection{Simulations par éléments finis}
Nous avons donc pris le parti de compléter notre étude expérimentale par des simulations numériques. En effet, il est possible avec ces simulations de faire varier tous les paramètres (période du réseau, taille et forme des particules) de manière indépendante, afin d'estimer les effets relatifs de ces paramètres. Les simulations, de types \og éléments finis \fg, ont été réalisées par Alexandre Baron (Centre de Recherche Paul Pascal) à l'aide du logiciel COMSOL. Nous présentons ici les résultats obtenus.\par 
	\subsubsection{Variation de la forme}
Nous comparons ici les spectres d'absorptions calculés pour deux systèmes, l'un avec des particules sphériques et l'autre avec des particules à base conique et sommet en forme de calotte hémisphérique (que nous désignerons simplement par \og particules coniques \fg). Le schéma~\ref{schemaBoiteSimul} permet de visualiser les différents paramètres employés dans la simulation. Nous avons choisi $P$ = 350~nm, $d$ = 150~nm, $D$ = 250~nm pour correspondre à un réseau de 350~nm. Il est important de noter que du fait de conditions périodiques aux bords, nous étudions bien un réseau de particules.\par 
Les résultats sont représentés sur la figure~\ref{simulCompForme}. Nous ne commenterons pas les valeurs de l'absorption en-dessous de 350 nm, car ces valeurs correspondent au substrat de verre. Pour les particules sphériques, on observe (par ordre de longueur d'onde croissante) deux pics proches à 411 et 441~nm, puis deux pics proches à 525 et 579~nm. Pour les particules coniques, on observe un pic à 420~nm et un pic à 540~nm.\par 
Lorsque les particules sont sphériques, on observe donc davantage de pics. Cependant, ceux-ci sont très proches~; lorsque l'on considère des particules coniques, deux pics proches sont remplacés par un unique pic à une valeur intermédiaire. La forme des particules influence la forme du spectre d'absorption en décalant les pics, ce qui donne lieu à leur superposition ou au contraire leur éclatement. Ces changements, cependant, sont limités à moins de 50 nm pour les cas étudiés, et l'allure globale du spectre est conservée. Nous remarquons, en particulier, que la présence du second pic (ou des seconds pics) aux alentours de 540 nm est observée, quelle que soit la forme des particules~: ce n'est pas le changement de forme qui induit ce pic.\par 
\begin{figure}[!htb]
\centering
\includegraphics[width=0.6\textwidth]{schemaBoiteSimul}
\caption{Schéma des boîtes de simulation employées dans les calculs COMSOL.}
\label{schemaBoiteSimul}
\end{figure}
\begin{figure}[!htb]
\centering
\includegraphics[width=0.6\textwidth]{simulCompForme}
\caption{Absorption simulée pour un réseau de particules sphériques ou coniques. Les paramètres sont indiqués dans le texte.}
\label{simulCompForme}
\end{figure}
Sur la figure~\ref{simulCompForme}, nous avons comparé les résultats de simulation des particules coniques avec nos résultats expérimentaux. On observe un bon accord entre la simulation et l'expérience, ce qui conforte bien la validité des simulations. La simulation a toutefois tendance à donner des pics plus fins, ce qui sans doute dû au fait que le système simulé est idéal, contrairement à nos réseaux expérimentaux qui ont une certaine polydispersité en taille et forme de particules.\par 
\begin{figure}[!htb]
\centering
\includegraphics[width=0.6\textwidth]{simulCompExp}
\caption{Spectres d'absorption : simulé pour des particules coniques (en rouge) ou obtenu expérimentalement (en bleu), pour un réseau de période 350~nm et des particules d'argent d'une taille moyenne D = 250~nm.}
\label{simulCompForme}
\end{figure}
	\subsubsection{Variation de la période}
\paragraph*{Remarque :} le temps de calcul aux petites longueurs d'onde dans les grandes périodes augmente rapidement. Pour cette raison, nous avons limité la gamme de longueurs d'onde calculées pour les périodes les plus grandes.\par \vspace{12pt}

Des simulations ont été effectuées en changeant uniquement la taille de la boîte de simulation (P) ; les spectres d'absorption obtenus sont présentés sur la figure~\ref{simulCompPeriode}. Nous avons gardé tous les autres paramètres constants (particules coniques, D = 250~nm, d = 150~nm). Considérons en premier lieu uniquement le spectre relatif à la période de 500~nm. On observe un pic à 420~nm, un pic à 570~nm et un pic à 770~nm. Le premier pic (420~nm) n'a donc pas bougé lorsque l'on a augmenté la période. Suite à notre remarque, nous ne pourrons pas vérifier sur les simulations à plus grande période s'il s'est déplacé ou non. Nous l'associons cependant au \og premier pic \fg{} observé sur les spectres expérimentaux. Lorsque la période augmente (600 et 750~nm), les deux autres pics se déplacent vers le rouge. Il semble qu'un nouveau signal émerge pour des longueurs d'onde plus faibles, mais la simulation est limitée dans cette gamme pour les grandes périodes.\par 
\begin{figure}[!htb]
\centering
\includegraphics[width=0.6\textwidth]{simulCompPeriode}
\caption{Spectres d'absorption simulés pour des particules coniques de 250 nm de diamètre et une profondeur de 150 nm, en fonction de la période du réseau.}
\label{simulCompPeriode}
\end{figure}
En résumé, nous observons principalement trois pics dans l'absorption simulée des réseaux de particules d'argent~:
\begin{itemize}
\item le \og premier pic \fg, à une longueur d'onde de 420~nm, indépendant de la période~;
\item le \og second pic \fg, dont la position est la plus décalée vers les grandes longueurs d'onde, et qui se déplace lorsque la période varie~;
\item le \og troisième pic \fg, situé entre les deux précédents, qui se déplace également lorsque la période varie. 
\end{itemize}
Le fait d'observer un pic supplémentaire (à 570~nm pour P = 500~nm) est inattendu. Si le pic le plus décalé vers le rouge peut correspondre au \og second pic \fg{} des spectres expérimentaux, le \og troisième pic \fg{} est une nouveauté. Cependant, nous avions remarqué expérimentalement que l'absorption entre le premier et le deuxième pic restait à un niveau de l'ordre de 20 à 30~\%. Ce troisième pic est donc possiblement inclus dans le signal, mais pas suffisamment résolu pour être observé expérimentalement. Remarquons de plus qu'il semble se décomposer en deux pics pour les plus grandes périodes.\par
Nous avons tracé sur la figure~\ref{simulPicPos} les positions du deuxième et du troisième pic en fonction de la période du réseau. Nous observons pour chaque pic une dépendance linéaire entre sa position et la période. Les pentes obtenues par régression sont respectivement 1,52 et 1,06. Pour rappel, la pente expérimentale obtenue pour le second pic était 1,44.\par 
\begin{figure}[!htb]
\centering
\includegraphics[width=0.6\textwidth]{simulPicPos}
\caption{Position des pics d'absorption des spectres simulés en fonction de la période du réseau.}
\label{simulPicPos}
\end{figure}

COMSOL nous permet de visualiser l'intensité du champ \textbf{E} dans la boîte de simulation aux longueurs d'onde de résonance. Les cartographies sont présentées sur la figure~\ref{simulModes}. Dans l'ordre d'attribution des pics : a) le premier mode (420 nm) présente une exaltation du champ tout autour de la particule, dans la silice comme dans l'air. Nous l'appellerons « mode hybride »~; b) Le second mode (que nous attribuons au second pic) présente une exaltation uniquement dans le substrat, nous l'appelleront \og mode du substrat \fg, tandis que les modes c) et d) qui constituent le troisième pic présentent au contraire une exaltation dans l'air et une exaltation hybride. Nous les regroupons sous le terme \og mode de l'air \fg.\par
\begin{figure}[!htb]
\centering
\includegraphics[width=0.8\textwidth]{simulModes}
\caption{Cartographie simulée de l'intensité du champ \textbf{E} aux longueurs d'onde correspondant aux pics a) au premier pic, à 420 nm, pour la période de 350~nm b) au second pic pour la période de 600~nm et c) et d) au troisième pic, pour la période de 600~nm.}
\label{simulModes}
\end{figure}

\paragraph*{Remarque :} Les cartographie de champ (Figure~\ref{simulModes}) présentent des similitudes à celles obtenues pour des sphères tronquées supportées sur un substrat dans le cadre de l'approximation quasi-statique~\cite{Lazzari02d, lazzari03, lazzari14b}. Les modes d'absorption alors obtenus correspondent à des modes propres de polarisation de la charge sur les particules qui sont activés suivant la direction du champ. Ils sont reproduits sur la figure~\ref{lazzariModes}.\par 
\begin{figure}[!htb]
\centering
\includegraphics[width=0.25\textwidth]{lazzariMode1}\includegraphics[width=0.25\textwidth]{lazzariMode2}
\caption{Cartographie simulée des potentiels associés aux vibrations des charges dans les modes propres de particules supportées, pour un champ électrique parallèle à la surface. Issu de~\cite{lazzari14b}}
\label{lazzariModes}
\end{figure}
	\subsubsection{Changement de métal : cas de l'or}
La simulation permet de changer de métal en remplaçant simplement la fonction diélectrique renseignée. Nous avons exécuté les mêmes calculs avec des particules d'or. Notons que procéder au démouillage de l'or sur un substrat de verre est difficilement réalisable en pratique~: les températures de recuit nécessaires au démouillage dans un délai raisonnable dépassent 650~$^\circ$C, la température de transition vitreuse du verre. Un tel recuit peut alors déformer le substrat.\par 
Le spectre d'absorption calculé est présenté sur la figure~\ref{simulCompMetal}. Concernant le spectre relatif à l'or, nous observons une absorption élevée entre 400 et 500~nm, un pic double à 520 et 570~nm, puis une décroissance de l'absorption. Un nouveau pic est observé à 770~nm. Par comparaison au spectre obtenu avec la fonction diélectrique de l'argent, nous pouvons identifier le double pic comme étant relatif au mode de l'air, et le pic à 770~nm au mode du substrat. Ces pics sont positionnés à la même longueur d'onde, ils sont donc indépendants du matériau décrit pour la simulation. En revanche, la différence d'absorption aux petites longueurs d'onde est une signature des différents métaux. \par  
\begin{figure}[!htb]
\centering
\includegraphics[width=0.7\textwidth]{simulCompMetal}
\caption{Spectres d'absorption simulés pour un réseau de particules de période 500~nm. La taille (D) des particules est 250~nm, la profondeur (p) est 150~nm. Deux fonctions diélectriques ont été employées~: l'argent (en bleu) et l'or (en rouge).}
\label{simulCompMetal}
\end{figure}

	\subsubsection{Variation de la taille}
Enfin, nous avions souligné dans le chapitre I que la taille des particules pouvait avoir une influence sur leur réponse optique. Nous avons réalisé deux simulations pour un réseau de particules de période 600~nm. Deux tailles (D) on été comparées, 250 et 400~nm. Les spectres d'absorption simulés sont présentés sur la figure~\ref{simulCompTaille}. La spectre d'absorption simulé pour $D = 400$~nm présente des pics aux mêmes positions que celui simulé pour $D = 250$~nm. Leur intensité est plus forte. La taille ne modifie donc pas la position des pics relatifs aux modes du substrat et de l'air, mais augmente leur intensité. Cette augmentation traduit sans doute l'augmentation de quantité de métal susceptible d'absorber. Le mode hybride semble affecté de la même manière (position identique mais augmentation de l'intensité avec la taille).\par 

\begin{figure}[!htb]
\centering
\includegraphics[width=0.7\textwidth]{simulCompTaille}
\caption{Spectres d'absorption simulés pour un réseau de particules d'argent de période 600~nm. La taille (D) des particules est 250~nm (en bleu) ou 400~nm (en rouge).}
\label{simulCompTaille}
\end{figure}

\subsection{Discussion}
Forts des observations expérimentales et des résultats de simulation, nous cherchons à identifier l'origine physique des différents pics d'absorption.\par 
Les particules d'argent obtenues par démouillage sur une surface texturée sont organisées en réseau (imposé par le substrat). Les mesures de la réponse optique ont mis en évidence deux pics d'absorption. Le premier est présent quelle que soit la texture (différentes périodes, voire surface plane), tandis que le second n'est observé que sur les textures en réseau. Il se déplace en fonction de la période du réseau avec une relation linéaire de pente 1,44.\par 
Les simulations de la réponse optique de tels réseaux par éléments finis fournissent des spectres d'absorption très proches de nos observations expérimentales. Elles nous permettent de plus de supposer l'existence d'un troisième pic, sans doute pas assez prononcé pour être observé expérimentalement. Nous avons identifié ces différents pics comme étant reliés à des modes différents, que nous appelons respectivement \og hybride \fg, \og du substrat \fg{} et \og de l'air \fg. Nous allons discuter des différents modes séparément.\par 
Nous avons observé expérimentalement que le mode hybride est indépendant de la période du réseau et n'est que peu modifié par la forme ou la taille des particules. Ce mode semble équivalent à la LSPR (résonance du plasmon de surface localisé) d'une particule d'argent supportée sur de la silice, observée par Tanyeli \textit{et al.}~\cite{tanyeli2013effect} (à 450~nm pour des particules d'une taille de 150~nm) ou Royer \textit{et al.}~\cite{royer1987substrate} (à 400~nm pour une taille de 50~nm). Le fait que ce mode soit hybride semble indiquer qu'il provient bien d'une interaction avec le milieu environnant (ce qui est le cas de LSPR). On se serait cependant attendu, selon la dépendance linéaire prévue par Tanyeli \textit{et al.} (voir figure~\ref{tanyeliSubstrateResonance} de la page~\pageref{tanyeliSubstrateResonance}) à l'observer à une plus grande longueur d'onde. D'un autre côté, dans le cadre de l'approximation quasi-statique, la polarisabilité d'une particule présente une résonance à une longueur d'onde indépendante de sa taille (mais l'intensité en dépend). Nos particules étant trop grosses pour rentrer dans le cadre de cette approximation, il convient d'être prudent quant à l'interprétation de ce résultat. L'augmentation en taille apporte notamment de la complexité avec des résonances multipolaires. Ceci pourrait notamment être la raison de l'absorption non nulle pour les plus grandes particules à de plus grandes longueurs d'onde que 420~nm. Nous supposerons en conclusion que le mode hybride est bien dû à une LSPR des particules d'argent supportées, mais la détermination exacte de sa nature nous reste à l'heure actuelle inaccessible car elle nécessiterait des études en polarisation.\par 
Les deux autres modes (mode du substrat et mode de l'air) se comportent de manière semblable. N'étant présents que lorsque les particules sont organisées en réseau, ils sont donc dépendants des interactions entre particules. Leur position est principalement influencée par la période, avec une relation linéaire de pente 1,52 et 1,06 selon la simulation, ou 1,44 selon l'expérience. 
Nous avons vu dans le chapitre I que deux modèles peuvent décrire ce comportement~: un couplage de la lumière aux particules, inaccessible sans diffraction~\cite{kravets2008extremely}, ou un effet de couplage d'une particule au réseau, exprimé par la sommes des dipôles retardés~\cite{zou2004narrow}. La dépendance linéaire pointe dans la direction du premier modèle.\par 
Ce modèle prévoit deux résonances, pour un réseau de particules à l'interface entre deux milieux, en incidence normale : à $\lambda_{air} = P/m$ et $\lambda_{sub} = n_{sub}P/m$, avec $m$ entier. Nos données expérimentales montrent que le pic relatif au mode du substrat se déplace selon une loi $\lambda = 1,44P $ (ou $\lambda = 1, 52P$ selon les simulations). Cette pente correspond exactement à l'indice typique de la silice sol-gel (mesuré par ellipsométrie). La loi est donc parfaitement respectée. Nous pouvons interpréter le pic d'absorption observé comme étant dû à un mode de diffraction présent en émergence rasante. Ce mode de diffraction a donc un chemin optique très grand dans la couche de particules, augmentant donc l'absorption qu'il subit. Remarquons de plus que le champ \textbf{E} est particulièrement exalté dans la silice, ce qui correspond effectivement à un mode de diffraction en émergence rasante à une longueur d'onde de 1,44P.\par 
L'équation $\lambda_{sub} = n_{sub}P/m$ est en réalité une simplification, puisque le réseau est à deux dimensions. Il conviendrait plutôt de considérer les distances entre particules dans toutes les directions en écrivant la loi de Bragg comme $\lambda_{sub} = n_{sub}P/\sqrt{m_1^2+m_2^2}$. Notamment, pour $m_1=m_2=1$, on trouve un mode diffracté en émergence rasante pour $\lambda = n_{sub}P/\sqrt{2}$. Numériquement, pour $n_{sub}=1,44$ (indice expérimental), on obtient la loi $\lambda = 1,02 P/$, et pour $n_{sub}=1,52$ (indice retrouvé par la simulation), on obtient la loi $\lambda = 1,07 P/$.\par 
Cette discussion permet d'interpréter la complexité du pic identifié comme mode de l'air. Nous avons souligné le fait que ce pic était de forme plus complexe, et la position du maximum est linéaire avec la période selon une pente de 1,06. Il s'agirait en réalité de deux pics bien disctincts. L'un est effectivement un mode dû au réseau dans l'air, tandis que le second est un mode dans la silice pour des particules distantes de $P\sqrt{2}$. La cartographie du champ \textbf{E}, réalisée sur les deux composantes pour $P = 600$~nm (figure~\ref{simulModes}), confirme cette hypothèse. Le mode de l'air est clairement identifiable pour $\lambda = 620$~nm (cartographie c), tandis que le mode de la silice, légèrement superposé au mode précédent est visible pour $\lambda = 645$~nm. Ainsi, le pic que nous identifions comme \og mode de l'air \fg{} est en réalité la superposition du véritable mode de l'air et d'un mode de silice issu d'un ordre de diffraction supérieur. Remarquons que ceci contribue d'autant plus à avoir une absorption non nulle entre le mode hybride et le mode de la silice.\par 

Enfin, nous avions observé un déplacement du pic d'absorption lié au mode de la silice dans le spectre mesuré sur un réseau hexagonal de particules, par comparaison avec la mesure effectuée sur un réseau carré (voir figure~\ref{optiqueCompareGeometry} de la page~\pageref{optiqueCompareGeometry}). Pour $P = 350$~nm et le réseau carré, il est positionné à 530~nm. Pour le réseau hexagonal (et $P=350$~nm), ce pic est déplacé à 490~nm. Or, la diffraction pour un réseau hexagonal ediffère légèrement de celle d'un réseau carré. Les rangées de particules sont espacées de $P\sqrt{3}/2$. Ainsi, le premier ordre diffracté en émergence rasante se trouve à une position théorique de 493~nm~: cela est conforme à nos observations.\par 

Ce modèle rend bien compte du changement de période, mais aussi de géométrie. Il explique les déplacements des différents pics d'absorption, aussi bien pour les données simulées que pour les mesures expérimentales.\par 

\paragraph*{Remarque~:} Nos mesures en sphère intégrante incluent une erreur systématique~: la transmission est bel et bien mesurée en incidence normale, mais la réflexion à un angle de $\theta$ = 8$^\circ$. Déduire directement l'absorption de ces deux grandeurs peut donc mener à des erreurs. Notamment pour les modes relatifs à la diffraction. Dans ce cas précis, considérons les premiers ordres de diffraction en émergence rasante produits par un réseau carré. Il y en a quatre, que l'on pourrait repérer par les indices <$10$>, <$01$>, <$\bar{1}0$> et <$0\bar{1}$>. L'inclinaison à 8$^\circ$ est uniquement horizontale dans la sphère, et nous alignons nos échantillons de manière à avoir les directions du réseaux horizontales et verticales~; l'inclinaison n'affecte ainsi que deux ordres. Les deux autres ordres de diffraction correspondent toujours à une longueur d'onde définie par $\lambda = n_iP$. Les deux ordres affectés, selon la loi de Bragg, sont présents pour des longueurs d'onde $\lambda_\pm = n_iP(1\pm\sin\theta)$. Au lieu d'avoir un seul pic, on obtient alors trois pics d'intensités relatives {1:2:1}. Celui du milieu correspond à la mesure obtenue pour une incidence normale, les autres sont décalés de $\pm\delta\lambda = n_iP\sin\theta$. Cette quantité représente 50~nm pour la période de 350~nm, pour le mode relatif à la silice (le \og second pic \fg)~; ces pics se recouvrent et donnent lieu à un unique pic large.\par
La position du pic n'est pas affecté, mais seulement sa largeur. C'est pour cette raison que nous observons malgré tout une bonne correspondance entre nos expériences et les simulations. La largeur des pics expérimentaux peut en partie s'expliquer par cette légère inclinaison imposée par la sphère intégrante.\par 
 

\section{Conclusion}
Nous avons étudié la possibilité d'organiser des réseaux de particules d'argent sur des couches de silice sol-gel texturées par démouillage. La compréhension du démouillage que nous avons développé sur les surfaces planes nous a permis de comprendre comment le rôle des grains était primordial dans la fabrication du réseau de particules. Contrairement à ce qu'avaient observé Giermann et Thompson dans le cas de l'or, le volume de métal de la couche d'argent peut être plus important que le volume des trous de la texture et malgré tout donner lieu à un réseau organisé de particules. La condition d'obtention d'un réseau idéal est d'avoir un grain (et un seul) en croissance au sein d'un motif de la texture. \par 
Cette conception nous a permis de proposer une optimisation des conditions de dépôt de la couche. Cette optimisation concerne l'épaisseur initiale, mais aussi la puissance électrique fournie pendant le dépôt. Nous avons développé un algorithme de traitement d'images pour quantifier la qualité d'un réseau de particules. Ceci nous permet de proposer des paramètres optimaux pour une texture donnée.\par 
Des particules d'argent réparties de manière aléatoire à la surface d'un substrat de silice donne lieu à un pic d'absorption centré à 420~nm. La réponse optique des réseaux organisés de particules diffère principalement par l'apparition d'un nouveau pic d'absorption. La position de ce pic dépend de la période du réseau avec une relation linéaire. Ce pic dépend également de la qualité du réseau obtenu par démouillage~: plus la qualité du réseau est grande, plus ce pic sera fin et intense. Le pic observé à 420~nm a une position constante.\par 
Des simulations numériques par éléments finis modélisant nos systèmes nous ont permis de discuter de la nature physique des différents pics d'absorption observés expérimentalement. Nous attribuons le pic qui ne se déplace pas à une LSPR des particules. Le second pic est attribué à l'absorption d'un pic de diffraction créé par le réseau et se propageant parallèlement à la surface. Son chemin optique dans la couche métallique absorbante est augmenté, menant donc à son absorption. Ce pic est centré sur une longueur d'onde égale à la période du réseau multipliée par l'indice du milieu ($\lambda_{abs}=n_{i}P$. Les particules étant à l'interface entre deux milieux, deux pics sont donc attendus ($i$=silice ou $i$=air). Les données expérimentales nous permettent d'observer le pic relatif au substrat uniquement.\par \vspace{12pt}

En termes d'applications, le contrôle des propriétés optiques des réseaux de particules offre de nouvelles perspectives. Le fait de pouvoir créer de l'absorption à une longueur d'onde désirée est intéressant, par exemple dans le cadre de l'amélioration de l'efficacité des cellules photovolatïques. Les cellules usuelles peuvent être limitées par un manque de conversion de lumière dans le proche infrarouge, où le spectre solaire est intense. Augmenter l'absorption, grâce à des particules à l'interface de la phase active, précisément dans la gamme visée permettrait d'augmenter l'efficacité globale de la cellule.\par 
Pour d'autres applications industrielles (notamment les vitrages bas-émissifs), créer de l'absorption dans le proche infrarouge est un enjeu important. Cependant, la présence d'absorption dans la gamme visible dans nos systèmes peut être un obstacle. Toutefois, nous avons montré que cette absorption dépend de la nature du matériau employé, contrairement aux modes créés par le réseau. Il serait envisageable de moduler séparément ces deux absorptions. D'autres pistes d'expérimentation sont envisageables~: une texture anisotrope (un paramètre de réseau différent selon la direction du plan) devrait donner lieu à deux pics d'absorption différents selon notre modèle. Changer le matériau qui compose le substrat permet également de modifier la position du mode du substrat.\par 
Le principal défi est lié à l'utilisation du démouillage, qui ne permet pas l'obtention de n'importe quel réseau. Dans l'exemple des réseaux anisotropes, les deux périodes ne peuvent pas être trop différentes~: nous avons vu que l'optimisation de l'épaisseur initiale de la couche métallique était dépendante de la texture. Si les deux périodes s'écartent trop, il ne sera plus possible d'avoir une épaisseur correspondant aux deux. Il conviendra alors de trouver le meilleur compromis.\par 


\newpage
\bibliographystyle{ieeetr}
\bibliography{biblio}


%%%%%%%%%%%%%%%%%%%%%%%%%%%%%%%%%%%%%%%%%%%%%%%%%%%%%%
%%
%% Chapitre 6 : Conclusion
%%
%%%%%%%%%%%%%%%%%%%%%%%%%%%%%%%%%%%%%%%%%%%%%%%%%%%%%%


\chapter*{Conclusion}
\begin{flushleft}
\textit{All models are wrong, but some are usefull.}
\end{flushleft}
\begin{flushright}
\textbf{George Box}, \textit{Robustness in the strategy of scientific model building, 1979.}
\end{flushright}
\pagestyle{Conclusion}
Notre travail était orienté selon deux axes~: la compréhension et le contrôle du démouillage à l'état solide des couches d'argent. Nous allons discuter de ces deux aspects l'un après l'autre.\par 
Nous avons mis en évidence lors de notre revue bibliographique la grande diversité des systèmes étudiés pendant le démouillage, ainsi que la diversité des morphologies de démouillage observées. Nous nous plaçons dans le cadre du démouillage des couches polycristallines. Il est usuellement décrit en deux étapes~: l'induction et la propagation des trous. Pendant l'induction, les grains de la couche grossissent, les joints de grains forment des sillons de plus en plus profonds et finalement des trous. Pendant le propagation, ces trous se propagent. L'évolution du front de démouillage est discutée en termes de courbure locale. Ceci donne lieu à la formation d'un bourrelet de démouillage~: le front est plus épais que le reste de la couche. Le moteur du démouillage est supposé être la réduction de l'énergie de surface de la couche métallique.\par 
Seuls quelques modèles récents font état du rôle des grains en tant qu'entités discrètes pendant le démouillage. Ces modèles impliquent également l'existence d'un bourrelet (constitué de grains). Ils prennent en compte la réduction de l'énergie à l'interface métal-substrat en plus de la réduction de l'énergie de surface dans la force motrice du démouillage.\par 
Nous avons mené des expériences \textit{in situ} de suivi du démouillage par plusieurs méthodes~: l'AFM, le MEB et l'éllipsométrie. Dans un premier temps, nous avons procédé à cette étude dans des atmopshères contenant de l'oxygène (air ou 400~Pa). Dans ce cadre, nous avons observé que certains grains spécifiques entraient en croissance extraordinaire dès l'induction. Ces grains accumulent tout le matériau provenant du démouillage et pilotent la morphologie de la couche lors de sa rétraction. Ceci implique qu'il n'y a pas de bourrelet de propagation~; la propagation des trous est un transport de matériau entre les grains consommés au niveau du front vers les grains en croissance. Le reste de la couche n'est pas modifié pendant la propagation, il n'y a pas de bourrelet. Les grains qui croissent donnent lieu aux particules observées à la fin du démouillage.\par 
Nous avons également pu observer que la propagation des trous ne donnait pas lieu à un état final, au sens où les particules obtenues à ce moment-là n'avaient pas la forme attendue par des conceptions thermodynamiques. Il existe en réalité un troisième régime dans le démouillage, que nous appelons frittage. Le frittage a une cinétique très lente, il correspond à l'évolution des particules d'argent isolées vers leur forme d'équilibre (ou du moins une forme plus proche de l'équilibre attendu). Le début du frittage est marqué par un ralentissement de la cinétique de l'évolution du taux de couverture. En termes de morphologie, c'est le moment où il ne reste plus aucun grain \og intact\fg{}: tous on soit disparu, soit crû. \par 
En outre, nous avons observé que le démouillage avait un impact sur le substrat~: celui-ci se déformait au niveau du front lorsque ce dernier marquait le début du régime de frittage.\par 
Nous avons procédé à une comparaison du démouillage en faisant varier l'atmosphère de recuit. Nous avons démontré que l'absence d'oxygène changeait radicalement la morphologie et la cinétique du démouillage. Sans oxygène, nous n'observons pas de croissance extréordinaire de certains grains~: plus de grains peuvent croitre, mais de manière modérée. Ces grains bloquent localement la propagation du front, lui donnant une forme dendritique.\par 
Nous avons développé et appliqué une procédure de traitement d'image pour mesurer la courbure locale du front, ainsi que la vitesse locale de propagation. Les données ainsi obtenues nous ont permis de démontrer que la courbure locale de la surface n'est pas le moteur de la propagation. Nous proposons que la force motrice de la propagation est la réorganisation des grains dans la couche qui se poursuit. \par 
L'influence de l'oxygène tient dans le fait qu'il abaisse l'énergie de surface de l'argent. Ceci permet une nucléation de trous plus aisée, et augmente ainsi la densité de trou observée lors du démouillage. Une étude en variations progressive de la pression partielle d'oxygène montre que cet effet a lieu dès des pressions de $10^{-7}$~atm ($10^{-2}$~Pa). Cette réduction d'énergie étant anisotrope, elle peut participer à l'émergence de croissance extraordinaire que nous n'observons que sous oxygène.\par 
D'un point de vue industriel, l'argent est systématiquement encapsulé dans des empilements. Nous avons cependant démontré que l'oxygène pouvait être apporté par la sous-couche. Ainsi, l'observation de défauts de type \og dômes \fg{} semble indiquer une présence de croissance extraordinaire caractéristique du démouillage sous oxygène. Au contraire, un démouillage de type \og dendrites \fg{} est au contraire similaire au démouillage en absence d'oxygène. Cela tend à montrer que la stabilité des couches d'argent dans les empilements est fortement dépendante de la composition des autres couches, même si elles ne sont pas directement en contact. La présence d'oxygène pouvant diffuser dans l'empilement est un facteur promoteur du démouillage.\par 
Nous avons appliqué ces connaissances au démouillage sur des surfaces texturées. Nos expériences étaient en désaccord avec le modèle proposé par Giermann et Thompson~\cite{giermann2005solid, giermann2011requirements}, qui prédit l'impossibilité d'une organisation des particules métalliques si le volume de métal excède le volume des trous accueillant les particules. Notre modèle implique que c'est la présence d'un grain en croissance extraordinaire au sein d'un trou qui détermine si une particule s'y trouve à la fin. Ceci permet alors d'avoir des particules de volume supérieur au volume des trous de la texture tout en gardant une très bonne organisation spatiale.\par 
Cette connaissance nous a permis d'optimiser les dépôt afin de fabriquer les meilleurs réseaux de particules possibles. Cette optimisation repose qsur trois facteurs~: l'épaisseur de la couche initiale, la puissance électrique fournie à la cathode lors du dépôt et la pression de gaz neutre constituant le plasma du dépôt.\par
Nous avons observé que les particules obtenues sur surface texturées épousent la forme des trous dans lesquels elles se situent. Elles présentent donc une partie en forme de pyramide, surmontée d'une hémisphère. Cette forme est différente de ce qu'observaient Giermann et Thompson.\par 
Nous avons mesuré la réponse optique des particules obtenues. Sur une surface plane, si l'on étudie l'absorption, seul un pic à 420~nm est visible. Sur une surface texturée, un deuxième pic émerge. La finesse de ce pic est dépendante de la qualité du réseau de particules, et sa position dépend de la période du réseau avec une relation linéaire. \par 
Des simulation optiques ont consolidé notre étude expérimentale. Elles retrouvent les même comportements, et indiquent la présence d'un troisième pic, lui aussi dépendant linéairement de la période du réseau. Ces simulations et ces observations nous permettent d'avancer que les pics d'absorption observés sont dûs au couplage de la lumière, via la diffraction, aux particules.\par 
Ce système est prometteur en termes d'applications~: il permet de positionner très précisément un pic d'absorption. Par exemple, en le positionnant dans l'IR, ceci permettrait de renforcer une propriétés de basse-émissivité d'un système. Pour le photovoltaïque, cette approche permettrait également de positionner un pic d'absorption en fonction du spectre solaire, afin de convertir le maximum de lumière en paires électrons-trous.\par 
\vspace{12pt}
La configuration de cette thèse était originale, puisqu'elle abordait la question du démouillage par un angle industriel. Cette approche imposait un système d'étude loin des cas modèles (une couche d'argent polycristalline sur une substrat amorphe), mais c'est précisément cette distance à l'idéalité qui nous a permis d'améliorer la compréhension du démouillage. En effet, observer d'aussi grandes différences morphologiques lors du démouillage en changeant seulement d'atmosphère est un résultat nouveau. Ceci nous a permis de proposer un nouveau modèle du démouillage.\par 
La connaissance du phénomène est précieuse en vue de l'amélioration des propriétés optiques des réseaux de particules obtenus par démouillage. Nous avons montré que le pic d'absorption généré par le réseau est indépendant du matériau qui constitue les particules. Ainsi, la recherche d'un autre métal à faire démouiller permettrait, tout en gardant le pic du réseau dont la position est contrôlée, de faire disparaître l'absorption dans les visible.\par 
Nous disposons d'un bon panorama des paramètres à prendre en compte en vue d'applications industrielles. À l'heure actuelle, la principale limite est la texturation de surface, qui ne s'applique pas à l'échelle indsutrielle (les dimensions typiques sont 6x3~m$^2$). Si ce verrou technologique est amené à disparaitre, nous disposons d'une base de travail solide pour optimiser la production de systèmes à forte valeur ajoutée. Les applications les plus immédiates correspondent au Bas-Émissif ainsi qu'aux cellules photovoltaïque.\par 

\vspace{12pt}
Du point de vue de la compréhension fondamentale, certaines zones d'ombre demeurent. Pourquoi certains grains croissent plutôt que d'autres, notamment. Il peut s'agir d'une délicate compétition de facteurs locaux (énergies de surface et d'interface au joints de grains) dont la mesure expérimentale est ardue. Notons de plus que la microscopie électronique, à ces échelles-là, perturbe beaucoup le phénomène. Des compléments de compréhension pourraient être apportés par des modélisations. La difficulté réside dans l'aspect multi-échelle du phénomène, aussi les simulations devront-elles sans doute être issues de différentes méthodes. Pour le moment, une simulation de type Vertex~\cite{weygand1998vertex} qui permet de prendre en compte la structure cristalline de la couche semble être un piste prometteuse.\par 
Concernant l'optique des réseaux, nous avons trouvé un modèle satisfaisant, mais la manière la plus directe de montrer sa pertinence serait de réaliser des études avec des angles d'incidence variables. Une étude dans ce sens, comme par exemple ce que l'ellipsométrie permet de réaliser~\cite{brakstad2015dispersion}, apporterait beaucoup d'éléments de compréhension.\par  
\newpage
\bibliographystyle{ieeetr}
\bibliography{biblio}

%


\end{document}